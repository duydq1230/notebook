\chapter{Service-Oriented Architecture}

View online \href{http://magizbox.com/training/computer_science/site/software_architecture/}{http://magizbox.com/training/computer_science/site/software_architecture/}

A service-oriented architecture (SOA) is an architectural pattern in computer software design in which application components provide services to other components via a communications protocol, typically over a network. The principles of service-orientation are independent of any vendor, product or technology. 2



Generally accepted view 1
Boundaries are explicit
Services are autonomous
Services share schema and contract, not class
Service compatibility is based on policy
Microservices
In computing, microservices is a software architecture style in which complex applications are composed of small, independent processes communicating with each other using language-agnostic APIs. These services are small building blocks, highly decoupled and focussed on doing a small task, facilitating a modular approach to system-building. One of concepts which integrates microservices as a software architecture style is dew computing. 1



Properties 2
Each running in its own process
Communicating with lightweight mechanisms, often an HTTP resource API
Build around business capabilities
Independently deployable
fully automated deployment
Maybe in a different programming language and use different data storage technologies
Monolith vs Microservice
Monolith	Microservice
Simplicity	Partial Deployment
Consistency	Availability
Inter-module refactoring	Preserve Modularity
Multiple Platforms
Benefits 4
Their small size enables developers to be most productive.
It's easy to comprehend and test each service.
You can correctly handle failure of any dependent service.
They reduce impact of correlated failures.
Web Service
RESTful API


REST Client
Sense (Beta)

A JSON aware developer console to ElasticSearch.

API Document and Client Generator
http://swagger.io/swagger-editor/

API Client
CRUD Pet

API 	 Client
Method	URL	Body	Return Body	Method
 GET	/pets	 	[Pet]	PetApi.list()
 POST	/pets/	Pet	Pet	PetApi.create(pet)
 GET	/pets/pet_id	 	Pet	PetApi.get(pet_id)
 PUT	 /pets/pet_id	Pet	Pet	PetApi.update(pet_id, pet)
 DELETE	/pets/pet_id	 	 	PetApi.delete(pet_id)


CRUD Store

GET /stores	StoreApi.list()
...	...
Relationships

Many to many

### Example [https://api.facebook.com/method/links.getStats?urls=%%URL%%&format=json](https://api.facebook.com/method/links.getStats?urls=%%URL%%&format=json)
Microservices ↩

Slide 11/42, Micro-servies ↩

Martin Fowler, Microservices, youtube ↩

Rick E. Osowski, Microservices in action, Part 1: Introduction to microservices, IBM developerworks ↩