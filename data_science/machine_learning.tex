\chapter{Học máy}

\begin{itemize}
  \item Vấn đề với HMM và CRF?
  \item Học MLE và MAP?
\end{itemize}


Machine Learning
Machine learning is a branch of science that deals with programming the systems in such a way that they automatically learn and improve with experience. Here, learning means recognizing and understanding the input data and making wise decisions based on the supplied data.

We can think of machine learning as approach to automate tasks like predictions or modelling. For example, consider an email spam filter system, instead of having programmers manually looking at the emails and coming up with spam rules. We can use a machine learning algorithm and feed it input data (emails) and it will automatically discover rules that are powerful enough to distinguish spam emails.

Machine learning is used in many application nowadays like spam detection in emails or movie recommendation systems that tells you movies that you might like based on your viewing history. The nice and powerful thing about machine learning is: It learns when it gets more data and hence it gets more and more powerful the more data we give them.

**Có bao nhiêu thuật toán Machine Learning?**

Có rất nhiều thuật toán Machine Learning, bài viết [Điểm qua các thuật toán Machine Learning hiện đại](https://ongxuanhong.wordpress.com/2015/10/22/diem-qua-cac-thuat-toan-machine-learning-hien-dai/) của Ông Xuân Hồng tổng hợp khá nhiều thuật toán. Theo đó, các thuật toán Machine Learning được chia thành các nhánh lớn như `regression`, `bayesian`, `regularization`, `decision tree`, `instance based`, `dimesionality reduction`, `clustering`, `deep learning`, `neural networks`, `associated rule`, `ensemble`... Ngoài ra thì còn có các cheatsheet của [sklearn](http://scikit-learn.org/stable/tutorial/machine_learning_map/index.html).

Việc biết nhiều thuật toán cũng giống như ra đường mà có nhiều lựa chọn về xe cộ. Tuy nhiên, quan trọng là có task để làm, sau đó thì cập nhật SOTA của task đó để biết các công cụ mới.

**Xây dựng model cần chú ý điều gì?**

Khi xây dựng một model cần chú ý đến vấn đề tối ưu hóa tham số (có thể sử dụng [GridSearchCV](sklearn.model_selection.GridSearchCV))

Bài phát biểu này có vẻ cũng rất hữu ích [PYCON UK 2017: Machine learning libraries you'd wish you'd known about](https://www.youtube.com/watch?v=nDF7_8FOhpI). Có đề cập đến

* [DistrictDataLabs/yellowbrick](https://github.com/DistrictDataLabs/yellowbrick) (giúp visualize model được train bởi sklearn)
* [marcotcr/lime](https://github.com/marcotcr/lime) (giúp inspect classifier)
* [TeamHG-Memex/eli5](https://github.com/TeamHG-Memex/eli5) (cũng giúp inspect classifier, hỗ trợ nhiều model như xgboost, crfsuite, đặc biệt có TextExplainer sử dụng thuật toán từ eli5)
* [rhiever/tpot](https://github.com/rhiever/tpot) (giúp tối ưu hóa pipeline)
* [dask/dask](https://github.com/dask/dask) (tính toán song song và lập lịch)

Ghi chú về các thuật toán trong xử lý ngôn ngữ tự nhiên tại [underthesea.flow/wiki](https://github.com/magizbox/underthesea.flow/wiki/Develop)

Framework để train, test hiện tại vẫn rất thoải mái sklearn. tensorboard cung cấp phần log cũng khá hay.

[Câu trả lời hay](https://www.quora.com/What-are-the-most-important-machine-learning-techniques-to-master-at-this-time/answer/Sean-McClure-3?srid=5O2u) cho câu hỏi [Những kỹ thuật machine learning nào quan trọng nhất để master?](https://www.quora.com/What-are-the-most-important-machine-learning-techniques-to-master-at-this-time), đặc biệt là dẫn đến bài [The State of ML and Data Science 2017](https://www.kaggle.com/surveys/2017) của Kaggle.

**Tài liệu học PGM**

[Playlist youtube](https://www.youtube.com/watch?v=WPSQfOkb1M8&amp;list=PL50E6E80E8525B59C) khóa học Probabilistic Graphical Models của cô Daphne Koller. Ngoài ra còn có một [tutorial](http://mensxmachina.org/files/software/demos/bayesnetdemo.html) dở hơi ở đâu về tạo Bayesian network

**[Chưa biết] Tại sao Logistic Regression lại là Linear Model?**

Trong quyển Deep Learning, chương 6, trang 165, tác giả có viết

```
Linear models, such as logistic regression and linear
regression, are appealing because they can be fit
efficiently and reliably, either in closed form or
with convex optimization
```

Mình tự hỏi tại sao logistic regression lại là linear, trong khi nó có sử dụng hàm logit (nonlinear)? Tìm hiểu hóa ra cũng có bạn hỏi giống mình trên [stats.stackexchange.com](https://stats.stackexchange.com/questions/93569/why-is-logistic-regression-a-linear-classifier). Ngoài câu trả lời trên stats.stackexchange, đọc một số cái khác [Generalized Linear Models, SPSS Statistics 22.0.0](https://www.ibm.com/support/knowledgecenter/en/SSLVMB_22.0.0/com.ibm.spss.statistics.help/spss/advanced/idh_idd_genlin_typeofmodel.htm)
 và [6.1 - Introduction to Generalized Linear Models, Analysis of Discrete Data, Pennsylvania State University](https://onlinecourses.science.psu.edu/stat504/node/216) cũng vẫn chưa hiểu lắm.

Hiện tại chỉ hiểu là các lớp model này chỉ có thể hoạt động trên các tập linear separable, có lẽ do việc map input x, luôn có một liên kết linear $latex wx$, trước khi đưa vào hàm non-linear.

**Các tập dữ liệu thú vị**

*Iris dataset*: dữ liệu về hoa iris

Là một ví dụ cho bài toán phân loại

*Weather problem*: dữ liệu thời tiết. Có thể tìm được ở trong quyển Data Mining: Practical Machine Learning Tools and Techniques

Là một ví dụ cho bài toán cây quyết định

## Deep Learning

**Tài liệu Deep Learning**

Lang thang thế nào lại thấy trang này [My Reading List for Deep Learning!](https://www.microsoft.com/en-us/research/wp-content/uploads/2017/02/DL_Reading_List.pdf) của một anh ở Microsoft. Trong đó, (đương nhiên) có Deep Learning của thánh Yoshua Bengio, có một vụ hay nữa là bài review "Deep Learning" của mấy thánh Yann Lecun, Yoshua Bengio, Geoffrey Hinton trên tạp chí Nature. Ngoài ra còn có nhiều tài liệu hữu ích khác.

### Các layer trong deep learning [^2]

#### Sparse Layers

[**nn.Embedding**](http://pytorch.org/docs/master/nn.html#embedding) ([hướng dẫn](http://pytorch.org/tutorials/beginner/nlp/word_embeddings_tutorial.html))
★ grep code: [Shawn1993/cnn-text-classification-pytorch](https://github.com/Shawn1993/cnn-text-classification-pytorch/blob/master/model.py#L18)
Đóng vai trò như một lookup table, map một word với dense vector tương ứng

#### Convolution Layers

[**nn.Conv1d**](http://pytorch.org/docs/master/nn.html#conv1d), [**nn.Conv2d**](http://pytorch.org/docs/master/nn.html#conv2d), [**nn.Conv3d**](http://pytorch.org/docs/master/nn.html#conv3d) [^1]
★ grep code: [Shawn1993/cnn-text-classification-pytorch](https://github.com/Shawn1993/cnn-text-classification-pytorch/blob/master/model.py#L20-L24), [galsang/CNN-sentence-classification-pytorch](https://github.com/galsang/CNN-sentence-classification-pytorch/blob/master/model.py#L36-L38)

Các tham số trong Convolution Layer

* `kernel_size` (hay là filter size)

Đối với NLP, kernel_size thường bằng region_size * word_dim (đối với conv1d) hay (region_size, word_dim) đối với conv2d

<small>Quá trình tạo feature map đối với region size bằng 2</small>
![](https://media.giphy.com/media/l2QE2y1UQP7vIgiti/giphy.gif)

* `in_channels`, `out_channels` (là số lượng `feature maps`)

Kênh (channels) là các cách nhìn (view) khác nhau đối với dữ liệu. Ví dụ, trong ảnh thường có 3 kênh RGB (red, green, blue), có thể áp dụng convolution giữa các kênh. Với văn bản cũng có thể có các kênh khác nhau, như khi có các kênh sử dụng các word embedding khác nhau (word2vec, GloVe), hoặc cùng một câu nhưng biểu diễn ở các ngôn ngữ khác nhau.

* `stride`

Định nghĩa bước nhảy của filter.

![](http://d3kbpzbmcynnmx.cloudfront.net/wp-content/uploads/2015/11/Screen-Shot-2015-11-05-at-10.18.08-AM-1024x251.png)

Hình minh họa sự khác biệt giữa các feature map đối với stride=1 và stride=2. Feature map đối với stride = 1 có kích thước là 5, feature map đối với stride = 3 có kích thước là 3. Stride càng lớn thì kích thước của feature map càng nhỏ.

Trong bài báo của Kim 2014, `stride = 1` đối với `nn.conv2d` và `stride = word_dim` đối với `nn.conv1d`

Toàn bộ tham số của mạng CNN trong bài báo Kim 2014,

![](http://d3kbpzbmcynnmx.cloudfront.net/wp-content/uploads/2015/11/Screen-Shot-2015-11-06-at-8.03.47-AM.png)

| Description         | Values          |
|---------------------|-----------------|
| input word vectors  | Google word2vec |
| filter region size  | (3, 4, 5)       |
| feature maps        | 100             |
| activation function | ReLU            |
| pooling             | 1-max pooling   |
| dropout rate        | 0.5             |
| $latex l&amp;s=2$2 norm constraint  | 3               |

Đọc thêm:

* [Lecture 13: Convolutional Neural Networks (for NLP). CS224n-2017](http://web.stanford.edu/class/cs224n/lectures/cs224n-2017-lecture13-CNNs.pdf)
* [DeepNLP-models-Pytorch - 8. Convolutional Neural Networks](https://nbviewer.jupyter.org/github/DSKSD/DeepNLP-models-Pytorch/blob/master/notebooks/08.CNN-for-Text-Classification.ipynb)
* [A Sensitivity Analysis of (and Practitioners’ Guide to) Convolutional Neural Networks for Sentence Classification. Zhang 2015](https://arxiv.org/pdf/1510.03820.pdf)

**BTS**

22/11/2017 - Phải nói quyển này hơi nặng so với mình. Nhưng thôi cứ cố gắng vậy.
24/11/2017 - Từ hôm nay, mỗi ngày sẽ ghi chú một phần (rất rất nhỏ) về Deep Learning [tại đây](https://docs.google.com/document/d/1KxDrw5s6uYHNLda7t0rhp0RM_TlUGxydQ-Qi1JOPFr8/edit?usp=sharing)

[^1]: [Understanding Convolutional Neural Networks for NLP](http://www.wildml.com/2015/11/understanding-convolutional-neural-networks-for-nlp)
[^2]: [http://pytorch.org/docs/master/nn.html](http://pytorch.org/docs/master/nn.html)

\section{Machine Learning Process}

The good life is a process, not a state of being. It is a direction not a destination.

Carl Rogers



I searched a framework fit for every data mining task, I found a good one from an article of Oracle.

And here is my summary. The data mining process has 4 steps:

Step 1. Problem Definition

This initial phase of a data mining project focuses on understanding the project objectives and requirements. Once you have specified the project from a business perspective, you can formulate it as a data mining problem and develop a preliminary implementation plan.

Step 2. Data Gathering & Preparation

The data understanding phase involves data collection and exploration. As you take a closer look at the data, you can determine how well it addresses the business problem. You might decide to remove some of the data or add additional data. This is also the time to identify data quality problems and to scan for patterns in the data.

 Data Access
 Data Sampling

 Data Transformation

Data in the real world is dirty [3]. They are often incomplete (lacking attribute values, lacking certain attributes of interest, or containing only aggregate data), noisy (containing errors or outliers),‰ inconsistent (containing discrepancies in codes or names). Step 3. Model Building In this phase, you select and apply various modeling techniques and calibrate the parameters to optimal values. If the algorithm requires data transformations, you will need to step back to the previous phase to implement them

 Create Model
 Test Model

  Evaluate & Interpret Model

Some important questions [2]:

Is at least one of predictors useful in predicting the response? (F-statistics)
Do all the predictors help to explain Y, or is only a subset of the predictors useful? (all subsets or best subsets)
How well does the model fit the data?
Given a set of predictor values, what response value should we predict, and how accurate is our prediction?
Step 4. Knowledge Deployment Knowledge deployment is the use of data mining within a target environment. In the deployment phase, insight and actionable information can be derived from data.
Model Apply
Custom Reports
External Applications
References
The Data Mining Process, Oracle
Trevor Hastie and Rob Tibshirani, Model Selection and Qualitative Predictors, URL:https://www.youtube.com/watch?v=3T6RXmIHbJ4
Nguyen Hung Son, Data cleaning and Data preprocessing, URL:http://www.mimuw.edu.pl/~son/datamining/DM/4-preprocess.pdf

\subsection{Problem Definition}

This initial phase of a data mining project focuses on understanding the project objectives and requirements. Once you have specified the project from a business perspective, you can formulate it as a data mining problem and develop a preliminary implementation plan.

For example, your business problem might be: "How can I sell more of my product to customers?" You might translate this into a data mining problem such as: "Which customers are most likely to purchase the product?" A model that predicts who is most likely to purchase the product must be built on data that describes the customers who have purchased the product in the past. Before building the model, you must assemble the data that is likely to contain relationships between customers who have purchased the product and customers who have not purchased the product. Customer attributes might include age, number of children, years of residence, owners/renters, and so on.

