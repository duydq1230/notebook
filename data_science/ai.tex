\chapter{Trí tuệ nhân tạo}

View online \href{http://magizbox.com/training/ai/site/}{http://magizbox.com/training/ai/site/}

Artificial intelligence (AI) is the intelligence exhibited by machines or software. It is also the name of the academic field of study which studies how to create computers and computer software that are capable of intelligent behavior. Major AI researchers and textbooks define this field as "the study and design of intelligent agents", in which an intelligent agent is a system that perceives its environment and takes actions that maximize its chances of success. John McCarthy, who coined the term in 1955, defines it as "the science and engineering of making intelligent machines".

\section{Autonomous Agents}

limited ability to perceive its environment
process the environment and calculate an action
no global plan / leader
Vehicles

Action / Selection
Steering
Locomotion
Steering Behavior 1 2


Steering = Desired - Velocity

Seek
Flow Filed Following
Path Following
Group Steering
https://github.com/shiffman/The-Nature-of-Code-Examples/tree/master/chp06_agents\

Massive Battle: Coordinated Movement of Autonomous Agents ↩

Craig Reynolds, Steering Behaviors For Autonomous Characters ↩

\section{Cellular Automator}

https://www.youtube.com/watch?v=DKGodqDs9sA&index=1&list=PLRqwX-V7Uu6YrWXvEQFOGbCt6cX83Xunm

Cellular Automata

Grid of cell
Each cell has state, neighborhood
cell state at time t defined by a function of neighborhood states at time t-1
Elementary Cellular Automata

\section{Fractal}

L-System

\section{The Pac-Man project}

Today I found an interesting AI project - The Pac-Man

http://ai.berkeley.edu/images/pacman_game.gif

Here is the project overview

The Pac-Man projects were developed for UC Berkeley's introductory artificial intelligence course, CS 188. They apply an array of AI techniques to playing Pac-Man. However, these projects don't focus on building AI for video games. Instead, they teach foundational AI concepts, such as informed state-space search, probabilistic inference, and reinforcement learning. These concepts underly real-world application areas such as natural language processing, computer vision, and robotics. We designed these projects with three goals in mind. The projects allow students to visualize the results of the techniques they implement. They also contain code examples and clear directions, but do not force students to wade through undue amounts of scaffolding. Finally, Pac-Man provides a challenging problem environment that demands creative solutions; real-world AI problems are challenging, and Pac-Man is too. In our course, these projects have boosted enrollment, teaching reviews, and student engagement. The projects have been field-tested, refined, and debugged over multiple semesters at Berkeley. We are now happy to release them to other universities for educational use.
In the next part of this post, I will show my works on this project

Project 1: Search in Pacman

[caption id="" align="alignleft" width="231"]DFS[/caption]

[caption id="" align="alignleft" width="233"]BFS[/caption]