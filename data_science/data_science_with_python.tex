\chapter{Data Science with Python}

View online \href{http://magizbox.com/training/ml_data_python/site/}{http://magizbox.com/training/ml_data_python/site/}

The ability to analyze data with Python is critical in data science. Learn the basics, and move on to create stunning visualizations.

\section{Get Started}

Get Started with Ubuntu
Requirements

numpy, scipy
matplotlib
pandas
scikit-learn
ipython
Install pip

sudo apt-get install python-pip
Install numpy scipy

sudo apt-get install python-numpy python-scipy \
    python-matplotlib python-pandas python-sympy python-nose
Install scikit-learn

pip install jupyter ipython pip install -U scikit-learn

\section{Data Transformation}

DataFrame is a 2-dimensional labeled data structure with columns of potentially different types. You can think of it like a spreadsheet or SQL table, or a dict of Series objects. It is generally the most commonly used pandas object

Create data frame
Create new data frame from lists

\begin{lstlisting}[language=Python]
import pandas as pd
students = pd.DataFrame({
  "name" : ["Kate", "John", "Tom", "Mark"],
  "age" : [20, 21, 19, 18]
})
#       age  name
#    0   20  Kate
#    1   21  John
#    2   19   Tom
#    3   18  Mark
\end{lstlisting}

Load dataframe
Load dataframe from datasets


\begin{lstlisting}[language=Python]
import pandas as pd
from sklearn import datasets
iris_data = datasets.load_iris()
iris = pd.DataFrame(data=iris_data.data,
                    columns=iris_data.feature_names)
iris

\end{lstlisting}

Selection

Select by column index

\begin{lstlisting}[language=Python]
students.iloc[1:3, :]
#       age  name
#    1   21  John
#    2   19   Tom
\end{lstlisting}

Filter

\begin{lstlisting}[language=Python]
students = pd.DataFrame({
  'math' : [90, 80, 95, 50],
  'physic' : [20, 50, 95, 60]
})
#    math  physic
# 0    90      20
# 1    80      50
# 2    95      95
# 3    50      60

students[students['math'] > 85]
#    math  physic
# 0    90      20
# 2    95      95

students[students['math'] == students['physic']]
#    math  physic
# 2    95      95
\end{lstlisting}

Create new column

\begin{lstlisting}[language=Python]
students = pd.DataFrame({
  'name' : ["Kate", "John", "Tom", "Mark"],
  'age' : [20, 21, 19, 18]
})
students["birthyear"] = students.apply(
  lambda row: 2016 - row['age'], axis=1)
students["birthyear"] = 2016 - students["age"]
#       age  name  birthyear
#    0   20  Kate       1996
#    1   21  John       1995
#    2   19   Tom       1997
#    3   18  Mark       1998
\end{lstlisting}

Delete column

\begin{lstlisting}[language=Python]
students = pd.DataFrame({
  'name' : ["Kate", "John", "Tom", "Mark"],
  'age' : [20, 21, 19, 18]
})
students = students.drop('age', 1)
\end{lstlisting}

References

Wes McKinney, 10-minute tour of pandas: video, notebook

DataFrame, Intro to Data Structures

\section{Data Preperation}

Normalization

Example

\begin{lstlisting}[language=Python]
import numpy
from sklearn.preprocessing import normalize
matrix = numpy.arange(0,27,3).reshape(3,3).astype(numpy.float64)

# array([[  0.,   3.,   6.],
#   [  9.,  12.,  15.],
#   [ 18.,  21.,  24.]])

normed_matrix = normalize(matrix, axis=1, norm=‘l1’)

# [[ 0.          0.33333333  0.66666667]
# [ 0.25        0.33333333  0.41666667]
# [ 0.28571429  0.33333333  0.38095238]]
\end{lstlisting}

Label Encoder

Encode labels (categorical variables) with value between 0 and n_classes-1.

\begin{lstlisting}[language=Python]
import sklearn
le = sklearn.preprocessing.LabelEncoder()
le.fit([“paris”, “paris”, “tokyo”, “amsterdam”])
# LabelEncoder()
list(le.classes_)
# [‘amsterdam’, ‘paris’, ‘tokyo’]
le.transform([“tokyo”, “tokyo”, “paris”])
# array([2, 2, 1]...)
list(le.inverse_transform([2, 2, 1]))
# [‘tokyo’, ‘tokyo’, ‘paris’]
\end{lstlisting}

References

How to normalize a 2-dimensional numpy array in python less verbose?
sklearn.preprocessing.LabelEncoder

\section{Data IO}

This post shows how to import data to Python from numerous resources

CSV

Read a csv file from local or from a server

\begin{lstlisting}[language=Python]
import numpy as np
import pandas as pd
# read data
df = pd.read_csv(“data.csv”, header = 0)
# write data
df.to_csv(“data.csv”, header=1, index=False)
\end{lstlisting}

Excel

\begin{lstlisting}[language=Python]
import pandas as pd
# read data
df = pd.read_excel(“data.xls”)
# write data
df = pd.to_excel(“data.xls”, index=False)
\end{lstlisting}

Sqlite

\begin{lstlisting}[language=Python]
import sqlite3

DB_NAME = “db.sqlite3”
SELECT_QUERY = “SELECT page_id, type FROM service_page”
# connect to sqlite3
db_connector = sqlite3.connect(DB_NAME)
# excute query
cursor = db_connector.execute(SELECT_QUERY)
# return dataset
data_set = cursor.fetchall()
\end{lstlisting}

References

pandas.read_excel
pandas.read_sqlite
sqlite3.read_sqlite

\section{Numpy}

NumPy

Use the following import convention:

\begin{lstlisting}[language=Python]
import numpy as np
\end{lstlisting}

Creating Arrays

\begin{lstlisting}[language=Python]
a = np.array([1, 2, 3])
b = np.array([(1.5, 2, 3), (4, 5, 6)], dtype=float)
c = np.array([[(1.5, 2, 3), (4, 5, 6)], [(3, 2, 1), (4, 5, 6)]], dtype=float)
\end{lstlisting}

Initial Placeholders

\begin{lstlisting}[language=Python]
# Create an array of zeros
np.zeros((3, 4))
array([[ 0., 0., 0., 0.], [ 0., 0., 0., 0.], [ 0., 0., 0., 0.]])
\end{lstlisting}

# Create an array of ones

\begin{lstlisting}[language=Python]
np.ones((2, 3, 4), dtype=np.int16)

    array([[[1, 1, 1, 1],
            [1, 1, 1, 1],
            [1, 1, 1, 1]],

           [[1, 1, 1, 1],
            [1, 1, 1, 1],
            [1, 1, 1, 1]]], dtype=int16)

# Create an array of evenly spaced values (step value)
np.arange(10, 25, 5)
array([10, 15, 20])
# Create an array of evenly spaced values (number of samples)
np.linspace(0, 2, 9)
array([ 0. , 0.25, 0.5 , 0.75, 1. , 1.25, 1.5 , 1.75, 2. ])
# Create a constant array
np.full((2, 2), 7)
# array([[ 7., 7.], [ 7., 7.]])

# Create a 2x2 identity matirx
np.eye(2)
array([[ 1., 0.], [ 0., 1.]])
# Create an array with random values
np.random.random((2, 2))
array([[ 0.11121701, 0.12191919], [ 0.61608418, 0.91899253]])
# Create an empty array
np.empty((3, 2))
array([[ 0., 0.], [ 0., 0.], [ 0., 0.]])
\end{lstlisting}

IO

Saving & Loading On Disk

\begin{lstlisting}[language=Python]
a = np.array([(1, 2), (3, 4)])
b = np.array([(5, 6), (7, 8)])
np.save(‘my_array’, a)
np.savez(‘arrays’, a, b)
np.load(‘arrays.npz’)
\end{lstlisting}

Saving & Loading Text Files

\begin{lstlisting}[language=Python]
np.loadtxt(“myfile.txt”)
array([[ 1., 2., 3.], [ 2., 3., 4.]])
np.genfromtxt(“my_file.csv”, delimiter=”,“)
array([[ 1., 2., 3.], [ 4., 5., 6.]])
a = np.array([(1.5, 2, 3), (4, 5, 6)], dtype=float)
np.savetxt(“myarray.txt”, a, delimiter=” “)
\end{lstlisting}

Data Types

\begin{lstlisting}[language=Python]
# Signed 64-bit integer types
np.int64
# Stardard double-precision floating point
np.float32
# Complex numbers represented by 128 floats
np.complex
# Boolean type storing TRUE and FALSE values
np.bool
# Python object type
np.object
# Fixed-length string type
np.string_
# Fixed-length unicode type
np.unicode_
numpy.unicode_
\end{lstlisting}

Inspecting Your Array

\begin{lstlisting}[language=Python]
a = np.array([(1.5, 2, 3), (4, 5, 6)], dtype=float)
# array dimensions
a.shape
(2L, 3L)
# length of array
len(a)
2
# number of array dimensions
a.ndim
2
# number of array elements
a.size
6
# data type of array elements
a.dtype
dtype(‘float64’)
# name of data type
a.dtype.name
‘float64’
# convert an array to a different type
a.astype(int)
array([[1, 2, 3], [4, 5, 6]])
\end{lstlisting}

Asking For Help

\begin{lstlisting}[language=Python]
np.info(np.ndarray.dtype)
\end{lstlisting}

Data-type of the array’s elements. Parameters ————— None Returns ———- d : numpy dtype object See Also ———— numpy.dtype Examples ———— >>> x array([[0, 1], [2, 3]]) >>> x.dtype dtype(‘int32’) >>> type(x.dtype)

Array Mathmatics

Arithmetic Operations

\begin{lstlisting}[language=Python]
a = np.random.random((2, 2))
b = np.random.random((2, 2))
# subtraction
np.subtract(a, b)
a - b
array([[-0.04906355, 0.24579184], [ 0.45085259, 0.55266361]])
# addition
np.add(b, a)
b + a
array([[ 0.11861634, 1.28886181], [ 0.84371684, 1.37134298]])
# division
np.divide(a, b)
a / b
array([[ 0.41479504, 1.47128543], [ 3.29520803, 2.35013443]])
# multiplication
np.multiply(a, b)
a * b
array([[ 0.00291565, 0.40018778], [ 0.12714751, 0.39378613]])
# exponentiation
np.exp(b)
array([[ 1.08745483, 1.68461152], [ 1.21705271, 1.50582314]])
# square root
np.exp(b)
array([[ 1.08745483, 1.68461152], [ 1.21705271, 1.50582314]])
# sines of an array
np.sin(a)
array([[ 0.03476938, 0.69421365], [ 0.60302258, 0.82033885]])
# cosine of an array
np.cos(b)
array([[ 0.99648749, 0.86705545], [ 0.98076917, 0.91738383]])
# natural algorithm
np.log(a)
array([[-3.35881648, -0.26484246], [-0.43496903, -0.03873741]])
# dot product
a.dot(b)
np.dot(a, b)
array([[ 0.15364329, 0.33223443], [ 0.24323666, 0.73136775]])
\end{lstlisting}

Comparison

\begin{lstlisting}[language=Python]
a = np.random.random((2, 2))
a
array([[ 0.20271908, 0.83347777], [ 0.61463859, 0.47298106]])
b = np.random.random((2, 2))
b
array([[ 0.71492635, 0.48317927], [ 0.83547998, 0.67228618]])
# element-wise comparison
a == b
array([[False, False], [False, False]], dtype=bool)
# element-wise comparison
a < 2
array([[ True, True], [ True, True]], dtype=bool)
# array-wise comparison
np.array_equal(a, b)
False
\end{lstlisting}

Aggregate Functions

\begin{lstlisting}[language=Python]
a = np.random.random((3, 3))
a
array([[ 0.71770831, 0.895387 , 0.58199526], [ 0.32399079, 0.24146174, 0.59422847], [ 0.9976845 , 0.36588863, 0.67375734]])
# array-wise sum
a.sum()
5.392102026407013
# array-wise minimum value
a.min()
0.2414617386336485
# maximum value of an array row
a.max(axis=0)
array([ 0.9976845 , 0.895387 , 0.67375734])
# cumulative sum of the elements
a.cumsum(axis=1)
array([[ 0.71770831, 1.61309531, 2.19509057], [ 0.32399079, 0.56545253, 1.15968099], [ 0.9976845 , 1.36357313, 2.03733047]])
# mean
a.mean()
0.59912244737855702
# median
np.median(a)
0.59422846515666305
# correlation coefficient
np.corrcoef(a)
array([[ 1. , -0.93042812, -0.55310242], [-0.93042812, 1. , 0.20930732], [-0.55310242, 0.20930732, 1. ]])
# stardard deviation
np.std(a)
0.24142891382802531
Copy Arrays
a = np.random.random((3, 3))
a
array([[ 0.25274882, 0.19042929, 0.16823795], [ 0.39392342, 0.05954749, 0.8608243 ], [ 0.99375507, 0.92845989, 0.45681322]])
a.view()
array([[ 0.25274882, 0.19042929, 0.16823795], [ 0.39392342, 0.05954749, 0.8608243 ], [ 0.99375507, 0.92845989, 0.45681322]])
np.copy(a)
array([[ 0.25274882, 0.19042929, 0.16823795], [ 0.39392342, 0.05954749, 0.8608243 ], [ 0.99375507, 0.92845989, 0.45681322]])
h = a.copy()
h
array([[ 0.25274882, 0.19042929, 0.16823795], [ 0.39392342, 0.05954749, 0.8608243 ], [ 0.99375507, 0.92845989, 0.45681322]])
\end{lstlisting}

Sorting Arrays

\begin{lstlisting}[language=Python]
a = np.random.random((3, 3))
a
array([[ 0.11422752, 0.30046885, 0.15876115], [ 0.89595996, 0.47878824, 0.41827471], [ 0.69593773, 0.52119338, 0.33048738]])
a.sort()
a
array([[ 0.11422752, 0.15876115, 0.30046885], [ 0.41827471, 0.47878824, 0.89595996], [ 0.33048738, 0.52119338, 0.69593773]])
a.sort(axis=0)
a
array([[ 0.11422752, 0.15876115, 0.30046885], [ 0.33048738, 0.47878824, 0.69593773], [ 0.41827471, 0.52119338, 0.89595996]])
\end{lstlisting}

Subsetting, Slicing, Indexing

Subsettings

\begin{lstlisting}[language=Python]
a = np.random.random((3, 3))
a
array([[ 0.07989823, 0.4180309 , 0.83932547], [ 0.06318651, 0.20509151, 0.08262809], [ 0.64938826, 0.531026 , 0.38633983]])
# select the element at the 2nd index
a[2]
array([ 0.64938826, 0.531026 , 0.38633983])
# select the element at row 0 column 2
a[1][2]
a[1, 2]
0.08262808937797228
\end{lstlisting}

Slicing

\begin{lstlisting}[language=Python]
# select items at index 0 and 1
a[0:2]
array([[ 0.07989823, 0.4180309 , 0.83932547], [ 0.06318651, 0.20509151, 0.08262809]])
# select items at rớ 0 and 1 in column 1
a[0:2, 1]
array([ 0.4180309 , 0.20509151])
# select all items at row 0
a[1, ...]
a[1, ]
array([ 0.06318651, 0.20509151, 0.08262809])
# reversed array a
a[::-1]
array([[ 0.64938826, 0.531026 , 0.38633983], [ 0.06318651, 0.20509151, 0.08262809], [ 0.07989823, 0.4180309 , 0.83932547]])
\end{lstlisting}

Boolean indexing

\begin{lstlisting}[language=Python]
# select elements from a less than 0.5
a[a < 0.5]
array([ 0.07989823, 0.4180309 , 0.06318651, 0.20509151, 0.08262809, 0.38633983])
\end{lstlisting}

Fancy indexing

\begin{lstlisting}[language=Python]
# select elements (1,0), (0,1), (1, 2) and (0,0)
a[[1, 0, 1, 0], [0, 1, 2, 0]]
array([ 0.06318651, 0.4180309 , 0.08262809, 0.07989823])
# select a subset of the matrix’s rows and columns
a[[1, 0, 1, 0]][:, [0, 1, 2, 0]]
array([[ 0.06318651, 0.20509151, 0.08262809, 0.06318651], [ 0.07989823, 0.4180309 , 0.83932547, 0.07989823], [ 0.06318651, 0.20509151, 0.08262809, 0.06318651], [ 0.07989823, 0.4180309 , 0.83932547, 0.07989823]])
\end{lstlisting}

Array Manipulation

Transposing Array

\begin{lstlisting}[language=Python]
a = np.random.random((2, 3))
a
array([[ 0.57430709, 0.64401188, 0.12761183], [ 0.0726823 , 0.7951682 , 0.54114093]])
# permulate array dimensions
i = np.transpose(a)
i
array([[ 0.57430709, 0.0726823 ], [ 0.64401188, 0.7951682 ], [ 0.12761183, 0.54114093]])
# permulate array dimensions
i.T
array([[ 0.57430709, 0.64401188, 0.12761183], [ 0.0726823 , 0.7951682 , 0.54114093]])
\end{lstlisting}

Changing Array Shape

\begin{lstlisting}[language=Python]
# flatten the array
a.ravel()
array([ 0.57430709, 0.64401188, 0.12761183, 0.0726823 , 0.7951682 , 0.54114093])
# reshape, but don’t change data
a.reshape(3, -2)
array([[ 0.57430709, 0.64401188], [ 0.12761183, 0.0726823 ], [ 0.7951682 , 0.54114093]])
\end{lstlisting}

Adding/Removing Elements

\begin{lstlisting}[language=Python]
# return a new array with shape (2, 6)
a.resize(2, 3)
a
array([[ 0.57430709, 0.64401188, 0.12761183], [ 0.0726823 , 0.7951682 , 0.54114093]])
# append items to an array
h = np.random.random((2, 3))
print “h:”, h
g = np.random.random((2, 3))
print “g:”, g
np.append(h, g)
h: [[ 0.67964404 0.09256795 0.90630423] [ 0.52906489 0.51567697 0.95132012]] g: [[ 0.03126344 0.84908154 0.74228134] [ 0.40333143 0.28595213 0.68416838]] array([ 0.67964404, 0.09256795, 0.90630423, 0.52906489, 0.51567697, 0.95132012, 0.03126344, 0.84908154, 0.74228134, 0.40333143, 0.28595213, 0.68416838])
# insert items in an array
a = np.random.random((1, 3))
print “a:”, a
np.insert(a, 1, 0.5)
a: [[ 0.76135438 0.30331334 0.91866363]] array([ 0.76135438, 0.5 , 0.30331334, 0.91866363])
# delete items from an array
a = np.random.random((1, 3))
print “a:”, a
np.delete(a, [1])
a: [[ 0.1034073 0.93066432 0.49608264]] array([ 0.1034073 , 0.49608264])
\end{lstlisting}

Combining Arrays

\begin{lstlisting}[language=Python]
# concatenate arrays
a = np.random.random((1, 3))
print a
b = np.random.random((1, 3))
print b
np.concatenate((a, b), axis=0)
[[ 0.34496986 0.59502574 0.43416152]] [[ 0.98921435 0.68832237 0.44286195]] array([[ 0.34496986, 0.59502574, 0.43416152], [ 0.98921435, 0.68832237, 0.44286195]])
# stack arrays vertically (row-wise)
a = np.random.random((1, 3))
print a
b = np.random.random((2, 3))
print b
np.vstack((a, b)) #  equivalent to np.r_[a, b]
[[ 0.78793841 0.9923401 0.96372077]] [[ 0.75537083 0.09781391 0.25327948] [ 0.20607759 0.03763863 0.30818643]] array([[ 0.78793841, 0.9923401 , 0.96372077], [ 0.75537083, 0.09781391, 0.25327948], [ 0.20607759, 0.03763863, 0.30818643]])
# stack arrays horizontally (column-wise)
a = np.random.random((3, 1))
print a
b = np.random.random((3, 2))
print b
np.hstack((a, b))
[[ 0.33728008] [ 0.1091688 ] [ 0.68714517]] [[ 0.61421635 0.49316384] [ 0.19072731 0.04383904] [ 0.30587218 0.28743208]] array([[ 0.33728008, 0.61421635, 0.49316384], [ 0.1091688 , 0.19072731, 0.04383904], [ 0.68714517, 0.30587218, 0.28743208]])
# equivalent to np.hstack
np.column_stack((a, b))
array([[ 0.33728008, 0.61421635, 0.49316384], [ 0.1091688 , 0.19072731, 0.04383904], [ 0.68714517, 0.30587218, 0.28743208]])
# equivalent to np.hstack
np.c_[a, b]
array([[ 0.33728008, 0.61421635, 0.49316384], [ 0.1091688 , 0.19072731, 0.04383904], [ 0.68714517, 0.30587218, 0.28743208]])
\end{lstlisting}

Spliting Arrays

\begin{lstlisting}[language=Python]
a = np.random.random((3, 4))
print a
[[ 0.64277816 0.75935599 0.64927247 0.80253242] [ 0.87630664 0.19748931 0.51895547 0.83645583] [ 0.03132085 0.043291 0.10945252 0.31883126]]
# split the array horizontally at the 3rd index
np.split(a, 3)
[array([[ 0.64277816, 0.75935599, 0.64927247, 0.80253242]]), array([[ 0.87630664, 0.19748931, 0.51895547, 0.83645583]]), array([[ 0.03132085, 0.043291 , 0.10945252, 0.31883126]])]
# split the array vertically at the 3rd index
np.vsplit(a, 3)
[array([[ 0.64277816, 0.75935599, 0.64927247, 0.80253242]]), array([[ 0.87630664, 0.19748931, 0.51895547, 0.83645583]]), array([[ 0.03132085, 0.043291 , 0.10945252, 0.31883126]])]
\end{lstlisting}

Suggested Readings
www.datacamp.com. Python For Data Science Cheat Sheet: Numpy Basics

\section{Data Wrangling}

Learn about data wrangling with pandas

Tiny Data
A foundation for wrangling in pandas

Create DataFrames
Specify values for each column

\begin{lstlisting}[language=Python]
import pandas as pd

df = pd.DataFrame({
“a”: [4, 5, 6],
“b”: [7, 8, 9],
“c”: [10, 11, 12]
}, index=[1, 2, 3])
df
a b c

1 4 7 10

2 5 8 11

3 6 9 12
\end{lstlisting}

Specify values for each row

\begin{lstlisting}[language=Python]
df = pd.DataFrame(
[[4, 5, 6],
[7, 8, 9],
[10, 11, 12]],
index=[1, 2, 3],
columns=[“a”, “b”, “c”])
df
a b c
1 4 5 6
2 7 8 9
3 10 11 12
\end{lstlisting}

Create DataFrame with a MultiIndex

df = pd.DataFrame({
“a”: [4, 5, 6],
“b”: [7, 8, 9],
“c”: [10, 11, 12]
})
index = pd.MultiIndex.from_tuples(
[(‘d’, 1), (‘d’, 2), (‘e’, 2)],
names=[‘n’,‘v’])
df
a b c

0 4 7 10

1 5 8 11

2 6 9 12

Reshaping Data
melt
“Unpivots” a DataFrame from wide format to long format, optionally leaving identifier variables set.

import pandas as pd

df = pd.DataFrame({
“a”: [4, 5],
“b”: [7, 8],
“c”: [10, 11]
})
df
a b c

0 4 7 10

1 5 8 11

pd.melt(df)
variable value

0 a 4

1 a 5

2 b 7

3 b 8

4 c 10

5 c 11

pivot
Reshape data (produce a “pivot” table) based on column values. Uses unique values from index / columns to form axes of the resulting DataFrame.

df = pd.DataFrame({‘foo’: [‘one’,‘one’,‘one’,‘two’,‘two’,‘two’],
‘bar’: [‘A’, ‘B’, ‘C’, ‘A’, ‘B’, ‘C’],
‘baz’: [1, 2, 3, 4, 5, 6]})
df
bar baz foo

0 A 1 one

1 B 2 one

2 C 3 one

3 A 4 two

4 B 5 two

5 C 6 two

df.pivot(index=‘foo’, columns=‘bar’, values=‘baz’)
bar A B C

foo

one 1 2 3

two 4 5 6

df.pivot(index=‘foo’, columns=‘bar’)[‘baz’]
bar A B C

foo

one 1 2 3

two 4 5 6

concat
Append rows of DataFrames

df1 = pd.DataFrame([[‘a’, 1], [‘b’, 2]],
columns=[‘letter’, ‘number’])
df1
letter number

0 a 1

1 b 2

df2 = pd.DataFrame([[‘c’, 3], [‘d’, 4]],
columns=[‘letter’, ‘number’])
pd.concat([df1, df2])
letter number

0 a 1

1 b 2

0 c 3

1 d 4

Append columns of DataFrames

df1 = pd.DataFrame([[‘a’, 1], [‘b’, 2]],
columns=[‘letter’, ‘number’])
df1
letter number

0 a 1

1 b 2

df2 = pd.DataFrame([[‘bird’, ‘polly’], [‘monkey’, ‘george’]],
columns=[‘animal’, ‘name’])
df2
animal name

0 bird polly

1 monkey george

pd.concat([df1, df2], axis=1)
letter number animal name

0 a 1 bird polly

1 b 2 monkey george

sort
df = pd.DataFrame([[‘a’, 10, 1], [‘b’, 10, 5], [‘c’, 30, 3]],
columns=[‘name’, ‘age’, ‘score’])
df
name age score

0 a 10 1

1 b 10 5

2 c 30 3

order rows by values of a column (low to high)

df.sort_values(‘age’)
name age score

0 a 10 1

1 b 10 5

2 c 30 3

order rows by values of a column (high to low)

df.sort_values(‘age’, ascending=False)
name age score

2 c 30 3

0 a 10 1

1 b 10 5

order rows by values of two column

df.sort_values([‘age’, ‘score’], ascending=[False, False])
name age score

2 c 30 3

1 b 10 5

0 a 10 1

sort the index of a DataFrame

df.sort_index()
name age score

0 a 10 1

1 b 10 5

2 c 30 3

Reset index of DataFrame to row numbers, moving index to columns

df.reset_index()
index name age score

0 0 a 10 1

1 1 b 10 5

2 2 c 30 3

drop
drop columns from DataFrame

df.drop([‘age’, ‘score’], axis=1)
name
0	a
1	b
2	c

\section{Visualization}

An introduction about data visualization techniques using Matplotlib and Seaborn.

Gallery
 line graph
Line Graph

 bar graph
Bar Graph

 pie graph
Pie Graph

 sratter plot
Scatter Plot

References
Patterns: The Data Visualisation Catalogue



