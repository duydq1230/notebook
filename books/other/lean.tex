\chapter{Lean}

View online \href{http://magizbox.com/training/lean/site/}{http://magizbox.com/training/lean/site/}

Lean startup is a method for developing businesses and products first proposed in 2008 by Eric Ries. Based on his previous experience working in several U.S. startups, Ries claims that startups can shorten their product development cycles by adopting a combination of business-hypothesis-driven experimentation, iterative product releases, and what he calls validated learning. Ries' overall claim is that if startups invest their time into iteratively building products or services to meet the needs of early customers, they can reduce the market risks and sidestep the need for large amounts of initial project funding and expensive product launches and failures.

\section{Lean Canvas}

\section{Workflow}

Life’s too short to build something nobody wants. What separates successful startups from unsuccessful ones is that they find a plan that works before running out of resources.

Build-Measure-Learn
The fundamental activity of a startup is to turn ideas into products, measure how customers respond, and then learn whether to pivot or persevere. All successful startup processes should be geared to accelerate that feedback loop.



Step 1: Document your Plan A.
Writing down the initial vision using Lean Canvas and then sharing it with at least one other person

Step 2: Identify the Riskiest Parts
Formulate Falsifiable Hypotheses

Falsifiable Hypothesis = [Specific Repeatable Action] will [Expected Measurable Action]
The biggest risk is building something nobody wants. The risks can be divided in:

Stage 1: Problem/Solution Fit.
Do I have a problem worth solving?

Answer these questions:

Is it something customers want? (must-have)
Will they pay for it? If not, who will? (viable)
Can it be solved? (feasible)
Stage 2: Product/market Fit.
Have I built something people want?"

Qualitative metrics (interviews)
Quantitative metrics for measuring product/market fit.
Stage 3: Scale.
How do I accelerate growth?

After product/market fit, raise a big round of funding to scale the business. Traction is needed! Seek External advice and ask specific questions.

Systematically test your Plan.
Based on the scientific method run experiments.

First build hypothesis that can be easily disproved and use artifacts in front of customers to “measure” their response using qualitative and quantitative data to derive “learning” to validate or refute the hypothesis.

The Lean Startup is also about being efficient. In all the stages we can take actions to be more efficient: eliminate don’t-needs at the MVP, deploy continuously, make feedback easy for customers, don’t push for features, etc.

The Lean Startup methodology is strongly rooted in the scientific method, and running experiments is a key activity. Your job isn’t just building the best solution, but owning the entire business model and making all the pieces fit.

\section{Validated Learning}

Startups exist not to make stuff, make money, or serve customers. They exist to learn how to build a sustainable business. This learning can be validated scientifically, by running experiments that allow us to test each element of our vision.
