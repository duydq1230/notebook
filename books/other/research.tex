\chapter{Nghiên cứu}

\diary{01/11/2017 Không biết mình có phải làm nghiên cứu không nữa? Vừa kiêm phát triển, vừa đọc paper mỗi ngày. Thôi, cứ (miễn cưỡng) cho là nghiên cứu viên đi.}

\section{Các công cụ}

\subsection{Google Scholar \& Semantic Scholar}

\href{https://scholar.google.com.vn/}{Google Scholar} vẫn là lựa chọn tốt

\begin{itemize}
  \item Tìm kiếm tác giả theo lĩnh vực nghiên cứu và quốc gia: sử dụng filter label: + đuôi
  \begin{itemize}
    \item ví dụ: \href{https://scholar.google.com.vn/citations?hl=en&amp;view_op=search_authors&amp;mauthors=label\%3Anatural_language_processing+.vn&amp;btnG=}{danh sách các nhà nghiên cứu Việt Nam thuộc lĩnh vực xử lý ngôn ngữ tự nhiên}
  \end{itemize}
  \item danh sách này đã sắp xếp theo lượng trích dẫn
\end{itemize}


Bên cạnh đó còn có \href{https://www.semanticscholar.org/}{semanticscholar} (một project của \href{http://allenai.org/}{allenai}) với các killer features

\begin{itemize}
  \item \href{https://www.semanticscholar.org/search?venue\%5B\%5D=ACL&amp;q=sentiment&amp;sort=relevance}{Tìm kiếm các bài báo khoa học với từ khóa và filter theo năm, tên hội nghị}
  \item \href{https://www.semanticscholar.org/author/Christopher-D-Manning/1812612}{Xem những người ảnh hưởng, ảnh hưởng bởi một nhà nghiên cứu, cũng như xem co-author, journals và conferences mà một nhà nghiên cứu hay gửi bài}
\end{itemize}

\subsection{Mendeley}

Mendeley rất tốt cho việc quản lý và lưu trữ. Tuy nhiên điểm hạn chế lại là không lưu thông tin về citation

\subsection{Hội nghị và tạp chí}

Các hội nghị tốt về xử lý ngôn ngữ tự nhiên

\begin{itemize}
  \item Rank A: ACL, EACL, NAACL, EMNLP, CoNLL
  \item Rank B: SemEval
\end{itemize}

Các tạp chí

\begin{itemize}
  \item \href{http://www.mitpressjournals.org/loi/coli}{Computational Linguistics (CL)}
\end{itemize}

\subsection{Câu chuyện của Scihub}

Sci-Hub được tạo ra vào ngày 5 tháng 9 năm 2011, do nhà nghiên cứu đến từ Kazakhstan, \href{https://en.wikipedia.org/wiki/Alexandra_Elbakyan}{Alexandra Elbakyan}

Hãy nghe chia sẻ của cô về sự ra đời của Sci-Hub

> Khi tôi còn là một sinh viên tại Đại học Kazakhstan, tôi không có quyền truy cập vào bất kỳ tài liệu nghiên cứu. Những bài báo tôi cần cho dự án nghiên cứu của tôi. Thanh toán 32 USD thì thật là điên rồ khi bạn cần phải đọc lướt hoặc đọc hàng chục hoặc hàng trăm tờ để làm nghiên cứu. Tôi có được những bài báo nhờ vào trộm chúng. Sau đó tôi thấy có rất nhiều và rất nhiều nhà nghiên cứu (thậm chí không phải sinh viên, nhưng các nhà nghiên cứu trường đại học) giống như tôi, đặc biệt là ở các nước đang phát triển. Họ đã tạo ra các cộng đồng trực tuyến (diễn đàn) để giải quyết vấn đề này. Tôi là một thành viên tích cực trong một cộng đồng như vậy ở Nga. Ở đây ai cần có một bài nghiên cứu, nhưng không thể trả tiền cho nó, có thể đặt một yêu cầu và các thành viên khác, những người có thể có được những giấy sẽ gửi nó cho miễn phí qua email. Tôi có thể lấy bất cứ bài nào, vì vậy tôi đã giải quyết nhiều yêu cầu và người ta luôn rất biết ơn sự giúp đỡ của tôi. Sau đó, tôi tạo Sci-Hub.org, một trang web mà chỉ đơn giản là làm cho quá trình này tự động và các trang web ngay lập tức đã trở thành phổ biến.

Về phần mình, là một nhà nghiên cứu trẻ, đương nhiên phải đọc liên tục. Các báo cáo ở Việt Nam về xử lý ngôn ngữ tự nhiên thì thường không tải lên các trang mở như arxiv.org, các kỷ yếu hội nghị cũng không public các proceedings. Thật sự scihub đã giúp mình rất nhiều.

\textbf{Scihub bị chặn}

Vào thời điểm này (12/2017), scihub bị chặn quyết liệt. Hóng được trên page facebook của scihub các cách truy cập scihub. Đã thử các domain khác như .tw, .hk. Mọi chuyện vẫn ổn cho đến hôm nay (21/12/2017), không thể truy cập vào nữa.

Đành phải cài tor để truy cập vào scihub ở địa chỉ \href{http://scihub22266oqcxt.onion/https://dl.acm.org/citation.cfm?id=1852627}{http://scihub22266oqcxt.onion}. Và mọi chuyện lại ổn.

\section{Làm sao để nghiên cứu tốt}

\begin{itemize}
  \item Làm việc mỗi ngày
  \item Cập nhật các kết quả từ các hội nghị, tạp chí
  \item Viết nhật ký nghiên cứu mỗi tuần (tổng kết công việc tuần trước, các ý tưởng mới, kế hoạch tuần này)
\end{itemize}

\section{Sách giáo khoa}

\href{https://gallery.mailchimp.com/dc3a7ef4d750c0abfc19202a3/files/Machine_Learning_Yearning_V0.5_01.pdf}{Machine Learning Yearning, by Andrew Ng}

\section{Lượm lặt}

\href{https://www.kdnuggets.com/2017/10/3-popular-courses-deep-learning.html}{Review các khóa học Deep Learning}

