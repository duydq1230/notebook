\chapter{Agile}

View online \href{http://magizbox.com/training/agile/site/}{http://magizbox.com/training/agile/site/}

\section{Introduction}

What is Agile? 2
Agile methodology is an alternative to traditional project management, typically used in software development. It helps teams respond to unpredictability through incremental, iterative work cadences, known as sprints. Agile methodologies are an alternative to waterfall, or traditional sequential development.

Agile Manifesto


Scrum Framework


A manifesto for small teams doing important work 4

What is the difference between Scrum and Agile Development? 1
Scrum is just one of the many iterative and incremental agile software development method. You can find here a very detailed description of the process.

In the SCRUM methodology a sprint is the basic unit of development. Each sprint starts with a planning meeting, where the tasks for the sprint are identified and an estimated commitment for the sprint goal is made. A Sprint ends with a review or retrospective meeting where the progress is reviewed and lessons for the next sprint are identified. During each sprint, the team creates finished portions of a product.

In the Agile methods each iteration involves a team working through a full software development cycle, including planning, requirements analysis, design, coding, unit testing, and acceptance testing when a working product is demonstrated to stakeholders.

Agile Stories and Teasers
2011, PRESENTATION: HOW OUR TEAM LIVES SCRUM
2010, The real life of a Scrum team – with photos
2009, How Scrum Helped Our Team
Agile Tools 3
Agilefant (4/ 3/ 3)
What is the difference between Scrum and Agile Development? ↩

Agile Methodology ↩

agile-tools.net/ ↩

A manifesto for small teams doing important work ↩

\section{Team}

Scrum Team 1
Within the Scrum Framework three roles are defined:

Product Owner
Scrum Master
Development team
Each of these roles has a defined set of responsibilities and only if they fulfill these responsibilities, closely interact and work together they can finish a project successfully.

 Scrum Roles & Stakeholders

Product Owner
One of the most important things for the success of scrum is the role of the Product Owner, who serves as an interface between the team and other involved parties (stakeholders). It can be said that in companies that use scrum, the tasks and responsibilities of the particular Product Owner are never the same. Starting with the choice of that person provided with the proper and necessary skills, make them take specific trainings, up to the responsibility they take; the role of the Product Owner –short PO- is the most complex one regarding that procedure.

Often the PO has to “fight” on both sides. Whereas the team can work a certain fraction of time (time boxed) “protected” by the Scrum Master, the Product Owner often needs to deal with marketing, management or the customers in order to be able to present the software requirements (User Stories) quite precisely to the team (see the box “criteria for User Stories).

Furthermore the Product Owner is responsible for the return on investment (ROI). He validates the solutions and verifies whether the quality is acceptable or not from the end-users’ point of view. He also has to decide over the importance of single features in order to prioritize these in their treatment and he has to tell the team what the product should look like in the end (product vision). Since one of the teams’ tasks is to work effectively, the Product Owner must react fast on call-backs. Hence he fulfills the role of a communicator, as he must be in contact with all stakeholders, sponsors and last but not least the team throughout a project. After all it is his task to coordinate the financial side of the product development, which is successful through his continuous work and prioritizing the advancing tasks (Product Backlog). All these diverse requests demonstrate how important the selection of the “right” person for the role of the PO is for the success of a project.

The nomenclature is definite. The Product Owner is at Scrum not only the manager of a product, but also the Owner and therefore he is the one responsible for the correct creation of a product. Being a Product Owner means:

You are responsible for the success of the outcome of the product delivered by the team. You make Business decisions of importance and priorities. You deliver the vision of the product to the team. You prepare the User Stories for the team of development. You should possess severe domain knowledge. You validate the outcomes and test them for their quality. You react promptly on callbacks. You communicate on a continual base with all Stakeholders, financiers and the team. You control the financial side of the project.

Scrum Master
The most obvious difference between a team leader and a Scrum Master is represented by the name itself though. Whereas one is leading the team and sets the tasks, the other one is in charge of observing that the team obeys the rules and realizes the method of Scrum entirely. The Scrum Master does not interfere into the decisions of the team regarding specifically the development, but rather is there for the team as an advisor. He only interferes actively when anybody of the team or any other participant of a project (Stakeholder) does not obey the rules of Scrum. Whereas a team leader often gives requirements and takes responsibility for the completion of those, an experienced Scrum Master gives only impulses and advises to the team to lead the correct way, to use the right method or to choose the right technology. In fine the Scrum Master acts more like a Team Coach than a team leader.

ScrumMaster and Impediments
Another important task of the Scrum Master is to get rid of all possible impediments that might disturb the work of the team. Usually problems can be classified in three different categories. The first one is problems the team cannot solve. E.g. the team cannot do any kind of performance-tests because the hardware is not in place, the IT-department does not provide Bug tracker, or the ordered software just still did not reach the team. Another impediment could be that the marketing or sales manager was there again demanding that another feature gets integrated “quickly”.

The second one regards impediments that result through the organizational structure or strategic decisions. Maybe the office is not capable of handling the important meetings or teamwork – e.g. because there is no media. One mistake that occurs quite often regards the problem that the Scrum Master is seen as the personnel responsible for the team members. This is often because of the classical role of a project leader, but using Scrum it only leads to conflicts of interests and is strongly against its major principle: The team owns a management role in the method of Scrum and is therefore coequal with the Scrum Master and the Product Owner. Another aspect can be the insufficient bandwidth of the internet for the new project.

The third problem refers to the individuals. Someone needs a hand with the debugging. Another one cannot solve a task alone and needs someone else for the pair programming. Someone else has to reset his computer....

Even though a Scrum Master can’t and shouldn’t realize some requirements himself, he is still responsible for solving and getting rid of problems and needs to give proper criteria. This task often takes up a lot of time and requires great authority and backbone. The Scrum Master has to create an optimal working-condition for the team and is responsible for this condition to be retained, in order to meet the goals of every sprint – i.e. for a short sprint the defined requirements.

Scrum Development Team
Different from other methods, in Scrum a team is not just the executive organ that receives its tasks from the project leader, it rather decides self dependent, which requirements or User Stories it can accomplish in one sprint. It constructs the tasks and is responsible for the permutation of those – the team becomes a manager. This new self-conception of the team and the therewith aligned tasks and responsibilities necessarily change the role of the team leader/project leader. The Scrum Master does not need to delegate all the work and to plan the project, he rather takes care that the team meets all conditions in order to reach the self-made goals. He cleans off any impediments, provides an ideal working environment for the team, coaches and is responsible for the observation of Scrum-rules – he becomes the so-called Servant Leader.

The changed role perception is one of the most important aspects, when someone wants to understand Scrum and with the intent to introduce it in their own company.

Scrum Roles ↩

\section{Artifacts}

Product Backlog 1


Force-ranked list of desired functionality
Visible to all stakeholders
Any stakeholder (including the Team) can add items
Constantly re-prioritized by the Product Owner
Items at top are more granular than items at bottom
Maintained during the Backlog Refinement Meeting


Product Backlog Item (PBI) 1


Specifies the what more than the how of a customer-centric feature
Often written in User Story form
Has a product-wide definition of done to prevent technical debt
May have item-specific acceptance criteria
Effort is estimated by the team, ideally in relative units (e.g., story points)
Effort is roughly 2-3 people 2-3 days, or smaller for advanced teams
Sprint Backlog 1


Consists of committed PBIs negotiated between the team and the Product Owner during the Sprint Planning Meeting

Scope commitment is fixed during Sprint Execution

Initial tasks are identified by the team during Sprint Planning Meeting

Team will discover additional tasks needed to meet the fixed scope commitment during Sprint execution

Visible to the team

Referenced during the Daily Scrum Meeting

Sprint Task 1


Specifies how to achieve the PBI’s what

Requires one day or less of work

Remaining effort is re-estimated daily, typically in hours

During Sprint Execution, a point person may volunteer to be primarily responsible for a task

Owned by the entire team; collaboration is expected

Personal Sprint Board with Sticky Notes (Windows)


Sprint Burndown Chart 1


Indicates total remaining team task hours within one Sprint
Re-estimated daily, thus may go up before going down
Intended to facilitate team self-organization
Fancy variations, such as itemizing by point person or adding trend lines, tend to reduce effectiveness at encouraging collaboration
Seemed like a good idea in the early days of Scrum, but in practice has often been misused as a management report, inviting intervention. The ScrumMaster should discontinue use of this chart if it becomes an impediment to team self-organization.


Product / Release Burndown Chart


Tracks the remaining Product Backlog effort from one Sprint to the next
May use relative units such as Story Points for Y axis
Depicts historical trends to adjust forecasts
scrum-reference-card ↩

\section{Meetings}

Meetings
Scrum Meetings 1
Sprint Planning
Daily Scrum
Sprint Review
Sprint Retrospective
Product Backlog Refinement


Sprint Planning 1


Goals
– Discuss and make sure the whole team understands the upcoming Work Items to deliver (quantity, complexity) – Select the work items to achieve during the Sprint (Create the Sprint Backlog) – Rationally forecast the amount of work and commit to accomplish it – Plan how this work will be done

Attendees
1st part (what?): Whole team
2nd part (how?): Even though the product owner does not attend, he should remain to answer questions.
Duration
The meeting is time-boxed: 2 hours / week of sprint duration.

Typical meeting roadmap
In Scrum, the sprint planning meeting has two parts:

1. What work will be done?

– the Product Owner presents the ordered product backlog items to the development team – the whole Scrum team collaborates to understand the work and select work items starting from the top of the product backlog

2. How will the work be accomplished?

– the development team discusses to define how to produce the next product increment in accordance with the current Definition of Done – The team does just the sufficient design and planning to be confident of completing the work during the sprint – The upcoming work to be done in the early days is broken down into small tasks of one day or less – Work to be done later may be left in larger units to be decomposed later (this is called Just-In-Time planning in Lean) – Final commitment to do the work

Important to know / Good practices
– The development team is alone responsible to determine the number of product backlog items that will be “pulled” to the sprint. Nobody else should interfere in that decision, based on the current state of the project, the past performance and the current availability of the team. – It is a good practice to set a sprint goal to keep focus on the big picture and not on the details. – It is necessary for the Sprint planning meeting success that the product backlog is well ordered and refined. This is the Product Owner’s responsibility. – The development team is responsible to decide how to do the work (self-organization). – There is no point, especially at the beginning, to try to make hourly estimates of the work and compare them to availability. This habit is inherited from traditional PM methods and may be counterproductive, as reduces ownership of the team’s commitments. The best for the team is to intuitively estimate its own capacity to do the work, reduce the amount of committed work to deliver, and get experienced at estimating during the first sprints.

Daily Scrum 1


Goals
Ensure, every day, at the same palace, that the team is on track for attaining the sprint goal and that team members are all on the same page.
Spot blocks and problems. The Scrum Master can resolve impediments that are beyond the team’s control.
This is NOT a management reporting meeting to anyone. The team members communicate together as a team.
Based on what comes up in the meeting, the development team reorganizes the work as needed to accomplish the sprint goal.
Create transparency, trust and better performance. Build the team’s self-organization and self-reliance.
Attendees
Development Team + Scrum Master.
The Product Owner presence is not required but it I almost always interesting for him / here to be present, especially to clarify requirements if needed.
Other stakeholders can attend to get a good and quick overview of the progress and project status, although having managerial presence may cause a “trigger” effect and pollute the meeting’s effectiveness.
Duration
No more than 15 minutes.
A good practice is to allow 2 minutes to each team member.
Typical meeting roadmap
Every team member answers 3 questions:



What I have accomplished since the last daily Scrum
What I plan to accomplish between now and the next daily Scrum
What (if anything) is impeding my progress
No discussion during the meeting. Only brief questions to clarify the previous list. Any subsequent discussion should take place after the meeting with the concerned team members.

Important to know / Good practices
The Daily Scrum is sometimes called “Daily Stand Up”. This name gives a good overview of the tone and shortness of this meeting.
Each team member moves the tasks in the taskboard while speaking (if not done before)
The Sprint Burndown chart can be updated by the Scrum Master at the end of the meeting
Having a “being blocked” task list is useful. Personally I add a dedicated column in the taskboard
Only team members speak during the daily Scrum. Nobody else.
TheDaily Scrum is a proof of the team’s sef-organizing capacity as it shows how much team members collaborate together as they address themselves to the whole team (and not only the Scrum Master)
It is quite common to uncover additional tasks during the Sprint to achieve the Sprint Goal
Sprint Review Meeting 1


Goals
After Sprint execution, the team holds a Sprint Review Meeting to demonstrate a working product increment to the Product Owner and everyone else who is interested.
The meeting should feature a live demonstration, not a report.
Attendees
Product Team
Product Owner
When
After Sprint execution
Duration
4 hours
given a 30-day Sprint (much longer than anyone recommends nowadays), the maximum time for a Sprint Review Meeting is 4 hours
Roadmap
Demonstration
After the demonstration, the Product Owner reviews the commitments made at the Sprint Planning Meeting and declares which items he now considers done.
For example, a software item that is merely “code complete” is considered not done, because untested software isn’t shippable. Incomplete items are returned to the Product Backlog and ranked according to the Product Owner’s revised priorities as candidates for future Sprints.
The ScrumMaster helps the Product Owner and stakeholders convert their feedback to new Product Backlog Items for prioritization by the Product Owner.
Often, new scope discovery outpaces the team’s rate of development. If the Product Owner feels that the newly discovered scope is more important than the original expectations, new scope displaces old scope in the Product Backlog.
The Sprint Review Meeting is the appropriate meeting for external stakeholders (even end users) to attend. It is the opportunity to inspect and adapt the product as it emerges, and iteratively refine everyone’s understanding of the requirements. New products, particularly software products, are hard to visualize in a vacuum. Many customers need to be able to react to a piece of functioning software to discover what they will actually want. Iterative development, a value-driven approach, allows the creation of products that couldn’t have been specified up front in a plan-driven approach.

Sprint Retrospective Meeting 1 2


Goals
At this meeting, the team reflects on its own process. They inspect their behavior and take action to adapt it for future Sprints.

When
Each Sprint ends with a retrospective.

Duration
45 minutes


Roadmap


Dedicated ScrumMasters will find alternatives to the stale, fearful meetings everyone has come to expect. An in-depth retrospective requires an environment of psychological safety not found in most organizations. Without safety, the retrospective discussion will either avoid the uncomfortable issues or deteriorate into blaming and hostility.

A common impediment to full transparency on the team is the presence of people who conduct performance appraisals.

Another impediment to an insightful retrospective is the human tendency to jump to conclusions and propose actions too quickly. Agile Retrospectives, the most popular book on this topic, describes a series of steps to slow this process down: Set the stage, gather data, generate insights, decide what to do, close the retrospective. (1) Another guide recommended for ScrumMasters, The Art of Focused Conversations, breaks the process into similar steps: Objective, reflective, interpretive, and decisional (ORID). (2)

A third impediment to psychological safety is geographic distribution. Geographically dispersed teams usually do not collaborate as well as those in team rooms.

Retrospectives often expose organizational impediments. Once a team has resolved the impediments within its immediate influence, the ScrumMaster should work to expand that influence, chipping away at the organizational impediments.

ScrumMasters should use a variety of techniques to facilitate retrospectives, including silent writing, timelines, and satisfaction histograms. In all cases, the goals are to gain a common understanding of multiple perspectives and to develop actions that will take the team to the next level.

Product Backlog Refinement Meeting 1


How to: A Great Product Backlog Refinement Workshop ↩

PRESENTATION: HOW OUR TEAM LIVES SCRUM ↩