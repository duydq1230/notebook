\chapter{Variables}

Although Java is object oriented, not all types are objects. It is built on top of basic variable types called primitives.

Here is a list of all primitives in Java:

byte (number, 1 byte)
short (number, 2 bytes)
int (number, 4 bytes)
long (number, 8 bytes)
float (float number, 4 bytes)
double (float number, 8 bytes)
char (a character, 2 bytes)
boolean (true or false, 1 byte) Java is a strong typed language, which means variables need to be defined before we use them.

\section{Numbers}

To declare and assign a number use the following syntax:

int myNumber;
myNumber = 5;
Or you can combine them:

int myNumber = 5;
To define a double floating point number, use the following syntax:

double d = 4.5;
d = 3.0;
If you want to use float, you will have to cast:

float f = (float) 4.5;
Or, You can use this:

float f = 4.5f (f is a shorter way of casting float)
Characters and Strings
In Java, a character is it's own type and it's not simply a number, so it's not common to put an ascii value in it, there is a special syntax for chars:

char c = 'g';
String is not a primitive. It's a real type, but Java has special treatment for String.

Here are some ways to use a string:

// Create a string with a constructor
String s1 = new String("Who let the dogs out?");
// Just using "" creates a string, so no need to write it the previous way.
String s2 = "Who who who who!";
// Java defined the operator + on strings to concatenate:
String s3 = s1 + s2;
There is no operator overloading in Java! The operator + is only defined for strings, you will never see it with other objects, only primitives.

You can also concat string to primitives:

int num = 5;
String s = "I have " + num + " cookies"; //Be sure not to use "" with primitives.
boolean
Every comparison operator in java will return the type boolean that not like other languages can only accept two special values: true or false.

boolean b = false;
b = true;

boolean toBe = false;
b = toBe || !toBe;
if (b) {
    System.out.println(toBe);
}

int children = 0;
b = children; // Will not work
if (children) { // Will not work
    // Will not work
}

\section{Operators}
Java provides a rich set of operators to manipulate variables. We can divide all the Java operators into the following groups:

Arithmetic Operators
Relational Operators
Bitwise Operators
Logical Operators
Assignment Operators
Misc Operators
The Arithmetic Operators
Arithmetic operators are used in mathematical expressions in the same way that they are used in algebra.

The following table lists the arithmetic operators:

Operator	Description	Example

\begin{lstlisting}
+ (Addition)	Adds values on either side of the operator	10 + 20 -> 30
- (Subtraction)	Subtracts right hand operand from left hand operand	10 - 20 -> -10
* ( Multiplication )	Multiplies values on either side of the operator	10 * 20 -> 200
/ (Division)	Divides left hand operand by right hand operand	20 / 10 -> 2
% (Modulus)	Divides left hand operand by right hand operand and returns remainder	7 % 3 -> 1
\end{lstlisting}

++ (Increment)	Increases the value of operand by 1
a = 20

a++ -> 21

-- ( Decrement )	Decreases the value of operand by 1
a = 20

a-- -> 19

The Relational Operators
There are following relational operators supported by Java language

== (equal to) Checks if the values of two operands are equal or not, if yes then condition becomes true.

Example: (A == B) is not true. 2 != (not equal to) Checks if the values of two operands are equal or not, if values are not equal then condition becomes true.

Example: (A != B) is true.

3 > (greater than) Checks if the value of left operand is greater than the value of right operand, if yes then condition becomes true.

Example: (A > B) is not true. 4 < (less than) Checks if the value of left operand is less than the value of right operand, if yes then condition becomes true.

Example: (A < B) is true. 5 >= (greater than or equal to) Checks if the value of left operand is greater than or equal to the value of right operand, if yes then condition becomes true.

Example (A >= B) is not true. 6 <= (less than or equal to) Checks if the value of left operand is less than or equal to the value of right operand, if yes then condition becomes true.

example(A <= B) is true.

The Bitwise Operators
Java defines several bitwise operators, which can be applied to the integer types, long, int, short, char, and byte.

Bitwise operator works on bits and performs bit-by-bit operation. Assume if a = 60; and b = 13; now in binary format they will be as follows:

a = 0011 1100

b = 0000 1101

a&b = 0000 1100

a|b = 0011 1101

a^b = 0011 0001

~a = 1100 0011

The following table lists the bitwise operators:

Assume integer variable A holds 60 and variable B holds 13 then:

& (bitwise and) Binary AND Operator copies a bit to the result if it exists in both operands.

Example: (A & B) will give 12 which is 0000 1100 2 | (bitwise or) Binary OR Operator copies a bit if it exists in either operand.

Example: (A | B) will give 61 which is 0011 1101 3 ^ (bitwise XOR) Binary XOR Operator copies the bit if it is set in one operand but not both.

Example: (A ^ B) will give 49 which is 0011 0001 4 ~ (bitwise compliment) Binary Ones Complement Operator is unary and has the effect of 'flipping' bits.

Example: (~A ) will give -61 which is 1100 0011 in 2's complement form due to a signed binary number. 5 << (left shift) Binary Left Shift Operator. The left operands value is moved left by the number of bits specified by the right operand

Example: A << 2 will give 240 which is 1111 0000 6 >> (right shift) Binary Right Shift Operator. The left operands value is moved right by the number of bits specified by the right operand.

Example: A >> 2 will give 15 which is 1111 7 >>> (zero fill right shift) Shift right zero fill operator. The left operands value is moved right by the number of bits specified by the right operand and shifted values are filled up with zeros.

Example: A >>>2 will give 15 which is 0000 1111

The Logical Operators
The following table lists the logical operators:

Assume Boolean variables A holds true and variable B holds false, then:

&& (logical and) Called Logical AND operator. If both the operands are non-zero, then the condition becomes true.

Example (A && B) is false. 2 || (logical or) Called Logical OR Operator. If any of the two operands are non-zero, then the condition becomes true.

Example (A || B) is true. 3 ! (logical not) Called Logical NOT Operator. Use to reverses the logical state of its operand. If a condition is true then Logical NOT operator will make false.

Example !(A && B) is true.

The Assignment Operators
There are following assignment operators supported by Java language:

Show Examples

SR.NO Operator and Description 1 = Simple assignment operator, Assigns values from right side operands to left side operand.

Example: C = A + B will assign value of A + B into C 2 += Add AND assignment operator, It adds right operand to the left operand and assign the result to left operand.

Example: C += A is equivalent to C = C + A 3 -= Subtract AND assignment operator, It subtracts right operand from the left operand and assign the result to left operand.

Example:C -= A is equivalent to C = C - A 4 *= Multiply AND assignment operator, It multiplies right operand with the left operand and assign the result to left operand.

Example: C *= A is equivalent to C = C * A 5 /= Divide AND assignment operator, It divides left operand with the right operand and assign the result to left operand

ExampleC /= A is equivalent to C = C / A 6 %= Modulus AND assignment operator, It takes modulus using two operands and assign the result to left operand.

Example: C %= A is equivalent to C = C % A 7 <<= Left shift AND assignment operator.

ExampleC <<= 2 is same as C = C << 2 8 >>= Right shift AND assignment operator

Example C >>= 2 is same as C = C >> 2 9 &= Bitwise AND assignment operator.

Example: C &= 2 is same as C = C & 2 10 ^= bitwise exclusive OR and assignment operator.

Example: C ^= 2 is same as C = C ^ 2 11 |= bitwise inclusive OR and assignment operator.

Example: C |= 2 is same as C = C | 2

Miscellaneous Operators
There are few other operators supported by Java Language.

Conditional Operator ( ? : ) Conditional operator is also known as the ternary operator. This operator consists of three operands and is used to evaluate Boolean expressions. The goal of the operator is to decide which value should be assigned to the variable. The operator is written as:

variable x = (expression) ? value if true : value if false
Following is the example:

public class Test {

   public static void main(String args[]){
      int a, b;
      a = 10;
      b = (a == 1) ? 20: 30;
      System.out.println( "Value of b is : " +  b );

      b = (a == 10) ? 20: 30;
      System.out.println( "Value of b is : " + b );
   }
}
This would produce the following result ?

Value of b is : 30
Value of b is : 20
Precedence of Operators
Operator precedence determines the grouping of terms in an expression. This affects how an expression is evaluated. Certain operators have higher precedence than others; for example, the multiplication operator has higher precedence than the addition operator:

For example, x = 7 + 3 * 2; here x is assigned 13, not 20 because operator * has higher precedence than +, so it first gets multiplied with 3*2 and then adds into 7.

Here, operators with the highest precedence appear at the top of the table, those with the lowest appear at the bottom. Within an expression, higher precedence operators will be evaluated first.

Category Operator Associativity Postfix () [] . (dot operator) Left toright Unary ++ - - ! ~ Right to left Multiplicative * / % Left to right Additive + - Left to right Shift >> >>> << Left to right Relational > >= < <= Left to right Equality == != Left to right Bitwise AND & Left to right Bitwise XOR ^ Left to right Bitwise OR | Left to right Logical AND && Left to right Logical OR || Left to right Conditional ?: Right to left Assignment = += -= *= /= %= >>= <<= &= ^= |= Right to left

Conditional
Java uses boolean variables to evaluate conditions. The boolean values true and false are returned when an expression is compared or evaluated. For example:

int a = 4;
boolean b = a == 4;

if (b) {
    System.out.println("It's true!");
}
Of course we don't normally assign a conditional expression to a boolean, we just use the short version:

int a = 4;

if (a == 4) {
    System.out.println("Ohhh! So a is 4!");
}
Boolean operators
There aren't that many operators to use in conditional statements and most of them are pretty strait forward:

int a = 4;
int b = 5;
boolean result;
result = a < b; // true
result = a > b; // false
result = a <= 4 // a smaller or equal to 4 - true
result = b >= 6 // b bigger or equal to 6 - false
result = a == b // a equal to b - false
result = a != b // a is not equal to b - true
result = a > b || a < b // Logical or - true
result = 3 < a && a < 6 // Logical and - true
result = !result // Logical not - false
if - else and between
The if, else statement in java is pretty simple.

if (a == b) {
    // a and b are equal, let's do something cool
}
And we can also add an else statement after an if, to do something if the condition is not true

if (a == b) {
    // We already know this part
} else {
    // a and b are not equal... :/
}
The if - else statements doesn't have to be in several lines with {}, if can be used in one line, or without the {}, for a single line statment.

if (a == b)
    System.out.println("Another line Wow!");
else
    System.out.println("Double rainbow!");
Although this method might be useful for making your code shorter by using fewer lines, we strongly recommend for beginners not to use this short version of statements and always use the full version with {}. This goes to every statement that can be shorted to a single line (for, while, etc).

The ugly side of if
There is a another way to write a one line if - else statement by using the operator ? :

int a = 4;
int result = a == 4 ? 1 : 8;

// result will be 1
// This is equivalent to
int result;

if (a == 4) {
    result = 1;
} else {
    result = 8;
}
Again, we strongly recommend for beginners not to use this version of if.

== and equals
The operator == works a bit different on objects than on primitives. When we are using objects and want to check if they are equal, the operator == will say if they are the same, if you want to check if they are logically equal, you should use the equals method on the object. For example:

String a = new String("Wow");
String b = new String("Wow");
String sameA = a;

boolean r1 = a == b;      // This is false, since a and b are not the same object
boolean r2 = a.equals(b); // This is true, since a and b are logically equals
boolean r3 = a == sameA;  // This is true, since a and sameA are really the same object

\chapter{Data Structure}

\section{Number}

Convert number to string

\begin{lstlisting}[language=Java]
String.valueOf(1000)
\end{lstlisting}

Make a random

\begin{lstlisting}[language=Java]
// create a random number from 0 to 99
(new Random()).nextInt(100)
\end{lstlisting}


\section{String}

Gộp một mảng các string vào một string

\begin{lstlisting}[language=Java]
List<String> strings = Arrays.asList("a", "b", "c");
String s = String.join(" ", strings);
\end{lstlisting}

\section{Collection}

\subsection{Arrays}

Arrays in Java are also objects. They need to be declared and then created. In order to declare a variable that will hold an array of integers, we use the following syntax:

int[] arr;
Notice there is no size, since we didn't create the array yet.

arr = new int[10];
This will create a new array with the size of 10. We can check the size by printing the array's length:

System.out.println(arr.length);
We can access the array and set values:

arr[0] = 4;
arr[1] = arr[0] + 5;
Java arrays are 0 based, which means the first element in an array is accessed at index 0 (e.g: arr[0], which accesses the first element). Also, as an example, an array of size 5 will only go up to index 4 due to it being 0 based.

int[] arr = new int[5]
//accesses and sets the first element
arr[0] = 4;
We can also create an array with values in the same line:

int[] arr = {1, 2, 3, 4, 5};
Don't try to print the array without a loop, it will print something nasty like [I@f7e6a96.

\subsection{Set}

\begin{lstlisting}[language=Java]
import java.util.HashSet;
import java.util.Set;

public class HelloWorld{

     public static void main(String []args){
         Set<Dog> dogs = new  HashSet<Dog>();
         Dog dog1 = new Dog("a", 1);
         Dog dog2 = new Dog("a", 2);
         Dog dog3 = new Dog("a", 1);
         Dog dog4 = new Dog("b", 1);
         dogs.add( dog1);
         dogs.add( dog2);
         dogs.add( dog3);
         dogs.add( dog4);
        System.out.println(dogs.size());
     }
}

// 3
public class Dog {
    public String name;
    public int age;
    public int value;
    public Dog(String name, int age){
        this.name = name;
        this.age = age;
        value = (this.name + String.valueOf(this.age)).hashCode();
    }

    @Override
    public int hashCode() {
        return value;
    }

    @Override
    public boolean equals(Object obj) {
        return (obj instanceof Dog && ((Dog) obj).value == this.value);
    }
}
\end{lstlisting}



List
List<String> places = Arrays.asList("Buenos Aires", "Córdoba", "La Plata");
Datetime
Calendar c = Calendar.getInstance();
Suggest Readings
Initialization of an ArrayList in one line
How to convert from int to String?