\chapter{OOP}

\subsection{Classes}

Java is an Object-Oriented Language. As a language that has the Object-Oriented feature, Java supports the following fundamental concepts

\begin{enumerate}
  \item Classes and Objects
  \item Inheritance
  \item Polymorphism
  \item Abstraction
  \item Instance
  \item Method
  \item Message Parsing
\end{enumerate}



In this chapter, we will look into the concepts - Classes and Objects.

Object − Objects have states and behaviors. Example: A dog has states - color, name, breed as well as behaviors – wagging the tail, barking, eating. An object is an instance of a class.
Class − A class can be defined as a template/blueprint that describes the behavior/state that the object of its type support.
Objects
Let us now look deep into what are objects. If we consider the real-world, we can find many objects around us, cars, dogs, humans, etc. All these objects have a state and a behavior.

If we consider a dog, then its state is - name, breed, color, and the behavior is - barking, wagging the tail, running.

If you compare the software object with a real-world object, they have very similar characteristics.

Software objects also have a state and a behavior. A software object's state is stored in fields and behavior is shown via methods.

So in software development, methods operate on the internal state of an object and the object-to-object communication is done via methods.

Classes
A class is a blueprint from which individual objects are created.

Following is a sample of a class.

Example

public class Dog {
   String breed;
   int ageC
   String color;

   void barking() {
   }

   void hungry() {
   }

   void sleeping() {
   }
}
A class can contain any of the following variable types.

Local variables − Variables defined inside methods, constructors or blocks are called local variables. The variable will be declared and initialized within the method and the variable will be destroyed when the method has completed.
Instance variables − Instance variables are variables within a class but outside any method. These variables are initialized when the class is instantiated. Instance variables can be accessed from inside any method, constructor or blocks of that particular class.
Class variables − Class variables are variables declared within a class, outside any method, with the static keyword.
A class can have any number of methods to access the value of various kinds of methods. In the above example, barking(), hungry() and sleeping() are methods.

Following are some of the important topics that need to be discussed when looking into classes of the Java Language.

Constructors
When discussing about classes, one of the most important sub topic would be constructors. Every class has a constructor. If we do not explicitly write a constructor for a class, the Java compiler builds a default constructor for that class.

Each time a new object is created, at least one constructor will be invoked. The main rule of constructors is that they should have the same name as the class. A class can have more than one constructor.

Following is an example of a constructor −

Example

public class Puppy {
   public Puppy() {
   }

   public Puppy(String name) {
      // This constructor has one parameter, name.
   }
}
Java also supports Singleton Classes where you would be able to create only one instance of a class.

Note − We have two different types of constructors. We are going to discuss constructors in detail in the subsequent chapters.

Creating an Object
As mentioned previously, a class provides the blueprints for objects. So basically, an object is created from a class. In Java, the new keyword is used to create new objects.

There are three steps when creating an object from a class −

Declaration − A variable declaration with a variable name with an object type.
Instantiation − The 'new' keyword is used to create the object.
Initialization − The 'new' keyword is followed by a call to a constructor. This call initializes the new object.
Following is an example of creating an object −

Example

\begin{lstlisting}[language=Java]
public class Puppy {
   public Puppy(String name) {
      // This constructor has one parameter, name.
      System.out.println("Passed Name is :" + name );
   }

   public static void main(String []args) {
      // Following statement would create an object myPuppy
      Puppy myPuppy = new Puppy( "tommy" );
   }
}
\end{lstlisting}

If we compile and run the above program, then it will produce the following result −

Passed Name is :tommy
Accessing Instance Variables and Methods
Instance variables and methods are accessed via created objects. To access an instance variable, following is the fully qualified path −

/* First create an object */
ObjectReference = new Constructor();

/* Now call a variable as follows */
ObjectReference.variableName;

/* Now you can call a class method as follows */
ObjectReference.MethodName();
Example

This example explains how to access instance variables and methods of a class.

public class Puppy {
   int puppyAge;

   public Puppy(String name) {
      // This constructor has one parameter, name.
      System.out.println("Name chosen is :" + name );
   }

   public void setAge( int age ) {
      puppyAge = age;
   }

   public int getAge( ) {
      System.out.println("Puppy's age is :" + puppyAge );
      return puppyAge;
   }

   public static void main(String []args) {
      /* Object creation */
      Puppy myPuppy = new Puppy( "tommy" );

      /* Call class method to set puppy's age */
      myPuppy.setAge( 2 );

      /* Call another class method to get puppy's age */
      myPuppy.getAge( );

      /* You can access instance variable as follows as well */
      System.out.println("Variable Value :" + myPuppy.puppyAge );
   }
}
If we compile and run the above program, then it will produce the following result

Output

Name chosen is :tommy
Puppy's age is :2
Variable Value :2
Source File Declaration Rules
As the last part of this section, let's now look into the source file declaration rules. These rules are essential when declaring classes, import statements and package statements in a source file.

There can be only one public class per source file.
A source file can have multiple non-public classes.
The public class name should be the name of the source file as well which should be appended by .java at the end. For example: the class name is public class Employee{} then the source file should be as Employee.java.
If the class is defined inside a package, then the package statement should be the first statement in the source file.
If import statements are present, then they must be written between the package statement and the class declaration. If there are no package statements, then the import statement should be the first line in the source file.
Import and package statements will imply to all the classes present in the source file. It is not possible to declare different import and/or package statements to different classes in the source file.
Classes have several access levels and there are different types of classes; abstract classes, final classes, etc. We will be explaining about all these in the access modifiers chapter.

Apart from the above mentioned types of classes, Java also has some special classes called Inner classes and Anonymous classes.

Java Package
In simple words, it is a way of categorizing the classes and interfaces. When developing applications in Java, hundreds of classes and interfaces will be written, therefore categorizing these classes is a must as well as makes life much easier.

Import Statements
In Java if a fully qualified name, which includes the package and the class name is given, then the compiler can easily locate the source code or classes. Import statement is a way of giving the proper location for the compiler to find that particular class.

For example, the following line would ask the compiler to load all the classes available in directory java_installation/java/io −

import java.io.*;
A Simple Case Study
For our case study, we will be creating two classes. They are Employee and EmployeeTest.

First open notepad and add the following code. Remember this is the Employee class and the class is a public class. Now, save this source file with the name Employee.java.

The Employee class has four instance variables - name, age, designation and salary. The class has one explicitly defined constructor, which takes a parameter.

Example

import java.io.*;
public class Employee {

   String name;
   int age;
   String designation;
   double salary;

   // This is the constructor of the class Employee
   public Employee(String name) {
      this.name = name;
   }

   // Assign the age of the Employee  to the variable age.
   public void empAge(int empAge) {
      age = empAge;
   }

   /* Assign the designation to the variable designation.*/
   public void empDesignation(String empDesig) {
      designation = empDesig;
   }

   /* Assign the salary to the variable salary.*/
   public void empSalary(double empSalary) {
      salary = empSalary;
   }

   /* Print the Employee details */
   public void printEmployee() {
      System.out.println("Name:"+ name );
      System.out.println("Age:" + age );
      System.out.println("Designation:" + designation );
      System.out.println("Salary:" + salary);
   }
}
As mentioned previously in this tutorial, processing starts from the main method. Therefore, in order for us to run this Employee class there should be a main method and objects should be created. We will be creating a separate class for these tasks.

Following is the EmployeeTest class, which creates two instances of the class Employee and invokes the methods for each object to assign values for each variable.

Save the following code in EmployeeTest.java file.

import java.io.*;
public class EmployeeTest {

   public static void main(String args[]) {
      /* Create two objects using constructor */
      Employee empOne = new Employee("James Smith");
      Employee empTwo = new Employee("Mary Anne");

      // Invoking methods for each object created
      empOne.empAge(26);
      empOne.empDesignation("Senior Software Engineer");
      empOne.empSalary(1000);
      empOne.printEmployee();

      empTwo.empAge(21);
      empTwo.empDesignation("Software Engineer");
      empTwo.empSalary(500);
      empTwo.printEmployee();
   }
}
Now, compile both the classes and then run EmployeeTest to see the result as follows −

Output

C:\> javac Employee.java
C:\> javac EmployeeTest.java
C:\> java EmployeeTest
Name:James Smith
Age:26
Designation:Senior Software Engineer
Salary:1000.0
Name:Mary Anne
Age:21
Designation:Software Engineer
Salary:500.0

\subsection{Encapsulation}

Encapsulation is one of the four fundamental OOP concepts. The other three are inheritance, polymorphism, and abstraction.

Encapsulation in Java is a mechanism of wrapping the data (variables) and code acting on the data (methods) together as a single unit. In encapsulation, the variables of a class will be hidden from other classes, and can be accessed only through the methods of their current class. Therefore, it is also known as data hiding.

Implementation
To achieve encapsulation in Java

Declare the variables of a class as private.
Provide public setter and getter methods to modify and view the variables values.
Example
Following is an example that demonstrates how to achieve Encapsulation in Java

/* File name : EncapTest.java */
public class EncapTest {
   private String name;
   private String idNum;
   private int age;

   public int getAge() {
      return age;
   }

   public String getName() {
      return name;
   }

   public String getIdNum() {
      return idNum;
   }

   public void setAge( int newAge) {
      age = newAge;
   }

   public void setName(String newName) {
      name = newName;
   }

   public void setIdNum( String newId) {
      idNum = newId;
   }
}
The public setXXX() and getXXX() methods are the access points of the instance variables of the EncapTest class. Normally, these methods are referred as getters and setters. Therefore, any class that wants to access the variables should access them through these getters and setters.

The variables of the EncapTest class can be accessed using the following program −

/* File name : RunEncap.java */
public class RunEncap {

   public static void main(String args[]) {
      EncapTest encap = new EncapTest();
      encap.setName("James");
      encap.setAge(20);
      encap.setIdNum("12343ms");

      System.out.print("Name : " + encap.getName() + " Age : " + encap.getAge());
   }
}
This will produce the following result −

Name : James Age : 20
Benefits
The fields of a class can be made read-only or write-only.
A class can have total control over what is stored in its fields.
The users of a class do not know how the class stores its data. A class can change the data type of a field and users of the class do not need to change any of their code.
Related Readings
"Java Inheritance". www.tutorialspoint.com. N.p., 2016. Web. 10 Dec. 2016.

\subsection{Inheritance}

In the preceding lessons, you have seen inheritance mentioned several times. In the Java language, classes can be derived from other classes, thereby inheriting fields and methods from those classes.

The idea of inheritance is simple but powerful: When you want to create a new class and there is already a class that includes some of the code that you want, you can derive your new class from the existing class. In doing this, you can reuse the fields and methods of the existing class without having to write (and debug!) them yourself.

A subclass inherits all the members (fields, methods, and nested classes) from its superclass. Constructors are not members, so they are not inherited by subclasses, but the constructor of the superclass can be invoked from the subclass.

Class Hierarchy
The Object class, defined in the java.lang package, defines and implements behavior common to all classes—including the ones that you write. In the Java platform, many classes derive directly from Object, other classes derive from some of those classes, and so on, forming a hierarchy of classes.


At the top of the hierarchy, Object is the most general of all classes. Classes near the bottom of the hierarchy provide more specialized behavior.

An Example
Here is the sample code for a possible implementation of a Bicycle class that was presented in the Classes and Objects lesson:

public class Bicycle {

    // the Bicycle class has three fields
    public int cadence;
    public int gear;
    public int speed;

    // the Bicycle class has one constructor
    public Bicycle(int startCadence, int startSpeed, int startGear) {
        gear = startGear;
        cadence = startCadence;
        speed = startSpeed;
    }

    // the Bicycle class has four methods
    public void setCadence(int newValue) {
        cadence = newValue;
    }

    public void setGear(int newValue) {
        gear = newValue;
    }

    public void applyBrake(int decrement) {
        speed -= decrement;
    }

    public void speedUp(int increment) {
        speed += increment;
    }

}
A class declaration for a MountainBike class that is a subclass of Bicycle might look like this:

public class MountainBike extends Bicycle {

    // the MountainBike subclass adds one field
    public int seatHeight;

    // the MountainBike subclass has one constructor
    public MountainBike(int startHeight,
                        int startCadence,
                        int startSpeed,
                        int startGear) {
        super(startCadence, startSpeed, startGear);
        seatHeight = startHeight;
    }

    // the MountainBike subclass adds one method
    public void setHeight(int newValue) {
        seatHeight = newValue;
    }
}
MountainBike inherits all the fields and methods of Bicycle and adds the field seatHeight and a method to set it. Except for the constructor, it is as if you had written a new MountainBike class entirely from scratch, with four fields and five methods. However, you didn't have to do all the work. This would be especially valuable if the methods in the Bicycle class were complex and had taken substantial time to debug.

What You Can Do in a Subclass
A subclass inherits all of the public and protected members of its parent, no matter what package the subclass is in. If the subclass is in the same package as its parent, it also inherits the package-private members of the parent. You can use the inherited members as is, replace them, hide them, or supplement them with new members:

The inherited fields can be used directly, just like any other fields.
You can declare a field in the subclass with the same name as the one in the superclass, thus hiding it (not * recommended).
You can declare new fields in the subclass that are not in the superclass.
The inherited methods can be used directly as they are.
You can write a new instance method in the subclass that has the same signature as the one in the superclass, thus overriding it.
You can write a new static method in the subclass that has the same signature as the one in the superclass, thus hiding it.
You can declare new methods in the subclass that are not in the superclass.
You can write a subclass constructor that invokes the constructor of the superclass, either implicitly or by using the keyword super.
The following sections in this lesson will expand on these topics.

Private Members in a Superclass
A subclass does not inherit the private members of its parent class. However, if the superclass has public or protected methods for accessing its private fields, these can also be used by the subclass.

A nested class has access to all the private members of its enclosing class—both fields and methods. Therefore, a public or protected nested class inherited by a subclass has indirect access to all of the private members of the superclass.

Casting Objects
We have seen that an object is of the data type of the class from which it was instantiated. For example, if we write

public MountainBike myBike = new MountainBike();
then myBike is of type MountainBike.

MountainBike is descended from Bicycle and Object. Therefore, a MountainBike is a Bicycle and is also an Object, and it can be used wherever Bicycle or Object objects are called for.

The reverse is not necessarily true: a Bicycle may be a MountainBike, but it isn't necessarily. Similarly, an Object may be a Bicycle or a MountainBike, but it isn't necessarily.

Casting shows the use of an object of one type in place of another type, among the objects permitted by inheritance and implementations. For example, if we write

Object obj = new MountainBike();
then obj is both an Object and a MountainBike (until such time as obj is assigned another object that is not a MountainBike). This is called implicit casting.

If, on the other hand, we write

MountainBike myBike = obj;
we would get a compile-time error because obj is not known to the compiler to be a MountainBike. However, we can tell the compiler that we promise to assign a MountainBike to obj by explicit casting:

MountainBike myBike = (MountainBike)obj;
This cast inserts a runtime check that obj is assigned a MountainBike so that the compiler can safely assume that obj is a MountainBike. If obj is not a MountainBike at runtime, an exception will be thrown.

Related Readings
"Inheritance". docs.oracle.com. N.p., 2016. Web. 8 Dec. 2016.
"Java Inheritance". www.tutorialspoint.com. N.p., 2016. Web. 8 Dec. 2016.
Friesen, Jeff. "Java 101: Inheritance In Java, Part 1". JavaWorld. N.p., 2016. Web. 8 Dec. 2016.

\subsection{Polymorphism}

Polymorphism is the ability of an object to take on many forms. The most common use of polymorphism in OOP occurs when a parent class reference is used to refer to a child class object.

Any Java object that can pass more than one IS-A test is considered to be polymorphic. In Java, all Java objects are polymorphic since any object will pass the IS-A test for their own type and for the class Object.

It is important to know that the only possible way to access an object is through a reference variable. A reference variable can be of only one type. Once declared, the type of a reference variable cannot be changed.

The reference variable can be reassigned to other objects provided that it is not declared final. The type of the reference variable would determine the methods that it can invoke on the object.

A reference variable can refer to any object of its declared type or any subtype of its declared type. A reference variable can be declared as a class or interface type.

Example
Let us look at an example.

public interface Vegetarian{}
public class Animal{}
public class Deer extends Animal implements Vegetarian{}
Now, the Deer class is considered to be polymorphic since this has multiple inheritance. Following are true for the above examples

A Deer IS-A Animal
A Deer IS-A Vegetarian
A Deer IS-A Deer
A Deer IS-A Object
When we apply the reference variable facts to a Deer object reference, the following declarations are legal

Deer d = new Deer();
Animal a = d;
Vegetarian v = d;
Object o = d;
All the reference variables d, a, v, o refer to the same Deer object in the heap.

Virtual Methods
In this section, I will show you how the behavior of overridden methods in Java allows you to take advantage of polymorphism when designing your classes.

We already have discussed method overriding, where a child class can override a method in its parent. An overridden method is essentially hidden in the parent class, and is not invoked unless the child class uses the super keyword within the overriding method.

/* File name : Employee.java */
public class Employee {
   private String name;
   private String address;
   private int number;

   public Employee(String name, String address, int number) {
      System.out.println("Constructing an Employee");
      this.name = name;
      this.address = address;
      this.number = number;
   }

   public void mailCheck() {
      System.out.println("Mailing a check to " + this.name + " " + this.address);
   }

   public String toString() {
      return name + " " + address + " " + number;
   }

   public String getName() {
      return name;
   }

   public String getAddress() {
      return address;
   }

   public void setAddress(String newAddress) {
      address = newAddress;
   }

   public int getNumber() {
      return number;
   }
}
Now suppose we extend Employee class as follows

/* File name : Salary.java */
public class Salary extends Employee {
   private double salary; // Annual salary

   public Salary(String name, String address, int number, double salary) {
      super(name, address, number);
      setSalary(salary);
   }

   public void mailCheck() {
      System.out.println("Within mailCheck of Salary class ");
      System.out.println("Mailing check to " + getName()
      + " with salary " + salary);
   }

   public double getSalary() {
      return salary;
   }

   public void setSalary(double newSalary) {
      if(newSalary >= 0.0) {
         salary = newSalary;
      }
   }

   public double computePay() {
      System.out.println("Computing salary pay for " + getName());
      return salary/52;
   }
}
Now, you study the following program carefully and try to determine its output

/* File name : VirtualDemo.java */
public class VirtualDemo {

   public static void main(String [] args) {
      Salary s = new Salary("Mohd Mohtashim", "Ambehta, UP", 3, 3600.00);
      Employee e = new Salary("John Adams", "Boston, MA", 2, 2400.00);
      System.out.println("Call mailCheck using Salary reference --");
      s.mailCheck();
      System.out.println("\n Call mailCheck using Employee reference--");
      e.mailCheck();
   }
}
This will produce the following result

Constructing an Employee
Constructing an Employee

Call mailCheck using Salary reference --
Within mailCheck of Salary class
Mailing check to Mohd Mohtashim with salary 3600.0

Call mailCheck using Employee reference--
Within mailCheck of Salary class
Mailing check to John Adams with salary 2400.0
Here, we instantiate two Salary objects. One using a Salary reference s, and the other using an Employee reference e.

While invoking s.mailCheck(), the compiler sees mailCheck() in the Salary class at compile time, and the JVM invokes mailCheck() in the Salary class at run time.

mailCheck() on e is quite different because e is an Employee reference. When the compiler sees e.mailCheck(), the compiler sees the mailCheck() method in the Employee class.

Here, at compile time, the compiler used mailCheck() in Employee to validate this statement. At run time, however, the JVM invokes mailCheck() in the Salary class.

This behavior is referred to as virtual method invocation, and these methods are referred to as virtual methods. An overridden method is invoked at run time, no matter what data type the reference is that was used in the source code at compile time.

Related Readings
"Java Polymorphism". www.tutorialspoint.com. N.p., 2016. Web. 10 Dec. 2016.

\subsection{Abstraction}

As per dictionary, abstraction is the quality of dealing with ideas rather than events. For example, when you consider the case of e-mail, complex details such as what happens as soon as you send an e-mail, the protocol your e-mail server uses are hidden from the user. Therefore, to send an e-mail you just need to type the content, mention the address of the receiver, and click send.

Likewise in Object-oriented programming, abstraction is a process of hiding the implementation details from the user, only the functionality will be provided to the user. In other words, the user will have the information on what the object does instead of how it does it.

In Java, abstraction is achieved using Abstract classes and interfaces.

Abstract Class
A class which contains the abstract keyword in its declaration is known as abstract class.

Abstract classes may or may not contain abstract methods, i.e., methods without body ( public void get(); )
But, if a class has at least one abstract method, then the class must be declared abstract.
If a class is declared abstract, it cannot be instantiated.
To use an abstract class, you have to inherit it from another class, provide implementations to the abstract methods in it.
If you inherit an abstract class, you have to provide implementations to all the abstract methods in it.
Example

This section provides you an example of the abstract class. To create an abstract class, just use the abstract keyword before the class keyword, in the class declaration.

/* File name : Employee.java */
public abstract class Employee {
   private String name;
   private String address;
   private int number;

   public Employee(String name, String address, int number) {
      System.out.println("Constructing an Employee");
      this.name = name;
      this.address = address;
      this.number = number;
   }

   public double computePay() {
     System.out.println("Inside Employee computePay");
     return 0.0;
   }

   public void mailCheck() {
      System.out.println("Mailing a check to " + this.name + " " + this.address);
   }

   public String toString() {
      return name + " " + address + " " + number;
   }

   public String getName() {
      return name;
   }

   public String getAddress() {
      return address;
   }

   public void setAddress(String newAddress) {
      address = newAddress;
   }

   public int getNumber() {
      return number;
   }
}
You can observe that except abstract methods the Employee class is same as normal class in Java. The class is now abstract, but it still has three fields, seven methods, and one constructor.

Now you can try to instantiate the Employee class in the following way

/* File name : AbstractDemo.java */
public class AbstractDemo {

   public static void main(String [] args) {
      /* Following is not allowed and would raise error */
      Employee e = new Employee("George W.", "Houston, TX", 43);
      System.out.println("\n Call mailCheck using Employee reference--");
      e.mailCheck();
   }
}
When you compile the above class, it gives you the following error

Employee.java:46: Employee is abstract; cannot be instantiated
      Employee e = new Employee("George W.", "Houston, TX", 43);
                   ^
1 error
Inheriting the Abstract Class
We can inherit the properties of Employee class just like concrete class in the following way

/* File name : Salary.java */
public class Salary extends Employee {
   private double salary;   // Annual salary

   public Salary(String name, String address, int number, double salary) {
      super(name, address, number);
      setSalary(salary);
   }

   public void mailCheck() {
      System.out.println("Within mailCheck of Salary class ");
      System.out.println("Mailing check to " + getName() + " with salary " + salary);
   }

   public double getSalary() {
      return salary;
   }

   public void setSalary(double newSalary) {
      if(newSalary >= 0.0) {
         salary = newSalary;
      }
   }

   public double computePay() {
      System.out.println("Computing salary pay for " + getName());
      return salary/52;
   }
}
Here, you cannot instantiate the Employee class, but you can instantiate the Salary Class, and using this instance you can access all the three fields and seven methods of Employee class as shown below.

/* File name : AbstractDemo.java */
public class AbstractDemo {

   public static void main(String [] args) {
      Salary s = new Salary("Mohd Mohtashim", "Ambehta, UP", 3, 3600.00);
      Employee e = new Salary("John Adams", "Boston, MA", 2, 2400.00);
      System.out.println("Call mailCheck using Salary reference --");
      s.mailCheck();
      System.out.println("\nCall mailCheck using Employee reference--");
      e.mailCheck();
   }
}
This produces the following result

Constructing an Employee
Constructing an Employee
Call mailCheck using Salary reference --
Within mailCheck of Salary class
Mailing check to Mohd Mohtashim with salary 3600.0

Call mailCheck using Employee reference--
Within mailCheck of Salary class
Mailing check to John Adams with salary 2400.0
Abstract Methods
If you want a class to contain a particular method but you want the actual implementation of that method to be determined by child classes, you can declare the method in the parent class as an abstract.

abstract keyword is used to declare the method as abstract.
You have to place the abstract keyword before the method name in the method declaration.
An abstract method contains a method signature, but no method body.
Instead of curly braces, an abstract method will have a semoi colon (;) at the end.
Following is an example of the abstract method.

public abstract class Employee {
   private String name;
   private String address;
   private int number;

   public abstract double computePay();
   // Remainder of class definition
}
Declaring a method as abstract has two consequences

The class containing it must be declared as abstract.
Any class inheriting the current class must either override the abstract method or declare itself as abstract.
Note − Eventually, a descendant class has to implement the abstract method; otherwise, you would have a hierarchy of abstract classes that cannot be instantiated.

Suppose Salary class inherits the Employee class, then it should implement the computePay() method as shown below

/* File name : Salary.java */
public class Salary extends Employee {
   private double salary;   // Annual salary

   public double computePay() {
      System.out.println("Computing salary pay for " + getName());
      return salary/52;
   }
   // Remainder of class definition
}
Related Readings
"Java Abstraction". www.tutorialspoint.com. N.p., 2016. Web. 10 Dec. 2016.