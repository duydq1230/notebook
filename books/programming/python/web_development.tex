\section{Web Development}

Django 1
Django is a high-level Python Web framework that encourages rapid development and clean, pragmatic design. Built by experienced developers, it takes care of much of the hassle of Web development, so you can focus on writing your app without needing to reinvent the wheel. It’s free and open source.

Project Folder Structure

project_folder/
├── your_project_name/
│   ├── your_project_name/
│   │   ├── static/
│   │   ├── models.py
│   │   ├── serializers.py
│   │   ├── settings.py
│   │   ├── urls.py
│   │   └── views.py
└   └── manage.py
Create (and use) REST API in 5 (+1) steps 1 2
Step 1: Install dependencies
pip install django
pip install djangorestframework
pip install markdown             # Markdown support for the browsable API.
pip install django-filter        # Filtering support
pip install django-cors-headers  # CORS support
Step 2: Create project
django-admin startproject your_project_name
Step 3: Config apps 3
Add 'your_project_name', 'rest_framework' to your INSTALLED_APPS setting in your_project_name/settings.py file

INSTALLED_APPS = (
    ...
    'your_project_name'
    'rest_framework',
)
Step 4: Model, View, Route 6
Step 4.1: Create model and serializer
You can go to Django: Model field reference page for more fields.

Step 4.1.1: Create Task class in your_project_name/models.py file
from django.db import models

class Task(models.Model):
    content = models.CharField(max_length=30)
    status = models.CharField(max_length=30)
Step 4.1.2: Create TaskSerializer class in your_project_name/serializers.py file
from your_project_name.models import Task
from rest_framework import serializers

class TaskSerializer(serializers.HyperlinkedModelSerializer):
    class Meta:
        model = Task
        fields = ('id', 'content', 'status')
Step 4.1.3: Create table in database 4
python manage.py syncdb
With django 1.9

python manage.py makemigrations your_project_name
python manage.py migrate
Step 4.2: Create TaskViewSet class in your_project_name/views.py file
from your_project_name.models import Task
from your_project_name.serializers import TaskSerializer
from rest_framework import viewsets

class TaskViewSet(viewsets.ModelViewSet):
    queryset = Task.objects.all()
    serializer_class = TaskSerializer
Step 4.3: Config route 5
Change your_project_name/urls.py file

from django.conf.urls import include, url
from django.contrib import admin
from rest_framework import routers
from your_project_name.views import TaskViewSet

router = routers.DefaultRouter()
router.register(r'api/tasks', TaskViewSet)
admin.autodiscover()

urlpatterns = [
    url(r'^admin/', include(admin.site.urls)),
    url(r'^', include(router.urls)),
    url(r'^api-auth/', include('rest_framework.urls', namespace='rest_framework'))
]
Step 5: Run Server
python manage.py runserver
Step 6. Use API
Step 6.1: Create a new task
curl -i -X POST -H "Content-Type:application/json" http://localhost:8000/api/tasks -d '{
  "content": "a",
  "status": "INIT"
}'
Step 6.2: List all tasks
curl http://localhost:8000/api/tasks
Step 6.3: Get detail of task 1
curl http://localhost:8000/api/tasks/1
Step 6.4: Delete task 1
curl -i -X DELETE http://localhost:8000/api/tasks/1
Step 7: CORS
Known Error: No 'Access-Control-Allow-Origin' header is present on the requested resource. Origin 'null' is therefore not allowed access.

Step 7.1: Install corsheader app
Add module corsheaders to your_project_name/settings.py

INSTALLED_APPS = (
    ...
    'corsheaders',
    ...
)
Step 7.2 Add middleware classes
Add middleware_classes to your_project_name/settings.py

MIDDLEWARE_CLASSES = (
    ...
    'corsheaders.middleware.CorsMiddleware',
    'django.middleware.common.CommonMiddleware',
    ...
)
Step 7.3 Configuration CORS Setting
Option 1: Allow All

Add this line to your_project_name/settings.py

CORS_ORIGIN_ALLOW_ALL: True
Step 8: https
You can use https://github.com/teddziuba/django-sslserver

Unicode
REST_FRAMEWORK = {
    'DEFAULT_RENDERER_CLASSES': (
        'rest_framework.renderers.JSONRenderer',
        'rest_framework.renderers.BrowsableAPIRenderer',
    )
}
Step 9: Paging
Add this module setting to your_project_name/settings.py


REST_FRAMEWORK = {
    'DEFAULT_PAGINATION_CLASS': 'rest_framework.pagination.LimitOffsetPagination',
}

API:


GET <>/?limit=<limit>&offset=<offset>

Step 10: Search by field in
import this to your viewsets.py


from rest_framework import filters

add this to your viewsets class


filter_backends = (filters.SearchFilter, )
search_fields = ('<field>','<field>',)

One-to-Many Relationship 7
from django.db import models

class User(models.Model):
    name = models.TextField()

    def __str__(self):
        return "{} - {}".format(str(self.id), self.name)


class Task(models.Model):
    name = models.TextField()
    assign = models.ForeignKey(User, on_delete=models.CASCADE)
Starting with Mysql
Add this database settings to your_project_name/settings.py


DATABASES = {
    'default': {
        'ENGINE': 'django.db.backends.mysql',
        'NAME': '[DB_NAME]',
        'USER': '[DB_USER]',
        'PASSWORD': '[PASSWORD]',
        'HOST': '[HOST]',   # Or an IP Address that your DB is hosted on
        'PORT': '3306',
    }
}

Install this module to your virtual environment


conda install mysql-python #if you are using virtual environment

pip install mysql-python #if you using are root environment

Custom View 8
from rest_framework import mixins


class CreateModelMixin(object):
    """
    Create a model instance.
    """
    def create(self, request, *args, **kwargs):
        event = request.data
        try:
            event['time'] = int(time.time())
        except Exception, e:
            print 'Set Time Error'
        serializer = self.get_serializer(data=request.data)
        serializer.is_valid(raise_exception=True)
        self.perform_create(serializer)
        headers = self.get_success_headers(serializer.data)
        return Response(serializer.data, status=status.HTTP_201_CREATED, headers=headers)

    def perform_create(self, serializer):
        serializer.save()

    def get_success_headers(self, data):
        try:
            return {'Location': data[api_settings.URL_FIELD_NAME]}
        except (TypeError, KeyError):
            return {}

class YourViewSet(CreateModelMixin,
                  mixins.RetrieveModelMixin,
                  mixins.UpdateModelMixin,
                  mixins.DestroyModelMixin,
                  mixins.ListModelMixin,
                  GenericViewSet):
    queryset = YourModel.objects.all()
    serializer_class = YourModelSerializer
Logging settings
Here is an example, put this settings dict into your settings.py file:


LOGGING = {
    'version': 1,
    'disable_existing_loggers': False,
    'formatters': {
        'verbose': {
            'format': '%(levelname)s %(asctime)s %(module)s %(process)d %(thread)d %(message)s'
        },
        'simple': {
            'format': '%(levelname)s %(message)s'
        },
    },
    'filters': {
        'special': {
            '()': 'project.logging.SpecialFilter',
            'foo': 'bar',
        },
        'require_debug_true': {
            '()': 'django.utils.log.RequireDebugTrue',
        },
    },
    'handlers': {
        'console': {
            'level': 'INFO',
            'filters': ['require_debug_true'],
            'class': 'logging.StreamHandler',
            'formatter': 'simple'
        },
        'mail_admins': {
            'level': 'ERROR',
            'class': 'django.utils.log.AdminEmailHandler',
            'filters': ['special']
        }
    },
    'loggers': {
        'django': {
            'handlers': ['console'],
            'propagate': True,
        },
        'django.request': {
            'handlers': ['mail_admins'],
            'level': 'ERROR',
            'propagate': False,
        },
        'myproject.custom': {
            'handlers': ['console', 'mail_admins'],
            'level': 'INFO',
            'filters': ['special']
        }
    }
}

Python: Build Python API Client package
Step 1: Write document on Swagger Editor1
Step 2: Genenrate Client --> Python --> save python-client.zip
Step 3: Extract zip
Step 4: Open project in Pycharm rename project directory, project name, swagger_client package
Step 5: 2
mkdir conda
cd conda
git clone https://github.com/hunguyen1702/condaBuildLocalTemplate.git
mv condaBuildLocalTemplate your_package_name
rm -rf .git README.md
Step 6: Edit meta.yaml file in your_package folder
6.1 Follow instruction inside meta.yaml
6.2 Replace these line
requirements:
  build:
    - python
    - setuptools
  run:
    - python
with:
requirements:
  build:
    - python
    - setuptools
    - six
    - certifi
    - python-dateutil
  run:
    - python
    - six
    - certifi
    - python-dateutil
Step 7:
cd ..
conda build your_package
Step 8:
mkdir channel
cd channel
conda convert --platform all ~/anaconda/conda-bld/linux-64/your_package_0.1.0-py27_0.tar.bz2
Step 9: Create virtual-env
name: your_env_name
dependencies:
- certifi=2016.2.28=py27_0
- openssl=1.0.2h=0
- pip=8.1.2=py27_0
- python=2.7.11=0
- python-dateutil=2.5.3=py27_0
- readline=6.2=2
- setuptools=20.7.0=py27_0
- six=1.10.0=py27_0
- tk=8.5.18=0
- wheel=0.29.0=py27_0
- zlib=1.2.8=0
- pip:
  - urllib3==1.15.1
Step 10: Install:
conda install --use-local your_package
Django ↩

Writing your first Django app, part 1 ↩

Django REST framework: Installation ↩

Django: Migrations ↩

Building a Simple REST API for Mobile Applications ↩

Django: Models ↩

How to show object details in Django Rest Framework browseable API? ↩

rest_framework:mixins ↩