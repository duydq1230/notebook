\chapter{Lời nói đầu}
\marginpar{Text will come here}
Đọc quyển Deep Learning quá xá hay luôn. Rồi lại đọc SLP 2. Thấy sao các thánh viết hay và chuẩn thế (đấy là lý do các thánh được gọi là ... các thánh chăng =))

Tính đến thời điểm này đã được 2 năm 10 tháng rồi. Quay lại với latex. Thỏa mãn được điều kiện của mình là một tool offline. Mình thích xuất ra pdf (có gì đọc lại hoặc tra cứu cũng dễ).

Hi vọng gắn bó với thằng này được lâu.


\noindent \textbf{Chào từ hồi magizbox.wordpress.com, cái này tồn tại được 77 ngày (hơn 2 tháng) (từ 01/11/2017 đến 17/01/2018)}

\framebox{\parbox{\dimexpr\linewidth-2\fboxsep-2\fboxrule}{
Chào Khách,

Mình là Vũ Anh. Tính đến thời điểm viết bài này thì đã lập trình được 7 năm (lập trình từ hồi năm 2010). Mình thích viết lách, bằng chứng là đã thay đổi host 2 lần datayo.wordpress.com, magizbox.com. Thành tựu ổn nhất hiện tại chỉ có một project <a href="https://github.com/magizbox/underthesea">underthesea</a>, xếp loại tạm được.

Blog này chứa những ghi chép loạn cào cào của mình về mọi thứ. Đúng với phong cách "vô tổ chức" của mình. Chắc chắn nó sẽ không hữu ích lắm với bạn. Tuy nhiên, cảm ơn bạn đã ghé qua.

Nếu Khách quan tâm, thì mình chỉ post bài xàm vào thứ 7 thôi nhé. Những ngày còn lại chỉ post bài nghiêm túc thôi. (bài này quá xàm nhưng được post vào thứ 5 nhé)

<strong>Làm sao để thực hiện blog này</strong>
<ul>
	<li>Viết markdown và latex hỗ trợ bởi wordpress</li>
	<li>Server cho phép lưu ảnh động <a href="https://giphy.com/">giphy</a></li>
	<li>Vấn đề lưu trữ ảnh: sử dụng tính năng live upload của github.com</li>
</ul>
}}

\noindent Bỏ cái này vì quá chậm. Không hỗ trợ tốt latex (công thức toán và reference). Mình vẫn thích một công cụ offline hơn.

\noindent \textbf{Chào từ hồi magizbox.com, cái này tồn tại được 488 ngày (1 năm 4 tháng. wow) (từ 01/07/2016 đến 01/11/2017)}

\framebox{\parbox{\dimexpr\linewidth-2\fboxsep-2\fboxrule}{
Hello World,

My name is Vu Anh. I'm a developer working at a startup in Hanoi, Vietnam. Coding and writing is fun, so I make this site to share my gists about computer science, data science, and more. It helps me keep my hobby and save information in case I forget. I wish it will be useful for you too.

PS: I always looking for collaboration. Feel free to contact me via email brother.rain.1024[at]gmail.com

Magizbox Stories


Oct 2, 2016: Wow. It's 524th day of my journey. I added some notes in README.md, index.html, changed structure of website. Today I feel like at begin day when I start writing at datayo.wordpress.com blog. In this pass, there are times when I want to make a professional website like tutorialpoints but it doesn't work that way. Because in my heart, I don't want it, I don't to make a professional website. I just love coding, writing and sharing my hobby with people around the world. So today I come back to starting point, I will keep my writing schedule, make some fun stuffs each week.

In July 2016, I turn to use HTML and mkdocs, and opensource magizbox.

In March 2015, I start writing blog with wordpress.
}}

\noindent Bỏ cái này vì thời gian build quá lằng nhằng. Quản lý dependencies các kiểu rất lâu. Muốn có một cái gì đó giúp viết thật nhanh và đơn giản.

\noindent \textbf{Chào từ hồi datayo.wordpress.com, cái này tồn tại được 489 ngày (1 năm 4 tháng) (từ 01/03/2015 đến 01/07/2016)}


\framebox{\parbox{\dimexpr\linewidth-2\fboxsep-2\fboxrule}{
I’m a junior data scientist, working as a researcher in big data team at a company in Vietnam. I love listening music when I’m writing code because it’s make me coding better. I love reading books before sleeping because it take me sleep easier and discover the world with data mining via beautiful language R.

I write this blog because I want to share my hobbies with everybody. I hope you will enjoy it. Feel free to contact me via twitter @rain_1024 or email brother.rain.1024@gmail.com (I will answer all emails for sure) for anything you want to help about data science. I love collaboration.

In case you find one of my posts could be better, don’t hesitate to drop me a line in comment. I’m very appreciated and I will try my best to make it better and better.
}}

\noindent Bỏ cái này. Bỏ wordpress. Vì muốn một site interative hơn.
