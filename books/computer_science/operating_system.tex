\chapter{Hệ điều hành}

\textbf{Những phần mềm không thể thiếu}

\begin{itemize}
  \item Adblock extension
  \item Trình duyệt Google Chrome (với các extensions Scihub, Mendeley Desktop, Adblock)
  \item Terminal (Oh-my-zsh)
  \item IDE Pycharm để code python
  \item Quản lý phiên bản code Git
  \item Bộ gõ ibus-unikey trong Ubuntu hoặc unikey (Windows) (Ctrl-Space để chuyển đổi ngôn ngữ)
  \item CUDA (lập trình trên GPU)
\end{itemize}

\textbf{Xem thông tin hệ thống}

Phiên bản `ubuntu 16.04`

\begin{lstlisting}
sudo apt-get install sysstat
\end{lstlisting}


Xem hoạt động (\%) của các core cpu

\begin{lstlisting}
mpstat -A
\end{lstlisting}


CPU của mình có bao nhiều core, bao nhiêu siblibngs

\begin{lstlisting}
cat /proc/cpuinfo

processor       : 23
vendor_id       : GenuineIntel
cpu family      : 6
model           : 62
model name      : Intel(R) Xeon(R) CPU E5-2430 v2 @ 2.50GHz
stepping        : 4
microcode       : 0x428
cpu MHz         : 1599.707
cache size      : 15360 KB
physical id     : 1
siblings        : 12
core id         : 5
cpu cores       : 6
apicid          : 43
initial apicid  : 43
fpu             : yes
fpu_exception   : yes
cpuid level     : 13
wp              : yes
flags           : fpu vme de pse tsc msr pae mce cx8 apic sep mtrr pge mca cmov pat pse36 clflush dts acpi mmx fxsr sse sse2 ss ht tm pbe syscall nx pdpe1gb rdtscp lm constant_tsc arch_perfmon pebs bts rep_good nopl xtopology nonstop_tsc aperfmperf eagerfpu pni pclmulqdq dtes64 monitor ds_cpl vmx smx est tm2 ssse3 cx16 xtpr pdcm pcid dca sse4_1 sse4_2 x2apic popcnt tsc_deadline_timer aes xsave avx f16c rdrand lahf_lm ida arat xsaveopt pln pts dtherm tpr_shadow vnmi flexpriority ept vpid fsgsbase smep erms
bogomips        : 5005.20
clflush size    : 64
cache_alignment : 64
address sizes   : 46 bits physical, 48 bits virtual
power management:
\end{lstlisting}

Kết quả cho thấy cpu của 6 core và 12 siblings

\textbf{Xem chương trình nào tốn ram}

\begin{lstlisting}
ps aux | awk '{print $2, $4, $11}' | sort -k2rn | head -n 20
\end{lstlisting}

\href{https://www.garron.me/en/go2linux/how-find-which-process-eating-ram-memory-linux.html}{https://www.garron.me/en/go2linux/how-find-which-process-eating-ram-memory-linux.html}

\textbf{Quản lý các thiết bị}

Câu lệnh blkid: xem các thiết bị 

```
$ blkid
/dev/sda1: UUID="6e1dba21-bbc0-437a-8052-81bb4e428990" TYPE="ext4" PARTUUID="317dca10-01"
/dev/sdb1: LABEL="data" UUID="7da6c37a-ce51-4610-8cbb-b2a3f5730dd6" TYPE="ext4" PARTUUID="e749bb4a-01"
```

Sử dụng /etc/fstab để quản lý thông tin mount

```
# <file system> <mount point>   <type>  <options>       <dump>  <pass>
# / was on /dev/sda1 during installation
UUID=6e1dba21-bbc0-437a-8052-81bb4e428990 /               ext4    errors=remount-ro 0       1
UUID=7da6c37a-ce51-4610-8cbb-b2a3f5730dd6 /data           ext4  defaults        0 0
```

Các tùy chọn khi mount.

Nếu có nhiều tùy chọn thì chúng được phân cách bởi dấu phẩy. Dưới đây là 1 số tùy chọn đáng chú ý:

- auto: tự động mount thiết bị khi máy tính khởi động.
- noauto: không tự động mount, nếu muốn sử dụng thiết bị thì sau khi khởi động vào hệ thống bạn cần chạy lệnh mount.
- user: cho phép người dùng thông thường được quyền mount.
- nouser: chỉ có người dùng root mới có quyền mount.
- exec: cho phép chạy các file nhị phân (binary) trên thiết bị.
- noexec: không cho phép chạy các file binary trên thiết bị.
- ro (read-only): chỉ cho phép quyền đọc trên thiết bị.
- rw (read-write): cho phép quyền đọc/ghi trên thiết bị.
- sync: thao tác nhập xuất (I/O) trên filesystem được đồng bộ hóa.
- async: thao tác nhập xuất (I/O) trên filesystem diễn ra không đồng bộ.
- defaults: tương đương với tập các tùy chọn rw, suid, dev, exec, auto, nouser, async

Tham khảo: https://manthang.wordpress.com/2010/11/27/cau-hinh-file-etc-fstab-de-quan-ly-viec-mount-thiet-bi-trong-linux/


