\chapter{Docker}

View online \href{http://magizbox.com/training/docker/site/}{http://magizbox.com/training/docker/site/}

Docker is an open platform for building, shipping and running distributed applications. It gives programmers, development teams and operations engineers the common toolbox they need to take advantage of the distributed and networked nature of modern applications.

\section{Get Started}

Docker Promise
Docker provides an integrated technology suite that enables development and IT operations teams to build, ship, and run distributed applications anywhere.



Build: Docker allows you to compose your application from microservices, without worrying about inconsistencies between development and production environments, and without locking into any platform or language.

Composable You want to be able to split up your application into multiple services.

Ship: Docker lets you design the entire cycle of application development, testing and distribution, and manage it with a consistent user interface.

Portable across providers You want to be able to move your application between different cloud providers and your own servers, or run it across several providers.

Run: Docker offers you the ability to deploy scalable services, securely and reliably, on a wide variety of platforms.

Portable across environments You want to be able to define how your application will run in development, and then run it seamlessly in testing, staging and production.

1. Docker Architecture


Docker Containers vs Virtual Hypervisor Model



In the Docker container model, the Docker engine sits atop a single host operating system. In contrast, with the traditional virtualization hypervisor mode, a separate guest operating system is required for each virtual machine.

Docker Images and Docker Containers



2.1 Docker Images
Docker images are the build component of Docker.

For example, an image could contain an Ubuntu operating system with Apache and your web application installed. Images are used to create Docker containers.

Docker provides a simple way to build new images or update existing images, or you can download Docker images that other people have already created.

Built by you or other Docker users

Stored in the Docker Hub or your local Registry

Images Tags

Images are specified by repository:tag
The same image my have multiple tags
The default tag is lastest
Look up the repository on Docker to see what tags are available
2.2 Docker Containers
Docker containers are the run component of Docker.

Docker containers are similar to a directory. A Docker container holds everything that is needed for an application to run. Each container is created from a Docker image. Docker containers can be run, started, stopped, moved, and deleted. Each container is an isolated and secure application platform.

2.3 Registry and Repository
Docker registries are the distribution component of Docker.

Docker registries

Docker registries hold images. These are public or private stores from which you upload or download images. The public Docker registry is provided with the Docker Hub. It serves a huge collection of existing images for your use. These can be images you create yourself or you can use images that others have previously created. Docker Hub

Docker Hub is the public registry that contains a large number of images available for your use.

2.3 Docker Networking
Docker use DNS instead etc/hosts
3. Docker Orchestration
Three tools for orchestrating distributed applications with Docker
Docker Machine
Tool that provisions Docker hosts and installs the Docker Engine on them
Docker Swarm
Tool that clusters many Engines and schedules containers
Docker Compose
Tool to create and manage multi-container applications
Installation
Docker only supports CentOS 7,  Windows 7.1, 8/8.1, Mac OSX 10.8 64bit

Windows, CentOS

Usage
Lab: Search for Images on Docker Hub
Installation: Ubuntu
Installation
ubuntu 14.04

Prerequisites

apt-get update
apt-get install apt-transport-https ca-certificates

apt-key adv --keyserver hkp://p80.pool.sks-keyservers.net:80 --recv-keys 58118E89F3A912897C070ADBF76221572C52609D
Edit /etc/apt/sources.list.d/docker.list

deb https://apt.dockerproject.org/repo ubuntu-trusty main
apt-get update
apt-get purge lxc-docker
apt-cache policy docker-engine
Install docker

apt-get update
apt-get install -y --force-yes docker-engine
service docker start
Run Docker

docker run hello-world
docker info
docker version
Installation
https://www.docker.com/products/docker-toolbox

Go to the Docker Toolbox page.
Click the installer link to download.
Install Docker Toolbox by double-clicking the installer.
Docker: CentOS 7
Installation
Install Docker

# make sure your existing yum packages are up-to-date.
sudo yum -y update

# add docker repo
sudo tee /etc/yum.repos.d/docker.repo <<-'EOF'
[dockerrepo]
name=Docker Repository
baseurl=https://yum.dockerproject.org/repo/main/centos/$releasever/
enabled=1
gpgcheck=1
gpgkey=https://yum.dockerproject.org/gpg
EOF

# install the Docker package.
sudo yum install -y docker-engine

# start the Docker daemon
sudo service docker start

# verify
sudo docker run hello-world
Install Docker Compose

sudo yum install epel-release
sudo yum install -y python-pip

sudo pip install docker-compose
Docker 1.10.3

Kitematic
Port Forwarding
Open Virtualbox

Volumes
Install Docker for Windows ↩

Self Paced Training, Docker Fundamentals ↩

Understand the architecture ↩

Understand the architecture ↩

Docker Compose and Networking ↩

https://www.docker.com/ ↩

Orchestrating Docker with Machine, Swarm and Compose ↩

\section{Dev}

Create New Image
# Create new image by commit
docker commit <container ID> <yourname>/curl:1.0

# Build image
docker build -t ubuntu/test:1.0 .
docker build -t ubuntu/test:1.0 --no-cache .
Import/Export Container 1 2
Export the contents of a container's filesystem as a tar archive

docker export [OPTIONS] CONTAINER
docker export red_panda > latest.tar

docker import file|URL|- [REPOSITORY[:TAG]]
docker import http://example.com/exampleimage.tgz
cat exampleimage.tgz | docker import - exampleimagelocal:new
Save/Load Image 3 4
Save one or more images to a tar archive

docker save [OPTIONS] IMAGE [IMAGE...]
docker save busybox > busybox.tar.gz

docker load [OPTIONS]
docker load < busybox.tar.gz
Push Images to Docker Hub
# Login to docker hub
docker login --username=yourhubusername --email=youremail@company.com

# push docker
docker push [repo:tag]

# tag an image into a repository
docker tag [OPTIONS] IMAGE[:TAG] [REGISTRYHOST/][USERNAME/]NAME[:TAG]
Windows
5 Useful Docker Tips and Tricks on Windows

Commandline Reference: export ↩

Commandline Reference: import ↩

Commandline Reference: save ↩

Commandline Reference: load ↩

\section{Dockerfile}

Build a song with Dockerfile

A dockerfile is a configuration file that contains instructions for building a Docker image.

Provides a more effective way to build images compared to using docker commit
Easily fits into your continuous integration and deployment process.
Command Line

# `FROM` instruction specifies what the base image should be
FROM java:7

# `RUN` instruction specifies a command to execute
RUN apt-get update && apt-get install -y \
      curl \
      vim \
      openjdk-7-jdk

# CMD
# Defines a default command to execute when a container is created
CMD ["ping", "127.0.0.1", "-c", "30"]

# ENTRYPOINT
# Defines the command that will run when a container is executed
ENTRYPOINT ["ping"]
Volumes

Configuration Files and Directories**

COPY <src>... <dest>
COPY hom* /mydir/        # adds all files starting with "hom"
COPY hom?.txt /mydir/    # ? is replaced with any single character, e.g., "home.txt"
Networking

Ports still need to be mapped when container is executed
# EXPOSE
# Configures which ports a container will listen on at runtime
FROM ubuntu:14.04
RUN apt-get update
RUN apt-get install -y nginx

EXPOSE 80 443

CMD ["nginx", "-g", "daemon off;"]
2.4 Linking Containers
Example: Web app container and Database container

Linking is a communication method between containers which allows them to securely transfer data from one to another

Source and recipient containers
Recipient containers have access to data on source containers
Links are established based on container names
Create a Link

Create the source container first
Create the recipient container and use the --link option

Best practice - give your containers meaningful names

Create the source container using the postgres
docker run -d --name database postgres

Create the recipient container and link it
docker run -d -P --name website --link database:db nginx
Uses of Linking

Containers can talk to each other without having to expose ports to the host
Essential for micro service application architecture
Example:
Container with the Tomcat running
Container with the MySQL running
Application on Tomcat needs to connect to MySQL
Operating Systems for Docker 1 2
Operating System	Features
CoreOS	Service Discovery, Cluster management, Auto-updates
CoreOS is designed for security, consistency, and reliability. Instead of installing packages via yum or apt, CoreOS uses Linux containers to manage your services at a higher level of abstraction. A single service’s code and all dependencies are packaged within a container that can be run on one or many CoreOS machines
The New Minimalist Operating Systems ↩

Top 5 operating systems for your Docker infrastructure ↩

\section{Container}

Lifecycle
docker create creates a container but does not start it.
docker rename allows the container to be renamed.
docker run creates and starts a container in one operation.
docker rm deletes a container.
docker update updates a container's resource limits.
Starting and Stopping
docker start starts a container so it is running.
docker stop stops a running container.
docker restart stops and starts a container.
docker pause pauses a running container, "freezing" it in place.
docker unpause will unpause a running container.
docker wait blocks until running container stops.
docker kill sends a SIGKILL to a running container.
docker attach will connect to a running container.
Info
docker ps shows running containers.
docker logs gets logs from container. (You can use a custom log driver, but logs is only available for json-file and * journald in 1.10).
docker inspect looks at all the info on a container (including IP address).
docker events gets events from container.
docker port shows public facing port of container.
docker top shows running processes in container.
docker stats shows containers' resource usage statistics.
docker diff shows changed files in the container's FS.
Import / Export
docker cp copies files or folders between a container and the local filesystem.
docker export turns container filesystem into tarball archive stream to STDOUT.
Example

docker cp foo.txt mycontainer:/foo.txt
docker cp mycontainer:/foo.txt foo.txt
Note that docker cp is not new in Docker 1.8. In older versions of Docker, the docker cp command only allowed copying files from a container to the host

Executing Commands
docker exec to execute a command in container.
To enter a running container, attach a new shell process to a running container called foo

docker exec -it foo /bin/bash.
Suggest Readings: docker cheatsheet

\section{Compose}

Compose a symphony

Compose is a tool for defining and running multi-container Docker applications. With Compose, you use a Compose file to configure your application’s services. Then, using a single command, you create and start all the services from your configuration. To learn more about all the features of Compose see the list of features. Docker Compose Describes the components of an application

Box your component

.component/
├── docker-compose.yml
├── app-1/
│   ├── app/
│   │   ├── file-1
│   │   └── file-2
│   ├── Dockerfile
│   └── run.sh
└── app-2/
    ├── app/
    │   ├── file-1
    │   └── file-2
    ├── Dockerfile
    └── run.sh
Example
docker-compose.yml example from docker-birthday-3 3

version: "2"

services:
  voting-app:
    build: ./voting-app/.
    volumes:
      - ./voting-app:/app
    ports:
      - "5000:80"
    networks:
      - front-tier
      - back-tier

  result-app:
    build: ./result-app/.
    volumes:
      - ./result-app:/app
    ports:
      - "5001:80"
    networks:
      - front-tier
      - back-tier

  worker:
    image: manomarks/worker
    networks:
      - back-tier

  redis:
    image: redis:alpine
    container_name: redis
    ports: ["6379"]
    networks:
      - back-tier

  db:
    image: postgres:9.4
    container_name: db
    volumes:
      - "db-data:/var/lib/postgresql/data"
    networks:
      - back-tier

volumes:
  db-data:

networks:
  front-tier:
  back-tier:
Networks
network_mode, ports, external_hosts, link, net

network_mode: "bridge"
network_mode: "host"
network_mode: "service:[service_name]"
network_mode: "container:[container name/id]"
Volumes
Ops
# rebuild your images
docker-compose build

# build or rebuild services
docker-compose build [SERVICE...]
docker-compose build --no-cache database


# start all the containers
docker-compose up -d
docker-compose up

# stops the containers
docker-compose stop [CONTAINER]
Keep a container running in compose 2

You can use one of those lines

command: "top"
command: "tail -f /dev/null"
command: "sleep infinity"
Docker Compose Files Version 2 ↩

github issue, Keep a container running in compose ↩

docker-compose.yml ↩

\section{Swarm}

Docker Swarm is native clustering for Docker. It turns a pool of Docker hosts into a single, virtual Docker host. Because Docker Swarm serves the standard Docker API, any tool that already communicates with a Docker daemon can use Swarm to transparently scale to multiple hosts 1

Architecture 2


Swarm Manager: Docker Swarm has a Master or Manager, that is a pre-defined Docker Host, and is a single point for all administration. Currently only a single instance of manager is allowed in the cluster. This is a SPOF for high availability architectures and additional managers will be allowed in a future version of Swarm with #598.

Swarm Nodes: The containers are deployed on Nodes that are additional Docker Hosts. Each Swarm Node must be accessible by the manager, each node must listen to the same network interface (TCP port). Each node runs a node agent that registers the referenced Docker daemon, monitors it, and updates the discovery backend with the node’s status. The containers run on a node.

Scheduler Strategy: Different scheduler strategies (binpack, spread, and random) can be applied to pick the best node to run your container. The default strategy is spread which optimizes the node for least number of running containers. There are multiple kinds of filters, such as constraints and affinity. This should allow for a decent scheduling algorithm.

Node Discovery Service: By default, Swarm uses hosted discovery service, based on Docker Hub, using tokens to discover nodes that are part of a cluster. However etcd, consul, and zookeeper can be also be used for service discovery as well. This is particularly useful if there is no access to Internet, or you are running the setup in a closed network. A new discovery backend can be created as explained here. It would be useful to have the hosted Discovery Service inside the firewall and #660 will discuss this.

Standard Docker API: Docker Swarm serves the standard Docker API and thus any tool that talks to a single Docker host will seamlessly scale to multiple hosts now. That means if you were using shell scripts using Docker CLI to configure multiple Docker hosts, the same CLI would can now talk to Swarm cluster and Docker Swarm will then act as proxy and run it on the cluster.

Create Swarm
Simple Swarm
Swarm with CentOS
Swarm with Windows
Swarm and Compose 5
Create docker-compose.yml file

Example

wget https://docs.docker.com/swarm/swarm_at_scale/docker-compose.yml
Discovery 4


Scheduling 3
Docker Swarm: CentOS
Prerequisites
Docker 1.10.3
CentOS 7
Create cluster 1
Receipts: consult

Configure daemon in each host 6

Edit /etc/systemd/system/docker.service.d/docker.conf

[Service]
ExecStart=
ExecStart=/usr/bin/docker daemon -H fd:// -D -H tcp://<node_ip>:2375 --cluster-store=consul://<consul_ip>:8500 --cluster-advertise=<node_ip>:2375
Restart docker

systemctl daemon-reload
systemctl restart docker
Verify docker run with config

ps aux | grep docker | grep -v grep
Run cluster

In discovery host

# run consul
docker run -d -p 8500:8500 --name=consul progrium/consul -server -bootstrap
In swarm manage host

# run swarm master
docker run -d -p 2376:2375 swarm manage consul://<consul_ip>:8500/swarm
In swarm node host

# open 2375 port
firewalld-cmd --get-active-zones
firewall-cmd --zone=public --add-port=2375/tcp --permanent
firewall-cmd --reload

# run swarm node
docker run -d swarm join --addr=<node_ip>:2375 consul://<consul_ip>:8500/swarm
Question: Why manager and consul cannot run in the same host? (because docker-proxy?)

Verify 2

# manage
docker -H :2376 info
export DOCKER_HOST=tcp://<ip>:12375
docker info
docker run -d -P redis

# debug a swarm
docker swarm --debug manage consul://<consul_ip>:8500
Docker Swarm: Simple Swarm
Simple Swarm in one node with token 1
Change file vi /etc/default/docker

DOCKER_OPTS="-H tcp://0.0.0.0:2375 -H unix:///var/run/docker.sock"

# Restart docker
service docker restart

# create swarm token
docker run --rm swarm create
> token_id

# Run swarm node
docker run -d swarm join --addr=ip:2375 token://token_id
docker ps

# Run swarm manager
docker run -d -p 12375:2375 swarm manage token://token_id

# Swarm info
docker -H :12375 info

export DOCKER_HOST=tcp://ip:12375
docker info
docker run -d -P redis
Docker Swarm: Windows
Prerequisites
Docker 1.10.3
Giới thiệu và chạy thử Docker Swarm ↩

Docker-swarm, Cannot connect to the docker engine endpoint ↩

Docker Swarm Part 3: Scheduling ↩

Docker Swarm Part 2: Discovery ↩

Deploy the application ↩

Configuring and running Docker on various distributions ↩

\section{UCP}

Docker Universal Control Plane

Docker Universal Control Plane provides an on-premises, or virtual private cloud (VPC) container management solution for Docker apps and infrastructure regardless of where your applications are running.

[embed]https://www.youtube.com/watch?v=woDzgqtZZKc[/embed]

\section{Labs}

Lab 1:

# 1. Create a container from an Ubuntu image and run a bash terminal
docker run -it ubuntu /bin/bash
# 2.Inside the container install curl
apt-get install curl
# 3. Exit the container terminal
Ctrl-P
# 4. Run `docker ps -a` and take not of your container ID
# 5. Save the container as a new image. For the repository name use <yourname>/curl. Tag the image as 1.0
# 6. Run `docker images` and verify that you can see your new image
Lab 2: Run Ubntu

FROM ubuntu:14.04
RUN apt-get update && apt-get install -y curl \ vim
# 1. Go into the test folder and open your Dockerfile from the previous exercise

# 2. Add the folloing line to the end
CMD ["ping", "127.0.0.1", "-c", "30"]

# 3. Build the image
docker build -t <yourname>/testimage:1.1 .

# 4. Execute a container from the image and observe the output
docker run <yourname>/testimage:1.1

# 5. Execute another container from the image and specify the echo command
docker run <yourname>/testimage:1.1 echo "hello world"

# 6. Observe how the container argument overrides the CMD instruction
Lab:

1) In your home directory, create a folder called test
2) In the test folder, create a file called `Dockerfile`
3) In the file, specify to use Ubuntu 14.04 as the base image
FROM ubuntu:14.04
4) Write an instruction to install curl and vim after an apt-get update
RUN apt-get update && apt-get install -y curl \ vim
5) Build an image from Dockerfile. Give it the repository <yourname>/testimage and tag it as 1.0
docker build -t johnnytu/testimage:1.0 .
6) Create a container using your newly build image and verify that curl and vim are installed.
Lab: Run a container and get Terminal Access

* Create a container using the ubuntu image and connect to STDIN and a terminal
docker run -i -t ubuntu /bin/bash
* In your container, create a new user using your first and last name as the
username
addusername username
* Exit the container
exit
* Notice how the container shutdown
* Once again run:
docker run -i -t ubuntu /bin/bash
* Try and find your user `cat /etc/passwd`
* Noice that it does not exist.
Lab: Run a web application inside a container

# The -P flag to map container ports to host ports
docker run -d -P tomcat:7
docker ps     # check your image details runing
go to <server url>:<port number> # verify that you can see the Tomcat page
Lab Create a Docker Hub Account

Lab Push Images to Docker Hub

# Login to docker hub
$ docker login --username=yourhubusername --email=youremail@company.com
Password:
WARNING: login credentials saved in C:\Users\sven\.docker\config.json
Login Succeeded

# Use `docker push` command
docker push [repo:tag]

# Local repo must have same name and tag as the Docker Hub repo

# Used  to rename a local image repository before pushing to Docker Hub
docker tag [image ID] [repo:tag]
docker tag [local repo:tag] [Docker Hub repo:tag]

> tag image with ID (trainingteam/testexample is the name of repository on Docker hub)
docker tag edfc212de17b trainingteam/testexample:1.0

> tag image using the local repository tag
docker tag johnnytu/testimage:1.5 trainingteam/testexample

\section{GUI}

Environment

Ubuntu 14.04
Docker Engine 1.11
note: I can't make it run on CentOS.

Dockerfile 1
[code]

Base docker image
FROM debian:sid MAINTAINER Jessica Frazelle jess@docker.com

ADD https://dl.google.com/linux/direct/google-talkplugin_current_amd64.deb /src/google-talkplugin_current_amd64.deb

Install Chrome
RUN echo 'deb http://httpredir.debian.org/debian testing main' >> /etc/apt/sources.list && \ apt-get update && apt-get install -y \ ca-certificates \ curl \ hicolor-icon-theme \ libgl1-mesa-dri \ libgl1-mesa-glx \ libv4l-0 \ -t testing \ fonts-symbola \ --no-install-recommends \ && curl -sSL https://dl.google.com/linux/linux_signing_key.pub | apt-key add - \ && echo "deb [arch=amd64] http://dl.google.com/linux/chrome/deb/ stable main" > /etc/apt/sources.list.d/google.list \ && apt-get update && apt-get install -y \ google-chrome-stable \ --no-install-recommends \ && dpkg -i '/src/google-talkplugin_current_amd64.deb' \ && apt-get purge --auto-remove -y curl \ && rm -rf /var/lib/apt/lists/ \ && rm -rf /src/.deb

COPY local.conf /etc/fonts/local.conf

RUN rm /etc/apt/sources.list.d/google-chrome.list

RUN apt-get update RUN apt-get install -y libcanberra-gtk-module RUN apt-get install -y dbus libdbus-1-dev libxml2-dev dbus-x11

Autorun chrome
ENTRYPOINT [ "google-chrome" ] CMD [ "--user-data-dir=/data" ] [/code]

Build Image
[code] docker build -t chrome . [/code]

Docker Run
[code] xhost + docker run -it --net host --cpuset-cpus 0 -v /tmp/.X11-unix:/tmp/.X11-unix -e DISPLAY=unix$DISPLAY -v $HOME/chrome/data/Downloads:/root/Downloads -v $HOME/chrome/data/:/data -v /deb/shm:/dev/shm --device /dev/snd chrome --user-data-dir=/data www.google.com [/code]

Known Issue
Issue 1: Failed to load module "canberra-gtk-module" 2

[code] Gtk-Message: Failed to load module "canberra-gtk-module" [/code]

Issue 2: Chrome crash 3

Solution: add -v /dev/shm:/dev/shm


\begin{itemize}
  \item \href{https://hub.docker.com/r/jess/chrome/~/dockerfile/}{https://hub.docker.com/r/jess/chrome/~/dockerfile/}
  \item \href{http://fabiorehm.com/blog/2014/09/11/running-gui-apps-with-docker/#comment-2515573749}{http://fabiorehm.com/blog/2014/09/11/running-gui-apps-with-docker/#comment-2515573749}
\end{itemize}


