\chapter{Trở thành giảng viên}

\diary{Hơi bị hay \href{https://techmaster.vn/khoa-hoc/25484/tro-thanh-giang-vien-gioi}{Ai cũng có thể trở thành giảng viên giỏi nếu} của anh Cường tại techmaster. Tưởng phải mua, hóa ra lại được set free. A hihi.}

\begin{itemize}
  \item Lên ý tưởng
  \item Chuẩn bị nội dung
  \item Code mẫu
  \item Xây dựng kịch bản
  \item Trình bày đơn giản, dễ hiểu
\end{itemize}


Một bài giảng

\section{Khác biệt giữa dạy online và offline}

Dạy online

\textbf{Ưu điểm}

\begin{enumerate}
  \item Khả năng mở rộng rất tốt
  \item Chi phí thấp
  \item Số lượng học viên không giới hạn
\end{enumerate}

\textbf{Nhược điểm}

\begin{itemize}
  \item Tỷ lệ bỏ học rất cao
  \item Tương tác chưa thực sự tốt
\end{itemize}

\subsection{Làm sao để giảm tỷ lệ bỏ học trực tuyến?}

\begin{itemize}
  \item Video ngắn < 10 phút
  \item Cấp chứng chỉ 70-90 USD lấy 1 chứng chỉ cho khóa học 4-6 tuần
  \item Chấm bài tập
  \item Facebook group kết nối giảng viên học viên
  \item Bài tập đủ dễ: chia nhỏ dự án khó ra
  \item Gamification: học như chơi
  \item Chịu khó email thông báo cho lớp
\end{itemize}

\section{Marketing}

Các hình thức marketing

1. Mạng xã hội: Facebook, forum
Phải boost quảng cáo. 500000 - 2000000 VND cho mỗi khóa học

2. SEO - Google Search

3. Qua hội thảo, meetup ~ hiệu quả thấp do số lượng người tham gia < 40 người

Conversation rate: số người mua hàng / số người tham quan

Nếu đến từ organic search CR = 1 / 200

Nếu qua chia sẻ bài viết mạng xã hội CR = 1 / 4000

Nếu qua giới thiệu, tư vấn người quen CR = 1 / 2

\section{Flip Learning}

Lớp học đảo ngược

Vấn đề học online

\begin{itemize}
  \item Bỏ học cao
  \item Tương tác face 2 face kém
  \item Thực hành không có, không kiểm soát
\end{itemize}

Vấn đề học offline

\begin{itemize}
  \item Chi phí cao
  \item Giao thông không thuận tiện
  \item Tuyển sinh khó
\end{itemize}

Flip Learning lớp học đảo ngược, kết hợp ưu điểm học trực tuyến và thực hành phòng lab.

- Khuyến khích học viên xem trước bài giảng video hướng dẫn giảng viên

- Đọc tìm hiểu thêm, trả lời trắc nghiệm qua Internet

- Tại buổi học, dành tối đa thời gian để \textbf{thực hành, hỏi đáp, hợp tác}

Khi đến lớp, học viên đã có kiến thức, biết rõ mình sẽ làm gì.

- Không thụ động, rèn tính tự học, tự đọc

- Không chống cằm nhìn giảng viên, không ghi chép

- Hỏi, làm, nói, giúp đỡ nhau

\textbf{Flip Learning:}

- Giảng viên phải chuẩn bị bài giảng

- Đến lớp trao đổi, hướng dẫn

\section{Thuyết trình}

Tự nhiên hôm nay (22/01/2018) lại đọc được bài \href{https://huynq.net/you-suck-at-powerpoint/}{You suck at PowerPoint}, thấy hay quá.

Sau đây là 10 lỗi thường gặp khi làm bài thuyết trình

\begin{enumerate}
  \item Quá nhiều chữ trong 1 slide.
  \item Màu chữ và màu nền không tương phản với nhau.
  \item Dùng clip art, word art.
  \item Hình ảnh sử dụng trong slide chất lượng kém, scale sai tỉ lệ.
  \item Sử dụng nhiều font chữ trong 1 slide.
  \item Lạm dụng quá nhiều hiệu ứng (animation/transition).
  \item Bài presentation không có cấu trúc.
  \item Slide không ăn nhập gì với nội dung trình bày.
  \item Không ghi rõ nguồn khi sử dụng tài liệu, hình ảnh của người khác.
  \item Ý thức của người làm slide
\end{enumerate}

\section{Xác định đối tượng học tập}

Học viên Techmaster là ai?

1- Sinh viên CNTT năm đầu, cần thực hành: 30%
2- Người thất nghiệp: 40%
3- Lập trình cần nâng cao kỹ năng 30%
4- Hầu hết là từ nông thôn 70%
5- Thời gian có hạn, cần tìm việc ngay

Họ cần gì?

1- Bài giảng thực tế, có nhiều ví dụ sinh động
2- Rất khác với ở trường đại học:
+ Không phải thi
+ Lý thuyết ít, thực hành nhiều, trừu tượng ít
+ Ví dụ phải cool
+ Có người hỗ trợ ngay
3- Trung tâm giới thiệu việc làm sau khi học

\section{Sai lầm cỗ hữu của giảng viên}

1- Nghĩ ai cũng biết như mình.

Học viên 80% từ nông thôn, 40% thất nghiệp
Giảng viên dùng toàn từ chuyên ngành tiếng Anh mà không giải thích cặn kẽ

2- Nói nhiều, lý thuyết nhiều, ít có ví dụ, thí nghiệm minh họa. Học viên đã rất chán kiểu dạy ở đại học

3- Khô cứng: giải thích về kế thừa

4- Ba hoa, bốc phét như bán hàng đa cấp. Lesson online giới hạn 10 phút, học offline chỉ có 140 phút.

5- Dùng nhiều tính từ, đại ngôn: rất, cực.. mà không có con số.

6- Cẩu thả: không chuẩn bị kỹ slide, slide toàn chữ
12 trang.

7- Mã nguồn, dự án ví dụ vô duyên, khó hiểu

\begin{lstlisting}
*
**
***
****
*****

*     *
 *   *
  * *
   *
\end{lstlisting}

8- Chia nhỏ các bài tập

\section{Xây dựng khung giáo trình}

1- Top Down tốt hơn Bottom Up

2- Tham khảo sách ebook, Udemy, PluralSights, Lynda...

3- Người bình thường không thể nhớ quá 5 mục trong một buổi học:
+ Offline: 1 buổi dạy 1 chủ đề chia thành 4 minitask
+ Online: 1 section gồm 4-5 video, mỗi video < 10 phút

4- Sử dụng MindMap

\section{Viết kịch bản nói}

1- Viết chi tiết chính xác đến từng câu, rồi đọc lại
+ Nói chưa lưu loát > viết chi tiết
2- Liệt kê ý chính, vừa demo, vừa nói
+ Nói tốt + lười
+ Hậu kỳ phải cắt bỏ nhiều đoạn nói nhịu, ề à
Câu từ dài dòng, rườm ra, tỷ lệ tiếp thu càng thấp
Học trực tuyến, học viên dùng mắt đọc - xem là chính 70\%, nghe là phụ, 30\%. Video qua 5 phút gây buồn ngủ.

Nhắc kịch bản

- Không nên in ra giấy !
- Màn hình rộng Dell Utra, để mở cửa sổ nhắc kịch bản
- Mở rộng thêm 1 màn hình mới
- Gõ kịch bản ra iPad, dựng iPad lên thành màn hình phụ

\section{Đồ nghề tạo giáo trình trực tuyến}

1- Web Cam

+ Logitech C920 hoặc Xioami Camera loại 650,000VND.
+ Tỷ lệ bỏ lửng bài video giảm 30\% nếu học viên thấy mặt giảng viên.
+ Mặt giảng viên đẹp trai, hấp dẫn, hài hước càng tốt
+ Mở sẵn cửa sổ ứng dụng cần demo. Giảm tối đa thời gian chết

2- Camtasia

+ Camtasia Studio trên Windows có nhiều chức năng hơn bản Camtasia Mac
+ Phiên bản Camtasia Mac không nén được video định dạng H264 phải dùng FFmpeg để nén
+ Thu ở khung hình 1280 * 720 pixels. Trên Mac có công cụ xScope để căn kích thước màn hình thu

3- Phòng thu âm cần yên tĩnh

+ Đóng chặt cửa sổ, cửa phòng khi quay chấp nhật nóng trong 10 phút quay.
+ Bật chế noise remove trong Camtasia

4- Sublime Text
5- Ink2Go
6- Lucid Chart
7- Adobe Photoshop
8- Skitch
9- PowerPoint
10- Slides.com

\section{Kinh nghiệm khi quay video}

Phòng cần yên tĩnh: nên tắt quạt hay các thiết bị điện tử gây ồn. Tắt điện thoại tránh cuộc gọi đến khi đang ghi hình
Để micro xa bàn phím để không lẫn tiếng lạch cạch

Tuyệt đối không nên để background wallpaper lòe loẹt, học viên không chú ý vào demo của bạn mà xem background thì hiệu quả truyền đạt rất tệ. Tốt nhất để màn hình background là đen 100%

Tắt Skype, chat, loại bỏ các icon không cần thiết trên màn hình desktop

Nên dùng web camera Logitech C920 để xa mồm để không ghi tiếng thở hổn hển của giảng viên

Nên quay cả mặt giảng viên, nhưng cut crop để không quay background của phòng thu. Hãy để học viên tập trung vào bài giảng

Không sử dụng âm thanh nên có tiếng hát hoặc giai điệu nhanh, phức tạp khiến học viên không tập trung

Lưu ý rằng:

Học viên thích xem demo sinh động và đọc chữ trên màn hình rất nhanh, đừng nói quá dài dòng, phức tạp.

\section{Thiết kế để học trực tuyến}

1- Một video < 10 phút. Tốt nhất 3 - 5 phút

2- Ví dụ là một dự án mẫu chạy được. Đơn giản tối đa

3- Học viên phấn khích khi thấy kết quả cuối cùng

4- Chia nhỏ task top-down:

+ Sản phẩm này làm gì?

+ Có chức năng gì? chỉ nên 1 hoặc 2

+ Màn hình chính

\section{Upload video và tạo bài giảng}

Tạo một lesson như thế nào?

- Có một video dài < 10 phút

- Tối thiểu 3 câu hỏi quiz

- Nên bổ sung mã nguồn hoặc slide

\section{Tính logic cấu trúc bài giảng}

Không nên:

- Tùy tiện theo ý thích hoặc kinh nghiệm cá nhân giảng viên
- Giập khuôn theo sách hoặc khóa học có sẵn
Hậu quả học nhiều, nhưng kỹ năng, kiến thức lộn xộn. Thà học ít mà dùng được nhiều, hiệu quả còn hơn

Nên:

- Thiết kế top down dùng Mindmap
- Các mẩu kiến thức có liên kết, sâu chuỗi

- Trước - Sau:
 thủ tục -> hướng đối tượng -> design pattern
 array -> dictionary -> linked list

 SQL raw query -> Objection Relation Mapping

- Kế thừa: Cha - Con, Chị - Em

 UIView > UIScollView > UITableView, UICollectionView

- Đối lập - so sánh:

 Postgresql <> MongoDB, FireBase <> RethinkDB
 Apple Push Notification <> Google Cloud Messaging

- Phân nhóm theo tiêu chí: Dễ học, dùng thường xuyên, học viên sướng. Sướng quyết định tất cả

Ví du:

- Lẻ tẻ, minh họa cụ thể từng cú pháp, API function. Học viên không thu hoạch được nhiều
- Một dự án kết hợp các bước sẽ thú vị hơn, dài hơn. Hay lúc đầu, buồn ngủ lúc sau.
