\chapter{Phương pháp làm việc}

Xây dựng phương pháp làm việc là một điều không đơn giản. Với kinh nghiệm 3 năm làm việc, trải qua 2 project. Mà vẫn chưa produce được sản phẩm cho khách hàng. Thiết nghĩ mình nên viết phương pháp làm việc ra để xem xét lại. Có lẽ sẽ có ích cho mọi người.

Làm sao để làm việc hiệu quả, hay xây dựng phương pháp làm việc hữu ích? Câu trả lời ngắn gọn là "Một công cụ không bao giờ đủ".

<!--more-->

### Nội dung

1. [Làm sao để đánh giá công việc trong khoảng thời gian dài hạn?](#section1)
2. [Làm sao để quản lý project?](#section2)
3. [Làm sao để công việc trôi chảy?](#section3)
4. [Làm sao để xem xét lại quá trình làm việc?](#section4)

<p id="section1">&nbsp;</p>

### Làm sao để đánh giá công việc trong khoảng thời gian dài hạn?

Câu trả lời OKR (Objectives and Key Results)

![](https://image.slidesharecdn.com/20170829-ale2017-okr-170829060043/95/agile-leadership-and-goal-management-with-objectives-key-results-okrs-ale-2017-prague-23-638.jpg?cb=1503986541)
*OKR Framework*

Đầu mỗi quý ⏱, nên dành vài ngày cho việc xây dựng mục tiêu và những kết quả quan trọng cho quý tới. Cũng như review lại ☕ kết quả quý trước.

Bước 1: Xây dựng mục tiêu cá nhân (Objectives)

Bước 2: Xây dựng các Key Results cho mục tiêu này

Bước 3: Lên kế hoạch để hiện thực hóa các Key Results

<p id="section2">&nbsp;</p>

### Làm sao để quản lý một project

Meistertask

![](https://focus.meisterlabs.com/wp-content/uploads/2015/03/Projectboard_MT_EN-1.png)
*Meister Task*

<p id="section3">&nbsp;</p>

### Làm sao để công việc trôi chảy?

Có vẻ trello là công cụ thích hợp

Bước 1: Tạo một team với một cái tên thật ấn tượng (của mình là Strong Coder)

Trong phần Description của team, nên viết Objectives and Key Results của quý này

Sau đây là một ví dụ

```
Objectives and Key Results

-> Build Vietnamese Sentiment Analysis
-> Develop underthesea
-> Deep Learning Book
```

Bước 2: Đầu mỗi tuần, tạo một board với tên là thời gian ứng với tuần đó (của mình là `2017 | Fight 02 (11/12 - 16/12)`)

Board này sẽ gồm 5 mục: "TODO", "PROGRESSING", "Early Fight", "Late Fight", "HABBIT", được lấy cảm hứng từ Kanban Board

![](https://mktgcdn.leankit.com/uploads/images/general/_xLarge/kanban_guide_print_KPO_bleed_board2.jpg)
*Trello Board example*

* ⏱ Mỗi khi không có việc gì làm, xem xét card trong "TODO"
* ⏱ [FOCUS] tập trung làm việc trong "PROGRESSING"
* ☕ Xem xét lại thói quen làm việc với "HABBIT"

Một Card cho Trello cần có

* Tên công việc (title)
* Độ quan trọng (thể hiện ở label xanh (chưa quan trọng), vàng (bình thường), đỏ (quan trọng))
* Hạn chót của công việc (due date)

Sắp xếp TODO theo thứ tự độ quan trọng và Due date

<p id="section4">&nbsp;</p>

### Làm sao để xem xét lại quá trình làm việc?

Nhật lý làm việc hàng tuần ☕. Việc này lên được thực hiện vào đầu tuần ⏱. Có 3 nội dung quan trọng trong nhật ký làm việc (ngoài gió mây trăng cảm xúc, quan hệ với đồng nghiệp...)

* Kết quả công việc tuần này
* Những công việc chưa làm? Lý do tại sao chưa hoàn thành?
* Dự định cho tuần tới

*Đang nghiên cứu*

**Làm sao để lưu lại các ý tưởng, công việc cần làm?**: Dùng chức năng checklist của card trong meister. Khi có ý tưởng mới, sẽ thêm một mục trong checklist

**Làm sao để tập trung vào công việc quan trọng?**: Dùng chức năng tag của meister, mỗi một công việc sẽ được đánh sao (với các mức 5 sao, 3 sao, 1 sao), thể hiện mức độ quan trọng của công việc. Mỗi một sprint nên chỉ tập trung vào 10 star, một product backlog chỉ nên có 30 star.

**Tài liệu của dự án**: Sử dụng Google Drive, tài liệu mô tả dự án sẽ được link vào card tương ứng trong meister.

\chapter{Làm sao để thành công?}

\diary{27/01/2018: Hôm nay xem Shark Tank Việt Nam có vụ nói về các yếu tố để thành công hay quá.}

Vòng tròn thành công gồm có 3 phần : ĐAM MÊ, NĂNG LỰC, LỢI NHUẬN

ĐAM MÊ: là cái mình yêu thích làm


NĂNG LỰC: là cái mình làm tốt

LỢI NHUẬN: là cái đem lại giá trị cho người khác

Người Nhật thì có khái niệm về Ikigai (Lẽ sống) rất hay 

Cân bằng giữa 

* Việc bạn yêu thích (What you LOVE)
* Cái thế giới cần (What the WORLD NEEDS)
* Việc làm ra tiền (What you can be PAID FOR)
* Việc bạn làm giỏi (What you're GOOD AT)

ĐAM MÊ = PASSION = What you LOVE + What you're GOOD AT
SỨ MỆNH = MISSION = What you LOVE + What the WORLD NEEDS
CHUYÊN GIA = PROFESSION = What you can be PAID FOR + What you're GOOD AT
VIỆC LÀM = VOCATION = What you can be PAID FOR + What the WORLD NEEDS

ĐAM MÊ + CHUYÊN GIA = Thiếu Cái thế giới cần 
+ Thỏa mãn 
- Cảm thấy vô dụng 

ĐAM MÊ + SỨ MỆNH = Thiếu Việc làm ra tiền
+ Vui vẻ và no đủ
- Không giàu được 

VIỆC LÀM + CHUYÊN GIA = Thiếu Việc bạn yêu thích
+ Thoải mái, đủ đầy
- Cảm giác trống rỗng

VIỆC LÀM + SỨ MẠNH = Thiếu Việc bạn làm giỏi
+ Thú vị và hài lòng
- Có cảm giác bất an và bấp bênh 

https://mymodernmet.com/ikigai-japanese-life-philosophy/
https://nuockangen.com/2017/09/10/khai-niem-song-cua-nhat-ban-ikigai-co-phai-la-bi-quyet-cua-mot-cuoc-song-dai-hanh-phuc-y-nghia/
http://tapchitrungnien.com/cong-thuc-ikigai-bi-quyet-hanh-phuc-va-truong-tho-cua-nguoi-nhat.html

\chapter{Đặt tên email}

Email hiện tại là anhv.ict91@gmail.com. Chắc tốt hơn là email brother.rain.1024@gmail.com. Vụ này đã được anh Lê Hồng Phương nhắc một lần rồi.
