\chapter{Git}

View online \href{http://magizbox.com/training/git/site/}{http://magizbox.com/training/git/site/}

Git is a free and open source distributed version control system designed to handle everything from small to very large projects with speed and efficiency.

\section{Get Started}

Windows
Step 1. Download the Git for Windows installer package.

Step 2. When you've successfully started the installer, you should see the Git Setup wizard screen. Follow the Next and Finish prompts to complete the installation.



Step 3.Open Git Bash.vbs from the Git folder of the Programs directory to open a command window.



Step 4. Configure your username using the following command

$ git config --global user.name "your_name"
Step 5. Configure your email address using the following command

$ git config --global user.email "your_email"
Linux
Step 1. Enter the following command to install Git:

$ sudo apt-get install git
Step 2. Verify the installation was successful by typing which git at the command line.

$ which git
/opt/local/bin/git
Step 3. Configure your username using the following command.

$ git config --global user.name "Emma Paris"
Step 4. Configure your email address using the following command.

$ git config --global user.email "eparis@atlassian.com"

\section{Git Basic}

\subsection{Getting a Git Repository}

If you can read only one chapter to get going with Git, this is it. This chapter covers every basic command you need to do the vast majority of the things you’ll eventually spend your time doing with Git. By the end of the chapter, you should be able to configure and initialize a repository, begin and stop tracking files, and stage and commit changes. We’ll also show you how to set up Git to ignore certain files and file patterns, how to undo mistakes quickly and easily, how to browse the history of your project and view changes between commits, and how to push and pull from remote repositories.

Getting a Git Repository
You can get a Git project using two main approaches. The first takes an existing project or directory and imports it into Git. The second clones an existing Git repository from another server.

Initializing a Repository in an Existing Directory

If you’re starting to track an existing project in Git, you need to go to the project’s directory. If you’ve never done this, it looks a little different depending on which system you’re running:

for Linux:

$ cd /home/user/your_repository
for Mac:

$ cd /Users/user/your_repository
for Windows:

$ cd /c/user/your_repository
and type:

$ git init
This creates a new subdirectory named .git that contains all of your necessary repository files – a Git repository skeleton. At this point, nothing in your project is tracked yet. (See Git Internals for more information about exactly what files are contained in the .git directory you just created.)

If you want to start version-controlling existing files (as opposed to an empty directory), you should probably begin tracking those files and do an initial commit. You can accomplish that with a few git add commands that specify the files you want to track, followed by a git commit:

$ git add *.c
$ git add LICENSE
$ git commit -m 'initial project version'
We’ll go over what these commands do in just a minute. At this point, you have a Git repository with tracked files and an initial commit.

Cloning an Existing Repository
If you want to get a copy of an existing Git repository – for example, a project you’d like to contribute to – the command you need is git clone. If you’re familiar with other VCS systems such as Subversion, you’ll notice that the command is "clone" and not "checkout". This is an important distinction – instead of getting just a working copy, Git receives a full copy of nearly all data that the server has. Every version of every file for the history of the project is pulled down by default when you run git clone. In fact, if your server disk gets corrupted, you can often use nearly any of the clones on any client to set the server back to the state it was in when it was cloned (you may lose some server-side hooks and such, but all the versioned data would be there – see Git on the Server for more details).

You clone a repository with git clone [url]. For example, if you want to clone the Git linkable library called libgit2, you can do so like this:

$ git clone https://github.com/libgit2/libgit2
That creates a directory named “libgit2”, initializes a .git directory inside it, pulls down all the data for that repository, and checks out a working copy of the latest version. If you go into the new libgit2 directory, you’ll see the project files in there, ready to be worked on or used. If you want to clone the repository into a directory named something other than “libgit2”, you can specify that as the next command-line option:

$ git clone https://github.com/libgit2/libgit2 mylibgit
That command does the same thing as the previous one, but the target directory is called mylibgit.

Git has a number of different transfer protocols you can use. The previous example uses the https:// protocol, but you may also see git:// or luser@server:path/to/repo.git, which uses the SSH transfer protocol. Git on the Server will introduce all of the available options the server can set up to access your Git repository and the pros and cons of each.

\section{Commit}

Recording Changes to the Repository

You have a bona fide Git repository and a checkout or working copy of the files for that project. You need to make some changes and commit snapshots of those changes into your repository each time the project reaches a state you want to record.

Remember that each file in your working directory can be in one of two states: tracked or untracked. Tracked files are files that were in the last snapshot; they can be unmodified, modified, or staged. Untracked files are everything else – any files in your working directory that were not in your last snapshot and are not in your staging area. When you first clone a repository, all of your files will be tracked and unmodified because Git just checked them out and you haven’t edited anything.

As you edit files, Git sees them as modified, because you’ve changed them since your last commit. You stage these modified files and then commit all your staged changes, and the cycle repeats.



Figure 8. The lifecycle of the status of your files.

Checking the Status of Your Files
The main tool you use to determine which files are in which state is the git status command. If you run this command directly after a clone, you should see something like this:

$ git status
On branch master
Your branch is up-to-date with 'origin/master'.
nothing to commit, working directory clean
This means you have a clean working directory – in other words, none of your tracked files are modified. Git also doesn’t see any untracked files, or they would be listed here. Finally, the command tells you which branch you’re on and informs you that it has not diverged from the same branch on the server. For now, that branch is always “master”, which is the default; you won’t worry about it here. Git Branching will go over branches and references in detail.

Let’s say you add a new file to your project, a simple README file. If the file didn’t exist before, and you run git status, you see your untracked file like so:

$ echo 'My Project' > README
$ git status
On branch master
Your branch is up-to-date with 'origin/master'.
Untracked files:
  (use "git add <file>..." to include in what will be committed)

    README
nothing added to commit but untracked files present (use "git add" to track)
You can see that your new README file is untracked, because it’s under the “Untracked files” heading in your status output. Untracked basically means that Git sees a file you didn’t have in the previous snapshot (commit); Git won’t start including it in your commit snapshots until you explicitly tell it to do so. It does this so you don’t accidentally begin including generated binary files or other files that you did not mean to include. You do want to start including README, so let’s start tracking the file.

Tracking New Files
In order to begin tracking a new file, you use the command git add. To begin tracking the README file, you can run this:

$ git add README
If you run your status command again, you can see that your README file is now tracked and staged to be committed:

$ git status

On branch master
Your branch is up-to-date with 'origin/master'.
Changes to be committed:
  (use "git reset HEAD <file>..." to unstage)

    new file:   README
You can tell that it’s staged because it’s under the “Changes to be committed” heading. If you commit at this point, the version of the file at the time you ran git add is what will be in the historical snapshot. You may recall that when you ran git init earlier, you then ran git add (files) – that was to begin tracking files in your directory. The git add command takes a path name for either a file or a directory; if it’s a directory, the command adds all the files in that directory recursively.

Staging Modified Files
Let’s change a file that was already tracked. If you change a previously tracked file called CONTRIBUTING.md and then run your git status command again, you get something that looks like this:

$ git status
On branch master
Your branch is up-to-date with 'origin/master'.
Changes to be committed:
  (use "git reset HEAD <file>..." to unstage)

    new file:   README

Changes not staged for commit:
  (use "git add <file>..." to update what will be committed)
  (use "git checkout -- <file>..." to discard changes in working directory)

    modified:   CONTRIBUTING.md
The CONTRIBUTING.md file appears under a section named “Changes not staged for commit” – which means that a file that is tracked has been modified in the working directory but not yet staged. To stage it, you run the git add command. git add is a multipurpose command – you use it to begin tracking new files, to stage files, and to do other things like marking merge-conflicted files as resolved. It may be helpful to think of it more as “add this content to the next commit” rather than “add this file to the project”. Let’s run git add now to stage the CONTRIBUTING.md file, and then run git status again:

$ git add CONTRIBUTING.md
$ git status
On branch master
Your branch is up-to-date with 'origin/master'.
Changes to be committed:
  (use "git reset HEAD <file>..." to unstage)

    new file:   README
    modified:   CONTRIBUTING.md
Both files are staged and will go into your next commit. At this point, suppose you remember one little change that you want to make in CONTRIBUTING.md before you commit it. You open it again and make that change, and you’re ready to commit. However, let’s run git status one more time:

$ vim CONTRIBUTING.md
$ git status
On branch master
Your branch is up-to-date with 'origin/master'.
Changes to be committed:
  (use "git reset HEAD <file>..." to unstage)

    new file:   README
    modified:   CONTRIBUTING.md

Changes not staged for commit:
  (use "git add <file>..." to update what will be committed)
  (use "git checkout -- <file>..." to discard changes in working directory)

    modified:   CONTRIBUTING.md
What the heck? Now CONTRIBUTING.md is listed as both staged and unstaged. How is that possible? It turns out that Git stages a file exactly as it is when you run the git add command. If you commit now, the version of CONTRIBUTING.md as it was when you last ran the git add command is how it will go into the commit, not the version of the file as it looks in your working directory when you run git commit. If you modify a file after you run git add, you have to run git add again to stage the latest version of the file:

$ git add CONTRIBUTING.md
$ git status
On branch master
Your branch is up-to-date with 'origin/master'.
Changes to be committed:
  (use "git reset HEAD <file>..." to unstage)

    new file:   README
    modified:   CONTRIBUTING.md
Short Status
While the git status output is pretty comprehensive, it’s also quite wordy. Git also has a short status flag so you can see your changes in a more compact way. If you run git status -s or git status --short you get a far more simplified output from the command:

$ git status -s
 M README
MM Rakefile
A  lib/git.rb
M  lib/simplegit.rb
?? LICENSE.txt
New files that aren’t tracked have a ?? next to them, new files that have been added to the staging area have an A, modified files have an M and so on. There are two columns to the output - the left-hand column indicates the status of the staging area and the right-hand column indicates the status of the working tree. So for example in that output, the README file is modified in the working directory but not yet staged, while the lib/simplegit.rb file is modified and staged. The Rakefile was modified, staged and then modified again, so there are changes to it that are both staged and unstaged.

Ignoring Files
Often, you’ll have a class of files that you don’t want Git to automatically add or even show you as being untracked. These are generally automatically generated files such as log files or files produced by your build system. In such cases, you can create a file listing patterns to match them named .gitignore. Here is an example .gitignore file:

$ cat .gitignore
*.[oa]
*~
The first line tells Git to ignore any files ending in “.o” or “.a” – object and archive files that may be the product of building your code. The second line tells Git to ignore all files whose names end with a tilde (~), which is used by many text editors such as Emacs to mark temporary files. You may also include a log, tmp, or pid directory; automatically generated documentation; and so on. Setting up a .gitignore file before you get going is generally a good idea so you don’t accidentally commit files that you really don’t want in your Git repository.

The rules for the patterns you can put in the .gitignore file are as follows:

Blank lines or lines starting with # are ignored.
Standard glob patterns work.
You can start patterns with a forward slash (/) to avoid recursivity.
You can end patterns with a forward slash (/) to specify a directory.
You can negate a pattern by starting it with an exclamation point (!).
Glob patterns are like simplified regular expressions that shells use. An asterisk (*) matches zero or more characters; [abc] matches any character inside the brackets (in this case a, b, or c); a question mark (`?) matches a single character; and brackets enclosing characters separated by a hyphen ([0-9]) matches any character between them (in this case 0 through 9). You can also use two asterisks to match nested directories;a/**/zwould matcha/z,a/b/z,a/b/c/z`, and so on.

Here is another example .gitignore file:

# no .a files
*.a

# but do track lib.a, even though you're ignoring .a files above
!lib.a

# only ignore the TODO file in the current directory, not subdir/TODO
/TODO

# ignore all files in the build/ directory
build/

# ignore doc/notes.txt, but not doc/server/arch.txt
doc/*.txt

# ignore all .pdf files in the doc/ directory and any of its subdirectories
doc/**/*.pdf
Tip

GitHub maintains a fairly comprehensive list of good .gitignore file examples for dozens of projects and languages at https://github.com/github/gitignore if you want a starting point for your project.

Viewing Your Staged and Unstaged Changes
If the git status command is too vague for you – you want to know exactly what you changed, not just which files were changed – you can use the git diff command. We’ll cover git diff in more detail later, but you’ll probably use it most often to answer these two questions: What have you changed but not yet staged? And what have you staged that you are about to commit? Although git status answers those questions very generally by listing the file names, git diff shows you the exact lines added and removed – the patch, as it were.

Let’s say you edit and stage the README file again and then edit the CONTRIBUTING.md file without staging it. If you run your git status command, you once again see something like this:

$ git status
On branch master
Your branch is up-to-date with 'origin/master'.
Changes to be committed:
  (use "git reset HEAD <file>..." to unstage)

    modified:   README

Changes not staged for commit:
  (use "git add <file>..." to update what will be committed)
  (use "git checkout -- <file>..." to discard changes in working directory)

    modified:   CONTRIBUTING.md
To see what you’ve changed but not yet staged, type git diff with no other arguments:

$ git diff
diff --git a/CONTRIBUTING.md b/CONTRIBUTING.md
index 8ebb991..643e24f 100644
--- a/CONTRIBUTING.md
+++ b/CONTRIBUTING.md
@@ -65,7 +65,8 @@ branch directly, things can get messy.
 Please include a nice description of your changes when you submit your PR;
 if we have to read the whole diff to figure out why you're contributing
 in the first place, you're less likely to get feedback and have your change
-merged in.
+merged in. Also, split your changes into comprehensive chunks if your patch is
+longer than a dozen lines.

 If you are starting to work on a particular area, feel free to submit a PR
 that highlights your work in progress (and note in the PR title that it's
That command compares what is in your working directory with what is in your staging area. The result tells you the changes you’ve made that you haven’t yet staged.

If you want to see what you’ve staged that will go into your next commit, you can use git diff --staged. This command compares your staged changes to your last commit:

$ git diff --staged
diff --git a/README b/README
new file mode 100644
index 0000000..03902a1
--- /dev/null
+++ b/README
@@ -0,0 +1 @@
+My Project
It’s important to note that git diff by itself doesn’t show all changes made since your last commit – only changes that are still unstaged. This can be confusing, because if you’ve staged all of your changes, git diff will give you no output.

For another example, if you stage the CONTRIBUTING.md file and then edit it, you can use git diff to see the changes in the file that are staged and the changes that are unstaged. If our environment looks like this:

$ git add CONTRIBUTING.md
$ echo '# test line' >> CONTRIBUTING.md
$ git status
On branch master
Your branch is up-to-date with 'origin/master'.
Changes to be committed:
  (use "git reset HEAD <file>..." to unstage)

    modified:   CONTRIBUTING.md

Changes not staged for commit:
  (use "git add <file>..." to update what will be committed)
  (use "git checkout -- <file>..." to discard changes in working directory)

    modified:   CONTRIBUTING.md
Now you can use git diff to see what is still unstaged:

$ git diff
diff --git a/CONTRIBUTING.md b/CONTRIBUTING.md
index 643e24f..87f08c8 100644
--- a/CONTRIBUTING.md
+++ b/CONTRIBUTING.md
@@ -119,3 +119,4 @@ at the
 ## Starter Projects

 See our [projects list](https://github.com/libgit2/libgit2/blob/development/PROJECTS.md).
+# test line
and git diff --cached to see what you’ve staged so far (--staged and --cached are synonyms):

$ git diff --cached
diff --git a/CONTRIBUTING.md b/CONTRIBUTING.md
index 8ebb991..643e24f 100644
--- a/CONTRIBUTING.md
+++ b/CONTRIBUTING.md
@@ -65,7 +65,8 @@ branch directly, things can get messy.
 Please include a nice description of your changes when you submit your PR;
 if we have to read the whole diff to figure out why you're contributing
 in the first place, you're less likely to get feedback and have your change
-merged in.
+merged in. Also, split your changes into comprehensive chunks if your patch is
+longer than a dozen lines.

 If you are starting to work on a particular area, feel free to submit a PR
 that highlights your work in progress (and note in the PR title that it's
Note

Git Diff in an External Tool

We will continue to use the git diff command in various ways throughout the rest of the book. There is another way to look at these diffs if you prefer a graphical or external diff viewing program instead. If you run git difftool instead of git diff, you can view any of these diffs in software like emerge, vimdiff and many more (including commercial products). Run git difftool --tool-help to see what is available on your system.

Committing Your Changes
Now that your staging area is set up the way you want it, you can commit your changes. Remember that anything that is still unstaged – any files you have created or modified that you haven’t run git add on since you edited them – won’t go into this commit. They will stay as modified files on your disk. In this case, let’s say that the last time you ran git status, you saw that everything was staged, so you’re ready to commit your changes. The simplest way to commit is to type git commit:

$ git commit
Doing so launches your editor of choice. (This is set by your shell’s $EDITOR environment variable – usually vim or emacs, although you can configure it with whatever you want using the git config --global core.editor command as you saw in Getting Started).

The editor displays the following text (this example is a Vim screen):

# Please enter the commit message for your changes. Lines starting
# with '#' will be ignored, and an empty message aborts the commit.
# On branch master
# Your branch is up-to-date with 'origin/master'.
#
# Changes to be committed:
#   new file:   README
#   modified:   CONTRIBUTING.md
#
~
~
~
".git/COMMIT_EDITMSG" 9L, 283C
You can see that the default commit message contains the latest output of the git status command commented out and one empty line on top. You can remove these comments and type your commit message, or you can leave them there to help you remember what you’re committing. (For an even more explicit reminder of what you’ve modified, you can pass the -v option to git commit. Doing so also puts the diff of your change in the editor so you can see exactly what changes you’re committing.) When you exit the editor, Git creates your commit with that commit message (with the comments and diff stripped out).

Alternatively, you can type your commit message inline with the commit command by specifying it after a -m flag, like this:

$ git commit -m "Story 182: Fix benchmarks for speed"
[master 463dc4f] Story 182: Fix benchmarks for speed
 2 files changed, 2 insertions(+)
 create mode 100644 README
Now you’ve created your first commit! You can see that the commit has given you some output about itself: which branch you committed to (master), what SHA-1 checksum the commit has (463dc4f), how many files were changed, and statistics about lines added and removed in the commit.

Remember that the commit records the snapshot you set up in your staging area. Anything you didn’t stage is still sitting there modified; you can do another commit to add it to your history. Every time you perform a commit, you’re recording a snapshot of your project that you can revert to or compare to later.

Skipping the Staging Area
Although it can be amazingly useful for crafting commits exactly how you want them, the staging area is sometimes a bit more complex than you need in your workflow. If you want to skip the staging area, Git provides a simple shortcut. Adding the -a option to the git commit command makes Git automatically stage every file that is already tracked before doing the commit, letting you skip the git add part:

$ git status
On branch master
Your branch is up-to-date with 'origin/master'.
Changes not staged for commit:
  (use "git add <file>..." to update what will be committed)
  (use "git checkout -- <file>..." to discard changes in working directory)

    modified:   CONTRIBUTING.md

no changes added to commit (use "git add" and/or "git commit -a")
$ git commit -a -m 'added new benchmarks'
[master 83e38c7] added new benchmarks
 1 file changed, 5 insertions(+), 0 deletions(-)
Notice how you don’t have to run git add on the CONTRIBUTING.md file in this case before you commit. That’s because the -a flag includes all changed files. This is convenient, but be careful; sometimes this flag will cause you to include unwanted changes.

Removing Files
To remove a file from Git, you have to remove it from your tracked files (more accurately, remove it from your staging area) and then commit. The git rm command does that, and also removes the file from your working directory so you don’t see it as an untracked file the next time around.

If you simply remove the file from your working directory, it shows up under the “Changed but not updated” (that is, unstaged) area of your git status output:

$ rm PROJECTS.md
$ git status
On branch master
Your branch is up-to-date with 'origin/master'.
Changes not staged for commit:
  (use "git add/rm <file>..." to update what will be committed)
  (use "git checkout -- <file>..." to discard changes in working directory)

        deleted:    PROJECTS.md

no changes added to commit (use "git add" and/or "git commit -a")
Then, if you run git rm, it stages the file’s removal:

$ git rm PROJECTS.md
rm 'PROJECTS.md'
$ git status
On branch master
Your branch is up-to-date with 'origin/master'.
Changes to be committed:
  (use "git reset HEAD <file>..." to unstage)

    deleted:    PROJECTS.md
The next time you commit, the file will be gone and no longer tracked. If you modified the file and added it to the staging area already, you must force the removal with the -f option. This is a safety feature to prevent accidental removal of data that hasn’t yet been recorded in a snapshot and that can’t be recovered from Git.

Another useful thing you may want to do is to keep the file in your working tree but remove it from your staging area. In other words, you may want to keep the file on your hard drive but not have Git track it anymore. This is particularly useful if you forgot to add something to your .gitignore file and accidentally staged it, like a large log file or a bunch of .a compiled files. To do this, use the --cached option:

$ git rm --cached README
You can pass files, directories, and file-glob patterns to the git rm command. That means you can do things such as:

$ git rm log/\*.log
Note the backslash () in front of the *. This is necessary because Git does its own filename expansion in addition to your shell’s filename expansion. This command removes all files that have the .log extension in the log/ directory. Or, you can do something like this:

$ git rm \*~
This command removes all files whose names end with a ~.

Moving Files
Unlike many other VCS systems, Git doesn’t explicitly track file movement. If you rename a file in Git, no metadata is stored in Git that tells it you renamed the file. However, Git is pretty smart about figuring that out after the fact – we’ll deal with detecting file movement a bit later.

Thus it’s a bit confusing that Git has a mv command. If you want to rename a file in Git, you can run something like:

$ git mv file_from file_to
and it works fine. In fact, if you run something like this and look at the status, you’ll see that Git considers it a renamed file:

$ git mv README.md README
$ git status
On branch master
Your branch is up-to-date with 'origin/master'.
Changes to be committed:
  (use "git reset HEAD <file>..." to unstage)

    renamed:    README.md -> README
However, this is equivalent to running something like this:

$ mv README.md README
$ git rm README.md
$ git add README
Git figures out that it’s a rename implicitly, so it doesn’t matter if you rename a file that way or with the mv command. The only real difference is that git mv is one command instead of three – it’s a convenience function. More importantly, you can use any tool you like to rename a file, and address the add/rm later, before you commit.

Cheatsheet


# Add file contents to the index
git add [filename]
# Record changes to the repository
git commit -m [message]
# Reset current HEAD to the specified state
git reset HEAD~1
# Switch branches or restore working tree files
git checkout HEAD~2
View Status

git status
git log
git log --since=2015-12-01
git log --until=2015-12-01
git log --author="magizbox"
git log --grep="Init"
git log --oneline origin/master
Suggested Readings
http://git-scm.com/book/ca/v1/Git-Basics-Recording-Changes-to-the-Repository

\section{Tagging}

Like most VCSs, Git has the ability to tag specific points in history as being important. Typically people use this functionality to mark release points (v1.0, and so on). In this section, you’ll learn how to list the available tags, how to create new tags, and what the different types of tags are.

Listing Your Tags
Listing the available tags in Git is straightforward. Just type git tag:

$ git tag
v0.1
v1.3
This command lists the tags in alphabetical order; the order in which they appear has no real importance.

You can also search for tags with a particular pattern. The Git source repo, for instance, contains more than 500 tags. If you’re only interested in looking at the 1.8.5 series, you can run this:

$ git tag -l "v1.8.5*"
v1.8.5
v1.8.5-rc0
v1.8.5-rc1
v1.8.5-rc2
v1.8.5-rc3
v1.8.5.1
v1.8.5.2
v1.8.5.3
v1.8.5.4
v1.8.5.5
Creating Tags
Git uses two main types of tags: lightweight and annotated.

A lightweight tag is very much like a branch that doesn’t change – it’s just a pointer to a specific commit.

Annotated tags, however, are stored as full objects in the Git database. They’re checksummed; contain the tagger name, email, and date; have a tagging message; and can be signed and verified with GNU Privacy Guard (GPG). It’s generally recommended that you create annotated tags so you can have all this information; but if you want a temporary tag or for some reason don’t want to keep the other information, lightweight tags are available too.

Annotated Tags
Creating an annotated tag in Git is simple. The easiest way is to specify -a when you run the tag command:

$ git tag -a v1.4 -m "my version 1.4"
$ git tag
v0.1
v1.3
v1.4
The -m specifies a tagging message, which is stored with the tag. If you don’t specify a message for an annotated tag, Git launches your editor so you can type it in.

You can see the tag data along with the commit that was tagged by using the git show command:

$ git show v1.4
tag v1.4
Tagger: Ben Straub <ben@straub.cc>
Date:   Sat May 3 20:19:12 2014 -0700

my version 1.4

commit ca82a6dff817ec66f44342007202690a93763949
Author: Scott Chacon <schacon@gee-mail.com>
Date:   Mon Mar 17 21:52:11 2008 -0700

    changed the version number
That shows the tagger information, the date the commit was tagged, and the annotation message before showing the commit information.

Lightweight Tags
Another way to tag commits is with a lightweight tag. This is basically the commit checksum stored in a file – no other information is kept. To create a lightweight tag, don’t supply the -a, -s, or -m option:

$ git tag v1.4-lw
$ git tag
v0.1
v1.3
v1.4
v1.4-lw
v1.5
This time, if you run git show on the tag, you don’t see the extra tag information. The command just shows the commit:

$ git show v1.4-lw
commit ca82a6dff817ec66f44342007202690a93763949
Author: Scott Chacon <schacon@gee-mail.com>
Date:   Mon Mar 17 21:52:11 2008 -0700

    changed the version number
Tagging Later
You can also tag commits after you’ve moved past them. Suppose your commit history looks like this:

$ git log --pretty=oneline
15027957951b64cf874c3557a0f3547bd83b3ff6 Merge branch 'experiment'
a6b4c97498bd301d84096da251c98a07c7723e65 beginning write support
0d52aaab4479697da7686c15f77a3d64d9165190 one more thing
6d52a271eda8725415634dd79daabbc4d9b6008e Merge branch 'experiment'
0b7434d86859cc7b8c3d5e1dddfed66ff742fcbc added a commit function
4682c3261057305bdd616e23b64b0857d832627b added a todo file
166ae0c4d3f420721acbb115cc33848dfcc2121a started write support
9fceb02d0ae598e95dc970b74767f19372d61af8 updated rakefile
964f16d36dfccde844893cac5b347e7b3d44abbc commit the todo
8a5cbc430f1a9c3d00faaeffd07798508422908a updated readme
Now, suppose you forgot to tag the project at v1.2, which was at the “updated rakefile” commit. You can add it after the fact. To tag that commit, you specify the commit checksum (or part of it) at the end of the command:

$ git tag -a v1.2 9fceb02
You can see that you’ve tagged the commit:

$ git tag
v0.1
v1.2
v1.3
v1.4
v1.4-lw
v1.5

$ git show v1.2
tag v1.2
Tagger: Scott Chacon <schacon@gee-mail.com>
Date:   Mon Feb 9 15:32:16 2009 -0800

version 1.2
commit 9fceb02d0ae598e95dc970b74767f19372d61af8
Author: Magnus Chacon <mchacon@gee-mail.com>
Date:   Sun Apr 27 20:43:35 2008 -0700

    updated rakefile
...
Sharing Tags
By default, the git push command doesn’t transfer tags to remote servers. You will have to explicitly push tags to a shared server after you have created them. This process is just like sharing remote branches – you can run git push origin [tagname].

$ git push origin v1.5
Counting objects: 14, done.
Delta compression using up to 8 threads.
Compressing objects: 100% (12/12), done.
Writing objects: 100% (14/14), 2.05 KiB | 0 bytes/s, done.
Total 14 (delta 3), reused 0 (delta 0)
To git@github.com:schacon/simplegit.git
 * [new tag]         v1.5 -> v1.5
If you have a lot of tags that you want to push up at once, you can also use the --tags option to the git push command. This will transfer all of your tags to the remote server that are not already there.

$ git push origin --tags
Counting objects: 1, done.
Writing objects: 100% (1/1), 160 bytes | 0 bytes/s, done.
Total 1 (delta 0), reused 0 (delta 0)
To git@github.com:schacon/simplegit.git
 * [new tag]         v1.4 -> v1.4
 * [new tag]         v1.4-lw -> v1.4-lw
Now, when someone else clones or pulls from your repository, they will get all your tags as well.

Checking out Tags
You can’t really check out a tag in Git, since they can’t be moved around. If you want to put a version of your repository in your working directory that looks like a specific tag, you can create a new branch at a specific tag with git checkout -b [branchname] [tagname]:

$ git checkout -b version2 v2.0.0
Switched to a new branch 'version2'
Of course if you do this and do a commit, your version2 branch will be slightly different than your v2.0.0 tag since it will move forward with your new changes, so do be careful.

Suggested Readings
Git-Basics-Tagging

\section{Sub Module}

Add new submodule
git submodule add https://github.com/chaconinc/DbConnector
Update submodule
git submodule init
git submodule update

\section{Git Configurations}

How do I force git to use LF instead of CR+LF under windows?
The proper way to get LF endings in Windows is to first set core.autocrlf to false:

git config --global core.autocrlf false
Suggested Readings
How do I force git to use LF instead of CR+LF under windows?

\section{Team Workflow}

In this article, I will represent our workflow with git to collaboration. As you can see, there are upstream, A, B repositories. Upstream repostiroy is main repository of project, owner by team leader. A and B repositories belong to developers. In upstream remote, there are master and develop branches. In developer's repositories, there are develop and feature-something branches.

resources/workflow.png

Step 1: Create new project
In step 1, leader create a repository.

resources/workflow-step-1.png

Step 2: Forking
In step 2, each developers create their own repository by forking main repository

resources/workflow-step-2.png

Step 3: Commits
In step 3, developers work on their branches, each peace of their works should be end by a commit

resources/workflow-step-3.png

Step 4: Merge Requests
After finish a feature, each developer will create a merge requests to main repository. Leader take responsibility for merging their requests

resources/workflow-step-4.png

Step 5: Fetch and Rebase
Developer will checkout to develop branch, fetch from upstream remote and rebase

resources/workflow-step-5.png

Step 6: Develop new features
Team sync. At this moment, developer can checkout from develop branch to create new feature.

resources/workflow-step-6.png

Step 7: New version
Leader take responsibility to merge from dev branch to master branch and create tag to release new version.

resources/workflow-step-7.png

Related Readings
"Git With Development, Staging And Production Branches". stackoverflow.com. N.p., 2016. Web. 28 Oct. 2016.
"A Successful Git Branching Model". nvie.com. N.p., 2016. Web. 28 Oct. 2016.

\section{Braching}

Branches
A branch represents an independent line of development. Branches serve as an abstraction for the edit > stage > commit process discussed in Git Basics, the first module of this series. You can think of them as a way to request a brand new working directory, staging area, and project history. New commits are recorded in the history for the current branch, which results in a fork in the history of the project.

git branches diagram


Local Branches

# view branches
git branch
# create new branch
git checkout -b [branchname]
# switch to a branch
git checkout [branchname]
# delete branch
git branch -D [branchname]
Merge Branches

git merge [branchname]
Remote Branches2

# delete remote branches
git push origin --delete [branchname]
Repositories
what is origin?

# view remote branches
git remote -v
# add remote branches
git remote add [repositoryname] [branch_url]
# clone a repository into a new directory
git clone
# push to repository
git push [repositoryname] [branchname]
git pull
Branch Naming Convention
Choose short and descriptive names:
# good
$ git checkout -b oauth-migration

# bad - too vague
$ git checkout -b login_fix
Identifiers from corresponding tickets in an external service (eg. a GitHub issue) are also good candidates for use in branch names. For example:
# GitHub issue #15
$ git checkout -b issue-15
Use dashes to separate words.

When several people are working on the same feature, it might be convenient to have personal feature branches and a team-wide feature branch. Use the following naming convention:

$ git checkout -b feature-a/master # team-wide branch
$ git checkout -b feature-a/maria  # Maria's personal branch
$ git checkout -b feature-a/nick   # Nick's personal branch
Merge at will the personal branches to the team-wide branch (see "Merging"). Eventually, the team-wide branch will be merged to "master".
Delete your branch from the upstream repository after it's merged, unless there is a specific reason not to.

Tip: Use the following command while being on "master", to list merged branches:

$ git branch --merged | grep -v "\*"
Create an archive
9 10

# Create an archive of files from a named tree
git archive --format=zip HEAD > app.zip
Fork & Request
11 12

In a repository you want to fork, click Fork button. It will create your own repository
Run this:
git remote add origin [your_repository] # that have created when you fork
git remote add upstream [original_repository] # the repository that you have fork
Code in your own repository & commit
Go to your Fork repository
Switch to your branch
Create Pull Request to send a merge request to the owner of the original repository
https://www.atlassian.com/git/tutorials/using-branches/ ↩

http://stackoverflow.com/questions/2003505/delete-a-git-branch-both-locally-and-remotely/2003515#2003515 ↩

https://guides.github.com/features/issues/ ↩

https://help.github.com/articles/using-pull-requests/ ↩

https://help.github.com/articles/about-github-wikis/ ↩

https://help.github.com/articles/about-repository-graphs/ ↩

Closing an issue in the same repository ↩

Git Tools - Submodules ↩

Git: how to get all the files changed and new files in a folder or zip? ↩

git-scm.com: git-archive ↩

Forking a Repo ↩

Using pull request ↩

\section{Github}

Power Tools with Github
 Issues
Issues are a great way to keep track of tasks, enhancements, and bugs for your projects. They re kind of like email except they can be shared and discussed with the rest of your team. Most software projects have a bug tracker of some kind. GitHub's tracker is called Issues, and has its own section in every repository.



Commit to fix issue

git commit -m "fix #34"
Milestones
Once you've collected a lot of issues, you may find it hard to find the ones you care about. Milestones, labels, and assignees are great features to filter and categorize issues. A milestone acts like a container for issues. This is useful for associating issues with specific features or project phases

 �[^5]

Pull Requests
Pull requests let you tell others about changes you've pushed to a repository on GitHub. Once a pull request is sent, interested parties can review the set of changes, discuss potential modifications, and even push follow-up commits if necessary.



Wiki
Just as writing good code and great tests are important, excellent documentation helps others use and extend your project.
Every GitHub repository comes equipped with a section for hosting documentation, called a wiki.



Repository Graphs
Every repository has graphs that display data about traffic, contributors, and commits.



