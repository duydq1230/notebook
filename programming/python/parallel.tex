\section{Lập trình song song}

Running several threads is similar to running several different programs concurrently, but with the following benefits

Multiple threads within a process share the same data space with the main thread and can therefore share information or communicate with each other more easily than if they were separate processes.
Threads sometimes called light-weight processes and they do not require much memory overhead; they are cheaper than processes.
A thread has a beginning, an execution sequence, and a conclusion. It has an instruction pointer that keeps track of where within its context it is currently running.

It can be pre-empted (interrupted)

It can temporarily be put on hold (also known as sleeping) while other threads are running - this is called yielding.
Starting a New Thread
To spawn another thread, you need to call following method available in thread module:

thread.start_new_thread ( function, args[, kwargs] )
This method call enables a fast and efficient way to create new threads in both Linux and Windows.

The method call returns immediately and the child thread starts and calls function with the passed list of args. When function returns, the thread terminates.

Here, args is a tuple of arguments; use an empty tuple to call function without passing any arguments. kwargs is an optional dictionary of keyword arguments.

Example

\begin{lstlisting}[language=Python]
#!/usr/bin/python

import thread
import time

# Define a function for the thread
def print_time( threadName, delay):
   count = 0
   while count < 5:
      time.sleep(delay)
      count += 1
      print "%s: %s" % ( threadName, time.ctime(time.time()) )

# Create two threads as follows
try:
   thread.start_new_thread( print_time, ("Thread-1", 2, ) )
   thread.start_new_thread( print_time, ("Thread-2", 4, ) )
except:
   print "Error: unable to start thread"

while 1:
   pass
\end{lstlisting}

When the above code is executed, it produces the following result

\begin{lstlisting}
Thread-1: Thu Jan 22 15:42:17 2009
Thread-1: Thu Jan 22 15:42:19 2009
Thread-2: Thu Jan 22 15:42:19 2009
Thread-1: Thu Jan 22 15:42:21 2009
Thread-2: Thu Jan 22 15:42:23 2009
Thread-1: Thu Jan 22 15:42:23 2009
Thread-1: Thu Jan 22 15:42:25 2009
Thread-2: Thu Jan 22 15:42:27 2009
Thread-2: Thu Jan 22 15:42:31 2009
Thread-2: Thu Jan 22 15:42:35 2009
\end{lstlisting}

Although it is very effective for low-level threading, but the thread module is very limited compared to the newer threading module.

\textbf{The Threading Module}

The newer threading module included with Python 2.4 provides much more powerful, high-level support for threads than the thread module discussed in the previous section.

The threading module exposes all the methods of the thread module and provides some additional methods:

threading.activeCount(): Returns the number of thread objects that are active.
threading.currentThread(): Returns the number of thread objects in the caller's thread control.
threading.enumerate(): Returns a list of all thread objects that are currently active.
In addition to the methods, the threading module has the Thread class that implements threading. The methods provided by the Thread class are as follows:

run(): The run() method is the entry point for a thread.
start(): The start() method starts a thread by calling the run method.
join([time]): The join() waits for threads to terminate.
isAlive(): The isAlive() method checks whether a thread is still executing.
getName(): The getName() method returns the name of a thread.
setName(): The setName() method sets the name of a thread.

\textbf{Creating Thread Using Threading Module}

To implement a new thread using the threading module, you have to do the following −

Define a new subclass of the Thread class.
Override the init(self [,args]) method to add additional arguments.
Then, override the run(self [,args]) method to implement what the thread should do when started.
Once you have created the new Thread subclass, you can create an instance of it and then start a new thread by invoking the start(), which in turn calls run() method.

Example

\begin{lstlisting}[language=Python]
#!/usr/bin/python

import threading
import time

exitFlag = 0

class myThread (threading.Thread):
    def __init__(self, threadID, name, counter):
        threading.Thread.__init__(self)
        self.threadID = threadID
        self.name = name
        self.counter = counter
    def run(self):
        print "Starting " + self.name
        print_time(self.name, self.counter, 5)
        print "Exiting " + self.name

def print_time(threadName, delay, counter):
    while counter:
        if exitFlag:
            threadName.exit()
        time.sleep(delay)
        print "%s: %s" % (threadName, time.ctime(time.time()))
        counter -= 1

# Create new threads
thread1 = myThread(1, "Thread-1", 1)
thread2 = myThread(2, "Thread-2", 2)

# Start new Threads
thread1.start()
thread2.start()

print "Exiting Main Thread"
\end{lstlisting}

When the above code is executed, it produces the following result

\begin{lstlisting}[language=Python]
Starting Thread-1
Starting Thread-2
Exiting Main Thread
Thread-1: Thu Mar 21 09:10:03 2013
Thread-1: Thu Mar 21 09:10:04 2013
Thread-2: Thu Mar 21 09:10:04 2013
Thread-1: Thu Mar 21 09:10:05 2013
Thread-1: Thu Mar 21 09:10:06 2013
Thread-2: Thu Mar 21 09:10:06 2013
Thread-1: Thu Mar 21 09:10:07 2013
Exiting Thread-1
Thread-2: Thu Mar 21 09:10:08 2013
Thread-2: Thu Mar 21 09:10:10 2013
Thread-2: Thu Mar 21 09:10:12 2013
Exiting Thread-2
\end{lstlisting}

Synchronizing Threads

The threading module provided with Python includes a simple-to-implement locking mechanism that allows you to synchronize threads. A new lock is created by calling the Lock() method, which returns the new lock.

The acquire(blocking) method of the new lock object is used to force threads to run synchronously. The optional blocking parameter enables you to control whether the thread waits to acquire the lock.

If blocking is set to 0, the thread returns immediately with a 0 value if the lock cannot be acquired and with a 1 if the lock was acquired. If blocking is set to 1, the thread blocks and wait for the lock to be released.

The release() method of the new lock object is used to release the lock when it is no longer required.

Example

\begin{lstlisting}[language=Python]
#!/usr/bin/python

import threading
import time

class myThread (threading.Thread):
    def __init__(self, threadID, name, counter):
        threading.Thread.__init__(self)
        self.threadID = threadID
        self.name = name
        self.counter = counter
    def run(self):
        print "Starting " + self.name
        # Get lock to synchronize threads
        threadLock.acquire()
        print_time(self.name, self.counter, 3)
        # Free lock to release next thread
        threadLock.release()

def print_time(threadName, delay, counter):
    while counter:
        time.sleep(delay)
        print "%s: %s" % (threadName, time.ctime(time.time()))
        counter -= 1

threadLock = threading.Lock()
threads = []

# Create new threads
thread1 = myThread(1, "Thread-1", 1)
thread2 = myThread(2, "Thread-2", 2)

# Start new Threads
thread1.start()
thread2.start()

# Add threads to thread list
threads.append(thread1)
threads.append(thread2)

# Wait for all threads to complete
for t in threads:
    t.join()
print "Exiting Main Thread"
\end{lstlisting}

When the above code is executed, it produces the following result

\begin{lstlisting}[language=Python]
Starting Thread-1
Starting Thread-2
Starting Thread-3
Thread-1 processing One
Thread-2 processing Two
Thread-3 processing Three
Thread-1 processing Four
Thread-2 processing Five
Exiting Thread-3
Exiting Thread-1
Exiting Thread-2
Exiting Main Thread
\end{lstlisting}

Related Readings
"Python Multithreaded Programming". www.tutorialspoint.com. N.p., 2016. Web. 13 Dec. 2016.
"An Introduction To Python Concurrency". dabeaz.com. N.p., 2016. Web. 14 Dec. 2016.

\section{Event Based Programming}

Introduction: pydispatcher 1 2

PyDispatcher provides the Python programmer with a multiple-producer-multiple-consumer signal-registration and routing infrastructure for use in multiple contexts. The mechanism of PyDispatcher started life as a highly rated recipe in the Python Cookbook. The project aims to include various enhancements to the recipe developed during use in various applications. It is primarily maintained by Mike Fletcher. A derivative of the project provides the Django web framework's "signal" system.

Used by Django community

Usage 1
# To set up a function to receive signals:
from pydispatch import dispatcher

SIGNAL = 'my-first-signal'


def handle_event(sender):
    """Simple event handler"""
    print 'Signal was sent by', sender


dispatcher.connect(handle_event, signal=SIGNAL, sender=dispatcher.Any)

# The use of the Any object allows the handler to listen for messages
# from any Sender or to listen to Any message being sent.
# To send messages:
first_sender = object()
second_sender = {}


def main():
    dispatcher.send(signal=SIGNAL, sender=first_sender)
    dispatcher.send(signal=SIGNAL, sender=second_sender)

    # Which causes the following to be printed:

    # Signal was sent by <object object at 0x196a090>
    # Signal was sent by {}
Messaging
Conda link Docker link Github - pubSubService Github - pubSubClient Pypi link

Python Publish - Subscribe Pattern Implementation:

Step by Step to run PubSub:
Step 1: Pull pubsub image from docker hub & run it:
docker pull hunguyen/pubsub:latest
docker run -d -p 8000:8000 hunguyen/pubsub
Step 2: To run client first install pyconfiguration from conda
conda install -c rain1024 pyconfiguration
Step 3: Install pubSubClient package from conda
conda install -c hunguyen pubsubclient
Step 4: Create config.json file
{
  "PUBLISH_SUBSCRIBE_SERVICE": "http://api.service.com"
}
Step 5: Run pubsubclient
# create and register or sync a publisher
publisher = Publisher('P1')
# create a new topic
topic = Topic('A')
# create an event of a topic
event = Event(topic)
# publisher publishes an event
publisher.publish(event)
# create and register or sync a subscriber
subscriber = Subscriber('S1')
# subscriber subscribes to a topic
subscriber.subscribe(topic)
# subscriber get all new events by time stamp of topics which it has subscribed
events = subscriber.get_events()
pydispatcher ↩

stackoverflow, Recommended Python publish/subscribe/dispatch module? ↩