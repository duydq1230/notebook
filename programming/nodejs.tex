\chapter{NodeJS}

View online \href{http://magizbox.com/training/nodejs/site/}{http://magizbox.com/training/nodejs/site/}

Node.js is an open-source, cross-platform JavaScript runtime environment for developing a diverse variety of tools and applications. Although Node.js is not a JavaScript framework, many of its basic modules are written in JavaScript, and developers can write new modules in JavaScript. The runtime environment interprets JavaScript using Google's V8 JavaScript engine. Node.js has an event-driven architecture capable of asynchronous I/O. These design choices aim to optimize throughput and scalability in Web applications with many input/output operations, as well as for real-time Web applications (e.g., real-time communication programs and browser games). Node.js was originally written in 2009 by Ryan Dahl. The initial release supported only Linux. Its development and maintenance was led by Dahl and later sponsored by Joyent.

\section{Get Started}

Installation
Windows
In this section I will show you how to Install Node.js® and NPM on Windows

Prerequisites
Node isn’t a program that you simply launch like Word or Photoshop: you won’t find it pinned to the taskbar or in your list of Apps. To use Node you must type command-line instructions, so you need to be comfortable with (or at least know how to start) a command-line tool like the Windows Command Prompt, PowerShell, Cygwin, or the Git shell (which is installed along with Github for Windows).

Installation Overview
Installing Node and NPM is pretty straightforward using the installer package available from the Node.js® web site.

Installation Steps
1. Download the Windows installer from the Nodes.js® web site.

2. Run the installer (the .msi file you downloaded in the previous step.)

3. Follow the prompts in the installer (Accept the license agreement, click the NEXT button a bunch of times and accept the default installation settings).



4. Restart your computer. You won’t be able to run Node.js® until you restart your computer.

Ubuntu
In this section I will show you how to Install Node.js® and NPM on Ubuntu

# update os
sudo apt-get update
# install node with apt-get
sudo apt-get install nodejs
# install npm with apt-get
sudo apt-get install npm
Test
Make sure you have Node and NPM installed by running simple commands to see what version of each is installed and to run a simple test program:

> node -v
v6.9.5

> npm -v
3.10.10
Suggested Readings
How To Install Node.js on an Ubuntu 14.04 server
How to Install Node.js® and NPM on Windows

\section{Basic Syntax}

Print
console.log("Hello World");
Conditional
if(you_smart){
    console.log("learn nodejs");
} else {
    console.log("go away");
}
Loop
for(var count = 0; count < 10; count++){
    console.log(count);
}
Function
function print_info(arg1, arg2){
    console.log(arg1);
    console.log(arg2);
}

\section{File System & IO}

File System & IO
Node implements File I/O using simple wrappers around standard POSIX functions. The Node File System (fs) module can be imported using the following syntax −

var fs = require("fs")
Synchronous vs Asynchronous
Every method in the fs module has synchronous as well as asynchronous forms. Asynchronous methods take the last parameter as the completion function callback and the first parameter of the callback function as error. It is better to use an asynchronous method instead of a synchronous method, as the former never blocks a program during its execution, whereas the second one does.

Example

Create a text file named input.txt with the following content −

Tutorials Point is giving self learning content
to teach the world in simple and easy way!!!!!
Let us create a js file named main.js with the following code −

var fs = require("fs");

// Asynchronous read
fs.readFile('input.txt', function (err, data) {
   if (err) {
      return console.error(err);
   }
   console.log("Asynchronous read: " + data.toString());
});

// Synchronous read
var data = fs.readFileSync('input.txt');
console.log("Synchronous read: " + data.toString());

console.log("Program Ended");
Now run the main.js to see the result −

$ node main.js
Verify the Output.

Synchronous read: Tutorials Point is giving self learning content
to teach the world in simple and easy way!!!!!

Program Ended
Asynchronous read: Tutorials Point is giving self learning content
to teach the world in simple and easy way!!!!!
The following sections in this chapter provide a set of good examples on major File I/O methods.
Open a File
Syntax

Following is the syntax of the method to open a file in asynchronous mode −

fs.open(path, flags[, mode], callback)
Parameters

Here is the description of the parameters used −

path − This is the string having file name including path.
flags − Flags indicate the behavior of the file to be opened. All possible values have been mentioned below.
mode − It sets the file mode (permission and sticky bits), but only if the file was created. It defaults to 0666, readable and writeable.
callback − This is the callback function which gets two arguments (err, fd).
Flags

Flags for read/write operations are −

r - Open file for reading. An exception occurs if the file does not exist.
r+ - Open file for reading and writing. An exception occurs if the file does not exist.
rs - Open file for reading in synchronous mode.
rs+ - Open file for reading and writing, asking the OS to open it synchronously. See notes for 'rs' about using this with caution.
w - Open file for writing. The file is created (if it does not exist) or truncated (if it exists).
wx - Like 'w' but fails if the path exists.
w+ - Open file for reading and writing. The file is created (if it does not exist) or truncated (if it exists).
wx+ - Like 'w+' but fails if path exists.
a - Open file for appending. The file is created if it does not exist.
ax - Like 'a' but fails if the path exists.
a+ - Open file for reading and appending. The file is created if it does not exist.
ax+ - Like 'a+' but fails if the the path exists.
Example

Let us create a js file named main.js having the following code to open a file input.txt for reading and writing.

var fs = require("fs");

// Asynchronous - Opening File
console.log("Going to open file!");
fs.open('input.txt', 'r+', function(err, fd) {
   if (err) {
      return console.error(err);
   }
  console.log("File opened successfully!");
});
Now run the main.js to see the result −

$ node main.js
Verify the Output.

Going to open file!
File opened successfully!
Get File Information
Syntax

Following is the syntax of the method to get the information about a file −

fs.stat(path, callback)
Parameters

Here is the description of the parameters used −

path − This is the string having file name including path.
callback − This is the callback function which gets two arguments (err, stats) where stats is an object of fs.Stats type which is printed below in the example.
Apart from the important attributes which are printed below in the example, there are several useful methods available in fs.Stats class which can be used to check file type. These methods are given in the following table.

Method Description

stats.isFile() - Returns true if file type of a simple file.
stats.isDirectory() - Returns true if file type of a directory.
stats.isBlockDevice() - Returns true if file type of a block device.
stats.isCharacterDevice() - Returns true if file type of a character device.
stats.isSymbolicLink() - Returns true if file type of a symbolic link.
stats.isFIFO() - Returns true if file type of a FIFO.
stats.isSocket() - Returns true if file type of asocket.
Example

Let us create a js file named main.js with the following code −

var fs = require("fs");

console.log("Going to get file info!");
fs.stat('input.txt', function (err, stats) {
   if (err) {
       return console.error(err);
   }
   console.log(stats);
   console.log("Got file info successfully!");

   // Check file type
   console.log("isFile ? " + stats.isFile());
   console.log("isDirectory ? " + stats.isDirectory());
});
Now run the main.js to see the result −

$ node main.js
Verify the Output.

Going to get file info!
{
   dev: 1792,
   mode: 33188,
   nlink: 1,
   uid: 48,
   gid: 48,
   rdev: 0,
   blksize: 4096,
   ino: 4318127,
   size: 97,
   blocks: 8,
   atime: Sun Mar 22 2015 13:40:00 GMT-0500 (CDT),
   mtime: Sun Mar 22 2015 13:40:57 GMT-0500 (CDT),
   ctime: Sun Mar 22 2015 13:40:57 GMT-0500 (CDT)
}
Got file info successfully!
isFile ? true
isDirectory ? false
Writing a File
Syntax

Following is the syntax of one of the methods to write into a file −

fs.writeFile(filename, data[, options], callback)
This method will over-write the file if the file already exists. If you want to write into an existing file then you should use another method available.

Parameters

Here is the description of the parameters used −

path − This is the string having the file name including path.
data − This is the String or Buffer to be written into the file.
options − The third parameter is an object which will hold {encoding, mode, flag}. By default. encoding is utf8, mode is octal value 0666. and flag is 'w'
callback − This is the callback function which gets a single parameter err that returns an error in case of any writing error.
Example

Let us create a js file named main.js having the following code −

var fs = require("fs");

console.log("Going to write into existing file");
fs.writeFile('input.txt', 'Simply Easy Learning!',  function(err) {
   if (err) {
      return console.error(err);
   }

   console.log("Data written successfully!");
   console.log("Let's read newly written data");
   fs.readFile('input.txt', function (err, data) {
      if (err) {
         return console.error(err);
      }
      console.log("Asynchronous read: " + data.toString());
   });
});
Now run the main.js to see the result −

$ node main.js
Verify the Output.

Going to write into existing file
Data written successfully!
Let's read newly written data
Asynchronous read: Simply Easy Learning!
Reading a File
Syntax

Following is the syntax of one of the methods to read from a file −

fs.read(fd, buffer, offset, length, position, callback)
This method will use file descriptor to read the file. If you want to read the file directly using the file name, then you should use another method available.

Parameters

Here is the description of the parameters used −

fd − This is the file descriptor returned by fs.open().
buffer − This is the buffer that the data will be written to.
offset − This is the offset in the buffer to start writing at.
length − This is an integer specifying the number of bytes to read.
position − This is an integer specifying where to begin reading from in the file. * If position is null, data will be read from the current file position. callback − This is the callback function which gets the three arguments, (err, bytesRead, buffer).
Example

Let us create a js file named main.js with the following code −

var fs = require("fs");
var buf = new Buffer(1024);

console.log("Going to open an existing file");
fs.open('input.txt', 'r+', function(err, fd) {
   if (err) {
      return console.error(err);
   }
   console.log("File opened successfully!");
   console.log("Going to read the file");
   fs.read(fd, buf, 0, buf.length, 0, function(err, bytes){
      if (err){
         console.log(err);
      }
      console.log(bytes + " bytes read");

      // Print only read bytes to avoid junk.
      if(bytes > 0){
         console.log(buf.slice(0, bytes).toString());
      }
   });
});
Now run the main.js to see the result −

$ node main.js
Verify the Output.

Going to open an existing file
File opened successfully!
Going to read the file
97 bytes read
Tutorials Point is giving self learning content
to teach the world in simple and easy way!!!!!
Closing a File
Syntax

Following is the syntax to close an opened file −

fs.close(fd, callback)
Parameters

Here is the description of the parameters used −

fd − This is the file descriptor returned by file fs.open() method.
callback − This is the callback function No arguments other than a possible exception are given to the completion callback.
Example Let us create a js file named main.js having the following code −

var fs = require("fs");
var buf = new Buffer(1024);

console.log("Going to open an existing file");
fs.open('input.txt', 'r+', function(err, fd) {
   if (err) {
      return console.error(err);
   }
   console.log("File opened successfully!");
   console.log("Going to read the file");

   fs.read(fd, buf, 0, buf.length, 0, function(err, bytes){
      if (err){
         console.log(err);
      }

      // Print only read bytes to avoid junk.
      if(bytes > 0){
         console.log(buf.slice(0, bytes).toString());
      }

      // Close the opened file.
      fs.close(fd, function(err){
         if (err){
            console.log(err);
         }
         console.log("File closed successfully.");
      });
   });
});
Now run the main.js to see the result −

$ node main.js
Verify the Output.

Going to open an existing file
File opened successfully!
Going to read the file
Tutorials Point is giving self learning content
to teach the world in simple and easy way!!!!!

File closed successfully.
Truncate a File
Syntax

Following is the syntax of the method to truncate an opened file −

fs.ftruncate(fd, len, callback)
Parameters

Here is the description of the parameters used −

fd − This is the file descriptor returned by fs.open().
len − This is the length of the file after which the file will be truncated.
callback − This is the callback function No arguments other than a possible ekxception are given to the completion callback.
Example

Let us create a js file named main.js having the following code −

var fs = require("fs");
var buf = new Buffer(1024);

console.log("Going to open an existing file");
fs.open('input.txt', 'r+', function(err, fd) {
   if (err) {
      return console.error(err);
   }
   console.log("File opened successfully!");
   console.log("Going to truncate the file after 10 bytes");

   // Truncate the opened file.
   fs.ftruncate(fd, 10, function(err){
      if (err){
         console.log(err);
      }
      console.log("File truncated successfully.");
      console.log("Going to read the same file");

      fs.read(fd, buf, 0, buf.length, 0, function(err, bytes){
         if (err){
            console.log(err);
         }

         // Print only read bytes to avoid junk.
         if(bytes > 0){
            console.log(buf.slice(0, bytes).toString());
         }

         // Close the opened file.
         fs.close(fd, function(err){
            if (err){
               console.log(err);
            }
            console.log("File closed successfully.");
         });
      });
   });
});
Now run the main.js to see the result −

$ node main.js
Verify the Output.

Going to open an existing file
File opened successfully!
Going to truncate the file after 10 bytes
File truncated successfully.
Going to read the same file
Tutorials
File closed successfully.
Delete a File
Syntax Following is the syntax of the method to delete a file −

fs.unlink(path, callback)
Parameters

Here is the description of the parameters used −

path − This is the file name including path.
callback − This is the callback function No arguments other than a possible exception are given to the completion callback.
Example

Let us create a js file named main.js having the following code −

var fs = require("fs");

console.log("Going to delete an existing file");
fs.unlink('input.txt', function(err) {
   if (err) {
      return console.error(err);
   }
   console.log("File deleted successfully!");
});
Now run the main.js to see the result −

$ node main.js
Verify the Output.

Going to delete an existing file
File deleted successfully!
Create a Directory
Syntax

Following is the syntax of the method to create a directory −

fs.mkdir(path[, mode], callback)
Parameters

Here is the description of the parameters used −

path − This is the directory name including path.
mode − This is the directory permission to be set. Defaults to 0777.
callback − This is the callback function No arguments other than a possible exception are given to the completion callback.
Example

Let us create a js file named main.js having the following code −

var fs = require("fs");

console.log("Going to create directory /tmp/test");
fs.mkdir('/tmp/test',function(err){
   if (err) {
      return console.error(err);
   }
   console.log("Directory created successfully!");
});
Now run the main.js to see the result −

$ node main.js
Verify the Output.

Going to create directory /tmp/test
Directory created successfully!
Read a Directory
Syntax

Following is the syntax of the method to read a directory −

fs.readdir(path, callback)
Parameters

Here is the description of the parameters used −

path − This is the directory name including path.
callback − This is the callback function which gets two arguments (err, files) where files is an array of the names of the files in the directory excluding '.' and '..'.
Example

Let us create a js file named main.js having the following code −

var fs = require("fs");

console.log("Going to read directory /tmp");
fs.readdir("/tmp/",function(err, files){
   if (err) {
      return console.error(err);
   }
   files.forEach( function (file){
      console.log( file );
   });
});
Now run the main.js to see the result −

$ node main.js
Verify the Output.

Going to read directory /tmp
ccmzx99o.out
ccyCSbkF.out
employee.ser
hsperfdata_apache
test
test.txt
Remove a Directory
Syntax

Following is the syntax of the method to remove a directory −

fs.rmdir(path, callback)
Parameters

Here is the description of the parameters used −

path − This is the directory name including path.
callback − This is the callback function No argume nts other than a possible exception are given to the completion callback.
Example

Let us create a js file named main.js having the following code −

var fs = require("fs");

console.log("Going to delete directory /tmp/test");
fs.rmdir("/tmp/test",function(err){
   if (err) {
      return console.error(err);
   }
   console.log("Going to read directory /tmp");

   fs.readdir("/tmp/",function(err, files){
      if (err) {
         return console.error(err);
      }
      files.forEach( function (file){
         console.log( file );
      });
   });
});
Now run the main.js to see the result −

$ node main.js
Verify the Output.

Going to read directory /tmp
ccmzx99o.out
ccyCSbkF.out
employee.ser
hsperfdata_apache
test.txt

\section{Package Manager}

Package Manager: NPM
Node Package Manager (NPM) provides two main functionalities −

Online repositories for node.js packages/modules which are searchable on search.nodejs.org
Command line utility to install Node.js packages, do version management and dependency management of Node.js packages.
NPM comes bundled with Node.js installables after v0.6.3 version. To verify the same, open console and type the following command and see the result −

$ npm --version
2.7.1
If you are running an old version of NPM then it is quite easy to update it to the latest version. Just use the following command from root −

$ sudo npm install npm -g
/usr/bin/npm -> /usr/lib/node_modules/npm/bin/npm-cli.js
npm@2.7.1 /usr/lib/node_modules/npm
Installing Modules
There is a simple syntax to install any Node.js module −

$ npm install <Module Name>
For example, following is the command to install a famous Node.js web framework module called express −

$ npm install express
Now you can use this module in your js file as following −

var express = require('express');
Global vs Local Installation
By default, NPM installs any dependency in the local mode. Here local mode refers to the package installation in node_modules directory lying in the folder where Node application is present. Locally deployed packages are accessible via require() method. For example, when we installed express module, it created node_modules directory in the current directory where it installed the express module.

$ ls -l
total 0
drwxr-xr-x 3 root root 20 Mar 17 02:23 node_modules
Alternatively, you can use npm ls command to list down all the locally installed modules.

Globally installed packages/dependencies are stored in system directory. Such dependencies can be used in CLI (Command Line Interface) function of any node.js but cannot be imported using require() in Node application directly. Now let's try installing the express module using global installation.

$ npm install express -g
This will produce a similar result but the module will be installed globally. Here, the first line shows the module version and the location where it is getting installed.

express@4.12.2 /usr/lib/node_modules/express
├── merge-descriptors@1.0.0
├── utils-merge@1.0.0
├── cookie-signature@1.0.6
├── methods@1.1.1
├── fresh@0.2.4
├── cookie@0.1.2
├── escape-html@1.0.1
├── range-parser@1.0.2
├── content-type@1.0.1
├── finalhandler@0.3.3
├── vary@1.0.0
├── parseurl@1.3.0
├── content-disposition@0.5.0
├── path-to-regexp@0.1.3
├── depd@1.0.0
├── qs@2.3.3
├── on-finished@2.2.0 (ee-first@1.1.0)
├── etag@1.5.1 (crc@3.2.1)
├── debug@2.1.3 (ms@0.7.0)
├── proxy-addr@1.0.7 (forwarded@0.1.0, ipaddr.js@0.1.9)
├── send@0.12.1 (destroy@1.0.3, ms@0.7.0, mime@1.3.4)
├── serve-static@1.9.2 (send@0.12.2)
├── accepts@1.2.5 (negotiator@0.5.1, mime-types@2.0.10)
└── type-is@1.6.1 (media-typer@0.3.0, mime-types@2.0.10)
You can use the following command to check all the modules installed globally −

$ npm ls -g
Using package.json
package.json is present in the root directory of any Node application/module and is used to define the properties of a package. Let's open package.json of express package present in node_modules/express/

{
   "name": "express",
      "description": "Fast, unopinionated, minimalist web framework",
      "version": "4.11.2",
      "author": {

         "name": "TJ Holowaychuk",
         "email": "tj@vision-media.ca"
      },

   "contributors": [{
      "name": "Aaron Heckmann",
      "email": "aaron.heckmann+github@gmail.com"
   },

   {
      "name": "Ciaran Jessup",
      "email": "ciaranj@gmail.com"
   },

   {
      "name": "Douglas Christopher Wilson",
      "email": "doug@somethingdoug.com"
   },

   {
      "name": "Guillermo Rauch",
      "email": "rauchg@gmail.com"
   },

   {
      "name": "Jonathan Ong",
      "email": "me@jongleberry.com"
   },

   {
      "name": "Roman Shtylman",
      "email": "shtylman+expressjs@gmail.com"
   },

   {
      "name": "Young Jae Sim",
      "email": "hanul@hanul.me"
   } ],
   "license": "MIT", "repository": {
      "type": "git",
      "url": "https://github.com/strongloop/express"
   },
   "homepage": "https://expressjs.com/", "keywords": [
      "express",
      "framework",
      "sinatra",
      "web",
      "rest",
      "restful",
      "router",
      "app",
      "api"
   ],
   "dependencies": {
      "accepts": "~1.2.3",
      "content-disposition": "0.5.0",
      "cookie-signature": "1.0.5",
      "debug": "~2.1.1",
      "depd": "~1.0.0",
      "escape-html": "1.0.1",
      "etag": "~1.5.1",
      "finalhandler": "0.3.3",
      "fresh": "0.2.4",
      "media-typer": "0.3.0",
      "methods": "~1.1.1",
      "on-finished": "~2.2.0",
      "parseurl": "~1.3.0",
      "path-to-regexp": "0.1.3",
      "proxy-addr": "~1.0.6",
      "qs": "2.3.3",
      "range-parser": "~1.0.2",
      "send": "0.11.1",
      "serve-static": "~1.8.1",
      "type-is": "~1.5.6",
      "vary": "~1.0.0",
      "cookie": "0.1.2",
      "merge-descriptors": "0.0.2",
      "utils-merge": "1.0.0"
   },
   "devDependencies": {
      "after": "0.8.1",
      "ejs": "2.1.4",
      "istanbul": "0.3.5",
      "marked": "0.3.3",
      "mocha": "~2.1.0",
      "should": "~4.6.2",
      "supertest": "~0.15.0",
      "hjs": "~0.0.6",
      "body-parser": "~1.11.0",
      "connect-redis": "~2.2.0",
      "cookie-parser": "~1.3.3",
      "express-session": "~1.10.2",
      "jade": "~1.9.1",
      "method-override": "~2.3.1",
      "morgan": "~1.5.1",
      "multiparty": "~4.1.1",
      "vhost": "~3.0.0"
   },
   "engines": {
      "node": ">= 0.10.0"
   },
   "files": [
      "LICENSE",
      "History.md",
      "Readme.md",
      "index.js",
      "lib/"
   ],
   "scripts": {
      "test": "mocha --require test/support/env
         --reporter spec --bail --check-leaks test/ test/acceptance/",
      "test-cov": "istanbul cover node_modules/mocha/bin/_mocha
         -- --require test/support/env --reporter dot --check-leaks test/ test/acceptance/",
      "test-tap": "mocha --require test/support/env
         --reporter tap --check-leaks test/ test/acceptance/",
      "test-travis": "istanbul cover node_modules/mocha/bin/_mocha
         --report lcovonly -- --require test/support/env
         --reporter spec --check-leaks test/ test/acceptance/"
   },
   "gitHead": "63ab25579bda70b4927a179b580a9c580b6c7ada",
   "bugs": {
      "url": "https://github.com/strongloop/express/issues"
   },
   "_id": "express@4.11.2",
   "_shasum": "8df3d5a9ac848585f00a0777601823faecd3b148",
   "_from": "express@*",
   "_npmVersion": "1.4.28",
   "_npmUser": {
      "name": "dougwilson",
      "email": "doug@somethingdoug.com"
   },
   "maintainers": [
      {
         "name": "tjholowaychuk",
         "email": "tj@vision-media.ca"
      },
      {
         "name": "jongleberry",
         "email": "jonathanrichardong@gmail.com"
      },
      {
         "name": "shtylman",
         "email": "shtylman@gmail.com"
      },
      {
         "name": "dougwilson",
         "email": "doug@somethingdoug.com"
      },
      {
         "name": "aredridel",
         "email": "aredridel@nbtsc.org"
      },
      {
         "name": "strongloop",
         "email": "callback@strongloop.com"
      },
      {
         "name": "rfeng",
         "email": "enjoyjava@gmail.com"
      }
   ],
   "dist": {
      "shasum": "8df3d5a9ac848585f00a0777601823faecd3b148",
      "tarball": "https://registry.npmjs.org/express/-/express-4.11.2.tgz"
   },
   "directories": {},
      "_resolved": "https://registry.npmjs.org/express/-/express-4.11.2.tgz",
      "readme": "ERROR: No README data found!"
}
Attributes of Package.json
name − name of the package
version − version of the package
description − description of the package
homepage − homepage of the package
author − author of the package
contributors − name of the contributors to the package
dependencies − list of dependencies. NPM automatically installs all the dependencies mentioned here in the node_module folder of the package. repository − repository type and URL of the package
main − entry point of the package
keywords − keywords
Uninstalling a Module
Use the following command to uninstall a Node.js module.

$ npm uninstall express
Once NPM uninstalls the package, you can verify it by looking at the content of /node_modules/ directory or type the following command −

$ npm ls
Updating a Module
Update package.json and change the version of the dependency to be updated and run the following command.

$ npm update express
Search a Module
Search a package name using NPM.

$ npm search express
Create a Module
Creating a module requires package.json to be generated. Let's generate package.json using NPM, which will generate the basic skeleton of the package.json.

$ npm init

This utility will walk you through creating a package.json file.
It only covers the most common items, and tries to guess sane defaults.

See 'npm help json' for definitive documentation on these fields
and exactly what they do.

Use 'npm install <pkg> --save' afterwards to install a package and
save it as a dependency in the package.json file.

Press ^C at any time to quit.
name: (webmaster)
You will need to provide all the required information about your module. You can take help from the above-mentioned package.json file to understand the meanings of various information demanded. Once package.json is generated, use the following command to register yourself with NPM repository site using a valid email address.

$ npm adduser
Username: mcmohd
Password:
Email: (this IS public) mcmohd@gmail.com
It is time now to publish your module −

$ npm publish
If everything is fine with your module, then it will be published in the repository and will be accessible to install using NPM like any other Node.js module.

\section{Command Line}

Pass command line arguments
The arguments are stored in process.argv

Here are the node docs on handling command line args:

 process.argv is an array containing the command line arguments. The first element will be 'node', the second element will be the name of the JavaScript file. The next elements will be any additional command line arguments.

// print process.argv
process.argv.forEach(function (val, index, array) {
  console.log(index + ': ' + val);
});
This will generate:

$ node process-2.js one two=three four
0: node
1: /Users/mjr/work/node/process-2.js
2: one
3: two=three
4: four




