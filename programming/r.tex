\chapter{R}

View online \href{http://magizbox.com/training/r/site/}{http://magizbox.com/training/r/site/}

R
R is a programming language and software environment for statistical computing and graphics supported by the R Foundation for Statistical Computing. The R language is widely used among statisticians and data miners for developing statistical software and data analysis. Polls, surveys of data miners, and studies of scholarly literature databases show that R's popularity has increased substantially in recent years.

R is a GNU package. The source code for the R software environment is written primarily in C, Fortran, and R. R is freely available under the GNU General Public License, and pre-compiled binary versions are provided for various operating systems. While R has a command line interface, there are several graphical front-ends available.[

R was created by Ross Ihaka and Robert Gentleman at the University of Auckland, New Zealand, and is currently developed by the R Development Core Team, of which Chambers is a member. R is named partly after the first names of the first two R authors and partly as a play on the name of S. The project was conceived in 1992, with an initial version released in 1995 and a stable beta version in 2000.

\section{R Courses}

I'm going to give a course about R, but it's take a lot of time to finish. I will give at least one lesson a week. You can track it here

(next) Data visualization with R
Everything you need to know about R
Read and Write Data
Importing data from JSON into R
Manipulate Data
Manipulate String and Datetime
Actually, beside my works, there are a lot of excellent and free courses in the internet for you

Beginner

tryr from codeschool

tryr is a course for beginners created by codeschool. This course contains R Syntax, Vectors, Matrices, Summary Statistics, Factors, Data Frames and Working With Real-World Data sections.

Introduction to R from datacamp

This course created by datacamp - a "online learning platform that focuses on building the best learning experience for Data Science in specific". Here is the introduction about this course quoted from authors "In this introduction to R, you will master the basics of this beautiful open source language such as factors, lists and data frames. With the knowledge gained in this course, you will be ready to undertake your first very own data analysis." It contains 6 chapters: Intro to basics, Vectors, Matrices, Factors, Data frames and Lists.

Intermediate and Advanced

R Programming of Johns Hopkins University in coursera Learn how to program in R and how to use R for effective data analysis. This is the second course in the Johns Hopkins Data Science Specialization. It's a 4-weeks course, contains: Overview of R, R data types and objects, reading and writing data (week 1),  Control structures, functions, scoping rules, dates and times (week 2), Loop functions, debugging tools (week 3) and Simulation, code profiling (week 4)

An Introduction to Statistical Learning with Applications in R of two experts Trevor Hastie and Rob Tibshirani from Standfor Unitiversity

This course was introduced by Kevin Markham in r-blogger in september 2014. "I found it to be an excellent course in statistical learning (also known as “machine learning”), largely due to the high quality of both the textbook and the video lectures. And as an R user, it was extremely helpful that they included R code to demonstrate most of the techniques described in the book." In this course you will learn about Statistical Learning, Linear Regression, Classification, Resampling Methods, Linear Model Selection and Regularization, Moving Beyond Linearity, Tree-Based Methods, Support Vector Machines and Unsupervised Learning

Cheatsheet – Python & R codes for common Machine Learning Algorithms

\section{Everything you need to know about R}

In this post I maintain all useful references for someone want to write nice R code.

Google’s R Style Guide at google
R is a high-level programming language used primarily for statistical computing and graphics. The goal of the R Programming Style Guide is to make our R code easier to read, share, and verify. The rules below were designed in collaboration with the entire R user community at Google.

Installing R packages at r-bloggers
https://www.r-bloggers.com/installing-r-packages/

This is a short post giving steps on how to actually install R packages.

Managing your projects in a reproducible fashion at nicercode
https://nicercode.github.io/blog/2013-04-05-projects/

Managing your projects in a reproducible fashion doesn’t just make your science reproducible, it makes your life easier.

Creating R Packages
http://cran.r-project.org/doc/contrib/Leisch-CreatingPackages.pdf

This tutorial gives a practical introduction to creating R packages. We discuss how object oriented programming and S formulas can be used to give R code the usual look and feel, how to start a package from a collection of R functions, and how to test the code once the package has been created. As running example we use functions for standard linear regression analysis which are developed from scratch

How to write trycatch in R
http://stackoverflow.com/questions/12193779/how-to-write-trycatch-in-r

Welcome to the R world 😉

Debugging with RStudio
https://support.rstudio.com/hc/en-us/articles/200713843-Debugging-with-RStudio

RStudio includes a visual debugger that can help you understand code and find bugs.

Optimising code
http://adv-r.had.co.nz/Profiling.html#performance-profiling

Optimising code to make it run faster is an iterative process:

Find the biggest bottleneck (the slowest part of your code). Try to eliminate it (you may not succeed but that’s ok). Repeat until your code is “fast enough.” This sounds easy, but it’s not.