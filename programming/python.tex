\chapter{Python}


\section{Cơ bản}

\textbf{Vấn đề với mảng}

\begin{item}
  \item Random Sampling \footnote{tham khảo [pytorch](http://pytorch.org/docs/master/torch.html?highlight=randn#torch.randn), [numpy](https://docs.scipy.org/doc/numpy-1.13.0/reference/routines.random.html))} - sinh ra một mảng ngẫu nhiên trong khoảng (0, 1), mảng ngẫu nhiên số nguyên trong khoảng (x, y), mảng ngẫu nhiên là permutation của số từ 1 đến n
\end{item}

\section{Quản lý gói với Anaconda}

\noindent Cài đặt package tại một branch của một project trên github

\begin{lstlisting}[language=bash]
  $ pip install git+https://github.com/tangentlabs/django-oscar-paypal.git@issue/34/oscar-0.6#egg=django-oscar-paypal
\end{lstlisting}

\noindent Trích xuất danh sách package

\begin{lstlisting}[language=bash]
 $ pip freeze > requirements.txt
\end{lstlisting}

\noindent \textbf{Chạy ipython trong environment anaconda}

\noindent Chạy đống lệnh này

\begin{lstlisting}[language=bash]
  conda install nb_conda
  source activate my_env
  python -m IPython kernelspec install-self --user
  ipython notebook
\end{lstlisting}

\noindent \textbf{Interactive programming với ipython}

\noindent Trích xuất ipython ra slide (không hiểu sao default `--to slides` không work nữa, lại phải thêm tham số `--reveal-prefix` [^1]

\begin{lstlisting}[language=bash]
jupyter nbconvert "file.ipynb"
  --to slides
  --reveal-prefix "https://cdnjs.cloudflare.com/ajax/libs/reveal.js/3.1.0"
\end{lstlisting}

**Tham khảo thêm**

* https://stackoverflow.com/questions/37085665/in-which-conda-environment-is-jupyter-executing
* https://github.com/jupyter/notebook/issues/541#issuecomment-146387578
* https://stackoverflow.com/a/20101940/772391

\noindent \textbf{python 3.4 hay 3.5}

Có lẽ 3.5 là lựa chọn tốt hơn (phải có của tensorflow, pytorch, hỗ trợ mock)

### Quản lý môi trường phát triển với conda

Chạy lệnh `remove` để xóa một môi trường

\begin{lstlisting}[language=bash]
conda remove --name flowers --all
\end{lstlisting}

\section{Test với python}

\textbf{Sử dụng những loại test nào?}

Hiện tại mình đang viết unittest với default class của python là Unittest. Thực ra toàn sử dụng `assertEqual` là chính!

Ngoài ra mình cũng đang sử dụng tox để chạy test trên nhiều phiên bản python (python 2.7, 3.5). Điều hay của tox là mình có thể thiết kế toàn bộ cài đặt project và các dependencies package trong file `tox.ini`

\textbf{Chạy test trên nhiều phiên bản python với tox}

Pycharm hỗ trợ debug tox (quá tuyệt!), chỉ với thao tác đơn giản là nhấn chuột phải vào file tox.ini của project.

\section{Xây dựng docs với readthedocs và sphinx}

\noindent \textbf{20/12/2017}: Tự nhiên hôm nay tất cả các class có khai báo kế thừa ở project languageflow không thể index được. Vãi thật. Làm thằng đệ không biết đâu mà build model.

Thử build lại chục lần, thay đổi file conf.py và package\_reference.rst chán chê không được. Giả thiết đầu tiên là do hai nguyên nhân (1) docstring ghi sai, (2) nội dung trong package\_reference.rst bị sai. Sửa chán chê cũng vẫn thể, thử checkout các commit của git. Không hoạt động!

Mất khoảng vài tiếng mới để ý thằng readthedocs có phần log cho từng build một. Lần mò vào build gần nhất và build (mình nhớ là) thành công cách đây 2 ngày

\noindent Log build gần nhất

\begin{lstlisting}[language=bash]
Running Sphinx v1.6.5
making output directory...
loading translations [en]... done
loading intersphinx inventory from https://docs.python.org/objects.inv...
intersphinx inventory has moved: https://docs.python.org/objects.inv -> https://docs.python.org/2/objects.inv
loading intersphinx inventory from http://docs.scipy.org/doc/numpy/objects.inv...
intersphinx inventory has moved: http://docs.scipy.org/doc/numpy/objects.inv -> https://docs.scipy.org/doc/numpy/objects.inv
building [mo]: targets for 0 po files that are out of date
building [readthedocsdirhtml]: targets for 8 source files that are out of date
updating environment: 8 added, 0 changed, 0 removed
reading sources... [ 12%] authors
reading sources... [ 25%] contributing
reading sources... [ 37%] history
reading sources... [ 50%] index
reading sources... [ 62%] installation
reading sources... [ 75%] package_reference
reading sources... [ 87%] readme
reading sources... [100%] usage

looking for now-outdated files... none found
pickling environment... done
checking consistency... done
preparing documents... done
writing output... [ 12%] authors
writing output... [ 25%] contributing
writing output... [ 37%] history
writing output... [ 50%] index
writing output... [ 62%] installation
writing output... [ 75%] package_reference
writing output... [ 87%] readme
writing output... [100%] usage
\end{lstlisting}

Log build hồi trước

\begin{lstlisting}[language=bash]
Running Sphinx v1.5.6
making output directory...
loading translations [en]... done
loading intersphinx inventory from https://docs.python.org/objects.inv...
intersphinx inventory has moved: https://docs.python.org/objects.inv -> https://docs.python.org/2/objects.inv
loading intersphinx inventory from http://docs.scipy.org/doc/numpy/objects.inv...
intersphinx inventory has moved: http://docs.scipy.org/doc/numpy/objects.inv -> https://docs.scipy.org/doc/numpy/objects.inv
building [mo]: targets for 0 po files that are out of date
building [readthedocs]: targets for 8 source files that are out of date
updating environment: 8 added, 0 changed, 0 removed
reading sources... [ 12%] authors
reading sources... [ 25%] contributing
reading sources... [ 37%] history
reading sources... [ 50%] index
reading sources... [ 62%] installation
reading sources... [ 75%] package_reference
reading sources... [ 87%] readme
reading sources... [100%] usage

/home/docs/checkouts/readthedocs.org/user_builds/languageflow/checkouts/develop/languageflow/transformer/count.py:docstring of languageflow.transformer.count.CountVectorizer:106: WARNING: Definition list ends without a blank line; unexpected unindent.
/home/docs/checkouts/readthedocs.org/user_builds/languageflow/checkouts/develop/languageflow/transformer/tfidf.py:docstring of languageflow.transformer.tfidf.TfidfVectorizer:113: WARNING: Definition list ends without a blank line; unexpected unindent.
../README.rst:7: WARNING: nonlocal image URI found: https://img.shields.io/badge/latest-1.1.6-brightgreen.svg
looking for now-outdated files... none found
pickling environment... done
checking consistency... done
preparing documents... done
writing output... [ 12%] authors
writing output... [ 25%] contributing
writing output... [ 37%] history
writing output... [ 50%] index
writing output... [ 62%] installation
writing output... [ 75%] package_reference
writing output... [ 87%] readme
writing output... [100%] usage
\end{lstlisting}

Đập vào mắt là sự khác biệt giữa documentation type

Lỗi

\begin{lstlisting}[language=bash]
building [readthedocsdirhtml]: targets for 8 source files that are out of date
\end{lstlisting}

Chạy

\begin{lstlisting}[language=bash]
building [readthedocs]: targets for 8 source files that are out of date
\end{lstlisting}

Hí ha hí hửng. Chắc trong cơn bất loạn sửa lại settings đây mà. Sửa lại nó trong phần Settings (Admin &gt; Settings &gt; Documentation type)

![](https://magizbox.files.wordpress.com/2017/10/screenshot-from-2017-12-20-09-54-23.png)

Khi chạy nó đã cho ra log đúng

\begin{lstlisting}[language=bash]
building [readthedocsdirhtml]: targets for 8 source files that are out of date
\end{lstlisting}

Nhưng vẫn lỗi. Vãi!!! Sau khoảng 20 phút tiếp tục bấn loạn, chửi bới readthedocs các kiểu. Thì để ý dòng này

Lỗi

\begin{lstlisting}[language=bash]
Running Sphinx v1.6.5
\end{lstlisting}


Chạy

\begin{lstlisting}[language=bash]
Running Sphinx v1.5.6
\end{lstlisting}

Ngay dòng đầu tiên mà không để ý, ngu thật. Aha, Hóa ra là thằng readthedocs nó tự động update phiên bản sphinx lên 1.6.5. Mình là mình chúa ghét thay đổi phiên bản (code đã mệt rồi, lại còn phải tương thích với nhiều phiên bản nữa thì ăn c** à). Đầu tiên search với Pycharm thấy dòng này trong `conf.py`

\begin{lstlisting}[language=bash]
# If your documentation needs a minimal Sphinx version, state it here.
# needs_sphinx = '1.0'
\end{lstlisting}

Đổi thành

\begin{lstlisting}[language=bash]
# If your documentation needs a minimal Sphinx version, state it here.
needs_sphinx = '1.5.6'
\end{lstlisting}

Vẫn vậy (holy sh*t). Thử sâu một tẹo (thực sự là rất nhiều tẹo). Thấy cái này trong trang Settings

![](https://magizbox.files.wordpress.com/2017/10/screenshot-from-2017-12-20-10-01-39.png)

Ờ há. Thằng đần này cho phép trỏ đường dẫn tới một file trong project để cấu hình dependency. Haha.
Tạo thêm một file `requirements` trong thư mục `docs` với nội dung

\begin{lstlisting}
sphinx==1.5.6
\end{lstlisting}


Sau đó cấu hình nó trên giao diện web của readthedocs

![](https://magizbox.files.wordpress.com/2017/10/screenshot-from-2017-12-20-10-04-49.png)

Build thử. Build thử thôi. Cảm giác đúng lắm rồi đấy. Và... nó chạy. Ahihi

![](https://magizbox.files.wordpress.com/2017/10/screenshot-from-2017-12-20-10-06-32.png)

\textbf{Kinh nghiệm}

* Khi không biết làm gì, hãy làm 3 việc. Đọc LOG. Phân tích LOG. Và cố gắng để LOG thay đổi theo ý mình.

PS: Trong quá trình này, cũng không thèm build thằng PDF với Epub nữa. Tiết kiệm được bao nhiêu thời gian.

\section{Pycharm Pycharm}

01/2018: Pycharm là trình duyệt ưa thích của mình trong suốt 3 năm vừa rồi.

Hôm nay tự nhiên lại gặp lỗi không tự nhận unittest, không resolve được package import bởi relative path. Vụ không tự nhận unittest sửa bằng cách xóa file .idea là xong. Còn vụ không resolve được package import bởi relative path thì vẫn chịu rồi. Nhìn code cứ đỏ lòm khó chịu thật.

\section{Vì sao lại code python?}

\textbf{01/11/2017}
Thích python vì nó quá đơn giản (và quá đẹp).

[^1]: https://github.com/jupyter/nbconvert/issues/91#issuecomment-283736634