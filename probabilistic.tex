\part{Xác suất}

\chapter{Các hàm phân phối thông dụng}

Phần này có thêm khảo \cite{Goodfellow-et-al-2016} và giáo trình xác suất thống kê của thạc sỹ Trần Thiện Khải, đại học Trà Vinh  \footnote{\href{http://www.ctec.tvu.edu.vn/ttkhai/xacsuatthongke_dh.htm}{http://www.ctec.tvu.edu.vn/ttkhai/xacsuatthongke\_dh.htm}}

\textbf{17/01/2018} Lòng vòng thế nào hôm nay lại tìm được của bạn Đỗ Minh Hải \footnote{\href{https://dominhhai.github.io/vi/2017/10/prob-com-var}{https://dominhhai.github.io/vi/2017/10/prob-com-var}}, rất hay

\subsection{Biến rời rạc}

\subsubsection{Phân phối đều - Discrete Uniform distribution}

Là phân phối mà xác suất xuất hiện của các sự kiện là như nhau.
\newline
Biến ngẫu nhiên $X$ tuân theo phân phối đều rời rạc
$$X \sim \mathcal{U} (a, b)$$
với tham số $a, b \in \mathbb Z; a < b$ là khoảng giá trị của $X$, đặt $n = b-a+1$
\newline
Ta sẽ có:
\newline
\begin{tabular}{ | l | l | }
  \hline
  Định nghĩa & Giá trị \\
  \hline
  PMF & $p(x)$ | $\dfrac{1}{n}, \forall x \in [a,b]$ \\
  \hline
  CDF - $F(x;a,b)$ & $\dfrac{x-a+1}{n}, \forall x \in [a,b]$ \\
  \hline
  Kỳ vọng - $E[X]$ & $\dfrac{a+b}{2}$ \\
  \hline
  Phương sai - $Var(X)$ & $\dfrac{n^2-1}{12}$ \\
  \hline
\end{tabular}
\newline
Ví dụ: Lịch chạy của xe buýt tại một trạm xe buýt như sau: chiếc xe buýt đầu tiên trong ngày sẽ khởi hành từ trạm này vào lúc 7 giờ, cứ sau mỗi 15 phút sẽ có một xe khác đến trạm. Giả sử một hành khách đến trạm trong khoảng thời gian từ 7 giờ đến 7 giờ 30. Tìm xác suất để hành khách này chờ:

a) Ít hơn 5 phút.

b) Ít nhất 12 phút.

\textbf{Giải}

Gọi X là số phút sau 7 giờ mà hành khách đến trạm.

Ta có: $X \sim R[0;30]$.

a) Hành khách sẽ chờ ít hơn 5 phút nếu đến trạm giữa 7 giờ 10 và 7 giờ 15 hoặc giữa 7 giờ 25 và 7 giờ 30. Do đó xác suất cần tìm là:

$$P(0<X<15) + P(25<X<30)=\frac{5}{30} + \frac{5}{30}=\frac{1}{3}$$

b) Hành khách chờ ít nhất 12 phút nếu đến trạm giữa 7 giờ và 7 giờ 3 phút hoặc giữa 7 giờ 15 phút và 7 giờ 18 phút. Xác suất cần tìm là:

$$P(0<X<3) + P(15<X<18)=\frac{3}{30} + \frac{3}{30}=\frac{1}{5}$$

\subsubsection{Phân phối Béc-nu-li - Bernoulli distribution}

Như đã đề cập về phép thử Béc-nu-li rằng mọi phép thử của nó chỉ cho 2 kết quả duy nhất là $A$ với xác suất $p$ và $\bar A$ với xác suất $q=1-p$
Biến ngẫu nhiên $X$ tuân theo phân phối Béc-nu-li
$$X \sim B(p)$$
với tham số $p \in \mathbb{R}, 0 \leq p \leq 1$ là xác suất xuất hiện của $A$ tại mỗi phép thử
\newline
\begin{tabular}{ | l | l | l | }
  \hline
  Định nghĩa & & Giá trị \\
  \hline
  PMF & p(x) & $p(x)$ | $p^x (1-p)^{1-x}, x \in \{0, 1\} $ \\
  \hline
  CDF & $F(x;p)$  &
  \begin{cases}
    0 & \text{for } x < 0 \\
    1-p & \text{for } 0 \leq x < 1 \\
    1 & \text{for } x \geq 1
  \end{cases} \\
  \hline
  Kỳ vọng & $E[X]$ & $p$ \\
  \hline
  Phương sai & $Var(X)$ & $p(1-p)$ \\
  \hline
\end{tabular}
\newline

\textbf{Ví dụ}

Tham khảo thêm các thuật toán khác tại \cite{hai_2018}