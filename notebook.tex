\documentclass[14pt, a4paper, oneside]{book}

\usepackage[T1]{fontenc}
\usepackage[utf8]{inputenc}
\usepackage[utf8]{vietnam}
\usepackage[vietnam]{babel}

\usepackage[unicode]{hyperref}
\usepackage{amsmath}
\usepackage{amssymb}
\usepackage{eucal}
\usepackage{mathtools}

\setlength{\emergencystretch}{2pt}

\usepackage{color}
\usepackage{graphicx}
\graphicspath{ {images/} }

\usepackage{courier}
\usepackage{listings}
\lstset{
basicstyle=\fontsize{10}\ttfamily,
columns=flexible,
frame=none,
breaklines=true,
postbreak=\mbox{\textcolor{red}{$\hookrightarrow$}\space},
inputencoding=utf8,
showstringspaces=false
}
\usepackage{courier}
\usepackage{textcomp}
\usepackage{listings}
\usepackage{color}
%\usepackage{framed}
\definecolor{lbcolor}{rgb}{0.9,0.9,0.9}
\lstset{
literate={-}{-}1,
basicstyle=\fontsize{10}\ttfamily,
columns=flexible,
frame=none,
breaklines=true,
postbreak=\mbox{\textcolor{red}{$\hookrightarrow$}\space},
inputencoding=utf8,
upquote=true,
showstringspaces=false,
xleftmargin=.25in,
backgroundcolor=\color{lbcolor},
framexleftmargin=5pt,
frame=tb, framerule=0pt
}
\lstdefinelanguage{JavaScript}{
  keywords={typeof, new, true, false, catch, function, return, null, catch, switch, var, if, in, while, do, else, case, break},
  keywordstyle=\color{black}\bfseries,
  ndkeywords={class, export, boolean, throw, implements, import, this},
  ndkeywordstyle=\color{darkgray}\bfseries,
  identifierstyle=\color{black},
  sensitive=false,
  comment=[l]{//},
  morecomment=[s]{/*}{*/},
  commentstyle=\color{purple}\ttfamily,
  stringstyle=\color{black}\ttfamily,
  morestring=[b]',
  morestring=[b]"
}

% ====== BEGING TESTING ======
\renewcommand{\arraystretch}{1}

% ============================ %
%           definition
% ============================ %
\newcommand{\definition}[2]{
{\setlength{\parindent}{0pt}
\begin{parcolumns}[nofirstindent,colwidths={1=2.8cm}]{2}
  \colchunk{
  \index{#1}
  \noindent \textcolor{blue}{#1}
  }
  \colchunk{ #2 }
\end{parcolumns}
}
}



% ============================ %
%            notes
% ============================ %
\usepackage{lipsum}                     % Dummytext
\usepackage{xargs}                      % Use more than one optional parameter in a new commands
\usepackage[pdftex,dvipsnames]{xcolor}  % Coloured text
\usepackage[colorinlistoftodos,prependcaption,textsize=tiny]{todonotes}

\newcommandx{\unsure}[2][1=]{\todo[linecolor=red,backgroundcolor=red!25,bordercolor=red,#1]{#2}}
\newcommandx{\change}[2][1=]{\todo[linecolor=blue,backgroundcolor=blue!25,bordercolor=blue,#1]{#2}}
\newcommandx{\info}[2][1=]{\todo[linecolor=OliveGreen,backgroundcolor=OliveGreen!25,bordercolor=OliveGreen,#1]{#2}}
\newcommandx{\diary}[2][1=]{\todo[linecolor=OliveGreen,backgroundcolor=OliveGreen!25,bordercolor=OliveGreen,#1]{#2}}
\newcommandx{\improvement}[2][1=]{\todo[linecolor=Plum,backgroundcolor=Plum!25,bordercolor=Plum,#1]{#2}}
\newcommandx{\thiswillnotshow}[2][1=]{\todo[disable,#1]{#2}}
\usepackage{parcolumns}
% ======   END  TESTING ======


\usepackage[makeindex]{imakeidx}
\usepackage[plain]{cite}
\usepackage{natbib}

\hypersetup{
colorlinks=true,
linkcolor=blue,
filecolor=magenta,
urlcolor=cyan,
}

\setlength{\marginparwidth}{2cm}

\title{Ghi chú của một coder}
\author{Vũ Anh}
\date{Tháng 01 năm 2018}
%\usepackage{index}
\makeindex

\begin{document}


  \maketitle

  \tableofcontents
  \addcontentsline{toc}{chapter}{\contentsname}

%  \chapter{Lập trình là gì?}

\section{Các vấn đề lập trình}

Các vấn đề lập trình với từng ngôn ngữ

\subsection{Nhập môn}

Phần 1: Cơ bản

\begin{lstlisting}
├── 1. introduction
├── 2. syntax
├── 3. data structure
├── 4. oop
\end{lstlisting}

Phần 2: Xây dựng ứng dụng

\begin{lstlisting}
├── 16. database
├── 5. networking
├── 6. os
├── 14. ui
├── 15. web
\end{lstlisting}

Phần 3: Các chủ đề nâng cao

\begin{lstlisting}
├── 7. parallel
├── 8. event based
├── 9. error handling
├── 10. logging
\end{lstlisting}

Phần 4: Phát triển phần mềm chuyên nghiệp

\begin{lstlisting}
├── 11. configuration
├── 12. documentation
├── 13. test
├── 17. ide
├── 18. package manager
├── 19. build tool
├── 20. make module
└── 21. production (docker)
\end{lstlisting}

\section{Introduction}

Installation (environment, IDE)

Hello world

Courses

Resources


\section{Syntax}

variables and expressions

conditional

loops and Iteration

functions

define, use

parameters

scope of variables

anonymous functions

callbacks

self-invoking functions, inner functions

functions that return functions, functions that redefined themselves

closures

naming convention

comment convention

\section{Cấu trúc dữ liệu}

Number

String

Collection

DateTime

Boolean

Object

\section{Lập trình hướng đối tượng}

Classes & Objects

Inheritance

Encapsulation

Abstraction

Polymorphism

For OOP Example: see Python: OOP

\subsection{Bài tập}

\textbf{Quản lý tài khoản ngân hàng}

\section{Networking}

REST (example with chat app sender, receiver, message)

\subsection{Bài tập}

Guess My Number Game

\section{GUI - Giao diện}

Quản lý hot girl

Quản lý truyện tranh

Create Analog Clock

Chương trình lịch âm dương

Chương trình học từ tiếng Anh

\section{Game}

\begin{itemize}
  \item Create Pong Game
  \item Create flappy bird
  \item Create Bouncing Game
\end{itemize}


\section{Cơ sở dữ liệu}

\subsection{Thử thách}


\section{How to ask a question}

Focus on questions about an actual problem you have faced. Include details about what you have tried and exactly what you are trying to do.

Ask about...

✔ Specific programming problems

✔ Software algorithms

✔ Coding techniques

✔ Software development tools

Not all questions work well in our format. Avoid questions that are primarily opinion-based, or that are likely to generate discussion rather than answers.

Don't ask about...

✖ Questions you haven't tried to find an answer for (show your work!)

✖ Product or service recommendations or comparisons

✖ Requests for lists of things, polls, opinions, discussions, etc.

✖ Anything not directly related to writing computer programs

\section{Các nguyên tắc lập trình}

Generic

KISS (Keep It Simple Stupid)

YAGNI

Do The Simplest Thing That Could Possibly Work

Keep Things DRY

Code For The Maintainer

Avoid Premature Optimization

Inter-Module/Class

Minimise Coupling

Law of Demeter

Composition Over Inheritance

Orthogonality

Module/Class

Maximise Cohesion

Liskov Substitution Principle

Open/Closed Principle

Single Responsibility Principle

Hide Implementation Details

Curly's Law

Software Quality Laws

First Law of Software Quality


\section{Các mô hình lập trình}

Main paradigm approaches 1

1. Imperative


Description:

Computation as statements that directly change a program state (datafields)

Main Characteristics:

Direct assignments, common data structures, global variables

Critics: Edsger W. Dijkstra, Michael A. Jackson

Examples: Assembly, C, C++, Java, PHP, Python

2. Structured

Description:

A style of imperative programming with more logical program structure

Main Characteristics:

Structograms, indentation, either no, or limited use of, goto statements

Examples: C, C++, Java, Python

3. Procedural

Description:

Derived from structured programming, based on the concept of modular programming or the procedure call

Main Characteristics:

Local variables, sequence, selection, iteration, and modularization

Examples: C, C++, Lisp, PHP, Python

4. Functional


Description:

Treats computation as the evaluation of mathematical functions avoiding state and mutable data

Main Characteristics:

Lambda calculus, compositionality, formula, recursion, referential transparency, no side effects

Examples: Clojure, Coffeescript, Elixir, Erlang, F#, Haskell, Lisp, Python, Scala, SequenceL, SML

5. Event-driven including time driven

Description:

Program flow is determined mainly by events, such as mouse clicks or interrupts including timer

Main Characteristics:

Main loop, event handlers, asynchronous processes

Examples: Javascript, ActionScript, Visual Basic

6. Object-oriented

Description:

Treats datafields as objects manipulated through pre-defined methods only

Main Characteristics:

Objects, methods, message passing, information hiding, data abstraction, encapsulation, polymorphism, inheritance, serialization-marshalling

Examples: Common Lisp, C++, C#, Eiffel, Java, PHP, Python, Ruby, Scala

7. Declarative

Description:

Defines computation logic without defining its detailed control flow

Main Characteristics:

4GLs, spreadsheets, report program generators

Examples: SQL, regular expressions, CSS, Prolog

8. Automata-based programming

Description:

Treats programs as a model of a finite state machine or any other formal automata

Main Characteristics:

State enumeration, control variable, state changes, isomorphism, state transition table

Examples: AsmL

\section{Testing}

\includegraphics{programming/introduction/unit_test_tdd}

1. Definition 1 2

Test-driven development (TDD) is a software development process that relies on the repetition of a very short development cycle:

\includegraphics[width=\linewidth]{programming/introduction/tdd.jpg}

Step 1: First the developer writes an (initially failing) automated test case that defines a desired improvement or new function,

Step 2: Then produces the minimum amount of code to pass that test,

Step 3: Finally refactors the new code to acceptable standards.

Kent Beck, who is credited with having developed or 'rediscovered' the technique, stated in 2003 that TDD encourages simple designs and inspires confidence.

2. Principles 2

Kent Beck defines

Never with a single line of code unless you have a failing automated test.
Eliminate duplication
Red: (Automated test fail) Green (Automated test pass because dev code has been written) Refactor (Eliminate duplication, Clean the code)

3. Assertions & Assert Framework

\includegraphics[width=\linewidth]{programming/introduction/tdd_assertion.png}

Assert that the expected results have occurred.
[code lang="java"] @Test public void test() { assertEquals(2, 1 + 1); } [/code]


4. Test Runners 3

\includegraphics[width=\linewidth]{programming/introduction/tdd_test_runner.png}

When testing a large real-world web app there may be tens or hundreds of test cases, and we certainly don't want to run each one manually. In such as scenario we need to use a test runner to find and execute the tests for us, and in this article we'll explore just that.

A test runner provides the a good basis for a real testing framework. A test runner is designed to run tests, tag tests with attributes (annotations), and provide reporting and other features.

In general you break your tests up into 3 standard sections; setUp(), tests, and tearDown(), typical for a test runner setup.

The setUp() and tearDown() methods are run automatically for every test, and contain respectively:

The setup steps you need to take before running the test, such as unlocking the screen and killing open apps.
The cooldown steps you need to run after the test, such as closing the Marionette session.

5. Test Frameworks

Language	Test Frameworks
C++/VisualStudio	C++: Test
Web Service	rest-assured
Web UI	SeleniumHQ

\section{Logging}

Logging is the process of recording application actions and state to a secondary interface.

\includegraphics[width=\linewidth]{programming/introduction/logging}

Logging System

\includegraphics[width=\linewidth]{programming/introduction/logging_system}

Levels

Level	When it’s used
DEBUG	Detailed information, typically of interest only when diagnosing problems.
INFO	Confirmation that things are working as expected.
WARNING	An indication that something unexpected happened, or indicative of some problem in the near future (e.g. ‘disk space low’). The software is still working as expected.

ERROR

Due to a more serious problem, the software has not been able to perform some function.
CRITICAL	A serious error, indicating that the program itself may be unable to continue running.
Best Practices 2 4 5
Logging should always be considered when handling an exception but should never take the place of a real handler.
Keep all logging code in your production code. Have an ability to enable more/less detailed logging in production, preferably per subsystem and without restarting your program.
Make logs easy to parse by grep and by eye. Stick to several common fields at the beginning of each line. Identify time, severity, and subsystem in every line. Clearly formulate the message. Make every log message easy to map to its source code line.
If an error happens, try to collect and log as much information as possible. It may take long but it's OK because normal processing has failed anyway. Not having to wait when the same condition happens in production with a debugger attached is priceless.

\section{Lập trình hàm}

Functional
Without mutable variables, assignment, conditional

Advantages 1
Most functional languages provide a nice, protected environment, somewhat like JavaLanguage. It's good to be able to catch exceptions instead of having CoreDumps in stability-critical applications.
FP encourages safe ways of programming. I've never seen an OffByOne mistake in a functional program, for example... I've seen one. Adding two lengths to get an index but one of them was zero-indexed. Easy to discover though. -- AnonymousDonor
Functional programs tend to be much more terse than their ImperativeLanguage counterparts. Often this leads to enhanced programmer productivity.
FP encourages quick prototyping. As such, I think it is the best software design paradigm for ExtremeProgrammers... but what do I know.
FP is modular in the dimension of functionality, where ObjectOrientedProgramming is modular in the dimension of different components.
Generic routines (also provided by CeePlusPlus) with easy syntax. ParametricPolymorphism
The ability to have your cake and eat it. Imagine you have a complex OO system processing messages - every component might make state changes depending on the message and then forward the message to some objects it has links to. Wouldn't it be just too cool to be able to easily roll back every change if some object deep in the call hierarchy decided the message is flawed? How about having a history of different states?
Many housekeeping tasks made for you: deconstructing data structures (PatternMatching), storing variable bindings (LexicalScope with closures), strong typing (TypeInference), * GarbageCollection, storage allocation, whether to use boxed (pointer-to-value) or unboxed (value directly) representation...
Safe multithreading! Immutable data structures are not subject to data race conditions, and consequently don't have to be protected by locks. If you are always allocating new objects, rather than destructively manipulating existing ones, the locking can be hidden in the allocation and GarbageCollection system.

\section{Lập trình song song}

Paralell/Concurrency Programming
1. Callback Pattern 2
Callback functions are derived from a programming paradigm known as functional programming. At a fundamental level, functional programming specifies the use of functions as arguments. Functional programming was—and still is, though to a much lesser extent today—seen as an esoteric technique of specially trained, master programmers.

Fortunately, the techniques of functional programming have been elucidated so that mere mortals like you and me can understand and use them with ease. One of the chief techniques in functional programming happens to be callback functions. As you will read shortly, implementing callback functions is as easy as passing regular variables as arguments. This technique is so simple that I wonder why it is mostly covered in advanced JavaScript topics.

[code lang="javascript"] function getN(){ return 10; }

var n = getN();

function getAsyncN(callback){ setTimeout(function(){ callback(10); }, 1000); }

function afterGetAsyncN(result){ var n = 10; console.log(n); }

getAsyncN(afterGetAsyncN); [/code]

2. Promise Pattern 1 3
What is a promise?
The core idea behind promises is that a promise represents the result of an asynchronous operation.

A promise is in one of three different states:

pending - The initial state of a promise.
fulfilled - The state of a promise representing a successful operation.
rejected - The state of a promise representing a failed operation.
Once a promise is fulfilled or rejected, it is immutable (i.e. it can never change again).


\begin{lstlisting}[language=Javscript]
function aPromise(message){
  return new Promise(function(fulfill, reject){
    if(message == "success"){
      fulfill("it is a success Promise");
    } if(message == "fail"){
      reject("it is a fail Promise");
    }
  });
}
\end{lstlisting}

Usage:

\begin{lstlisting}[language=Javascript]
aPromise("success").then(function(successMessage){
  console.log(successMessage) }, function(failMessage){
  // it is a success Promise
  console.log(failMessage)
})
\end{lstlisting}

\begin{lstlisting}[language=Javascript]
aPromise("fail").then(function(successMessage){
  console.log(successMessage) }, function(failMessage){
  console.log(failMessage)
}) // it is a fail Promise
\end{lstlisting}

\section{IDE - Môi trường phát triển tích hợp}

An integrated development environment (IDE) is a software application that provides comprehensive facilities to computer programmers for software development. An IDE normally consists of a source code editor, build automation tools and a debugger. Most modern IDEs have intelligent code completion.

1. Navigation

Word Navigation Line Navigation File Navigation

2. Editing

Auto Complete Code Complete Multicursor Template (Snippets)

3. Formatting

Debugging
Custom Rendering for Object
%  \part{Lập trình}

\chapter{Lập trình là gì?}

\section{Các vấn đề lập trình}

Các vấn đề lập trình với từng ngôn ngữ

\subsection{Nhập môn}

Phần 1: Cơ bản

\begin{lstlisting}
├── 1. introduction
├── 2. syntax
├── 3. data structure
├── 4. oop
\end{lstlisting}

Phần 2: Xây dựng ứng dụng

\begin{lstlisting}
├── 16. database
├── 5. networking
├── 6. os
├── 14. ui
├── 15. web
\end{lstlisting}

Phần 3: Các chủ đề nâng cao

\begin{lstlisting}
├── 7. parallel
├── 8. event based
├── 9. error handling
├── 10. logging
\end{lstlisting}

Phần 4: Phát triển phần mềm chuyên nghiệp

\begin{lstlisting}
├── 11. configuration
├── 12. documentation
├── 13. test
├── 17. ide
├── 18. package manager
├── 19. build tool
├── 20. make module
└── 21. production (docker)
\end{lstlisting}

\section{Introduction}

Installation (environment, IDE)

Hello world

Courses

Resources


\section{Syntax}

variables and expressions

conditional

loops and Iteration

functions

define, use

parameters

scope of variables

anonymous functions

callbacks

self-invoking functions, inner functions

functions that return functions, functions that redefined themselves

closures

naming convention

comment convention

\section{Cấu trúc dữ liệu}

Number

String

Collection

DateTime

Boolean

Object

\section{Lập trình hướng đối tượng}

Classes & Objects

Inheritance

Encapsulation

Abstraction

Polymorphism

For OOP Example: see Python: OOP

\subsection{Bài tập}

\textbf{Quản lý tài khoản ngân hàng}

\section{Networking}

REST (example with chat app sender, receiver, message)

\subsection{Bài tập}

Guess My Number Game

\section{GUI - Giao diện}

Quản lý hot girl

Quản lý truyện tranh

Create Analog Clock

Chương trình lịch âm dương

Chương trình học từ tiếng Anh

\section{Game}

\begin{itemize}
  \item Create Pong Game
  \item Create flappy bird
  \item Create Bouncing Game
\end{itemize}


\section{Cơ sở dữ liệu}

\subsection{Thử thách}


\section{How to ask a question}

Focus on questions about an actual problem you have faced. Include details about what you have tried and exactly what you are trying to do.

Ask about...

✔ Specific programming problems

✔ Software algorithms

✔ Coding techniques

✔ Software development tools

Not all questions work well in our format. Avoid questions that are primarily opinion-based, or that are likely to generate discussion rather than answers.

Don't ask about...

✖ Questions you haven't tried to find an answer for (show your work!)

✖ Product or service recommendations or comparisons

✖ Requests for lists of things, polls, opinions, discussions, etc.

✖ Anything not directly related to writing computer programs

\section{Các nguyên tắc lập trình}

Generic

KISS (Keep It Simple Stupid)

YAGNI

Do The Simplest Thing That Could Possibly Work

Keep Things DRY

Code For The Maintainer

Avoid Premature Optimization

Inter-Module/Class

Minimise Coupling

Law of Demeter

Composition Over Inheritance

Orthogonality

Module/Class

Maximise Cohesion

Liskov Substitution Principle

Open/Closed Principle

Single Responsibility Principle

Hide Implementation Details

Curly's Law

Software Quality Laws

First Law of Software Quality


\section{Các mô hình lập trình}

Main paradigm approaches 1

1. Imperative


Description:

Computation as statements that directly change a program state (datafields)

Main Characteristics:

Direct assignments, common data structures, global variables

Critics: Edsger W. Dijkstra, Michael A. Jackson

Examples: Assembly, C, C++, Java, PHP, Python

2. Structured

Description:

A style of imperative programming with more logical program structure

Main Characteristics:

Structograms, indentation, either no, or limited use of, goto statements

Examples: C, C++, Java, Python

3. Procedural

Description:

Derived from structured programming, based on the concept of modular programming or the procedure call

Main Characteristics:

Local variables, sequence, selection, iteration, and modularization

Examples: C, C++, Lisp, PHP, Python

4. Functional


Description:

Treats computation as the evaluation of mathematical functions avoiding state and mutable data

Main Characteristics:

Lambda calculus, compositionality, formula, recursion, referential transparency, no side effects

Examples: Clojure, Coffeescript, Elixir, Erlang, F#, Haskell, Lisp, Python, Scala, SequenceL, SML

5. Event-driven including time driven

Description:

Program flow is determined mainly by events, such as mouse clicks or interrupts including timer

Main Characteristics:

Main loop, event handlers, asynchronous processes

Examples: Javascript, ActionScript, Visual Basic

6. Object-oriented

Description:

Treats datafields as objects manipulated through pre-defined methods only

Main Characteristics:

Objects, methods, message passing, information hiding, data abstraction, encapsulation, polymorphism, inheritance, serialization-marshalling

Examples: Common Lisp, C++, C#, Eiffel, Java, PHP, Python, Ruby, Scala

7. Declarative

Description:

Defines computation logic without defining its detailed control flow

Main Characteristics:

4GLs, spreadsheets, report program generators

Examples: SQL, regular expressions, CSS, Prolog

8. Automata-based programming

Description:

Treats programs as a model of a finite state machine or any other formal automata

Main Characteristics:

State enumeration, control variable, state changes, isomorphism, state transition table

Examples: AsmL

\section{Testing}

\includegraphics{programming/introduction/unit_test_tdd}

1. Definition 1 2

Test-driven development (TDD) is a software development process that relies on the repetition of a very short development cycle:

\includegraphics[width=\linewidth]{programming/introduction/tdd.jpg}

Step 1: First the developer writes an (initially failing) automated test case that defines a desired improvement or new function,

Step 2: Then produces the minimum amount of code to pass that test,

Step 3: Finally refactors the new code to acceptable standards.

Kent Beck, who is credited with having developed or 'rediscovered' the technique, stated in 2003 that TDD encourages simple designs and inspires confidence.

2. Principles 2

Kent Beck defines

Never with a single line of code unless you have a failing automated test.
Eliminate duplication
Red: (Automated test fail) Green (Automated test pass because dev code has been written) Refactor (Eliminate duplication, Clean the code)

3. Assertions & Assert Framework

\includegraphics[width=\linewidth]{programming/introduction/tdd_assertion.png}

Assert that the expected results have occurred.
[code lang="java"] @Test public void test() { assertEquals(2, 1 + 1); } [/code]


4. Test Runners 3

\includegraphics[width=\linewidth]{programming/introduction/tdd_test_runner.png}

When testing a large real-world web app there may be tens or hundreds of test cases, and we certainly don't want to run each one manually. In such as scenario we need to use a test runner to find and execute the tests for us, and in this article we'll explore just that.

A test runner provides the a good basis for a real testing framework. A test runner is designed to run tests, tag tests with attributes (annotations), and provide reporting and other features.

In general you break your tests up into 3 standard sections; setUp(), tests, and tearDown(), typical for a test runner setup.

The setUp() and tearDown() methods are run automatically for every test, and contain respectively:

The setup steps you need to take before running the test, such as unlocking the screen and killing open apps.
The cooldown steps you need to run after the test, such as closing the Marionette session.

5. Test Frameworks

Language	Test Frameworks
C++/VisualStudio	C++: Test
Web Service	rest-assured
Web UI	SeleniumHQ

\section{Logging}

Logging is the process of recording application actions and state to a secondary interface.

\includegraphics[width=\linewidth]{programming/introduction/logging}

Logging System

\includegraphics[width=\linewidth]{programming/introduction/logging_system}

Levels

Level	When it’s used
DEBUG	Detailed information, typically of interest only when diagnosing problems.
INFO	Confirmation that things are working as expected.
WARNING	An indication that something unexpected happened, or indicative of some problem in the near future (e.g. ‘disk space low’). The software is still working as expected.

ERROR

Due to a more serious problem, the software has not been able to perform some function.
CRITICAL	A serious error, indicating that the program itself may be unable to continue running.
Best Practices 2 4 5
Logging should always be considered when handling an exception but should never take the place of a real handler.
Keep all logging code in your production code. Have an ability to enable more/less detailed logging in production, preferably per subsystem and without restarting your program.
Make logs easy to parse by grep and by eye. Stick to several common fields at the beginning of each line. Identify time, severity, and subsystem in every line. Clearly formulate the message. Make every log message easy to map to its source code line.
If an error happens, try to collect and log as much information as possible. It may take long but it's OK because normal processing has failed anyway. Not having to wait when the same condition happens in production with a debugger attached is priceless.

\section{Lập trình hàm}

Functional
Without mutable variables, assignment, conditional

Advantages 1
Most functional languages provide a nice, protected environment, somewhat like JavaLanguage. It's good to be able to catch exceptions instead of having CoreDumps in stability-critical applications.
FP encourages safe ways of programming. I've never seen an OffByOne mistake in a functional program, for example... I've seen one. Adding two lengths to get an index but one of them was zero-indexed. Easy to discover though. -- AnonymousDonor
Functional programs tend to be much more terse than their ImperativeLanguage counterparts. Often this leads to enhanced programmer productivity.
FP encourages quick prototyping. As such, I think it is the best software design paradigm for ExtremeProgrammers... but what do I know.
FP is modular in the dimension of functionality, where ObjectOrientedProgramming is modular in the dimension of different components.
Generic routines (also provided by CeePlusPlus) with easy syntax. ParametricPolymorphism
The ability to have your cake and eat it. Imagine you have a complex OO system processing messages - every component might make state changes depending on the message and then forward the message to some objects it has links to. Wouldn't it be just too cool to be able to easily roll back every change if some object deep in the call hierarchy decided the message is flawed? How about having a history of different states?
Many housekeeping tasks made for you: deconstructing data structures (PatternMatching), storing variable bindings (LexicalScope with closures), strong typing (TypeInference), * GarbageCollection, storage allocation, whether to use boxed (pointer-to-value) or unboxed (value directly) representation...
Safe multithreading! Immutable data structures are not subject to data race conditions, and consequently don't have to be protected by locks. If you are always allocating new objects, rather than destructively manipulating existing ones, the locking can be hidden in the allocation and GarbageCollection system.

\section{Lập trình song song}

Paralell/Concurrency Programming
1. Callback Pattern 2
Callback functions are derived from a programming paradigm known as functional programming. At a fundamental level, functional programming specifies the use of functions as arguments. Functional programming was—and still is, though to a much lesser extent today—seen as an esoteric technique of specially trained, master programmers.

Fortunately, the techniques of functional programming have been elucidated so that mere mortals like you and me can understand and use them with ease. One of the chief techniques in functional programming happens to be callback functions. As you will read shortly, implementing callback functions is as easy as passing regular variables as arguments. This technique is so simple that I wonder why it is mostly covered in advanced JavaScript topics.

[code lang="javascript"] function getN(){ return 10; }

var n = getN();

function getAsyncN(callback){ setTimeout(function(){ callback(10); }, 1000); }

function afterGetAsyncN(result){ var n = 10; console.log(n); }

getAsyncN(afterGetAsyncN); [/code]

2. Promise Pattern 1 3
What is a promise?
The core idea behind promises is that a promise represents the result of an asynchronous operation.

A promise is in one of three different states:

pending - The initial state of a promise.
fulfilled - The state of a promise representing a successful operation.
rejected - The state of a promise representing a failed operation.
Once a promise is fulfilled or rejected, it is immutable (i.e. it can never change again).


\begin{lstlisting}[language=Javscript]
function aPromise(message){
  return new Promise(function(fulfill, reject){
    if(message == "success"){
      fulfill("it is a success Promise");
    } if(message == "fail"){
      reject("it is a fail Promise");
    }
  });
}
\end{lstlisting}

Usage:

\begin{lstlisting}[language=Javascript]
aPromise("success").then(function(successMessage){
  console.log(successMessage) }, function(failMessage){
  // it is a success Promise
  console.log(failMessage)
})
\end{lstlisting}

\begin{lstlisting}[language=Javascript]
aPromise("fail").then(function(successMessage){
  console.log(successMessage) }, function(failMessage){
  console.log(failMessage)
}) // it is a fail Promise
\end{lstlisting}

\section{IDE - Môi trường phát triển tích hợp}

An integrated development environment (IDE) is a software application that provides comprehensive facilities to computer programmers for software development. An IDE normally consists of a source code editor, build automation tools and a debugger. Most modern IDEs have intelligent code completion.

1. Navigation

Word Navigation Line Navigation File Navigation

2. Editing

Auto Complete Code Complete Multicursor Template (Snippets)

3. Formatting

Debugging
Custom Rendering for Object
%\chapter{Python}

\section{Giới thiệu}

\begin{item}
  \item `Python` is a widely used general-purpose, high-level programming language. Its design philosophy emphasizes code readability, and its syntax allows programmers to express concepts in fewer lines of code than would be possible in languages such as C++ or Java.
  \item The language provides constructs intended to enable clear programs on both a small and large scale.
\end{item}

Python Tutorial
Python is a general-purpose interpreted, interactive, object-oriented, and high-level programming language. It was created by Guido van Rossum during 1985- 1990. Like Perl, Python source code is also available under the GNU General Public License (GPL). This tutorial gives enough understanding on Python programming language.

Python is Interpreted

Python is processed at runtime by the interpreter. You do not need to compile your program before executing it. This is similar to PERL and PHP.

Python is Interactive

You can actually sit at a Python prompt and interact with the interpreter directly to write your programs.

Python is Object-Oriented

Python supports Object-Oriented style or technique of programming that encapsulates code within objects.

Python is Beginner Friendly

Python is a great language for the beginner-level programmers and supports the development of a wide range of applications from simple text processing to WWW browsers to games.

Audience
This tutorial is designed for software programmers who need to learn Python programming language from scratch.


\textbf{Sách}

\href{https://docs.google.com/document/d/1gQFMXZtynpuTenoOQNGCHttArT4NspTWcyJQr5ps9Mk/edit?usp=sharing}{Tập hợp các sách python}

\textbf{Khoá học}

\href{1frO9QYhgsXbMzcyXoA4czWkxTWF8RBTJVf9uoO1rElU}{Tập hợp các khóa học python}

\textbf{Tham khảo}

\href{http://blog.tryolabs.com/2015/12/15/top-10-python-libraries-of-2015/}{Top 10 Python Libraries Of 2015}

\section{Cài đặt}

Get Started
Welcome! This tutorial details how to get started with Python.

For Windows
Anaconda 4.3.0
Anaconda is BSD licensed which gives you permission to use Anaconda commercially and for redistribution.

1. Download the installer
2. Optional: Verify data integrity with MD5 or SHA-256
3. Double-click the .exe file to install Anaconda and follow the instructions on the screen
Python 3.6 version
64-BIT INSTALLER
Python 2.7 version
64-BIT INSTALLER
Step 2. Discover the Map

https://docs.python.org/2/library/index.html

For CentOS
Developer tools
The Development tools will allow you to build and compile software from source code. Tools for building RPMs are also included, as well as source code management tools like Git, SVN, and CVS.

\begin{lstlisting}[language=bash]
yum groupinstall "Development tools"
yum install zlib-devel
yum install bzip2-devel
yum install openssl-devel
yum install ncurses-devel
yum install sqlite-devel
\end{lstlisting}

Python & Anaconda
Anaconda is BSD licensed which gives you permission to use Anaconda commercially and for redistribution.

\begin{lstlisting}[language=bash]
cd /opt
wget --no-check-certificate https://www.python.org/ftp/python/2.7.6/Python-2.7.6.tar.xz
tar xf Python-2.7.6.tar.xz
cd Python-2.7.6
./configure --prefix=/usr/local
make && make altinstall
## link
ln -s /usr/local/bin/python2.7 /usr/local/bin/python
# final check
which python
python -V
# install Anaconda
cd ~/Downloads
wget https://repo.continuum.io/archive/Anaconda-2.3.0-Linux-x86_64.sh
bash ~/Downloads/Anaconda-2.3.0-Linux-x86_64.sh
\end{lstlisting}

\section{Cơ bản}

\section{Cú pháp cơ bản}

Print, print

\begin{lstlisting}[language=python]
print "Hello World"
\end{lstlisting}


Conditional

\begin{lstlisting}[language=Python]
if you_smart:
    print "learn python"
else:
    print "go away"
\end{lstlisting}

Loop

In general, statements are executed sequentially: The first statement in a function is executed first, followed by the second, and so on. There may be a situation when you need to execute a block of code several number of times.

Programming languages provide various control structures that allow for more complicated execution paths. A loop statement allows us to execute a statement or group of statements multiple times. The following diagram illustrates a loop statement


Python programming language provides following types of loops to handle looping requirements.

while loop	Repeats a statement or group of statements while a given condition is TRUE. It tests the condition before executing the loop body.
for loop	Executes a sequence of statements multiple times and abbreviates the code that manages the loop variable.
nested loops	You can use one or more loop inside any another while, for or do..while loop.
While Loop
A while loop statement in Python programming language repeatedly executes a target statement as long as a given condition is true.

Syntax

The syntax of a while loop in Python programming language is

\begin{lstlisting}[language=Python]
while expression:
   statement(s)
\end{lstlisting}

Example

\begin{lstlisting}[language=Python]
count = 0
while count < 9:
   print 'The count is:', count
   count += 1
print "Good bye!"
\end{lstlisting}


For Loop

It has the ability to iterate over the items of any sequence, such as a list or a string.

Syntax

\begin{lstlisting}[language=Python]
for iterating_var in sequence:
   statements(s)
\end{lstlisting}

If a sequence contains an expression list, it is evaluated first. Then, the first item in the sequence is assigned to the iterating variable iterating_var. Next, the statements block is executed. Each item in the list is assigned to iterating_var, and the statement(s) block is executed until the entire sequence is exhausted.

Example

\begin{lstlisting}[language=Python]
for i in range(10):
    print "hello", i

for letter in 'Python':
   print 'Current letter :', letter

fruits = ['banana', 'apple',  'mango']
for fruit in fruits:
   print 'Current fruit :', fruit

print "Good bye!"
\end{lstlisting}

Yield and Generator

Yield is a keyword that is used like return, except the function will return a generator.

\begin{lstlisting}[language=Python]
def createGenerator():
    yield 1
    yield 2
    yield 3
mygenerator = createGenerator() # create a generator
print(mygenerator) # mygenerator is an object!
# <generator object createGenerator at 0xb7555c34>
for i in mygenerator:
    print(i)
# 1
# 2
# 3
\end{lstlisting}


Visit Yield and Generator explained for more information

Functions

Variable-length arguments

\begin{lstlisting}[language=Python]
def functionname([formal_args,] *var_args_tuple ):
   "function_docstring"
   function_suite
   return [expression]
\end{lstlisting}

Example

\begin{lstlisting}[language=Python]
#!/usr/bin/python

# Function definition is here
def printinfo( arg1, *vartuple ):
   "This prints a variable passed arguments"
   print "Output is: "
   print arg1
   for var in vartuple:
      print var
   return;

# Now you can call printinfo function
printinfo( 10 )
printinfo( 70, 60, 50 )
\end{lstlisting}

Coding Convention
Code layout
Indentation: 4 spaces

Suggest Readings

"Python Functions". www.tutorialspoint.com
"Python Loops". www.tutorialspoint.com
"What does the “yield” keyword do?". stackoverflow.com
"Improve Your Python: 'yield' and Generators Explained". jeffknupp.com

\textbf{Vấn đề với mảng}

\begin{item}
  \item Random Sampling \footnote{tham khảo [pytorch](http://pytorch.org/docs/master/torch.html?highlight=randn#torch.randn), [numpy](https://docs.scipy.org/doc/numpy-1.13.0/reference/routines.random.html))} - sinh ra một mảng ngẫu nhiên trong khoảng (0, 1), mảng ngẫu nhiên số nguyên trong khoảng (x, y), mảng ngẫu nhiên là permutation của số từ 1 đến n
\end{item}

\section{Yield and Generators}

Coroutines and Subroutines
When we call a normal Python function, execution starts at function's first line and continues until a return statement, exception, or the end of the function (which is seen as an implicit return None) is encountered. Once a function returns control to its caller, that's it. Any work done by the function and stored in local variables is lost. A new call to the function creates everything from scratch.

This is all very standard when discussing functions (more generally referred to as subroutines) in computer programming. There are times, though, when it's beneficial to have the ability to create a "function" which, instead of simply returning a single value, is able to yield a series of values. To do so, such a function would need to be able to "save its work," so to speak.

I said, "yield a series of values" because our hypothetical function doesn't "return" in the normal sense. return implies that the function is returning control of execution to the point where the function was called. "Yield," however, implies that the transfer of control is temporary and voluntary, and our function expects to regain it in the future.

In Python, "functions" with these capabilities are called generators, and they're incredibly useful. generators (and the yield statement) were initially introduced to give programmers a more straightforward way to write code responsible for producing a series of values. Previously, creating something like a random number generator required a class or module that both generated values and kept track of state between calls. With the introduction of generators, this became much simpler.

To better understand the problem generators solve, let's take a look at an example. Throughout the example, keep in mind the core problem being solved: generating a series of values.

Note: Outside of Python, all but the simplest generators would be referred to as coroutines. I'll use the latter term later in the post. The important thing to remember is, in Python, everything described here as a coroutine is still a generator. Python formally defines the term generator; coroutine is used in discussion but has no formal definition in the language.

Example: Fun With Prime Numbers
Suppose our boss asks us to write a function that takes a list of ints and returns some Iterable containing the elements which are prime1 numbers.

Remember, an Iterable is just an object capable of returning its members one at a time.

"Simple," we say, and we write the following:

\begin{lstlisting}[language=Python]
def get_primes(input_list):
    result_list = list()
    for element in input_list:
        if is_prime(element):
            result_list.append()

    return result_list
\end{lstlisting}

or better yet...

\begin{lstlisting}[language=Python]
def get_primes(input_list):
    return (element for element in input_list if is_prime(element))

# not germane to the example, but here's a possible implementation of
# is_prime...

def is_prime(number):
    if number > 1:
        if number == 2:
            return True
        if number % 2 == 0:
            return False
        for current in range(3, int(math.sqrt(number) + 1), 2):
            if number % current == 0:
                return False
        return True
    return False
\end{lstlisting}

Either get_primes implementation above fulfills the requirements, so we tell our boss we're done. She reports our function works and is exactly what she wanted.

Dealing With Infinite Sequences
Well, not quite exactly. A few days later, our boss comes back and tells us she's run into a small problem: she wants to use our get_primes function on a very large list of numbers. In fact, the list is so large that merely creating it would consume all of the system's memory. To work around this, she wants to be able to call get_primes with a start value and get all the primes larger than start (perhaps she's solving Project Euler problem 10).

Once we think about this new requirement, it becomes clear that it requires more than a simple change to get_primes. Clearly, we can't return a list of all the prime numbers from start to infinity (operating on infinite sequences, though, has a wide range of useful applications). The chances of solving this problem using a normal function seem bleak.

Before we give up, let's determine the core obstacle preventing us from writing a function that satisfies our boss's new requirements. Thinking about it, we arrive at the following: functions only get one chance to return results, and thus must return all results at once. It seems pointless to make such an obvious statement; "functions just work that way," we think. The real value lies in asking, "but what if they didn't?"

Imagine what we could do if get_primes could simply return the next value instead of all the values at once. It wouldn't need to create a list at all. No list, no memory issues. Since our boss told us she's just iterating over the results, she wouldn't know the difference.

Unfortunately, this doesn't seem possible. Even if we had a magical function that allowed us to iterate from n to infinity, we'd get stuck after returning the first value:

def get_primes(start):
    for element in magical_infinite_range(start):
        if is_prime(element):
            return element
Imagine get_primes is called like so:

def solve_number_10():
    # She *is* working on Project Euler #10, I knew it!
    total = 2
    for next_prime in get_primes(3):
        if next_prime < 2000000:
            total += next_prime
        else:
            print(total)
            return
Clearly, in get_primes, we would immediately hit the case where number = 3 and return at line 4. Instead of return, we need a way to generate a value and, when asked for the next one, pick up where we left off.

Functions, though, can't do this. When they return, they're done for good. Even if we could guarantee a function would be called again, we have no way of saying, "OK, now, instead of starting at the first line like we normally do, start up where we left off at line 4." Functions have a single entry point: the first line.

Enter the Generator
This sort of problem is so common that a new construct was added to Python to solve it: the generator. A generator "generates" values. Creating generators was made as straightforward as possible through the concept of generator functions, introduced simultaneously.

A generator function is defined like a normal function, but whenever it needs to generate a value, it does so with the yield keyword rather than return. If the body of a def contains yield, the function automatically becomes a generator function (even if it also contains a return statement). There's nothing else we need to do to create one.

generator functions create generator iterators. That's the last time you'll see the term generator iterator, though, since they're almost always referred to as "generators". Just remember that a generator is a special type of iterator. To be considered an iterator, generators must define a few methods, one of which is next(). To get the next value from a generator, we use the same built-in function as for iterators: next().

This point bears repeating: to get the next value from a generator, we use the same built-in function as for iterators: next().

(next() takes care of calling the generator's next() method). Since a generator is a type of iterator, it can be used in a for loop.

So whenever next() is called on a generator, the generator is responsible for passing back a value to whomever called next(). It does so by calling yield along with the value to be passed back (e.g. yield 7). The easiest way to remember what yield does is to think of it as return (plus a little magic) for generator functions.**

Again, this bears repeating: yield is just return (plus a little magic) for generator functions.

Here's a simple generator function:

>>> def simple_generator_function():
>>>    yield 1
>>>    yield 2
>>>    yield 3
And here are two simple ways to use it:

>>> for value in simple_generator_function():
>>>     print(value)
1
2
3
>>> our_generator = simple_generator_function()
>>> next(our_generator)
1
>>> next(our_generator)
2
>>> next(our_generator)
3
Magic?
What's the magic part? Glad you asked! When a generator function calls yield, the "state" of the generator function is frozen; the values of all variables are saved and the next line of code to be executed is recorded until next() is called again. Once it is, the generator function simply resumes where it left off. If next() is never called again, the state recorded during the yield call is (eventually) discarded.

Let's rewrite get_primes as a generator function. Notice that we no longer need the magical_infinite_range function. Using a simple while loop, we can create our own infinite sequence:

def get_primes(number):
    while True:
        if is_prime(number):
            yield number
        number += 1
If a generator function calls return or reaches the end its definition, a StopIteration exception is raised. This signals to whoever was calling next() that the generator is exhausted (this is normal iterator behavior). It is also the reason the while True: loop is present in get_primes. If it weren't, the first time next() was called we would check if the number is prime and possibly yield it. If next() were called again, we would uselessly add 1 to number and hit the end of the generator function (causing StopIteration to be raised). Once a generator has been exhausted, calling next() on it will result in an error, so you can only consume all the values of a generator once. The following will not work:

>>> our_generator = simple_generator_function()
>>> for value in our_generator:
>>>     print(value)

>>> # our_generator has been exhausted...
>>> print(next(our_generator))
Traceback (most recent call last):
  File "<ipython-input-13-7e48a609051a>", line 1, in <module>
    next(our_generator)
StopIteration

>>> # however, we can always create a new generator
>>> # by calling the generator function again...

>>> new_generator = simple_generator_function()
>>> print(next(new_generator)) # perfectly valid
1
Thus, the while loop is there to make sure we never reach the end of get_primes. It allows us to generate a value for as long as next() is called on the generator. This is a common idiom when dealing with infinite series (and generators in general).

Visualizing the flow
Let's go back to the code that was calling get_primes: solve_number_10.

def solve_number_10():
    # She *is* working on Project Euler #10, I knew it!
    total = 2
    for next_prime in get_primes(3):
        if next_prime < 2000000:
            total += next_prime
        else:
            print(total)
            return
It's helpful to visualize how the first few elements are created when we call get_primes in solve_number_10's for loop. When the for loop requests the first value from get_primes, we enter get_primes as we would in a normal function.

We enter the while loop on line 3
The if condition holds (3 is prime)
We yield the value 3 and control to solve_number_10.
Then, back in solve_number_10:

The value 3 is passed back to the for loop
The for loop assigns next_prime to this value
next_prime is added to total
The for loop requests the next element from get_primes
This time, though, instead of entering get_primes back at the top, we resume at line 5, where we left off.

def get_primes(number):
    while True:
        if is_prime(number):
            yield number
        number += 1 # <<<<<<<<<<
Most importantly, number still has the same value it did when we called yield (i.e. 3). Remember, yield both passes a value to whoever called next(), and saves the "state" of the generator function. Clearly, then, number is incremented to 4, we hit the top of the while loop, and keep incrementing number until we hit the next prime number (5). Again we yield the value of number to the for loop in solve_number_10. This cycle continues until the for loop stops (at the first prime greater than 2,000,000).

Moar Power
In PEP 342, support was added for passing values into generators. PEP 342 gave generators the power to yield a value (as before), receive a value, or both yield a value and receive a (possibly different) value in a single statement.

To illustrate how values are sent to a generator, let's return to our prime number example. This time, instead of simply printing every prime number greater than number, we'll find the smallest prime number greater than successive powers of a number (i.e. for 10, we want the smallest prime greater than 10, then 100, then 1000, etc.). We start in the same way as get_primes:

def print_successive_primes(iterations, base=10):
    # like normal functions, a generator function
    # can be assigned to a variable

    prime_generator = get_primes(base)
    # missing code...
    for power in range(iterations):
        # missing code...

def get_primes(number):
    while True:
        if is_prime(number):
        # ... what goes here?
The next line of get_primes takes a bit of explanation. While yield number would yield the value of number, a statement of the form other = yield foo means, "yield foo and, when a value is sent to me, set other to that value." You can "send" values to a generator using the generator's send method.

def get_primes(number):
    while True:
        if is_prime(number):
            number = yield number
        number += 1
In this way, we can set number to a different value each time the generator yields. We can now fill in the missing code in print_successive_primes:

def print_successive_primes(iterations, base=10):
    prime_generator = get_primes(base)
    prime_generator.send(None)
    for power in range(iterations):
        print(prime_generator.send(base ** power))
Two things to note here: First, we're printing the result of generator.send, which is possible because send both sends a value to the generator and returns the value yielded by the generator (mirroring how yield works from within the generator function).

Second, notice the prime_generator.send(None) line. When you're using send to "start" a generator (that is, execute the code from the first line of the generator function up to the first yield statement), you must send None. This makes sense, since by definition the generator hasn't gotten to the first yield statement yet, so if we sent a real value there would be nothing to "receive" it. Once the generator is started, we can send values as we do above.

Round-up
In the second half of this series, we'll discuss the various ways in which generators have been enhanced and the power they gained as a result. yield has become one of the most powerful keywords in Python. Now that we've built a solid understanding of how yield works, we have the knowledge necessary to understand some of the more "mind-bending" things that yield can be used for.

Believe it or not, we've barely scratched the surface of the power of yield. For example, while send does work as described above, it's almost never used when generating simple sequences like our example. Below, I've pasted a small demonstration of one common way send is used. I'll not say any more about it as figuring out how and why it works will be a good warm-up for part two.

\begin{lstlisting}[language=Python]
import random

def get_data():
    """Return 3 random integers between 0 and 9"""
    return random.sample(range(10), 3)

def consume():
    """Displays a running average across lists of integers sent to it"""
    running_sum = 0
    data_items_seen = 0

    while True:
        data = yield
        data_items_seen += len(data)
        running_sum += sum(data)
        print('The running average is {}'.format(running_sum / float(data_items_seen)))

def produce(consumer):
    """Produces a set of values and forwards them to the pre-defined consumer
    function"""
    while True:
        data = get_data()
        print('Produced {}'.format(data))
        consumer.send(data)
        yield

if __name__ == '__main__':
    consumer = consume()
    consumer.send(None)
    producer = produce(consumer)

    for _ in range(10):
        print('Producing...')
        next(producer)
\end{lstlisting}

Remember...
There are a few key ideas I hope you take away from this discussion:

generators are used to generate a series of values
yield is like the return of generator functions
The only other thing yield does is save the "state" of a generator function
A generator is just a special type of iterator
Like iterators, we can get the next value from a generator using next()
for gets values by calling next() implicitly

\section{Cấu trúc dữ liệu}

Number
Basic Operation

\begin{lstlisting}[language=Python]
1
1.2
1 + 2
abs(-5)
\end{lstlisting}


\section{Quản lý gói với Anaconda}

\noindent Cài đặt package tại một branch của một project trên github

\begin{lstlisting}[language=Python]
$ pip install git+https://github.com/tangentlabs/django-oscar-paypal.git@issue/34/oscar-0.6#egg=django-oscar-paypal
\end{lstlisting}

\noindent Trích xuất danh sách package

\begin{lstlisting}
$ pip freeze > requirements.txt
\end{lstlisting}

\noindent \textbf{Chạy ipython trong environment anaconda}

\noindent Chạy đống lệnh này

\begin{lstlisting}[language=bash]
  conda install nb_conda
  source activate my_env
  python -m IPython kernelspec install-self --user
  ipython notebook
\end{lstlisting}

\noindent \textbf{Interactive programming với ipython}

\noindent Trích xuất ipython ra slide (không hiểu sao default `--to slides` không work nữa, lại phải thêm tham số `--reveal-prefix` [^1]

\begin{lstlisting}[language=bash]
jupyter nbconvert "file.ipynb"
  --to slides
  --reveal-prefix "https://cdnjs.cloudflare.com/ajax/libs/reveal.js/3.1.0"
\end{lstlisting}

**Tham khảo thêm**

* https://stackoverflow.com/questions/37085665/in-which-conda-environment-is-jupyter-executing
* https://github.com/jupyter/notebook/issues/541#issuecomment-146387578
* https://stackoverflow.com/a/20101940/772391

\noindent \textbf{python 3.4 hay 3.5}

Có lẽ 3.5 là lựa chọn tốt hơn (phải có của tensorflow, pytorch, hỗ trợ mock)

### Quản lý môi trường phát triển với conda

Chạy lệnh `remove` để xóa một môi trường

\begin{lstlisting}[language=bash]
conda remove --name flowers --all
\end{lstlisting}

\section{Test với python}

\textbf{Sử dụng những loại test nào?}

Hiện tại mình đang viết unittest với default class của python là Unittest. Thực ra toàn sử dụng `assertEqual` là chính!

Ngoài ra mình cũng đang sử dụng tox để chạy test trên nhiều phiên bản python (python 2.7, 3.5). Điều hay của tox là mình có thể thiết kế toàn bộ cài đặt project và các dependencies package trong file `tox.ini`

\textbf{Chạy test trên nhiều phiên bản python với tox}

Pycharm hỗ trợ debug tox (quá tuyệt!), chỉ với thao tác đơn giản là nhấn chuột phải vào file tox.ini của project.

\section{Xây dựng docs với readthedocs và sphinx}

\noindent \textbf{20/12/2017}: Tự nhiên hôm nay tất cả các class có khai báo kế thừa ở project languageflow không thể index được. Vãi thật. Làm thằng đệ không biết đâu mà build model.

Thử build lại chục lần, thay đổi file conf.py và package\_reference.rst chán chê không được. Giả thiết đầu tiên là do hai nguyên nhân (1) docstring ghi sai, (2) nội dung trong package\_reference.rst bị sai. Sửa chán chê cũng vẫn thể, thử checkout các commit của git. Không hoạt động!

Mất khoảng vài tiếng mới để ý thằng readthedocs có phần log cho từng build một. Lần mò vào build gần nhất và build (mình nhớ là) thành công cách đây 2 ngày

\noindent Log build gần nhất

\begin{lstlisting}
Running Sphinx v1.6.5
making output directory...
loading translations [en]... done
loading intersphinx inventory from https://docs.python.org/objects.inv...
intersphinx inventory has moved: https://docs.python.org/objects.inv -> https://docs.python.org/2/objects.inv
loading intersphinx inventory from http://docs.scipy.org/doc/numpy/objects.inv...
intersphinx inventory has moved: http://docs.scipy.org/doc/numpy/objects.inv -> https://docs.scipy.org/doc/numpy/objects.inv
building [mo]: targets for 0 po files that are out of date
building [readthedocsdirhtml]: targets for 8 source files that are out of date
updating environment: 8 added, 0 changed, 0 removed
reading sources... [ 12%] authors
reading sources... [ 25%] contributing
reading sources... [ 37%] history
reading sources... [ 50%] index
reading sources... [ 62%] installation
reading sources... [ 75%] package_reference
reading sources... [ 87%] readme
reading sources... [100%] usage

looking for now-outdated files... none found
pickling environment... done
checking consistency... done
preparing documents... done
writing output... [ 12%] authors
writing output... [ 25%] contributing
writing output... [ 37%] history
writing output... [ 50%] index
writing output... [ 62%] installation
writing output... [ 75%] package_reference
writing output... [ 87%] readme
writing output... [100%] usage
\end{lstlisting}

Log build hồi trước

\begin{lstlisting}[language=bash]
Running Sphinx v1.5.6
making output directory...
loading translations [en]... done
loading intersphinx inventory from https://docs.python.org/objects.inv...
intersphinx inventory has moved: https://docs.python.org/objects.inv -> https://docs.python.org/2/objects.inv
loading intersphinx inventory from http://docs.scipy.org/doc/numpy/objects.inv...
intersphinx inventory has moved: http://docs.scipy.org/doc/numpy/objects.inv -> https://docs.scipy.org/doc/numpy/objects.inv
building [mo]: targets for 0 po files that are out of date
building [readthedocs]: targets for 8 source files that are out of date
updating environment: 8 added, 0 changed, 0 removed
reading sources... [ 12%] authors
reading sources... [ 25%] contributing
reading sources... [ 37%] history
reading sources... [ 50%] index
reading sources... [ 62%] installation
reading sources... [ 75%] package_reference
reading sources... [ 87%] readme
reading sources... [100%] usage

/home/docs/checkouts/readthedocs.org/user_builds/languageflow/checkouts/develop/languageflow/transformer/count.py:docstring of languageflow.transformer.count.CountVectorizer:106: WARNING: Definition list ends without a blank line; unexpected unindent.
/home/docs/checkouts/readthedocs.org/user_builds/languageflow/checkouts/develop/languageflow/transformer/tfidf.py:docstring of languageflow.transformer.tfidf.TfidfVectorizer:113: WARNING: Definition list ends without a blank line; unexpected unindent.
../README.rst:7: WARNING: nonlocal image URI found: https://img.shields.io/badge/latest-1.1.6-brightgreen.svg
looking for now-outdated files... none found
pickling environment... done
checking consistency... done
preparing documents... done
writing output... [ 12%] authors
writing output... [ 25%] contributing
writing output... [ 37%] history
writing output... [ 50%] index
writing output... [ 62%] installation
writing output... [ 75%] package_reference
writing output... [ 87%] readme
writing output... [100%] usage
\end{lstlisting}

Đập vào mắt là sự khác biệt giữa documentation type

Lỗi

\begin{lstlisting}[language=bash]
building [readthedocsdirhtml]: targets for 8 source files that are out of date
\end{lstlisting}

Chạy

\begin{lstlisting}[language=bash]
building [readthedocs]: targets for 8 source files that are out of date
\end{lstlisting}

Hí ha hí hửng. Chắc trong cơn bất loạn sửa lại settings đây mà. Sửa lại nó trong phần Settings (Admin &gt; Settings &gt; Documentation type)

![](https://magizbox.files.wordpress.com/2017/10/screenshot-from-2017-12-20-09-54-23.png)

Khi chạy nó đã cho ra log đúng

\begin{lstlisting}[language=bash]
building [readthedocsdirhtml]: targets for 8 source files that are out of date
\end{lstlisting}

Nhưng vẫn lỗi. Vãi!!! Sau khoảng 20 phút tiếp tục bấn loạn, chửi bới readthedocs các kiểu. Thì để ý dòng này

Lỗi

\begin{lstlisting}[language=bash]
Running Sphinx v1.6.5
\end{lstlisting}


Chạy

\begin{lstlisting}[language=bash]
Running Sphinx v1.5.6
\end{lstlisting}

Ngay dòng đầu tiên mà không để ý, ngu thật. Aha, Hóa ra là thằng readthedocs nó tự động update phiên bản sphinx lên 1.6.5. Mình là mình chúa ghét thay đổi phiên bản (code đã mệt rồi, lại còn phải tương thích với nhiều phiên bản nữa thì ăn c** à). Đầu tiên search với Pycharm thấy dòng này trong `conf.py`

\begin{lstlisting}[language=bash]
# If your documentation needs a minimal Sphinx version, state it here.
# needs_sphinx = '1.0'
\end{lstlisting}

Đổi thành

\begin{lstlisting}[language=bash]
# If your documentation needs a minimal Sphinx version, state it here.
needs_sphinx = '1.5.6'
\end{lstlisting}

Vẫn vậy (holy sh*t). Thử sâu một tẹo (thực sự là rất nhiều tẹo). Thấy cái này trong trang Settings

![](https://magizbox.files.wordpress.com/2017/10/screenshot-from-2017-12-20-10-01-39.png)

Ờ há. Thằng đần này cho phép trỏ đường dẫn tới một file trong project để cấu hình dependency. Haha.
Tạo thêm một file `requirements` trong thư mục `docs` với nội dung

\begin{lstlisting}
sphinx==1.5.6
\end{lstlisting}


Sau đó cấu hình nó trên giao diện web của readthedocs

![](https://magizbox.files.wordpress.com/2017/10/screenshot-from-2017-12-20-10-04-49.png)

Build thử. Build thử thôi. Cảm giác đúng lắm rồi đấy. Và... nó chạy. Ahihi

![](https://magizbox.files.wordpress.com/2017/10/screenshot-from-2017-12-20-10-06-32.png)

\textbf{Kinh nghiệm}

* Khi không biết làm gì, hãy làm 3 việc. Đọc LOG. Phân tích LOG. Và cố gắng để LOG thay đổi theo ý mình.

PS: Trong quá trình này, cũng không thèm build thằng PDF với Epub nữa. Tiết kiệm được bao nhiêu thời gian.

\section{Pycharm Pycharm}

01/2018: Pycharm là trình duyệt ưa thích của mình trong suốt 3 năm vừa rồi.

Hôm nay tự nhiên lại gặp lỗi không tự nhận unittest, không resolve được package import bởi relative path. Vụ không tự nhận unittest sửa bằng cách xóa file .idea là xong. Còn vụ không resolve được package import bởi relative path thì vẫn chịu rồi. Nhìn code cứ đỏ lòm khó chịu thật.

\section{Vì sao lại code python?}

\textbf{01/11/2017}
Thích python vì nó quá đơn giản (và quá đẹp).

[^1]: https://github.com/jupyter/nbconvert/issues/91#issuecomment-283736634
%\chapter{C++}


C++ is a general-purpose programming language. It has imperative, object-oriented and generic programming features, while also providing facilities for low-level memory manipulation. It was designed with a bias toward system programming and embedded, resource-constrained and large systems, with performance, efficiency and flexibility of use as its design highlights. C++ has also been found useful in many other contexts, with key strengths being software infrastructure and resource-constrained applications, including desktop applications, servers (e.g. e-commerce, web search or SQL servers), and performance-critical applications (e.g. telephone switches or space probes). C++ is a compiled language, with implementations of it available on many platforms and provided by various organizations, including the Free Software Foundation (FSF's GCC), LLVM, Microsoft, Intel and IBM.

View online \href{http://magizbox.com/training/cpp/site/}{http://magizbox.com/training/cpp/site/}

\section{Get Started}

What do I need to start with CLion?
In general to develop in C/C++ with CLion you need:

CMake, 2.8.11+ (Check JetBrains guide for updates)
GCC/G++/Clang (Linux) or
MinGW 3. or MinGW — w64 3.-4. or Cygwin 1.7.32 (minimum required) up to 2.0. (Windows)
Downloading and Installing CMake
Downloading and installing CMake is pretty simple, just go to the website, download and install by following the recommended guide there or the on Desktop Wizard.

Download and install file cmake-3.9.0-win64-x65.msi
> cmake
Usage

  cmake [options] <path-to-source>
  cmake [options] <path-to-existing-build>

Specify a source directory to (re-)generate a build system for it in the
current working directory.  Specify an existing build directory to
re-generate its build system.

Run 'cmake --help' for more information.
Downloading and Getting Cygwin
Cygwin is a large collection of GNU and Open Source tools which provide functionality similar to a Linux distribution on Windows

Download file setup-x86_64.exe from the website https://cygwin.com/install.html

Install setup-x86_64.exe file



This is the root directory where Cygwin will be located, usually the recommended C:\ works



Choose where to install LOCAL DOWNLOAD PACKAGES: This is not the same as root directory, but rather where packages (ie. extra C libraries and tools) you download using Cygwin will be located



Follow the recommended instructions until you get to packages screen:



Once you get to the packages screen, this is where you customize what libraries or tools you will install. From here on I followed the above guide but here’s the gist:

From this window, choose the Cygwin applications to install. For our purposes, you will select certain GNU C/C++ packages.

Click the + sign next to the Devel category to expand it.

You will see a long list of possible packages that can be downloaded. Scroll the list to see more packages.

Pick each of the following packages by clicking its corresponding “Skip” marker.

gcc-core: C compiler subpackage
gcc-g++: C++ subpackage
libgcc1: C runtime library
gdb: The GNU Debugger
make: The GNU version of the ‘make’ utility
libmpfr4 : A library for multiple-precision floating-point arithmetic with exact rounding
Download and install CLion
Download file CLion-2017.2.exe from website https://www.jetbrains.com/clion/download/#section=windows



Config environment File > Settings... > Build, Execution, Deployment

Choose Cygwin home: C:\cygwin64
Choose CMake executable: Bundled CMake 3.8.2
Run your first C++ program with CLion

\section{Basic Syntax}

C/C++
Hello World
#include <iostream>
using namespace std;

int main() {
    cout << "hello world";
}
Convention
Naming
variable_name_like_this
class_data_memeber_name_like_this_
kConstantNamesLikeThis
ClassNameLikeThis
filenamelikethis_myusefulclass_test.cc
Comment
Class Comment
// Iterates over the contents of a GargantuanTable.
// Example:
//    GargantuanTableIterator* iter = table->NewIterator();
//    for (iter->Seek("foo"); !iter->done(); iter->Next()) {
//      process(iter->key(), iter->value());
//    }
//    delete iter;
class GargantuanTableIterator {
  ...
};
Todo Comment
// TODO(kl@gmail.com): Use a "*" here for concatenation operator.
// TODO(Zeke) change this to use relations.

\section{Cấu trúc dữ liệu}

Data Structure
Number
C++ offer the programmer a rich assortment of built-in as well as user defined data types. Following table lists down seven basic C++ data types:

Boolean - bool
Character - char
Integer - int
Floating point - float
Double floating point - double
Valueless - void
Wide character - wchar_t
Several of the basic types can be modified using one or more of these type modifiers: signed, unsigned, short, long

Following is the example, which will produce correct size of various data types on your computer.

#include <iostream>
using namespace std;

int main() {
   cout << "Size of char : " << sizeof(char) << endl;
   cout << "Size of int : " << sizeof(int) << endl;
   cout << "Size of short int : " << sizeof(short int) << endl;
   cout << "Size of long int : " << sizeof(long int) << endl;
   cout << "Size of float : " << sizeof(float) << endl;
   cout << "Size of double : " << sizeof(double) << endl;
   cout << "Size of wchar_t : " << sizeof(wchar_t) << endl;
   return 0;
}
String
String Basic

#include <iostream>
#include <string>
using namespace std ;

// assign a string
string s1 = "www.java2s.com\n";
cout << s1;

// input a string
string s2;
cin >> s2;

// concatenate two strings
string s_c = s1 + s2;

// compare strings
s1 == s2;
Collection
Pointer
A pointer is a variable whose value is the address of another variable. Like any variable or constant, you must declare a pointer before you can work with it.

The general form of a pointer variable declaration is:

type *variable_name;
// example
int    *ip;    // pointer to an integer
double *dp;    // pointer to a double
float  *fp;    // pointer to a float
char   *ch;    // pointer to character
Pointer Lab



#include <iostream>
using namespace std;

/*
 * Look at these lines
 */
int* a;
a = new int[3];
a[0] = 10;
a[1] = 2;
cout << "Address of pointer a: &a = " << &a << endl;
cout << "Value   of pointer a:  a = " << a << endl << endl;
cout << "Address of a[0]: &a[0] = " << &a[0] << endl;
cout << "Value   of a[0]: a[0]  = " << a[0]  << endl;
cout << "Value   of a[0]: *a    = " << *a    << endl << endl;
cout << "Address of a[1]: &a[1] = " << &a[1] << endl;
cout << "Value   of a[1]: a[1]  = " << a[1]  << endl;
cout << "Value   of a[1]: *(a+1)= " << *(a+1)<< endl << endl;
cout << "Address of a[2]: &a[2] = " << &a[2] << endl;
cout << "Value   of a[2]: a[2]  = " << a[2]  << endl;
cout << "Value   of a[2]: *(a+2)= " << *(a+2)<< endl << endl;
Result:

Address of pointer a: &a = 008FF770
Value   of pointer a:  a = 00C66ED0

Address of a[0]: &a[0] = 00C66ED0
Value   of a[0]: a[0]  = 10
Value   of a[0]: *a    = 10

Address of a[1]: &a[1] = 00C66ED4
Value   of a[1]: a[1]  = 2
Value   of a[1]: *(a+1)= 2

Address of a[2]: &a[2] = 00C66ED8
Value   of a[2]: a[2]  = -842150451
Value   of a[2]: *(a+2)= -842150451
Stack, Queue, Linked List, Array, Deque, List, Map, Set

Datetime
The C++ standard library does not provide a proper date type. C++ inherits the structs and functions for date and time manipulation from C. To access date and time related functions and structures, you would need to include header file in your C++ program.

There are four time-related types: clock_t, time_t, size_t, and tm. The types clock_t, size_t and time_t are capable of representing the system time and date as some sort of integer.

The structure type tm holds the date and time in the form of a C structure having the following elements:

struct tm {
   int tm_sec;   // seconds of minutes from 0 to 61
   int tm_min;   // minutes of hour from 0 to 59
   int tm_hour;  // hours of day from 0 to 24
   int tm_mday;  // day of month from 1 to 31
   int tm_mon;   // month of year from 0 to 11
   int tm_year;  // year since 1900
   int tm_wday;  // days since sunday
   int tm_yday;  // days since January 1st
   int tm_isdst; // hours of daylight savings time
}
Current date and time

Consider you want to retrieve the current system date and time, either as a local time or as a Coordinated Universal Time (UTC). Following is the example to achieve the same:

#include <iostream>
#include <ctime>

using namespace std;

int main( ) {
   // current date/time based on current system
   time_t now = time(0);

   // convert now to string form
   char* dt = ctime(&now);

   cout << "The local date and time is: " << dt << endl;

   // convert now to tm struct for UTC
   tm *gmtm = gmtime(&now);
   dt = asctime(gmtm);
   cout << "The UTC date and time is:"<< dt << endl;
}
When the above code is compiled and executed, it produces the following result:

The local date and time is: Sat Jan  8 20:07:41 2011

The UTC date and time is:Sun Jan  9 03:07:41 2011

\section{Lập trình hướng đối tượng}

Object Oriented Programming
Classes and Objects
#include <iostream>
using namespace std;

class Pacman {

    private:
      int x;
      int y;
    public:
    Pacman(int x, int y);
    void show();
};

Pacman::Pacman(int x, int y){
    this->x = x;
    this->y = y;
}

void Pacman::show(){
    std::cout << "(" << this->x << ", " << this->y << ")";
}

int main() {
    // your code goes here
    Pacman p = Pacman(2, 3);
    p.show();
    return 0;
}
Template
Function Template

#include <iostream>
#include <string>

using namespace std;

template <typename T>

T Max(T a, T b)
{
    return a < b ? b : a;
}

int main()
{

    int i = 39;
    int j = 20;
    cout << Max(i, j) << endl;

    double f1 = 13.5;
    double f2 = 20.7;
    cout << Max(f1, f2) << endl;

    string s1 = "Hello";
    string s2 = "World";
    cout << Max(s1, s2) << endl;

    double n1 = 20.3;
    float n2 = 20.4;
    // it will show an error
    // Error: no instance of function template "Max" matches the argument list
    //        arguments types are: (double, float)
    cout << Max(n1, n2) << endl;
    return 0;
}

\section{Cơ sở dữ liệu}


Database
Sqlite with Visual Studio 2013
Step 1: Create new project 1.1 Create a new C++ Win32 Console application.

Step 2: Download Sqlite DLL

2.1. Download the native SQLite DLL from: http://sqlite.org/sqlite-dll-win32-x86-3070400.zip 2.2. Unzip the DLL and DEF files and place the contents in your project’s source folder (an easy way to find this is to right click on the tab and click the “Open Containing Folder” menu item.

Step 3: Build LIB file

3.1. Open a “Developer Command Prompt” and navigate to your source folder. (If you can't find this tool, follow this post in stackoverflow Where is Developer Command Prompt for VS2013? to create it) 3.2. Create an import library using the following command line: LIB /DEF:sqlite3.def

Step 4: Add Dependencies

4.1. Add the library (i.e. sqlite3.lib) to your Project Properties -> Configuration Properties -> Linker -> Input -> Additional Dependencies. 4.2. Download http://sqlite.org/sqlite-amalgamation-3070400.zip 4.3. Unzip the sqlite3.h header file and place into your source directory. 4.4. Include the the sqlite3.h header file in your source code. 4.5. You will need to include the sqlite3.dll in the same directory as your program (or in a System Folder).

Step 5: Run test code

#include "stdafx.h"
#include <ios>
#include <iostream>
#include "sqlite3.h"

using namespace std;

int _tmain(int argc, _TCHAR* argv[])
{
   int rc;
   char *error;

   // Open Database
   cout << "Opening MyDb.db ..." << endl;
   sqlite3 *db;
   rc = sqlite3_open("MyDb.db", &db);
   if (rc)
   {
      cerr << "Error opening SQLite3 database: " << sqlite3_errmsg(db) << endl << endl;
      sqlite3_close(db);
      return 1;
   }
   else
   {
      cout << "Opened MyDb.db." << endl << endl;
   }

   // Execute SQL
   cout << "Creating MyTable ..." << endl;
   const char *sqlCreateTable = "CREATE TABLE MyTable (id INTEGER PRIMARY KEY, value STRING);";
   rc = sqlite3_exec(db, sqlCreateTable, NULL, NULL, &error);
   if (rc)
   {
      cerr << "Error executing SQLite3 statement: " << sqlite3_errmsg(db) << endl << endl;
      sqlite3_free(error);
   }
   else
   {
      cout << "Created MyTable." << endl << endl;
   }

   // Execute SQL
   cout << "Inserting a value into MyTable ..." << endl;
   const char *sqlInsert = "INSERT INTO MyTable VALUES(NULL, 'A Value');";
   rc = sqlite3_exec(db, sqlInsert, NULL, NULL, &error);
   if (rc)
   {
      cerr << "Error executing SQLite3 statement: " << sqlite3_errmsg(db) << endl << endl;
      sqlite3_free(error);
   }
   else
   {
      cout << "Inserted a value into MyTable." << endl << endl;
   }

   // Display MyTable
   cout << "Retrieving values in MyTable ..." << endl;
   const char *sqlSelect = "SELECT * FROM MyTable;";
   char **results = NULL;
   int rows, columns;
   sqlite3_get_table(db, sqlSelect, &results, &rows, &columns, &error);
   if (rc)
   {
      cerr << "Error executing SQLite3 query: " << sqlite3_errmsg(db) << endl << endl;
      sqlite3_free(error);
   }
   else
   {
      // Display Table
      for (int rowCtr = 0; rowCtr <= rows; ++rowCtr)
      {
         for (int colCtr = 0; colCtr < columns; ++colCtr)
         {
            // Determine Cell Position
            int cellPosition = (rowCtr * columns) + colCtr;

            // Display Cell Value
            cout.width(12);
            cout.setf(ios::left);
            cout << results[cellPosition] << " ";
         }

         // End Line
         cout << endl;

         // Display Separator For Header
         if (0 == rowCtr)
         {
            for (int colCtr = 0; colCtr < columns; ++colCtr)
            {
               cout.width(12);
               cout.setf(ios::left);
               cout << "~~~~~~~~~~~~ ";
            }
            cout << endl;
         }
      }
   }
   sqlite3_free_table(results);

   // Close Database
   cout << "Closing MyDb.db ..." << endl;
   sqlite3_close(db);
   cout << "Closed MyDb.db" << endl << endl;

   // Wait For User To Close Program
   cout << "Please press any key to exit the program ..." << endl;
   cin.get();

   return 0;
}

\section{Testing}

Create Unit Test in Visual Studio 2013
Step 1. Create TDDLab Solution
1.1 Open Visual Studio 2013

1.2 File ->  New Project... ->

Click Visual C++ -> Win32

Choose Win32 Console Application

Fill to Name input text: TDDLab

Click OK -> Next

1.3 In project settings, remove options:

Precompiled Header
Securirty Develoment Lifecyde(SQL) check
1.4 Click Finish

Step 2. Create Counter Class
2.1 Right-click TDDLab -> Add -> Class...

2.2 Choose Visual C++ -> C++ Class -> Add

2.3 Fill in Class name box Counter -> Finish

2.4 In Counter.h file, add this below function

int add(int a, int b);
2.5 In Counter.cpp, add this below function

int Counter::add(int a, int b) {
  return a+b;
}
Your Counter class should look like this



Step 3. Create TDDLabTest Project
3.1 Right-click Solution 'TDDLab' -> Add -> New Project...

3.2 Choose Visual C++ -> Test

3.3 Choose Native Unit Test Project

3.4 Fill to Name input text: TDDLabTest

Step 4. Write unit test
4.1 In unittest1.cpp, add header of Counter class

#include "../TDDLab/Counter.h"
4.2 In TEST_METHOD function

{
  Counter counter;
  Assert::AreEqual(2, counter.add(1, 1));
}
4.3 Click TEST in menu bar -> Run -> `All Test (Ctrl + R, A)

Step 5. Fix error LNK 2019: unresolved external symbol
5.1 Change Configuration Type of TDDLab project

Right click  TDDLab project -> Properties
General -> Configuration Type -> Static library (.lib) -> OK
5.2 Add Reference to TDDLabTest project

Right click TDDLabTest solution -> Properties -> Common Properties -> Add New Reference
Choose TDDLab -> OK -> OK
Step 6. Run Tests
Click TEST in menu bar -> Run -> `All Test (Ctrl + R, A)

Test should be passed.

\section{IDE & Debugging}

Visual Studio 2013
Install Extension

VsVim

googletest guide

Folder Structure with VS 2013

solution
│   README.md
│
|───project1
|   │   file011.txt
|   │   file012.txt
|   │
|───project2
|   │   file011.txt
|   │   file012.txt
|   │
Auto Format

Ctrl + K, Ctrl + D
Git in Visual Studio

https://git-scm.com/book/en/v2/Git-in-Other-Environments-Git-in-Visual-Studio

Online IDE
codechef ide


%\chapter{Javascript}

View online \href{http://magizbox.com/training/java/site/}{http://magizbox.com/training/java/site/}

What is Javascript?
JavaScript is a high-level, dynamic, untyped, and interpreted programming language. It has been standardized in the ECMAScript language specification. Alongside HTML and CSS, it is one of the three core technologies of World Wide Web content production; the majority of websites employ it and it is supported by all modern Web browsers without plug-ins. JavaScript is prototype-based with first-class functions, making it a multi-paradigm language, supporting object-oriented, imperative, and functional programming styles. It has an API for working with text, arrays, dates and regular expressions, but does not include any I/O, such as networking, storage, or graphics facilities, relying for these upon the host environment in which it is embedded.


\section{Installation}

Google Chrome
Pycharm

\section{IDE}

Google Chrome Developer Tools

The Chrome Developer Tools (DevTools for short), are a set of web authoring and debugging tools built into Google Chrome. The DevTools provide web developers deep access into the internals of the browser and their web application. Use the DevTools to efficiently track down layout issues, set JavaScript breakpoints, and get insights for code optimization.

\section{Basic Syntax}

1. Code Formatting
Indent with 2 spaces

// Object initializer.
var inset = {
  top: 10,
  right: 20,
  bottom: 15,
  left: 12
};

// Array initializer.
this.rows_ = [
  '"Slartibartfast" <fjordmaster@magrathea.com>',
  '"Zaphod Beeblebrox" <theprez@universe.gov>',
  '"Ford Prefect" <ford@theguide.com>',
  '"Arthur Dent" <has.no.tea@gmail.com>',
  '"Marvin the Paranoid Android" <marv@googlemail.com>',
  'the.mice@magrathea.com'
];

// Used in a method call.
goog.dom.createDom(goog.dom.TagName.DIV, {
  id: 'foo',
  className: 'some-css-class',
  style: 'display:none'
}, 'Hello, world!');
2. Naming
functionNamesLikeThis
variableNamesLikeThis
ClassNamesLikeThis
EnumNamesLikeThis
methodNamesLikeThis
CONSTANT_VALUES_LIKE_THIS
foo.namespaceNamesLikeThis.bar
filenameslikethis.js.
3. Comment
Use JSDoc

3.1 Class Comment
/**
 * Class making something fun and easy.
 * @param {string} arg1 An argument that makes this more interesting.
 * @param {Array.<number>} arg2 List of numbers to be processed.
 * @constructor
 * @extends {goog.Disposable}
 */
project.MyClass = function(arg1, arg2) {
  // ...
};
goog.inherits(project.MyClass, goog.Disposable);
3.2 Method Comment
/**
 * Operates on an instance of MyClass and returns something.
 * @param {project.MyClass} obj Instance of MyClass which leads to a long
 *     comment that needs to be wrapped to two lines.
 * @return {boolean} Whether something occurred.
 */
function PR_someMethod(obj) {
  // ...
}
4. Expression and Statements
Expression
A fragment of code that produces a value is called an Expression

22
"this is an epression"
(5 > 6) ? false : true
Statements
The Simplest kind of stagement is an expression with a semi colon

!false;
5 + 6;
5. Loop and iteration
while
var number = 0;
while (number <= 12) {
  console.log(number);
  number = number + 2;
}
do..while
do {
  var yourName = prompt("Who are you?");
} while (!yourName);
console.log(yourName);
for
for (var i = 0; i < 10; i++) {
  console.log(i);
}
6. Function
6.1 Defining a Function
var square = function(x) {
  return x * x;
};
square(5);
6.2 Scope
Scope is the area where contains all variable or function are living.
Scope has some rules:
Child Scope can access all variable and function in parent Scope. (E.g: Local Scope can access Global Scope)
function saveName(firstName) {
    var temp = "temp";
    function capitalizeName() {
        temp = temp + " here";
        return firstName.toUpperCase();
    }
    var capitalized = capitalizeName();
    return capitalized;
}
alert(saveName("Robert"));
But parent Scope can access variable and function inside children scope (E.g: Global Scope cannot acces to local Scope)
function talkDirty () {
    var saying = "Oh, you little VB lover, you";
    return alert(saying);
}
alert(saying); //->Error
6.3 Call Stack
The storage where computer stores context is called CALL STACK.

// CALL STACK
function greet(who) {
    console.log("Hello " + who);
    ask("How are you?");
    console.log("I'm fine");
};

function ask(question) {
    console.log("well, " + question);
};

greet("Harry");
console.log("Bye");
Out of Call Stack

function chicken() {
    return egg();
}

function egg() {
    return chicken();
}
console.log(chicken() + " came first");
6.4. Optional Argument
We can pass too many or too few arguments to the function without any SyntaxError.
If we pass too much arguments, the extra ones are ignored
If we pass to few arguments, the missing ones get value undefined
function power(base, exponent) {
    if (exponent == undefined) {
        exponent = 2;
    }
    var result = 1;
    for (var count = 0; count < exponent; count++) {
        result = result * base;
    }
    return result;
}
console.log(power(4));
console.log(power(4,3));
upside: flexible
downside: hard to control the error

6.5 Closure
Look at this example:

function sayHello(name){
    var text = 'Hello' + name;
    var say = function(){
        console.log(text);
    }
    return say;
}
var say2 = sayHello("ahaha");
say2();
if in C program, does say2() work?
The answer is nope! Because in C program, when a function returns, the Stack-flame will be destroyed, and all the local variable such as text will undefinded. So, when say2() is called, there is no text anymore, and the error, will be shown!
But, in JavaScript, This code works!! Because, it provides for us an Object called Closure! Closure is borned when we define a function in another function, it keep all the live local variable. So, when say2() is called, the closure will give all the value of local variable outside it, and text will be identity.!

var globalVariable = 10;
function func(){
    var name = "xxx";
    function getName(){
        return name;
    }
    function speak(){
        var sound = "alo";
        function scream(){
            console.log(globalVariable);
            console.log(name);
            return "aaaaaaaaaa!";
        }
        function talk(){
            var voice = getName() + " speak " + sound;
            console.log(voice);
            return voice;
        }
        scream();
        talk();
    }
    speak();
}
func();
6.6. Recursion
Recursion is function can call itself, as long as it is not overflow

function power(base, exponent){
    if (exponent == 0){
        return 1;
    }
    else{
        return base * power(base, exponent -1);
    }
}
console.log(power(2,3));

function FindSolution(target){
    function Find(start, history){
        if (start == target){
            return history;
        }
        else if (start > target){
            return null;
        }
        else{
            return Find(start + 5, "(" + history + " + 5 ") ||
            Find(start * 3, "(" + history + " * 3)");
        }
    }
    return Find(1, "1");
}
console.log(FindSolution(25));
6.7. Arguments object
The arguments object contains all parameters you pass to a function.

function argumentCounter() {
    console.log("you gave me", arguments.length, "argument.");
}
argumentCounter("Straw man", "Tautology", "Ad hominem");
6.8. Higher-Order Function
###Filter array
var ancestry = JSON.parse(ANCESTRY_FILE);
console.log(ancestry.length);

function filter(array, test) {
    var passed = [];
    for (var i = 0; i < array.length; i++){
        if (test(array[i])){
            passed.push(array[i]);
        }
    }
    return passed;
}
console.log(filter(ancestry, function(person){
    return person.born > 1900 && person.born < 1925;
}));

### TRANSFORMING WITH A MAP
function map(array, transform) {
  var mapped = [];
  for (var i = 0; i < array.length; i++)
    mapped.push(transform(array[i]));
  return mapped;
}

var overNinety = ancestry.filter(function(person) {
  return person.died - person.born > 90;
});
console.log(map(overNinety, function(person) {
  return person.name;
}));


### REDUCE
function reduce(array, combine, start) {
  var current = start;
  for (var i = 0; i < array.length; i++)
    current = combine(current, array[i]);
  return current;
}
console.log(reduce([1, 2, 3, 4], function(a, b) {
  return a + b;
}, 0));
#Problem: using map and reduce, transform [1,2,3,4] to [1,2],[3,4]

var a = [1, 2, 3, 4]
a = _.map(a, function(i){
    if(i % 2 == 0){
        return [[],[i]]
    } else {
        return [[i], []]
    }
});
a = _.reduce(a, function(x, y){
   return [x[0].concat(y[0]), x[1].concat(y[1])]
})


### BINDING FUNCTION
var theSet = ["Carel Haverbeke", "Maria van Brussel",
              "Donald Duck"];
function isInSet(set, person) {
  return set.indexOf(person.name) > -1;
}

console.log(ancestry.filter(function(person) {
  return isInSet(theSet, person);
}));
console.log(ancestry.filter(isInSet.bind(null, theSet)));
What's the cleanest way to write a multiline string in JavaScript? [duplicate] ↩

Google JavaScript Style Guide ↩

\section{Data Structure}

\subsection{Number}

Some example of number: 10, 1.234, 1.99e9, NaN, Infinity, -Infinity

console.log(2.99e9);
console.log(0 /0);
console.log(1 /0);
console.log(-1 /0);
Automatic Conversion

console.log(8 * null); // -> 0
console.log("5" - 1); // -> 4
console.log("5" + 1); //-> 51
console.log(false == 0) //-> true

\subsection{String}

sprintf
In index.html

<script src="cdnjs.cloudflare.com/ajax/libs/sprintf/1.0.3/sprintf.js"/>

In script.js

// arguments
sprintf("%1$s %2$s", "hello", "sprintf")
# hello sprintf

// object
var user = {
    name: "Dolly"
}
sprintf("Hello %(name)s", user)
# Hello Dolly

// array of object
var users = [
    {name: "Dolly"},
    {name: "Molly"}
]
sprintf("Hello %(users[0].name)s and %(users[1].name)s", {users: users})
# Hello Dolly and Molly
Multiline String
str = "\
line 1\
line 2\
line 3";
Regular Expression in JavaScript
This lab is based on Chapter9: EloquentJavaScript

Creating a regular expression
There are 2 ways:

var re1 = new RegExp("abc");
var re2 = /abc/
there are some special characters such as question mark, or plus sign. If you want to use them, you have to use backslash. Like this:

var eighteen = /eighteen\+/;
var question = /question\?/;
Testing for match
Regular Express has a number of method. Simplest is test

console.log(/abc/.test("abcd"));
console.log(/abc/.test("abxde"));
Matching a set of character []: Put a set of characters between 2 square bracket

console.log(/[0123456789]/.test("1245"));
console.log(/[0-9]/.test("1");
console.log(/[0-9]/.test("acd");
console.log(/[0-9]/.test("aaascacas1"));
There are some special character:
\d Any digit character (Like [0-9])

var datetime = /\d\d-\d\d-\d\d\d\d\s\d\d:\d\d/;
console.log(datetime.test("16-06-2016 14:09"));
console.log(dateTime.test("30-jan-2003 15:20"));
\w An alphanumeric character (“word character”)

var word = /\w/;
console.log(word.test("@#@#"));
\s Any whitespace character (space, tab, newline, and similar)

var space = /\d\.\s+abc/;
console.log(space.test("1. abd"));
console.log(space.test("1.     abd"));
console.log(space.test("1.abd"));
\D A character that is not a digit

var notDigit = /\D/;
console.log(notDigit.test("ww"));
console.log(notDigit.test("1a"));
console.log(notDigit.test("1124"));
\W A nonalphanumeric character

var nonAlphanumbericChar = /\W/;
console.log(nonAlphanumbericChar.test("abc12231"));
console.log(nonAlphanumbericChar.test("!@#%{}_"));
\S A nonwhitespace character

var nonWhiteSpace = /\S/;
console.log(nonWhiteSpace.test("abc123"));
console.log(nonWhiteSpace.test("1.  abcd"));
console.log(nonWhiteSpace.test("  "));
"." Any character except for newline

var anyThing = /...\./;
console.log(anyThing.test("abc."));
console.log(anyThing.test("acbacd."));
console.log(anyThing.test("acba"));
"^" Using caret character to match any except the ones

var notBinary = /[^01]/;
console.log(notBinary.test("01101011100"));
console.log(notBinary.test("01021010010"));
Repeating parts of Pattern
The square bracket [] above only match 1 digit. How can regex match more than 1 digit?
"+" Match one or more
"*" Match zero or more

console.log(/\d+/.test(1234));
console.log(/\d+/.test());

console.log(/\d*/.test(1234));
console.log(/\d*/.test())
"?" Question mark test a character exist or not is still oke

var ball = /bal?l/;
console.log(ball.test("ball"));
console.log(ball.test("bal"));
{a,b} the character before exist from a to b times. Check datetime:

var datetime = /\d{1,2}-\d{1,2}-\d{4} \d{1,2}:\d{1,2}/;
console.log(datetime.test("20-12-2015 14:09"));
var checkTimes = /waz{3,5}up/;
console.log(checkTimes.test("wazzzzzup"));
console.log(checkTimes.test("wazzzup"));
console.log(checkTimes.test("wazup"));
Grouping Subexpressions
() using prentheses to make whole group like one character

var cartoonCrying = /boo+(hoo+)+/i; //i to match all Captalize or normal text
console.log(cartoonCrying.test("Boohoooohoohooo"));
console.log(cartoonCrying.test("boohoooohooOOO"));
Matches and group
Test is a simplest method, and it only return true or false.
exec (execute) is anther method in regex. It returns null if no match, and object if match.

var match  = /\d+/.exec("one two 100");
console.log(match);
console.log(match.input);
console.log(match.index);
if in the expression has a group subexpression, then it will return the text contain this subexpress, and the text match this subexpress:

var quotedText = /'([^']*)'/;
console.log(quotedText.exec("she said 'hello'"));
and if the subexpression appears one more times, then the result will be displayed the last match one.

console.log(/bad(ly)?/.exec("bad"));
console.log(/(\d)+/.exec("123"));
The date type
create new Date(). return the current time

var date =  new Date();
console.log(new Date(2009, 11, 9);
console.log(new Date(2009, 11, 9, 23, 59, 61));
<!--TimeStamp-->
console.log(new Date(2009, 11, 9, 23, 59, 61).getTime());
console.log(new Date(1260378001000));
<!--getFullYear, getMonth,...-->
var date = new Date();
console.log(date.getFullYear());
console.log(date.getMonth());
console.log(date.getDate());
console.log(date.getHours());
console.log(date.getMinutes());
console.log(date.getSeconds());
Word and string boundaries
console.log(/cat/.test("concatenate"));
console.log(/cat/.test("con123cat-129e0enate"));
console.log(/\bcat\b/.test("concatenate"));
console.log(/\bcat\b/.test("con123cat-129e0enate"));
Choice patterm
Only one in the list beween the "|" match

var animalCount = /\b\d+ (pig|cow|chicken)s?\b/;
console.log(animalCount.test("15 pigs"));
console.log(animalCount.test("15 pigchickens"));
Replace
Replace will find the first match and replace.if we want to replace all matches, using "g" behind the expresssion

console.log("papa".replace("p", "m"));
console.log("Borobudur".replace(/[ou]/, "a"));
console.log("Borobudur".replace(/[ou]/g, "a"));
Replace can refer back to the matched, and using them

console.log("Le, Khanh\nNguyen, Hung\nDuong, Bach".replace(/([\w]+), ([\w]+)/g, "$1 $2"));
Greed
function stripComments(code) {
  return code.replace(/\/\/.*|\/\*[^]*\*\//g, "");
}
console.log(stripComments("1 + /* 2 */3"));
// → 1 + 3
console.log(stripComments("x = 10;// ten!"));
// → x = 10;
console.log(stripComments("1 /* a */+/* b */ 1"));
// → 1  1
Search method
Search method return the first index if the regular expression match.
And return -1 if not found

console.log("  word".search(/\S/));
// → 2
console.log("    ".search(/\S/));
// → -1
The last index property
In the regular expression has a property is lastIndex. And when this Regex do some method, it will start from the lastIndex. And after doing something, the lastIndex will update to the behind the index of the match exec.

var pattern = /y/g;
pattern.lastIndex = 3; //lastIndex update to 3
var match = pattern.exec("xyzzy"); //lastIndex update to 5
console.log(pattern.lastIndex);

match = pattern.exec("xyzzyxxx"); //Not match any "y" from index 5
console.log(match.index);
console.log(pattern.lastIndex);
Looping Over the Line
Applying the hepoloris of lastIndex, we can using while to do something like this:

var input = "A string with 3 numbers in it... 42 and 88.";
var number = /\b(\d+)\b/g;
var match;
while (match = number.exec(input))
  console.log("Found", match[1], "at", match.index);

\subsection{Collection}

Some useful methods with array
push and pop
var a = [1,2,3,4];
console.log(a.pop(), a);
console.log(a.push(3), a);
shift and unshift
console.log(a.shift(), a);
console.log(a.unshift(1), a);
indexOf and lastIndexOf
var b = [1,2,3,4,2,3,1];
console.log(b.indexOf(1));
console.log(b.lastIndexOf(1));
slice
console.log([0,1,2,3,4].slice(2,4));
console.log([0,1,2,3,4].slice(2));
concat
var a = [1,2,3];
var b = [4,5,6];
a.concat(b);
console.log(a);

\subsection{Datetime}

Current Time
moment().format('MMMM Do YYYY, h:mm:ss a');
Moment.js ↩

\subsection{Boolean}

Boolean has only 2 values: true and false

console.log("Abc" < "Abcd") // -> true
console.log("abc" < "Abcd") // -> false
console.log("123" == "123") // -> true
console.log(NaN == NaN) // -> false
what is the different?

console.log("5" == 5);
console.log("5" === 5);

\subsection{Object}

Object
Define an object
var object = {
  number: 10,
  string: "string",
  array: [1,2,3],
  object: {a: 1, b: 2}
}
Add new property to object
object.newProperty = "value";
object['key'] = 'value';
delete property
delete object.newProperty;
Window object (global object)
The Global scope is stored in an object which called window

function test(){
    var local = 10;
    console.log("local" in window);
    console.log(window.local);
}
test();
var global = 10;
console.log("global" in window);
console.log(window.global);

\section{OOP}


1. Classes and Objects
Constructor
function Ball(position){
    this.position = position;
    this.display = function(){
        console.log(this.position[0], ", ", this.position[1]);
    }
}

ball = new Ball([2, 3]);
ball.display();
2. Inheritance
Person = function (name, birthday, job) {
  this.name = name;
  this.birthday = birthday;
  this.job = job;
};

Person.prototype.display = function () {
  console.log(this.name, "\n");
  console.log(this.birthday, "\n");
  console.log(this.job, "\n");
};

Politician = function (name, birthday) {
  Person.call(this, name, birthday, "Politician");
};
Politician.prototype = Object.create(Person.prototype);
Politician.prototype.constructor = Politician;

var person1 = new Person("Barack Obama", "04/08/1961", "Politician");
var person2 = new Politician("David Cameron", "09/10/1966");
person1.display();
person2.display();

Object-Oriented Programming
var rabbit = {};
rabbit.speak = function(line){
console.log("The rabit says:'" + line + "'");
 };
rabbit.speak("I'm alive");

function speak(line){
   console.log("The "+ this.type + " rabbit says '" + line + "'");
 }

var whiteRabbit = {type: "white", speak: speak};
var fatRabbit = {type: "fat", speak: speak};
whiteRabbit.speak("Oh my ears and whiskers, " + "how late it's getting!");
fatRabbit.speak("I could sure use a carrot right now");

// Prototype
// Prototype is another object that is used as a fallback source of properties
// When object request a property that it does not have, its prototype will be searched for the property
var empty = {};
console.log(empty.toString);
console.log(empty.toString);

// Get prototype of an object 2 ways:
console.log(Object.getPrototypeOf({}) == Object.prototype);
console.log(Object.getPrototypeOf(Object.prototype));

// Using Object.create to create an object with an specific prototype
var protoRabbit = {
  speak: function(line){
    console.log("The " + this.type + " rabbit says '" + line + "'");
  }
};

var killerRabbit = Object.create(protoRabbit);
killerRabbit.type = "Killer";
killerRabbit.speak("Skreeee!");

// Constructor
function Rabbit(type){
   this.type = type;
}
var killerRabbit = new Rabbit("Killer");
var blackRabbit = new Rabbit("black");
console.log(blackRabbit.type);

// using prototype to add a new method
Rabbit.prototype.speak = function(line) {
  console.log("The " + this.type + " rabit says '" + line + "'");
};
blackRabbit.speak("Doom...");


// OVERRIDING DERIVED PROPERTIES
Rabbit.prototype.teeth = "small";
console.log(killerRabbit.teeth);

killerRabbit.teeth = "Long, sharp, and bloody";
console.log(killerRabbit.teeth);
console.log(blackRabbit.teeth);
console.log(Rabbit.prototype.teeth);

// PROTOTYPE INTERFERENCE
// A prototype can be used at any time to add methods, properties
// to all objects based on it
Rabbit.prototype.dance = function (){
  console.log("The " + this.type + " rabbit dances a jig");
};
killerRabbit.dance();
// but there is a problem:
var map = {};
function storePhi(event, phi){
   map[event] = phi;
}

storePhi("pizza", 0.069);
storePhi("touched tree", -0.081);
console.log(map);

Object.prototype.nonsense = "hi";
for (var name in map) {
  console.log(name);
}
console.log("nonsense" in map);
console.log("toString" in map);
delete Object.prototype.nonsense;
//  we can use Object.defineProperty to solve it
Object.defineProperty(Object.prototype, "hiddenNonsense", {
   enumerable: false,
   value: "hi"
});

for (var name in map) {
  console.log(name);
}
console.log(map.hiddenNonsense);
// but there still has a problem
console.log("toString" in map);
console.log(map.hasOwnProperty("toString"));

// PROTOTYPE-LESS OBJECTS
// if we only want to create an fresh object, without prototype then we tranform null to create
var map = Object.create(null);
map["pizza"] = 0.09;
console.log("toString" in map);
console.log("pizza" in map);

// POLYMORPHISM
// laying out a table: example for polymorphism
function rowHeights(rows) {
    return rows.map(function(row){
        return row.reduce(function(max, cell) {
            return Math.max(max, cell.minHeight());
        }, 0);
    });
}

function colWidths(rows) {
    return rows[0].map(function(_, i) {
        return rows.reduce(function(max, row){
            return Math.max(max, row[i].minWidth());
        }, 0);
    });
}

function drawTable(rows) {
    var heights = rowHeights(rows);
    var widths = colWidths(rows);

    function drawLine(blocks, lineNo) {
        return blocks.map(function(block) {
            return block[lineNo];
        }).join(" ");
    }

    function drawRow(row, rowNum){
        var blocks = row.map(function(cell, colNum) {
           return cell.draw(widths[colNum], heights[rowNum]);
        });
        return blocks[0].map(function(_, lineNo) {
            return drawLine(blocks, lineNo);
        }).join("\n");
    }

    return rows.map(drawRow).join("\n");
}

function repeat(string, times){
    var result = "";
    for (var i = 0; i < times; i++){
        result += string;
    }
    return result;
}

function TextCell(text){
    this.text = text.split("\n");
}
TextCell.prototype.minWidth = function(){
    return this.text.reduce(function(width, line){
        return Math.max(width, line.lenght);
    }, 0);
};
TextCell.prototype.minHeight = function(){
    return this.text.length;
}
TextCell.prototype.minHeight = function(){
    return this.text.lenght;
}
TextCell.prototype.minHeight = function(){
    return this.text.length;
}
TextCell.prototype.draw = function(width, height){
    var result = [];
    for (var i = 0; i < height; i++){
        var line = this.text[i] || "";
        result.push(line + repeat(" ", width - line.length));
    }
    return result;
}

var rows = [];
for (var i = 0; i < 5; i++){
    var row = [];
    for (var j = 0; j < 5; j++){
        if ((i + j) % 2 == 0){
            row.push(new TextCell("1234"));
        } else{
            row.push(new TextCell("5"));
        }
    }
    rows.push(row);
}
console.log(drawTable(rows));

// // GETTERS AND SETTERS
// var pile = {
//     elements: ["eggshell", "orange peel", "worm"],
//     get height(){
//         return this.elements.length;
//     },
//     set height(value) {
//         console.log("Ignoring attemp to set high to ", value);
//     }
// };

// console.log(pile.height);
// pile.height = 100;
// console.log(pile.height);

[1]: Introduction to Object-Oriented JavaScript [2]: How to call parent constructor?

\section{Networking}

POST
$.ajax({
  type: "POST",
  url: "http://service.com/items",
  data: JSON.stringify({"name": "new item"}),
  contentType: 'application/json'
}).done(function (data) {
  console.log(data)
}).fail(function () {
});

\section{Logging}


Javascript Logging
Having a fancy JavaScript debugger is great, but sometimes the fastest way to find bugs is just to dump as much information to the console as you can.

console.log
console.assert
console.error

\section{Documentation}

Components
jsdoc (with docdash template)

JSDoc is an API documentation generator for JavaScript, similar to JavaDoc or PHPDoc. You add documentation comments directly to your source code, right along side the code itself. The JSDoc Tool will scan your source code, and generate a complete HTML documentation website for you.

gulp, PyCharm

Usage
Step 1. Install gulp-jsdoc
npm install --save-dev gulp gulp-jsdoc docdash
Step 2. Create documentation task
Create documentation task in gulpfile.js file

var template = {
  "path": "./node_modules/docdash"
};

gulp.task('docs', function(){
  return gulp.src("./src/*.js")
    .pipe(jsdoc('./docs', template));
});
Step 3. Refresh Gulp tasks
In pycharm, click to refresh button in gulp window.

Step 4. Add comment to your code
Add comment to your code, You can see an example: should.js

/**
 * Simple utility function for a bit more easier should assertion
 * extension
 * @param {Function} f So called plugin function. It should accept
 * 2 arguments: `should` function and `Assertion` constructor
 * @memberOf should
 * @returns {Function} Returns `should` function
 * @static
 * @example
 *
 * should.use(function(should, Assertion) {
 *   Assertion.add('asset', function() {
 *      this.params = { operator: 'to be asset' };
 *
 *      this.obj.should.have.property('id').which.is.a.Number();
 *      this.obj.should.have.property('path');
 *  })
 * })
 */
should.use = function(f) {
  f(should, should.Assertion);
  return this;
};

Types: boolean, string, number, Array (see more)

Step 5. Run docs task
In pycharm, click to docs task in gulp window.

\section{Error Handling}

In javascript bugs may be displayed is NaN or underfined and program still run but after that, the wrong value can cause some mistake when we use it So, finding bugs and fix them is the quiet hard work in javascript But we can do, and this job is called debugging

STRICT MODE
This is the way to find errors that javascript ignores. Example is using an undefined variable. if we dont use strick mode, then everything will be ok, but if using, the error will be shown

function SpotProblem(){
//     "use strict";
    for (counter = 0; counter < 10; counter++){
        console.log("Good!");
    }
}
SpotProblem();
console.log(counter);
strick mode can find error when using this in local, but it is still in global. Example: When we forget to declare the key word "new" when create an new Object

"use strict";
function Person(name){
    this.name = name;
}
var john = Person("John");
console.log(name);
And there are another cases, that trick mode is not allowed: Delete an object is not allowed

"use strict";
var x = 3.14;
delete x;

"use strict";
var obj = {v1: 3, v2: 4};
delete pbj;

"use strict";
var func = function(){};
delete func;
Duplicate parameter is not allowed

"use strict";
var func = function(a1, a1){
    console.log(a1);
}
Reserve Word is not allowed to name variable

"use strict";
var arguments = 5;
var eval = 6;
console.log(arguments);
console.log(eval);
TESTING
Testing makes sure that the program working well, and if there are any changes, testing will automatic show us the error, thus, we know where need to fix

function Vector(x, y){
    this.x = x;
    this.y = y;
}
Vector.prototype.plus = function(other){
    return new Vector(this.x + other.x, this.y + other.y);
}

function TestVector(){
    var p1 = new Vector(10, 20);
    var p2 = new Vector(-10, 5);
    var p3 = p1.plus(p2);

    if (p1.x !== 10) return "fail: x property";
    if (p1.y !== 20) return "fail: y property";
    if (p2.x !== -10) return "fail: nagative x property";
    if (p2.y !== 5) return "fail: y property";
    if (p3.x !== 0) return "fail: x property from plus";
    if (p3.y !== 25) return "fail: y property from plus";
    return "Vector is Oke";
}
TestVector();
DEBUGGING
when the testing is fail, we have to debug to find the bugs.
The first we should guess the bug. And then we put break point in the line, we assume it make bug
If that is the exactly bug we want to find, then we fix it, and write more test for this case
In this example code below, the function convert the number in the decima to another. we run and see the result is wrong, so we guess that the error may be caused by the result variable, then we put break point in the line contains result variable.

function ConvertNumber(n, base) {
  var result = "", sign = "";
  if (n < 0) {
    sign = "-";
    n = -n;
  }
  do {
    result = String(n % base) + result;
    n /= base; //-> n = Math.floor(n / base);
  } while (n > 0);
  return sign + result;
}
console.log(ConvertNumber(13, 10));
console.log(ConvertNumber(14, 2));
ERROR PROPAGATION
Sometime our code is working well with normal input. But with special one, they can cause error. So, we have to consider all situation can make Flaws, and handling them.
This example code below has an if..else to handle the wrong input if user types not a number in the prompt input

function promptNumber(question) {
  var result = Number(prompt(question, ""));
  if (isNaN(result)) return null;
  else return result;
}
console.log(promptNumber("How many trees do you see?"));
EXCEPTION
In the Error Propagation, we can control the errors if we know them. But what will happen if we don't know the error? For solving this problem, javascript provides for us an try...catch.. to control error we dont know or not sure

try {
    throw new Error("Invalid defination");
} catch (error){
    console.log(error);
}

function promtDirection(question){
    var result = prompt(question, "");
    if (result.toLowerCase() == "left") return "L";
    if (result.toLowerCase() == "right") return "R";
    throw new Error("Invalid direction: " + result);
}

function look(){
    if (promtDirection("Which way?") == "L") {
        return "a house";
    }
    else{
        return "two angry bears";
    }
}

try {
    console.log("you see", look());
} catch (error) {
    console.log("Something went wrong: " + error);
}
CLEAN UP AFTER EXCEPTIONS
We have a block of code below:

var context = null;
function withContext(newContext, body){
  var oldContext = context;
  context = newContext;
  var result = body();
  context = oldContext;
  return result;
}
withContext("new", function(){
  var a = b/0;
  return a;
});
What would happend with context? It cannot be excute the last line code, because in withContext function, it will throw off the stack by an exception. So javascript provides a try...finally...

var context = null;
function withContext(newContext, body){
  var oldContext = context;
  context = newContext;
  try{
    return body();
  } finally {
    context = oldContext;
  }
}
withContext("new", function(){
  var a = b/0;
  return a;
});
SELECTIVE CATCHING
There are some errors cannot handle by environment. So, if we let the error go through, it can cause broken program.
For examnple, the Error() in environment cannot catch the infinitive loop in the try block, if we dont catch this problem, the programm will crash soon

for (;;) {
  try {
    var dir = promtDirection("Where?");
    console.log("You chose ", dir);
    break;
  } catch (e) {
    console.log("Not a valid direction. Try again.");
  }
}
The loop will break out if the promptDirection() can excute.
But it doesn't. Because it is not defined before, so the environment catch it and go through the catch to show error
The circle again and again will make the program crash.
So we will create a special Exception.

function InputError(message){
  this.message = message;
  this.stack = (new Error()).stack;
}
InputError.prototype = Object.create(Error.prototype);
InputError.prototype.name = "InputError";
Error: has an property is stack. it contains all exception, which environment can catch. Then, we have the promptDirection function to return the result if Enter valid format, or an exception if invalid

function promptDirection(question){
  var result = prompt(question, "");
  if (result.toLowerCase() == "left") return "L";
  if (result.toLowerCase() == "right") return "R";
  throw new InputError("Invalid direction: " + result);
}
Finally, we can catch all exception we want

for (;;){
  try {
    var dir = promptDirection("Where?");
    console.log("You choose ", dir);
    break;
  } catch(e) {
    if (e instanceof InputError){
      console.log("Not a valid direction. Try again. ");
    }
    else {
      throw e;
    }
  }
}
ASSERTIONS
function AssertionFailed(message) {
  this.message = message;
}
AssertionFailed.prototype = Object.create(Error.prototype);

function assert(test, message) {
  if (!test)
    throw new AssertionFailed(message);
}

function lastElement(array) {
  assert(array.length > 0, "empty array in lastElement");
  return array[array.length - 1];
}

\section{Testing}

Mocha
Mocha is a feature-rich JavaScript test framework running on Node.js and the browser, making asynchronous testing simple and fun. Mocha tests run serially, allowing for flexible and accurate reporting, while mapping uncaught exceptions to the correct test cases.

Installation
bower install -D mocha chai
Usage
Step 1. Make index.html

<!DOCTYPE html>
<html>
<head>
  <meta charset="utf-8">
  <title>Tests</title>
  <link rel="stylesheet" media="all" href="mocha.css">
</head>
<body>
  <div id="mocha"></div>
  <script src="mocha.js"></script>
  <script src="chai.js"></script>
  <script src="functions.js"></script>
  <script>mocha.setup('bdd'); chai.should();</script>
  <script src="tests.js"></script>
  <script>mocha.run();</script>
</body>
</html>
Step 2. Edit functions.js

function sum(a, b){
  return a + b;
}

function asynchronusSum(a, b){
  return new Promise(function(fulfill, reject){
    fulfill(a + b);
  });
}
Step 3. Edit tests.js

describe('Calculator', function() {
  this.timeout(5000);
  describe('#sum()', function() {
    it('should return sum of two number', function() {
      sum(2, 3).should.equal(5)
    });
  });

  describe('#asynchronusSum()', function() {
    it('should return sum of two number', function(done) {
      asynchronusSum(2, 3).then(function(output){
          output.should.equal(5);
          done();
      })
    });
  });
});

\section{Package Manager}

Bower
A package manager for the web

Web sites are made of lots of things — frameworks, libraries, assets, utilities, and rainbows. Bower manages all these things for you.

Bower works by fetching and installing packages from all over, taking care of hunting, finding, downloading, and saving the stuff you’re looking for. Bower keeps track of these packages in a manifest file, bower.json. How you use packages is up to you. Bower provides hooks to facilitate using packages in your tools and workflows.

Bower is optimized for the front-end. Bower uses a flat dependency tree, requiring only one version for each package, reducing page load to a minimum.

http://bower.io/

[code] bower install jquery underscore moment sprintf -S [/code]

HTML <bower based>

<script src="./bower_components/jquery/dist/jquery.js"></script>
<script src="./bower_components/moment/moment.js"></script>
<script src="./bower_components/underscore/underscore.js"></script>
<script src="./bower_components/sprintf/src/sprintf.js"></script>
HTML <cdn based>

<script src="//cdnjs.cloudflare.com/ajax/libs/jquery/3.0.0-beta1/jquery.js"></script>
<script src="//cdnjs.cloudflare.com/ajax/libs/underscore.js/1.8.3/underscore.js"></script>
<script src="//cdnjs.cloudflare.com/ajax/libs/sprintf/1.0.3/sprintf.js"></script>

\section{Build Tool}

Gulp

Automate and enhance your workflow

Here's some of the sweet stuff you try out with this repo.

Compile CoffeeScript (with source maps!)
Compile Handlebars Templates
Compile SASS with Compass
LiveReload
require non-CommonJS code, with dependencies
Set up module aliases
Run a static Node server (with logging)
Pop open your app in a Browser
Report Errors through Notification Center
Image processing
Installation
npm install -S gulp gulp-concat
Usage
Watch

var gulp = require('gulp');
var concat = require('gulp-concat');
var uglify = require('gulp-uglify');
var jsdoc = require("gulp-jsdoc");

var third_parties = [
  "bower_components/jquery/dist/jquery.js",
  "bower_components/bootstrap/dist/js/bootstrap.js",
  "bower_components/underscore/underscore.js",
  "bower_components/ring/ring.js",
  "bower_components/moment/moment.js",
  "bower_components/sprintf/src/sprintf.js",
  "bower_components/uri.js/src/URI.js",
  "bower_components/run/run.js"
];

var modules = [
  "modules/your_script.js"
];

gulp.watch(third_parties, ['js_thirdparty']);
gulp.watch(modules, ['js_modules']);

gulp.task('js_thirdparty', function () {
  return gulp
    .src(third_parties)
    .pipe(concat('third_party.uglify.js'))
    .pipe(uglify())
    .pipe(gulp.dest('./scripts'));
});

gulp.task('js_modules', function () {
  return gulp
    .src(modules)
    .pipe(concat('modules.uglify.js'))
    //.pipe(uglify())
    .pipe(gulp.dest('./scripts'));
});

gulp.task('documentation', function () {
  return gulp
    .src("./modules/*/*.js")
    .pipe(jsdoc('./documentation'));
});

gulp.task('default', ['js_thirdparty', 'js_modules']);
http://gulpjs.com/

Deprecated
grunt

\section{Make Module}

Make Module
sample modules: underscore, momentjs

Folder Structure
|- docs
|- test
|- src
|   |-- your_module.js
|- .gitignore
|- bower.json




%\chapter{Java}

01/11/2017: Java đơn giản là gay nhé. Không chơi. Viết java chỉ viết thế này thôi. Không viết hơn. Thề!
\chapter{PHP}

PHP là ngôn ngữ lập trình web dominate tất cả các anh tài khác mà (chắc là) chỉ dịu đi khi mô hình REST xuất hiện. Nhớ lần đầu gặp bạn Laravel mà cảm giác cuộc đời sang trang.

Cuối tuần này lại phải xem làm sao cài được xdebug vào PHPStorm cho thằng em tập tành lập trình. Haizzz

### Tương tác với cơ sở dữ liệu

Liệt kê danh sách các bản ghi trong bảng groups

```
$sql = "SELECT * FROM `groups`";
$groups = mysqli_query($conn, $sql);
```

Xóa một bản ghi trong bảng groups

```
$sql = "DELETE FROM `groups` WHERE id = `5`";
mysqli_query($conn, $sql);
```

### Cài đặt debug trong PHPStorm

https://www.youtube.com/watch?v=mEJ21RB0F14

(1) XAMPP

- Download XAMPP (cho PHP 7.1.x - do XDebug chưa chính thức hỗ trợ 7.2.0)
https://www.apachefriends.org/xampp-files/7.1.12/xampp-win32-7.1.12-0-VC14-installer.exe
- Install XAMPP xampp-win32-7.1.12-0-VC14-installer.exe
- Truy cập vào địa chỉ http://localhost/dashboard/phpinfo.php để kiểm tra cài đặt đã thành công chưa

(2) Tải và cài đặt PHPStorm

- Download PHPStorm https://download-cf.jetbrains.com/webide/PhpStorm-2017.3.2.exe
- Install PHPStorm

(3) Tạo một web project trong PHPStorm
- Chọn interpreter trỏ đến PHP trong xampp

(4) Viết một chương trình add.php

```php
$a = 2;
$b = 3;
$c = $a + $b;

echo $c;
```

Click vào `add.php`, chọn Debug, PHPStorm sẽ báo chưa cài XDebug

(5) Cài đặt XDebug theo hướng dẫn tại https://gist.github.com/odan/1abe76d373a9cbb15bed

Click vào add.php, chọn Debug

(6) Cài đặt XDebug với PHPStorm Marklets
Vào trang https://www.jetbrains.com/phpstorm/marklets/
Trong phần Zend Debugger
- chọn cổng 9000
- IP: 127.0.0.1
Nhấn nút Generate

Bookmark các link &quot;Start debugger&quot;, &quot;Stop debugger&quot; lên trình duyệt

(7) Debug PHP từ trình duyệt

* Vào trang http://localhost/untitled/add.php
* Click vào bookmark Start debugger
* Trong PHPStorm, nhấn vào biểu tượng &quot;Start Listening for PHP Debug Connections&quot;
* Đặt breakpoint tại dòng thứ 5
* Refresh lại trang http://localhost/untitled/add.php, lúc này, breakpoint sẽ dừng ở dòng 5

%\chapter{R}

View online \href{http://magizbox.com/training/r/site/}{http://magizbox.com/training/r/site/}

R
R is a programming language and software environment for statistical computing and graphics supported by the R Foundation for Statistical Computing. The R language is widely used among statisticians and data miners for developing statistical software and data analysis. Polls, surveys of data miners, and studies of scholarly literature databases show that R's popularity has increased substantially in recent years.

R is a GNU package. The source code for the R software environment is written primarily in C, Fortran, and R. R is freely available under the GNU General Public License, and pre-compiled binary versions are provided for various operating systems. While R has a command line interface, there are several graphical front-ends available.[

R was created by Ross Ihaka and Robert Gentleman at the University of Auckland, New Zealand, and is currently developed by the R Development Core Team, of which Chambers is a member. R is named partly after the first names of the first two R authors and partly as a play on the name of S. The project was conceived in 1992, with an initial version released in 1995 and a stable beta version in 2000.

\section{R Courses}

I'm going to give a course about R, but it's take a lot of time to finish. I will give at least one lesson a week. You can track it here

(next) Data visualization with R
Everything you need to know about R
Read and Write Data
Importing data from JSON into R
Manipulate Data
Manipulate String and Datetime
Actually, beside my works, there are a lot of excellent and free courses in the internet for you

Beginner

tryr from codeschool

tryr is a course for beginners created by codeschool. This course contains R Syntax, Vectors, Matrices, Summary Statistics, Factors, Data Frames and Working With Real-World Data sections.

Introduction to R from datacamp

This course created by datacamp - a "online learning platform that focuses on building the best learning experience for Data Science in specific". Here is the introduction about this course quoted from authors "In this introduction to R, you will master the basics of this beautiful open source language such as factors, lists and data frames. With the knowledge gained in this course, you will be ready to undertake your first very own data analysis." It contains 6 chapters: Intro to basics, Vectors, Matrices, Factors, Data frames and Lists.

Intermediate and Advanced

R Programming of Johns Hopkins University in coursera Learn how to program in R and how to use R for effective data analysis. This is the second course in the Johns Hopkins Data Science Specialization. It's a 4-weeks course, contains: Overview of R, R data types and objects, reading and writing data (week 1),  Control structures, functions, scoping rules, dates and times (week 2), Loop functions, debugging tools (week 3) and Simulation, code profiling (week 4)

An Introduction to Statistical Learning with Applications in R of two experts Trevor Hastie and Rob Tibshirani from Standfor Unitiversity

This course was introduced by Kevin Markham in r-blogger in september 2014. "I found it to be an excellent course in statistical learning (also known as “machine learning”), largely due to the high quality of both the textbook and the video lectures. And as an R user, it was extremely helpful that they included R code to demonstrate most of the techniques described in the book." In this course you will learn about Statistical Learning, Linear Regression, Classification, Resampling Methods, Linear Model Selection and Regularization, Moving Beyond Linearity, Tree-Based Methods, Support Vector Machines and Unsupervised Learning

Cheatsheet – Python & R codes for common Machine Learning Algorithms

\section{Everything you need to know about R}

In this post I maintain all useful references for someone want to write nice R code.

Google’s R Style Guide at google
R is a high-level programming language used primarily for statistical computing and graphics. The goal of the R Programming Style Guide is to make our R code easier to read, share, and verify. The rules below were designed in collaboration with the entire R user community at Google.

Installing R packages at r-bloggers
https://www.r-bloggers.com/installing-r-packages/

This is a short post giving steps on how to actually install R packages.

Managing your projects in a reproducible fashion at nicercode
https://nicercode.github.io/blog/2013-04-05-projects/

Managing your projects in a reproducible fashion doesn’t just make your science reproducible, it makes your life easier.

Creating R Packages
http://cran.r-project.org/doc/contrib/Leisch-CreatingPackages.pdf

This tutorial gives a practical introduction to creating R packages. We discuss how object oriented programming and S formulas can be used to give R code the usual look and feel, how to start a package from a collection of R functions, and how to test the code once the package has been created. As running example we use functions for standard linear regression analysis which are developed from scratch

How to write trycatch in R
http://stackoverflow.com/questions/12193779/how-to-write-trycatch-in-r

Welcome to the R world 😉

Debugging with RStudio
https://support.rstudio.com/hc/en-us/articles/200713843-Debugging-with-RStudio

RStudio includes a visual debugger that can help you understand code and find bugs.

Optimising code
http://adv-r.had.co.nz/Profiling.html#performance-profiling

Optimising code to make it run faster is an iterative process:

Find the biggest bottleneck (the slowest part of your code). Try to eliminate it (you may not succeed but that’s ok). Repeat until your code is “fast enough.” This sounds easy, but it’s not.
%\chapter{Scala}

View online \href{http://magizbox.com/training/scala/site/}{http://magizbox.com/training/scala/site/}

Scala is a programming language for general software applications. Scala has full support for functional programming and a very strong static type system. This allows programs written in Scala to be very concise and thus smaller in size than other general-purpose programming languages. Many of Scala's design decisions were inspired by criticism of the shortcomings of Java.

\section{Installation}

Windows
Step 1. Download scala from http://www.scala-lang.org/downloads

Step 2. Run installer

Step 3. Verify

Open terminal and check which version of scala

$ scala -version

Scala code runner version 2.11.5 -- Copyright 2002-2013, LAMP/EPFL

\section{IDE}

I use IntelliJ IDEA 2016.2 as scala IDE

IntelliJ IDEA Installation Guide
Online IDE
You can use tryscala as an online IDE

http://www.tryscala.com/

\section{Basic Syntax}

Print print
> println("Hello, Scala!");

Hello, Scala!
Conditional
if Statement

if statement consists of a Boolean expression followed by one or more statements.

var x = 10;
if( x < 20 ){
 println("This is if statement");
}
if-else Statement

var x = 30
if( x < 20 ){
  println("This is if statement");
} else {
  println("This is else statement");
}
if-else if-else Statement

 var x = 30;
if( x == 10 ){
 println("Value of X is 10");
} else if( x == 20 ){
 println("Value of X is 20");
} else if( x == 30 ){
 println("Value of X is 30");
} else{
 println("This is else statement");
}
Coding Convention 1
Keep It Simple
Don't pack two much in one expression
/*
 * It's amazing what you can get done in a single statement
 * But that does not mean you have to do it.
 */
jp.getRawClasspath.filter(
  _.getEntryKind == IClasspathEntry.CPE_SOURCE).
  iterator.flatMap(entry =>
    flatten(ResourcesPlugin.getWorkspace.
      getRoot.findMember(entry.getPath)))
Refactor
There's a lot of value in meaningfull names.
Easy to add them using inline vals and defs
val sources = jp.getRawClasspath.filter(
  _.getEntryKind == IClasspathEntry.CPE_SOURCE)
def workspaceRoot =
  ResourcesPlugin.getWorkspace.getRoot
def filesOfEntry(entry: Set[File]) =
  flatten(worspaceRoot.findMember(entry.getPath)
sources.iterator flatMap filesOfEntry
Prefer Functional
By default

use vals, not vars
use recursions or combinators, not loops
use immutable collections
concentrate on transformations, not CRUD
When to deviate from the default - sometimes, mutable gives better performance. - sometimes (but not that often!) it adds convenience

But don't diablolize local state
Why does mutable state lead to complexity?

It interacts with different program parts in ways that are hard to track.

=> Local state is less harmful than global state.

"Var" Shortcuts
var interfaces = parseClassHeader()...
if (isAnnotation) interfaces += ClassFileAnnotation
Refactor

val parsedIfaces = parseClassHeader()
val interfaces =
  if (isAnnotation) parsedIfaces + ClassFileAnnotation
  else parsedIfaces
Martin Odersky - Scala with Style ↩
%\chapter{NodeJS}

View online \href{http://magizbox.com/training/nodejs/site/}{http://magizbox.com/training/nodejs/site/}

Node.js is an open-source, cross-platform JavaScript runtime environment for developing a diverse variety of tools and applications. Although Node.js is not a JavaScript framework, many of its basic modules are written in JavaScript, and developers can write new modules in JavaScript. The runtime environment interprets JavaScript using Google's V8 JavaScript engine. Node.js has an event-driven architecture capable of asynchronous I/O. These design choices aim to optimize throughput and scalability in Web applications with many input/output operations, as well as for real-time Web applications (e.g., real-time communication programs and browser games). Node.js was originally written in 2009 by Ryan Dahl. The initial release supported only Linux. Its development and maintenance was led by Dahl and later sponsored by Joyent.

\section{Get Started}

Installation
Windows
In this section I will show you how to Install Node.js® and NPM on Windows

Prerequisites
Node isn’t a program that you simply launch like Word or Photoshop: you won’t find it pinned to the taskbar or in your list of Apps. To use Node you must type command-line instructions, so you need to be comfortable with (or at least know how to start) a command-line tool like the Windows Command Prompt, PowerShell, Cygwin, or the Git shell (which is installed along with Github for Windows).

Installation Overview
Installing Node and NPM is pretty straightforward using the installer package available from the Node.js® web site.

Installation Steps
1. Download the Windows installer from the Nodes.js® web site.

2. Run the installer (the .msi file you downloaded in the previous step.)

3. Follow the prompts in the installer (Accept the license agreement, click the NEXT button a bunch of times and accept the default installation settings).



4. Restart your computer. You won’t be able to run Node.js® until you restart your computer.

Ubuntu
In this section I will show you how to Install Node.js® and NPM on Ubuntu

# update os
sudo apt-get update
# install node with apt-get
sudo apt-get install nodejs
# install npm with apt-get
sudo apt-get install npm
Test
Make sure you have Node and NPM installed by running simple commands to see what version of each is installed and to run a simple test program:

> node -v
v6.9.5

> npm -v
3.10.10
Suggested Readings
How To Install Node.js on an Ubuntu 14.04 server
How to Install Node.js® and NPM on Windows

\section{Basic Syntax}

Print
console.log("Hello World");
Conditional
if(you_smart){
    console.log("learn nodejs");
} else {
    console.log("go away");
}
Loop
for(var count = 0; count < 10; count++){
    console.log(count);
}
Function
function print_info(arg1, arg2){
    console.log(arg1);
    console.log(arg2);
}

\section{File System & IO}

File System & IO
Node implements File I/O using simple wrappers around standard POSIX functions. The Node File System (fs) module can be imported using the following syntax −

var fs = require("fs")
Synchronous vs Asynchronous
Every method in the fs module has synchronous as well as asynchronous forms. Asynchronous methods take the last parameter as the completion function callback and the first parameter of the callback function as error. It is better to use an asynchronous method instead of a synchronous method, as the former never blocks a program during its execution, whereas the second one does.

Example

Create a text file named input.txt with the following content −

Tutorials Point is giving self learning content
to teach the world in simple and easy way!!!!!
Let us create a js file named main.js with the following code −

var fs = require("fs");

// Asynchronous read
fs.readFile('input.txt', function (err, data) {
   if (err) {
      return console.error(err);
   }
   console.log("Asynchronous read: " + data.toString());
});

// Synchronous read
var data = fs.readFileSync('input.txt');
console.log("Synchronous read: " + data.toString());

console.log("Program Ended");
Now run the main.js to see the result −

$ node main.js
Verify the Output.

Synchronous read: Tutorials Point is giving self learning content
to teach the world in simple and easy way!!!!!

Program Ended
Asynchronous read: Tutorials Point is giving self learning content
to teach the world in simple and easy way!!!!!
The following sections in this chapter provide a set of good examples on major File I/O methods.
Open a File
Syntax

Following is the syntax of the method to open a file in asynchronous mode −

fs.open(path, flags[, mode], callback)
Parameters

Here is the description of the parameters used −

path − This is the string having file name including path.
flags − Flags indicate the behavior of the file to be opened. All possible values have been mentioned below.
mode − It sets the file mode (permission and sticky bits), but only if the file was created. It defaults to 0666, readable and writeable.
callback − This is the callback function which gets two arguments (err, fd).
Flags

Flags for read/write operations are −

r - Open file for reading. An exception occurs if the file does not exist.
r+ - Open file for reading and writing. An exception occurs if the file does not exist.
rs - Open file for reading in synchronous mode.
rs+ - Open file for reading and writing, asking the OS to open it synchronously. See notes for 'rs' about using this with caution.
w - Open file for writing. The file is created (if it does not exist) or truncated (if it exists).
wx - Like 'w' but fails if the path exists.
w+ - Open file for reading and writing. The file is created (if it does not exist) or truncated (if it exists).
wx+ - Like 'w+' but fails if path exists.
a - Open file for appending. The file is created if it does not exist.
ax - Like 'a' but fails if the path exists.
a+ - Open file for reading and appending. The file is created if it does not exist.
ax+ - Like 'a+' but fails if the the path exists.
Example

Let us create a js file named main.js having the following code to open a file input.txt for reading and writing.

var fs = require("fs");

// Asynchronous - Opening File
console.log("Going to open file!");
fs.open('input.txt', 'r+', function(err, fd) {
   if (err) {
      return console.error(err);
   }
  console.log("File opened successfully!");
});
Now run the main.js to see the result −

$ node main.js
Verify the Output.

Going to open file!
File opened successfully!
Get File Information
Syntax

Following is the syntax of the method to get the information about a file −

fs.stat(path, callback)
Parameters

Here is the description of the parameters used −

path − This is the string having file name including path.
callback − This is the callback function which gets two arguments (err, stats) where stats is an object of fs.Stats type which is printed below in the example.
Apart from the important attributes which are printed below in the example, there are several useful methods available in fs.Stats class which can be used to check file type. These methods are given in the following table.

Method Description

stats.isFile() - Returns true if file type of a simple file.
stats.isDirectory() - Returns true if file type of a directory.
stats.isBlockDevice() - Returns true if file type of a block device.
stats.isCharacterDevice() - Returns true if file type of a character device.
stats.isSymbolicLink() - Returns true if file type of a symbolic link.
stats.isFIFO() - Returns true if file type of a FIFO.
stats.isSocket() - Returns true if file type of asocket.
Example

Let us create a js file named main.js with the following code −

var fs = require("fs");

console.log("Going to get file info!");
fs.stat('input.txt', function (err, stats) {
   if (err) {
       return console.error(err);
   }
   console.log(stats);
   console.log("Got file info successfully!");

   // Check file type
   console.log("isFile ? " + stats.isFile());
   console.log("isDirectory ? " + stats.isDirectory());
});
Now run the main.js to see the result −

$ node main.js
Verify the Output.

Going to get file info!
{
   dev: 1792,
   mode: 33188,
   nlink: 1,
   uid: 48,
   gid: 48,
   rdev: 0,
   blksize: 4096,
   ino: 4318127,
   size: 97,
   blocks: 8,
   atime: Sun Mar 22 2015 13:40:00 GMT-0500 (CDT),
   mtime: Sun Mar 22 2015 13:40:57 GMT-0500 (CDT),
   ctime: Sun Mar 22 2015 13:40:57 GMT-0500 (CDT)
}
Got file info successfully!
isFile ? true
isDirectory ? false
Writing a File
Syntax

Following is the syntax of one of the methods to write into a file −

fs.writeFile(filename, data[, options], callback)
This method will over-write the file if the file already exists. If you want to write into an existing file then you should use another method available.

Parameters

Here is the description of the parameters used −

path − This is the string having the file name including path.
data − This is the String or Buffer to be written into the file.
options − The third parameter is an object which will hold {encoding, mode, flag}. By default. encoding is utf8, mode is octal value 0666. and flag is 'w'
callback − This is the callback function which gets a single parameter err that returns an error in case of any writing error.
Example

Let us create a js file named main.js having the following code −

var fs = require("fs");

console.log("Going to write into existing file");
fs.writeFile('input.txt', 'Simply Easy Learning!',  function(err) {
   if (err) {
      return console.error(err);
   }

   console.log("Data written successfully!");
   console.log("Let's read newly written data");
   fs.readFile('input.txt', function (err, data) {
      if (err) {
         return console.error(err);
      }
      console.log("Asynchronous read: " + data.toString());
   });
});
Now run the main.js to see the result −

$ node main.js
Verify the Output.

Going to write into existing file
Data written successfully!
Let's read newly written data
Asynchronous read: Simply Easy Learning!
Reading a File
Syntax

Following is the syntax of one of the methods to read from a file −

fs.read(fd, buffer, offset, length, position, callback)
This method will use file descriptor to read the file. If you want to read the file directly using the file name, then you should use another method available.

Parameters

Here is the description of the parameters used −

fd − This is the file descriptor returned by fs.open().
buffer − This is the buffer that the data will be written to.
offset − This is the offset in the buffer to start writing at.
length − This is an integer specifying the number of bytes to read.
position − This is an integer specifying where to begin reading from in the file. * If position is null, data will be read from the current file position. callback − This is the callback function which gets the three arguments, (err, bytesRead, buffer).
Example

Let us create a js file named main.js with the following code −

var fs = require("fs");
var buf = new Buffer(1024);

console.log("Going to open an existing file");
fs.open('input.txt', 'r+', function(err, fd) {
   if (err) {
      return console.error(err);
   }
   console.log("File opened successfully!");
   console.log("Going to read the file");
   fs.read(fd, buf, 0, buf.length, 0, function(err, bytes){
      if (err){
         console.log(err);
      }
      console.log(bytes + " bytes read");

      // Print only read bytes to avoid junk.
      if(bytes > 0){
         console.log(buf.slice(0, bytes).toString());
      }
   });
});
Now run the main.js to see the result −

$ node main.js
Verify the Output.

Going to open an existing file
File opened successfully!
Going to read the file
97 bytes read
Tutorials Point is giving self learning content
to teach the world in simple and easy way!!!!!
Closing a File
Syntax

Following is the syntax to close an opened file −

fs.close(fd, callback)
Parameters

Here is the description of the parameters used −

fd − This is the file descriptor returned by file fs.open() method.
callback − This is the callback function No arguments other than a possible exception are given to the completion callback.
Example Let us create a js file named main.js having the following code −

var fs = require("fs");
var buf = new Buffer(1024);

console.log("Going to open an existing file");
fs.open('input.txt', 'r+', function(err, fd) {
   if (err) {
      return console.error(err);
   }
   console.log("File opened successfully!");
   console.log("Going to read the file");

   fs.read(fd, buf, 0, buf.length, 0, function(err, bytes){
      if (err){
         console.log(err);
      }

      // Print only read bytes to avoid junk.
      if(bytes > 0){
         console.log(buf.slice(0, bytes).toString());
      }

      // Close the opened file.
      fs.close(fd, function(err){
         if (err){
            console.log(err);
         }
         console.log("File closed successfully.");
      });
   });
});
Now run the main.js to see the result −

$ node main.js
Verify the Output.

Going to open an existing file
File opened successfully!
Going to read the file
Tutorials Point is giving self learning content
to teach the world in simple and easy way!!!!!

File closed successfully.
Truncate a File
Syntax

Following is the syntax of the method to truncate an opened file −

fs.ftruncate(fd, len, callback)
Parameters

Here is the description of the parameters used −

fd − This is the file descriptor returned by fs.open().
len − This is the length of the file after which the file will be truncated.
callback − This is the callback function No arguments other than a possible ekxception are given to the completion callback.
Example

Let us create a js file named main.js having the following code −

var fs = require("fs");
var buf = new Buffer(1024);

console.log("Going to open an existing file");
fs.open('input.txt', 'r+', function(err, fd) {
   if (err) {
      return console.error(err);
   }
   console.log("File opened successfully!");
   console.log("Going to truncate the file after 10 bytes");

   // Truncate the opened file.
   fs.ftruncate(fd, 10, function(err){
      if (err){
         console.log(err);
      }
      console.log("File truncated successfully.");
      console.log("Going to read the same file");

      fs.read(fd, buf, 0, buf.length, 0, function(err, bytes){
         if (err){
            console.log(err);
         }

         // Print only read bytes to avoid junk.
         if(bytes > 0){
            console.log(buf.slice(0, bytes).toString());
         }

         // Close the opened file.
         fs.close(fd, function(err){
            if (err){
               console.log(err);
            }
            console.log("File closed successfully.");
         });
      });
   });
});
Now run the main.js to see the result −

$ node main.js
Verify the Output.

Going to open an existing file
File opened successfully!
Going to truncate the file after 10 bytes
File truncated successfully.
Going to read the same file
Tutorials
File closed successfully.
Delete a File
Syntax Following is the syntax of the method to delete a file −

fs.unlink(path, callback)
Parameters

Here is the description of the parameters used −

path − This is the file name including path.
callback − This is the callback function No arguments other than a possible exception are given to the completion callback.
Example

Let us create a js file named main.js having the following code −

var fs = require("fs");

console.log("Going to delete an existing file");
fs.unlink('input.txt', function(err) {
   if (err) {
      return console.error(err);
   }
   console.log("File deleted successfully!");
});
Now run the main.js to see the result −

$ node main.js
Verify the Output.

Going to delete an existing file
File deleted successfully!
Create a Directory
Syntax

Following is the syntax of the method to create a directory −

fs.mkdir(path[, mode], callback)
Parameters

Here is the description of the parameters used −

path − This is the directory name including path.
mode − This is the directory permission to be set. Defaults to 0777.
callback − This is the callback function No arguments other than a possible exception are given to the completion callback.
Example

Let us create a js file named main.js having the following code −

var fs = require("fs");

console.log("Going to create directory /tmp/test");
fs.mkdir('/tmp/test',function(err){
   if (err) {
      return console.error(err);
   }
   console.log("Directory created successfully!");
});
Now run the main.js to see the result −

$ node main.js
Verify the Output.

Going to create directory /tmp/test
Directory created successfully!
Read a Directory
Syntax

Following is the syntax of the method to read a directory −

fs.readdir(path, callback)
Parameters

Here is the description of the parameters used −

path − This is the directory name including path.
callback − This is the callback function which gets two arguments (err, files) where files is an array of the names of the files in the directory excluding '.' and '..'.
Example

Let us create a js file named main.js having the following code −

var fs = require("fs");

console.log("Going to read directory /tmp");
fs.readdir("/tmp/",function(err, files){
   if (err) {
      return console.error(err);
   }
   files.forEach( function (file){
      console.log( file );
   });
});
Now run the main.js to see the result −

$ node main.js
Verify the Output.

Going to read directory /tmp
ccmzx99o.out
ccyCSbkF.out
employee.ser
hsperfdata_apache
test
test.txt
Remove a Directory
Syntax

Following is the syntax of the method to remove a directory −

fs.rmdir(path, callback)
Parameters

Here is the description of the parameters used −

path − This is the directory name including path.
callback − This is the callback function No argume nts other than a possible exception are given to the completion callback.
Example

Let us create a js file named main.js having the following code −

var fs = require("fs");

console.log("Going to delete directory /tmp/test");
fs.rmdir("/tmp/test",function(err){
   if (err) {
      return console.error(err);
   }
   console.log("Going to read directory /tmp");

   fs.readdir("/tmp/",function(err, files){
      if (err) {
         return console.error(err);
      }
      files.forEach( function (file){
         console.log( file );
      });
   });
});
Now run the main.js to see the result −

$ node main.js
Verify the Output.

Going to read directory /tmp
ccmzx99o.out
ccyCSbkF.out
employee.ser
hsperfdata_apache
test.txt

\section{Package Manager}

Package Manager: NPM
Node Package Manager (NPM) provides two main functionalities −

Online repositories for node.js packages/modules which are searchable on search.nodejs.org
Command line utility to install Node.js packages, do version management and dependency management of Node.js packages.
NPM comes bundled with Node.js installables after v0.6.3 version. To verify the same, open console and type the following command and see the result −

$ npm --version
2.7.1
If you are running an old version of NPM then it is quite easy to update it to the latest version. Just use the following command from root −

$ sudo npm install npm -g
/usr/bin/npm -> /usr/lib/node_modules/npm/bin/npm-cli.js
npm@2.7.1 /usr/lib/node_modules/npm
Installing Modules
There is a simple syntax to install any Node.js module −

$ npm install <Module Name>
For example, following is the command to install a famous Node.js web framework module called express −

$ npm install express
Now you can use this module in your js file as following −

var express = require('express');
Global vs Local Installation
By default, NPM installs any dependency in the local mode. Here local mode refers to the package installation in node_modules directory lying in the folder where Node application is present. Locally deployed packages are accessible via require() method. For example, when we installed express module, it created node_modules directory in the current directory where it installed the express module.

$ ls -l
total 0
drwxr-xr-x 3 root root 20 Mar 17 02:23 node_modules
Alternatively, you can use npm ls command to list down all the locally installed modules.

Globally installed packages/dependencies are stored in system directory. Such dependencies can be used in CLI (Command Line Interface) function of any node.js but cannot be imported using require() in Node application directly. Now let's try installing the express module using global installation.

$ npm install express -g
This will produce a similar result but the module will be installed globally. Here, the first line shows the module version and the location where it is getting installed.

express@4.12.2 /usr/lib/node_modules/express
├── merge-descriptors@1.0.0
├── utils-merge@1.0.0
├── cookie-signature@1.0.6
├── methods@1.1.1
├── fresh@0.2.4
├── cookie@0.1.2
├── escape-html@1.0.1
├── range-parser@1.0.2
├── content-type@1.0.1
├── finalhandler@0.3.3
├── vary@1.0.0
├── parseurl@1.3.0
├── content-disposition@0.5.0
├── path-to-regexp@0.1.3
├── depd@1.0.0
├── qs@2.3.3
├── on-finished@2.2.0 (ee-first@1.1.0)
├── etag@1.5.1 (crc@3.2.1)
├── debug@2.1.3 (ms@0.7.0)
├── proxy-addr@1.0.7 (forwarded@0.1.0, ipaddr.js@0.1.9)
├── send@0.12.1 (destroy@1.0.3, ms@0.7.0, mime@1.3.4)
├── serve-static@1.9.2 (send@0.12.2)
├── accepts@1.2.5 (negotiator@0.5.1, mime-types@2.0.10)
└── type-is@1.6.1 (media-typer@0.3.0, mime-types@2.0.10)
You can use the following command to check all the modules installed globally −

$ npm ls -g
Using package.json
package.json is present in the root directory of any Node application/module and is used to define the properties of a package. Let's open package.json of express package present in node_modules/express/

{
   "name": "express",
      "description": "Fast, unopinionated, minimalist web framework",
      "version": "4.11.2",
      "author": {

         "name": "TJ Holowaychuk",
         "email": "tj@vision-media.ca"
      },

   "contributors": [{
      "name": "Aaron Heckmann",
      "email": "aaron.heckmann+github@gmail.com"
   },

   {
      "name": "Ciaran Jessup",
      "email": "ciaranj@gmail.com"
   },

   {
      "name": "Douglas Christopher Wilson",
      "email": "doug@somethingdoug.com"
   },

   {
      "name": "Guillermo Rauch",
      "email": "rauchg@gmail.com"
   },

   {
      "name": "Jonathan Ong",
      "email": "me@jongleberry.com"
   },

   {
      "name": "Roman Shtylman",
      "email": "shtylman+expressjs@gmail.com"
   },

   {
      "name": "Young Jae Sim",
      "email": "hanul@hanul.me"
   } ],
   "license": "MIT", "repository": {
      "type": "git",
      "url": "https://github.com/strongloop/express"
   },
   "homepage": "https://expressjs.com/", "keywords": [
      "express",
      "framework",
      "sinatra",
      "web",
      "rest",
      "restful",
      "router",
      "app",
      "api"
   ],
   "dependencies": {
      "accepts": "~1.2.3",
      "content-disposition": "0.5.0",
      "cookie-signature": "1.0.5",
      "debug": "~2.1.1",
      "depd": "~1.0.0",
      "escape-html": "1.0.1",
      "etag": "~1.5.1",
      "finalhandler": "0.3.3",
      "fresh": "0.2.4",
      "media-typer": "0.3.0",
      "methods": "~1.1.1",
      "on-finished": "~2.2.0",
      "parseurl": "~1.3.0",
      "path-to-regexp": "0.1.3",
      "proxy-addr": "~1.0.6",
      "qs": "2.3.3",
      "range-parser": "~1.0.2",
      "send": "0.11.1",
      "serve-static": "~1.8.1",
      "type-is": "~1.5.6",
      "vary": "~1.0.0",
      "cookie": "0.1.2",
      "merge-descriptors": "0.0.2",
      "utils-merge": "1.0.0"
   },
   "devDependencies": {
      "after": "0.8.1",
      "ejs": "2.1.4",
      "istanbul": "0.3.5",
      "marked": "0.3.3",
      "mocha": "~2.1.0",
      "should": "~4.6.2",
      "supertest": "~0.15.0",
      "hjs": "~0.0.6",
      "body-parser": "~1.11.0",
      "connect-redis": "~2.2.0",
      "cookie-parser": "~1.3.3",
      "express-session": "~1.10.2",
      "jade": "~1.9.1",
      "method-override": "~2.3.1",
      "morgan": "~1.5.1",
      "multiparty": "~4.1.1",
      "vhost": "~3.0.0"
   },
   "engines": {
      "node": ">= 0.10.0"
   },
   "files": [
      "LICENSE",
      "History.md",
      "Readme.md",
      "index.js",
      "lib/"
   ],
   "scripts": {
      "test": "mocha --require test/support/env
         --reporter spec --bail --check-leaks test/ test/acceptance/",
      "test-cov": "istanbul cover node_modules/mocha/bin/_mocha
         -- --require test/support/env --reporter dot --check-leaks test/ test/acceptance/",
      "test-tap": "mocha --require test/support/env
         --reporter tap --check-leaks test/ test/acceptance/",
      "test-travis": "istanbul cover node_modules/mocha/bin/_mocha
         --report lcovonly -- --require test/support/env
         --reporter spec --check-leaks test/ test/acceptance/"
   },
   "gitHead": "63ab25579bda70b4927a179b580a9c580b6c7ada",
   "bugs": {
      "url": "https://github.com/strongloop/express/issues"
   },
   "_id": "express@4.11.2",
   "_shasum": "8df3d5a9ac848585f00a0777601823faecd3b148",
   "_from": "express@*",
   "_npmVersion": "1.4.28",
   "_npmUser": {
      "name": "dougwilson",
      "email": "doug@somethingdoug.com"
   },
   "maintainers": [
      {
         "name": "tjholowaychuk",
         "email": "tj@vision-media.ca"
      },
      {
         "name": "jongleberry",
         "email": "jonathanrichardong@gmail.com"
      },
      {
         "name": "shtylman",
         "email": "shtylman@gmail.com"
      },
      {
         "name": "dougwilson",
         "email": "doug@somethingdoug.com"
      },
      {
         "name": "aredridel",
         "email": "aredridel@nbtsc.org"
      },
      {
         "name": "strongloop",
         "email": "callback@strongloop.com"
      },
      {
         "name": "rfeng",
         "email": "enjoyjava@gmail.com"
      }
   ],
   "dist": {
      "shasum": "8df3d5a9ac848585f00a0777601823faecd3b148",
      "tarball": "https://registry.npmjs.org/express/-/express-4.11.2.tgz"
   },
   "directories": {},
      "_resolved": "https://registry.npmjs.org/express/-/express-4.11.2.tgz",
      "readme": "ERROR: No README data found!"
}
Attributes of Package.json
name − name of the package
version − version of the package
description − description of the package
homepage − homepage of the package
author − author of the package
contributors − name of the contributors to the package
dependencies − list of dependencies. NPM automatically installs all the dependencies mentioned here in the node_module folder of the package. repository − repository type and URL of the package
main − entry point of the package
keywords − keywords
Uninstalling a Module
Use the following command to uninstall a Node.js module.

$ npm uninstall express
Once NPM uninstalls the package, you can verify it by looking at the content of /node_modules/ directory or type the following command −

$ npm ls
Updating a Module
Update package.json and change the version of the dependency to be updated and run the following command.

$ npm update express
Search a Module
Search a package name using NPM.

$ npm search express
Create a Module
Creating a module requires package.json to be generated. Let's generate package.json using NPM, which will generate the basic skeleton of the package.json.

$ npm init

This utility will walk you through creating a package.json file.
It only covers the most common items, and tries to guess sane defaults.

See 'npm help json' for definitive documentation on these fields
and exactly what they do.

Use 'npm install <pkg> --save' afterwards to install a package and
save it as a dependency in the package.json file.

Press ^C at any time to quit.
name: (webmaster)
You will need to provide all the required information about your module. You can take help from the above-mentioned package.json file to understand the meanings of various information demanded. Once package.json is generated, use the following command to register yourself with NPM repository site using a valid email address.

$ npm adduser
Username: mcmohd
Password:
Email: (this IS public) mcmohd@gmail.com
It is time now to publish your module −

$ npm publish
If everything is fine with your module, then it will be published in the repository and will be accessible to install using NPM like any other Node.js module.

\section{Command Line}

Pass command line arguments
The arguments are stored in process.argv

Here are the node docs on handling command line args:

 process.argv is an array containing the command line arguments. The first element will be 'node', the second element will be the name of the JavaScript file. The next elements will be any additional command line arguments.

// print process.argv
process.argv.forEach(function (val, index, array) {
  console.log(index + ': ' + val);
});
This will generate:

$ node process-2.js one two=three four
0: node
1: /Users/mjr/work/node/process-2.js
2: one
3: two=three
4: four





%\input{programming/octave.tex}
%\chapter{Toolbox}

View online \href{http://magizbox.com/training/toolbox/site/}{http://magizbox.com/training/toolbox/site/}

Toolbox by MG
The Toolbox contains all the little tools you never know where to find.

Text Editor
Vim : Vim is a clone of Bill Joy's vi text editor program for Unix. It was written by Bram Moolenaar based on source for a port of the Stevie editor to the Amiga and first released publicly in 1991. Vim is designed for use both from a command-line interface and as a standalone application in a graphical user interface. Vim is free and open source software and is released under a license that includes some charityware clauses, encouraging users who enjoy the software to consider donating to children in Uganda. The license is compatible with the GNU General Public License. Although it was originally released for the Amiga, Vim has since been developed to be cross-platform, supporting many other platforms. In 2006, it was voted the most popular editor amongst Linux Journal readers; in 2015 the Stack Overflow developer survey found it to be the third most popular text editor; and in 2016 the Stack Overflow developer survey found it to be the fourth most popular development environment.

Virtual Machine
VirtualBox : Oracle VM VirtualBox (formerly Sun VirtualBox, Sun xVM VirtualBox and Innotek VirtualBox) is a free and open-source hypervisor for x86 computers currently being developed by Oracle Corporation. Developed initially by Innotek GmbH, it was acquired by Sun Microsystems in 2008 which was in turn acquired by Oracle in 2010. VirtualBox may be installed on a number of host operating systems, including: Linux, macOS, Windows, Solaris, and OpenSolaris. There are also ports to FreeBSD and Genode. It supports the creation and management of guest virtual machines running versions and derivations of Windows, Linux, BSD, OS/2, Solaris, Haiku, OSx86 and others, and limited virtualization of macOS guests on Apple hardware. For some guest operating systems, a "Guest Additions" package of device drivers and system applications is available which typically improves performance, especially of graphics.

VMWare : VMware, Inc. is a subsidiary of Dell Technologies that provides cloud computing and platform virtualization software and services. It was the first commercially successful company to virtualize the x86 architecture. VMware's desktop software runs on Microsoft Windows, Linux, and macOS, while its enterprise software hypervisor for servers, VMware ESXi, is a bare-metal hypervisor that runs directly on server hardware without requiring an additional underlying operating system.

\section{Vim}

Vim
Running Vim for the First Time
To start Vim, enter this command:

gvim file.txt
In UNIX you can type this at any command prompt. If you are running Microsoft Windows, open an MS-DOS prompt window and enter the command. In either case, Vim starts editing a file called file.txt. Because this is a new file, you get a blank window. This is what your screen will look like:

+---------------------------------------+
|#                  |
|~                  |
|~                  |
|~                  |
|~                  |
|"file.txt" [New file]          |
+---------------------------------------+
    ('#" is the cursor position.)
The tilde (~) lines indicate lines not in the file. In other words, when Vim runs out of file to display, it displays tilde lines. At the bottom of the screen, a message line indicates the file is named file.txt and shows that you are creating a new file. The message information is temporary and other information overwrites it.

THE VIM COMMAND

The gvim command causes the editor to create a new window for editing. If you use this command:

vim file.txt
the editing occurs inside your command window. In other words, if you are running inside an xterm, the editor uses your xterm window. If you are using an MS-DOS command prompt window under Microsoft Windows, the editing occurs inside this window. The text in the window will look the same for both versions, but with gvim you have extra features, like a menu bar. More about that later.

Inserting text
The Vim editor is a modal editor. That means that the editor behaves differently, depending on which mode you are in. The two basic modes are called Normal mode and Insert mode. In Normal mode the characters you type are commands. In Insert mode the characters are inserted as text. Since you have just started Vim it will be in Normal mode. To start Insert mode you type the "i" command (i for Insert). Then you can enter the text. It will be inserted into the file. Do not worry if you make mistakes; you can correct them later. To enter the following programmer's limerick, this is what you type:

iA very intelligent turtle
Found programming UNIX a hurdle
After typing "turtle" you press the key to start a new line. Finally you press the key to stop Insert mode and go back to Normal mode. You now have two lines of text in your Vim window:

+---------------------------------------+
|A very intelligent turtle      |
|Found programming UNIX a hurdle    |
|~                  |
|~                  |
|                   |
+---------------------------------------+
WHAT IS THE MODE?

To be able to see what mode you are in, type this command:

:set showmode
You will notice that when typing the colon Vim moves the cursor to the last line of the window. That's where you type colon commands (commands that start with a colon). Finish this command by pressing the <Enter> key (all commands that start with a colon are finished this way). Now, if you type the "i" command Vim will display --INSERT-- at the bottom of the window. This indicates you are in Insert mode.

+---------------------------------------+
|A very intelligent turtle      |
|Found programming UNIX a hurdle    |
|~                  |
|~                  |
|-- INSERT --               |
+---------------------------------------+
If you press <Esc> to go back to Normal mode the last line will be made blank.

GETTING OUT OF TROUBLE

One of the problems for Vim novices is mode confusion, which is caused by forgetting which mode you are in or by accidentally typing a command that switches modes. To get back to Normal mode, no matter what mode you are in, press the key. Sometimes you have to press it twice. If Vim beeps back at you, you already are in Normal mode.

==============================================================================

Moving around
After you return to Normal mode, you can move around by using these keys:

h   left                        *hjkl*
j   down
k   up
l   right
At first, it may appear that these commands were chosen at random. After all, who ever heard of using l for right? But actually, there is a very good reason for these choices: Moving the cursor is the most common thing you do in an editor, and these keys are on the home row of your right hand. In other words, these commands are placed where you can type them the fastest (especially when you type with ten fingers).

Note:
You can also move the cursor by using the arrow keys.  If you do,
however, you greatly slow down your editing because to press the arrow
keys, you must move your hand from the text keys to the arrow keys.
Considering that you might be doing it hundreds of times an hour, this
can take a significant amount of time.
   Also, there are keyboards which do not have arrow keys, or which
locate them in unusual places; therefore, knowing the use of the hjkl
keys helps in those situations.
One way to remember these commands is that h is on the left, l is on the right and j points down. In a picture:

           k
       h     l
         j
The best way to learn these commands is by using them. Use the "i" command to insert some more lines of text. Then use the hjkl keys to move around and insert a word somewhere. Don't forget to press to go back to Normal mode. The |vimtutor| is also a nice way to learn by doing.

For Japanese users, Hiroshi Iwatani suggested using this:

        Komsomolsk
            ^
            |
   Huan Ho  <--- --->  Los Angeles
(Yellow river)      |
            v
          Java (the island, not the programming language)
==============================================================================

Deleting characters
To delete a character, move the cursor over it and type "x". (This is a throwback to the old days of the typewriter, when you deleted things by typing xxxx over them.) Move the cursor to the beginning of the first line, for example, and type xxxxxxx (seven x's) to delete "A very ". The result should look like this:

+---------------------------------------+
|intelligent turtle         |
|Found programming UNIX a hurdle    |
|~                  |
|~                  |
|                   |
+---------------------------------------+
Now you can insert new text, for example by typing:

iA young <Esc>
This begins an insert (the i), inserts the words "A young", and then exits insert mode (the final ). The result:

+---------------------------------------+
|A young intelligent turtle     |
|Found programming UNIX a hurdle    |
|~                  |
|~                  |
|                   |
+---------------------------------------+
DELETING A LINE

To delete a whole line use the "dd" command. The following line will then move up to fill the gap:

+---------------------------------------+
|Found programming UNIX a hurdle    |
|~                  |
|~                  |
|~                  |
|                   |
+---------------------------------------+
DELETING A LINE BREAK

In Vim you can join two lines together, which means that the line break between them is deleted. The "J" command does this. Take these two lines:

A young intelligent
turtle
Move the cursor to the first line and press "J":

A young intelligent turtle
==============================================================================

Undo and Redo
Suppose you delete too much. Well, you can type it in again, but an easier way exists. The "u" command undoes the last edit. Take a look at this in action: After using "dd" to delete the first line, "u" brings it back. Another one: Move the cursor to the A in the first line:

A young intelligent turtle
Now type xxxxxxx to delete "A young". The result is as follows:

 intelligent turtle
Type "u" to undo the last delete. That delete removed the g, so the undo restores the character.

g intelligent turtle
The next u command restores the next-to-last character deleted:

ng intelligent turtle
The next u command gives you the u, and so on:

ung intelligent turtle
oung intelligent turtle
young intelligent turtle
 young intelligent turtle
A young intelligent turtle

Note:
If you type "u" twice, and the result is that you get the same text
back, you have Vim configured to work Vi compatible.  Look here to fix
this: |not-compatible|.
   This text assumes you work "The Vim Way".  You might prefer to use
the good old Vi way, but you will have to watch out for small
differences in the text then.
REDO

If you undo too many times, you can press CTRL-R (redo) to reverse the preceding command. In other words, it undoes the undo. To see this in action, press CTRL-R twice. The character A and the space after it disappear:

young intelligent turtle
There's a special version of the undo command, the "U" (undo line) command. The undo line command undoes all the changes made on the last line that was edited. Typing this command twice cancels the preceding "U".

A very intelligent turtle
  xxxx              Delete very

A intelligent turtle
          xxxxxx        Delete turtle

A intelligent
                Restore line with "U"
A very intelligent turtle
                Undo "U" with "u"
A intelligent
The "U" command is a change by itself, which the "u" command undoes and CTRL-R redoes. This might be a bit confusing. Don't worry, with "u" and CTRL-R you can go to any of the situations you had. More about that in section |32.2|.

Reference: http://vimdoc.sourceforge.net/htmldoc/usr_02.html

\section{Virtual Box}

Virtual Box
Export and Import VirtualBox VM images?
Export
Open VirtualBox and enter into the File option to choice Export Appliance...



You will then get an assistance window to help you generating the image.

Select the VM to export
Enter the output file path and name


You can choice a format, which I always leave the default OVF 1.

Finally you can write metadata like Version and Description
Now you have an OVA file that you can carry to whatever machine to use it.

Import
Open VirtualBox and enter into the File option to choice Import

You will then get an assistance window to help you loading the image.

Enter the path to the file that you have previously exported


Then you can modify the settings of the VM like RAM size, CPU, etc.


My recommendation on this is to enable the Reinitialize the MAC address of all the network cards option

Press Import and done!
Now you have cloned the VM from the host machine into another one

Reference: https://askubuntu.com/questions/588426/how-to-export-and-import-virtualbox-vm-images

Install Guest Additions
Guest Additions installs on the guest system and includes device drivers and system applications that optimize performance of the machine. Launch the guest OS in VirtualBox and click on Devices and Install Guest Additions.



The AutoPlay window opens on the guest OS and click on the Run VBox Windows Additions executable.



Click yes when the UAC screen comes up.



Now simply follow through the installation wizard.



During the installation wizard you can choose the Direct3D acceleration if you would like it. Remember this is going to take up more of your Host OS’s resources and is still experimental possibly making the guest unstable.



When the installation starts you will need to allow the Sun display adapters to be installed.



After everything has completed a reboot is required.

\section{VMWare}

VMWare
VMware Workstation is a program that allows you to run a virtual computer within your physical computer. The virtual computer runs as if it was its own machine. A virtual machine is great for trying out new operating systems such as Linux, visiting websites you don't trust, creating a computing environment specifically for children, testing the effects of computer viruses, and much more. You can even print and plug in USB drives. Read this guide to get the most out of VMware Workstation.

Installing VMware Workstation


1. Make sure your computer meets the system requirements. Because you will be running an operating system from within your own operating system, VMware Workstation has fairly high system requirements. If you don’t meet these, you may not be able to run VMware effectively. You must have a 64-bit processor. VMware supports Windows and Linux operating systems. You must have enough memory to run your operating system, the virtual operating system, and any programs inside that operating system. 1 GB is the minimum, but 3 or more is recommended. You must have a 16-bit or 32-bit display adapter. 3D effects will most likely not work well inside the virtual operating system, so gaming is not always efficient. You need at least 1.5 GB of free space to install VMware Workstation, along with at least 1 GB per operating system that you install.



2. Download the VMware software. You can download the VMware installer from the Download Center on the VMware website. Select the newest version and click the link for the installer. You will need to login with your VMware username. You will be asked to read and review the license agreement before you can download the file. You can only have one version of VMware Workstation installed at a time.



3. Install VMware Workstation. Once you have downloaded the file, right-click on the file and select “Run as administrator”. You will be asked to review the license again. Most users can use the Typical installation option. At the end of the installation, you will be prompted for your license key. Once the installation is finished, restart the computer. Part

Installing an Operating System


1. Open VMware. Installing a virtual operating system is much like installing it on a regular PC. You will need to have the installation disc or ISO image as well as any necessary licenses for the operating system that you want to install.

You can install most distributions of Linux as well as any version of Windows.


2. Click File. Select New Virtual Machine and then choose Typical. VMware will prompt you for the installation media. If it recognizes the operating system, it will enable Easy Installation:

Physical disc – Insert the installation disc for the operating system you want to install and then select the drive in VMware.
ISO image – Browse to the location of the ISO file on your computer.
Install operating system later. This will create a blank virtual disk. You will need to manually install the operating system later.


3. Enter in the details for the operating system. For Windows and other licensed operating systems, you will need to enter your product key. You will also need to enter your preferred username and a password if you want one. * If you are not using Easy Install, you will need to browse the list for the operating system you are installing.



4. Name your virtual machine. The name will help you identify it on your physical computer. It will also help distinguish between multiple virtual computers running different operating systems.



5. Set the disk size. You can allocate any amount of free space on your computer to the virtual machine to act as the installed operating system’s hard drive. Make sure to set enough to install any programs that you want to run in the virtual machine.



6. Customize your virtual machine’s virtual hardware. You can set the virtual machine to emulate specific hardware by clicking the “Customize Hardware” button. This can be useful if you are trying to run an older program that only supports certain hardware. Setting this is optional.



7. Set the virtual machine to start. Check the box labeled “Power on this virtual machine after creation” if you want the virtual machine to start up as soon as you finish making it. If you don’t check this box, you can select your virtual machine from the list in VMware and click the Power On button.



8. Wait for your installation to complete. Once you’ve powered on the virtual machine for the first time, the operating system will begin to install automatically. If you provided all of the correct information during the setup of the virtual machine, then you should not have to do anything. If you didn’t enter your product key or create a username during the virtual machine setup, you will most likely be prompted during the installation of the operating system.



9. Check that VMware Tools is installed. Once the operating system is installed, the program VMware Tools should be automatically installed. Check that it appears on the desktop or in the program files for the newly installed operating system.

VMware tools are configuration options for your virtual machine, and keeps your virtual machine up to date with any software changes.

Navigating VMware


1. Start a virtual machine. To start a virtual machine, click the VM menu and select the virtual machine that you want to turn on. You can choose to start the virtual machine normally, or boot directly to the virtual BIOS.



2. Stop a virtual machine. To stop a virtual machine, select it and then click the VM menu. Select the Power option.

Power Off – The virtual machine turns off as if the power was cut out.
Shut Down Guest – This sends a shutdown signal to the virtual machine which causes the virtual machine to shut down as if you had selected the shutdown option.
You can also turn off the virtual machine by using the shutdown option in the virtual operating system.


3. Move files between the virtual machine and your physical computer. Moving files between your computer and the virtual machine is as simple as dragging and dropping. Files can be moved in both directions between the computer and the virtual machine, and can also be dragged from one virtual machine to another.

When you drag and drop, the original will stay in the original location and a copy will be created in the new location.
You can also move files by copying and pasting.
Virtual machines can connect to shared folders as well.


4. Add a printer to your virtual machine. You can add any printer to your virtual machine without having to install any extra drivers, as long as it is already installed on your host computer.

Select the virtual machine that you want to add the printer to.
Click the VM menu and select Settings.
Click the Hardware tab, and then click Add. This will start the Add Hardware wizard.
Select Printer and then click Finish. Your virtual printer will be enabled the next time you turn the virtual machine on.


5. Connect a USB drive to the virtual machine. Virtual machines can interact with a USB drive the same way that your normal operating system does. The USB drive cannot be accessed on both the host computer and the virtual machine at the same time.

If the virtual machine is the active window, the USB drive will be automatically connected to the virtual machine when it is plugged in.
If the virtual machine is not the active window or is not running, select the virtual machine and click the VM menu. Select Removable Devices and then click Connect. The USB drive will automatically connect to your virtual machine.


6. Take a snapshot of a virtual machine. A snapshot is a saved state and will allow you to load the virtual machine to that precise moment as many times as you need.

Select your virtual machine, click the VM menu, hover over Snapshot and select Take Snapshot.
Give your Snapshot a name. You can also give it a description, though this is optional.
Click OK to save the Snapshot.
Load a saved Snapshot by clicking the VM menu and then selecting Snapshot. Choose the Snapshot you wish to load from the list and click Go To.


7. Become familiar with keyboard shortcuts. A combination of the "Ctrl" and other keys are used to navigate virtual machines. For example, "Ctrl," "Alt" and "Enter" puts the current virtual machine in full screen mode or moves through multiple machines. "Ctrl," "Alt" and "Tab" will move between virtual machines when the mouse is being used by 1 machine.


%  \part{Toán}

\chapter{Xác suất}

\diary{19/01/2018 Nay tìm được khóa \href{https://web.stanford.edu/class/archive/cs/cs109/cs109.1166/ppt/}{CS 109} của bác \href{https://web.stanford.edu/class/archive/cs/cs109/cs109.1166/}{Chris Piech} quá hay}

Xác suất là công cụ để mô hình hóa các sự vật tưởng như ngẫu nhiên.


\section{Probability Distributions}

When we use the word “probability” in day-to-day life, we refer to a degree of confidence that an event of an uncertain nature will occur. For example, the weather report might say “there is a low probability of light rain in the afternoon.” Probability theory deals with the formal foundations for discussing such estimates and the rules they should obey. Before we discuss the representation of probability, we need to define what the events are to which we want to assign a probability. These events might be different outcomes of throwing a die, the outcome of a horse race, the weather configurations in California, or the possible failures of a piece of machinery.

\subsection{Event Spaces}

event Formally, we define events by assuming that there is an agreed upon space of possible outcomes, outcome space which we denote by Ω. For example, if we consider dice, we might set Ω = {1, 2, 3, 4, 5, 6}. In the case of a horse race, the space might be all possible orders of arrivals at the finish line, a much larger space.

measurable event In addition, we assume that there is a set of measurable events S to which we are willing to assign probabilities. Formally, each event α ∈ S is a subset of Ω. In our die example, the event {6} represents the case where the die shows 6, and the event {1, 3, 5} represents the case of an odd outcome. In the horse-race example, we might consider the event “Lucky Strike wins,” which contains all the outcomes in which the horse Lucky Strike is first. Probability theory requires that the event space satisfy three basic properties: • It contains the empty event ∅, and the trivial event Ω. • It is closed under union. That is, if α, β ∈ S, then so is α ∪ β. • It is closed under complementation. That is, if α ∈ S, then so is Ω − α. The requirement that the event space is closed under union and complementation implies that it is also closed under other Boolean operations, such as intersection and set dierence.

\subsection{Probability Distributions}

Definition 2.1 A probability distribution P over (Ω, S) is a mapping from events in S to real values that satisfies probability distribution the following conditions: • P(α) ≥ 0 for all α ∈ S. • P(Ω) = 1. • If α, β ∈ S and α ∩ β = ∅, then P(α ∪ β) = P(α) + P(β). The first condition states that probabilities are not negative. The second states that the “trivial event,” which allows all possible outcomes, has the maximal possible probability of 1. The third condition states that the probability that one of two mutually disjoint events will occur is the sum of the probabilities of each event. These two conditions imply many other conditions. Of particular interest are P(∅) = 0, and P(α ∪ β) = P(α) + P(β) − P(α ∩ β).


1.3 Interpretations of Probability
Before we continue to discuss probability distributions, we need to consider the interpretations that we might assign to them. Intuitively, the probability P(α) of an event α quantifies the degree of confidence that α will occur. If P(α) = 1, we are certain that one of the outcomes in α occurs, and if P(α) = 0, we consider all of them impossible. Other probability values represent options that lie between these two extremes. This description, however, does not provide an answer to what the numbers mean. There are two common interpretations for probabilities. frequentist The frequentist interpretation views probabilities as frequencies of events. More precisely, the interpretation probability of an event is the fraction of times the event occurs if we repeat the experiment indefinitely. For example, suppose we consider the outcome of a particular die roll. In this case, the statement P(α) = 0.3, for α = {1, 3, 5}, states that if we repeatedly roll this die and record the outcome, then the fraction of times the outcomes in α will occur is 0.3. More precisely, the limit of the sequence of fractions of outcomes in α in the first roll, the first two rolls, the first three rolls, . . ., the first n rolls, . . . is 0.3.

The frequentist interpretation gives probabilities a tangible semantics. When we discuss concrete physical systems (for example, dice, coin flips, and card games) we can envision how these frequencies are defined. It is also relatively straightforward to check that frequencies must satisfy the requirements of proper distributions. The frequentist interpretation fails, however, when we consider events such as “It will rain tomorrow afternoon.” Although the time span of “Tomorrow afternoon” is somewhat ill defined, we expect it to occur exactly once. It is not clear how we define the frequencies of such events. Several attempts have been made to define the probability for such an event by finding a reference class reference class of similar events for which frequencies are well defined; however, none of them has proved entirely satisfactory. Thus, the frequentist approach does not provide a satisfactory interpretation for a statement such as “the probability of rain tomorrow afternoon is 0.3.”  An alternative interpretation views probabilities as subjective degrees of belief. Under subjective interpretation this interpretation, the statement P(α) = 0.3 represents a subjective statement about one’s own degree of belief that the event α will come about. Thus, the statement “the probability of rain tomorrow afternoon is 50 percent” tells us that in the opinion of the speaker, the chances of rain and no rain tomorrow afternoon are the same. Although tomorrow afternoon will occur only once, we can still have uncertainty about its outcome, and represent it using numbers (that is, probabilities). This description still does not resolve what exactly it means to hold a particular degree of belief. What stops a person from stating that the probability that Bush will win the election is 0.6 and the probability that he will lose is 0.8? The source of the problem is that we need to explain how subjective degrees of beliefs (something that is internal to each one of us) are reflected in our actions. This issue is a major concern in subjective probabilities. One possible way of attributing degrees of beliefs is by a betting game. Suppose you believe that P(α) = 0.8. Then you would be willing to place a bet of \$1 against \$3. To see this, note that with probability 0.8 you gain a dollar, and with probability 0.2 you lose \$3, so on average this bet is a good deal with expected gain of 20 cents. In fact, you might be even tempted to place a bet of \$1 against \$4. Under this bet the average gain is 0, so you should not mind. However, you would not consider it worthwhile to place a bet \$1 against

    4 and 10 cents, since that would have negative expected gain. Thus, by finding which bets you are willing to place, we can assess your degrees of beliefs. The key point of this mental game is the following. If you hold degrees of belief that do not satisfy the rule of probability, then by a clever construction we can find a series of bets that would result in a sure negative outcome for you. Thus, the argument goes, a rational person must hold degrees of belief that satisfy the rules of probability.1 In the remainder of the book we discuss probabilities, but we usually do not explicitly state their interpretation. Since both interpretations lead to the same mathematical rules, the technical definitions hold for both interpretations.

\section{Basic Concepts in Probability}

\subsection{Conditional Probability}

To use a concrete example, suppose we consider a distribution over a population of students taking a certain course. The space of outcomes is simply the set of all students in the population. Now, suppose that we want to reason about the students’ intelligence and their final grade. We can define the event \alpha to denote “all students with grade A,” and the event β to denote “all students with high intelligence.” Using our distribution, we can consider the probability of these events, as well as the probability of α ∩ β (the set of intelligent students who got grade A). This, however, does not directly tell us how to update our beliefs given new evidence. Suppose we learn that a student has received the grade A; what does that tell us about her intelligence? This kind of question arises every time we want to use distributions to reason about the real world. More precisely, after learning that an event α is true, how do we change our probability conditional about β occurring? The answer is via the notion of conditional probability. Formally, the probability conditional probability of β given α is defined as P(β | α) = P(α ∩ β) P(α) (2.1) That is, the probability that β is true given that we know α is the relative proportion of outcomes satisfying β among these that satisfy α. (Note that the conditional probability is not defined when P(α) = 0.) The conditional probability given an event (say α) satisfies the properties of definition 2.1 (see exercise 2.4), and thus it is a probability distribution by its own right. Hence, we can think of the conditioning operation as taking one distribution and returning another over the same probability space.

\section{Chain Rule and Bayes Rule}

From the definition of the conditional distribution, we immediately see that P(α ∩ β) = P(α)P(β | α). (2.2) chain rule This equality is known as the chain rule of conditional probabilities. More generally, if α1, . . . , αk are events, then we can write P(α1 ∩ . . . ∩ αk) = P(α1)P(α2 | α1) · · · P(αk | α1 ∩ . . . ∩ αk−1). (2.3) In other words, we can express the probability of a combination of several events in terms of the probability of the first, the probability of the second given the first, and so on. It is important to notice that we can expand this expression using any order of events — the result will remain the same. Bayes’ rule Another immediate consequence of the definition of conditional probability is Bayes’ rule P(α | β) = P(β | α)P(α) P(β)

A more general conditional version of Bayes’ rule, where all our probabilities are conditioned on some background event γ, also holds: P(α | β ∩ γ) = P(β | α ∩ γ)P(α | γ) P(β | γ) . Bayes’ rule is important in that it allows us to compute the conditional probability P(α | β) from the “inverse” conditional probability P(β | α). Example 2.1 Consider the student population, and let Smart denote smart students and GradeA denote students who got grade A. Assume we believe (perhaps based on estimates from past statistics) that P(GradeA | Smart) = 0.6, and now we learn that a particular student received grade A. Can we estimate the probability that the student is smart? According to Bayes’ rule, this depends on prior our prior probability for students being smart (before we learn anything about them) and the prior probability of students receiving high grades. For example, suppose that P(Smart) = 0.3 and P(GradeA) = 0.2, then we have that P(Smart | GradeA) = 0.6 ∗ 0.3/0.2 = 0.9. That is, an A grade strongly suggests that the student is smart. On the other hand, if the test was easier and high grades were more common, say, P(GradeA) = 0.4 then we would get that P(Smart | GradeA) = 0.6 ∗ 0.3/0.4 = 0.45, which is much less conclusive about the student. Another classic example that shows the importance of this reasoning is in disease screening. To see this, consider the following hypothetical example (none of the mentioned figures are related to real statistics). Example 2.2 Suppose that a tuberculosis (TB) skin test is 95 percent accurate. That is, if the patient is TB-infected, then the test will be positive with probability 0.95, and if the patient is not infected, then the test will be negative with probability 0.95. Now suppose that a person gets a positive test result. What is the probability that he is infected? Naive reasoning suggests that if the test result is wrong 5 percent of the time, then the probability that the subject is infected is 0.95. That is, 95 percent of subjects with positive results have TB. If we consider the problem by applying Bayes’ rule, we see that we need to consider the prior probability of TB infection, and the probability of getting positive test result. Suppose that 1 in 1000 of the subjects who get tested is infected. That is, P(TB) = 0.001. What is the probability of getting a positive test result? From our description, we see that 0.001 · 0.95 infected subjects get a positive result, and 0.999·0.05 uninfected subjects get a positive result. Thus, P(Positive) = 0.0509. Applying Bayes’ rule, we get that P(TB | Positive) = 0.001·0.95/0.0509 ≈ 0.0187. Thus, although a subject with a positive test is much more probable to be TB-infected than is a random subject, fewer than 2 percent of these subjects are TB-infected.

\section{Random Variables and Joint Distributions}

\subsection{Motivation}

Our discussion of probability distributions deals with events. Formally, we can consider any event from the set of measurable events. The description of events is in terms of sets of outcomes. In many cases, however, it would be more natural to consider attributes of the outcome. For example, if we consider a patient, we might consider attributes such as “age,”

“gender,” and “smoking history” that are relevant for assigning probability over possible diseases and symptoms. We would like then consider events such as “age > 55, heavy smoking history, and suers from repeated cough.” To use a concrete example, consider again a distribution over a population of students in a course. Suppose that we want to reason about the intelligence of students, their final grades, and so forth. We can use an event such as GradeA to denote the subset of students that received the grade A and use it in our formulation. However, this discussion becomes rather cumbersome if we also want to consider students with grade B, students with grade C, and so on. Instead, we would like to consider a way of directly referring to a student’s grade in a clean, mathematical way. The formal machinery for discussing attributes and their values in dierent outcomes are random variable random variables. A random variable is a way of reporting an attribute of the outcome. For example, suppose we have a random variable Grade that reports the final grade of a student, then the statement P (Grade = A) is another notation for P (GradeA).

n the statement P (Grade = A) is another notation for P (GradeA).

\subsection{What Is a Random Variable?}

Formally, a random variable, such as Grade, is defined by a function that associates with each outcome in Ω a value. For example, Grade is defined by a function fGrade that maps each person in Ω to his or her grade (say, one of A, B, or C). The event Grade = A is a shorthand for the event {ω ∈ Ω : fGrade(ω) = A}. In our example, we might also have a random variable Intelligence that (for simplicity) takes as values either “high” or “low.” In this case, the event “Intelligence = high” refers, as can be expected, to the set of smart (high intelligence) students. Random variables can take dierent sets of values. We can think of categorical (or discrete) random variables that take one of a few values, as in our two preceding examples. We can also talk about random variables that can take infinitely many values (for example, integer or real values), such as Height that denotes a student’s height. We use Val(X) to denote the set of values that a random variable X can take. In most of the discussion in this book we examine either categorical random variables or random variables that take real values. We will usually use uppercase roman letters X, Y, Z to denote random variables. In discussing generic random variables, we often use a lowercase letter to refer to a value of a random variable. Thus, we use x to refer to a generic value of X. For example, in statements such as “P (X = x) ≥ 0 for all x ∈ Val(X).” When we discuss categorical random variables, we use the notation x1, . . . , xk, for k = |Val(X)| (the number of elements in Val(X)), when we need to enumerate the specific values of X, for example, in statements such as kX i =1 P (X = xi) = 1. multinomial The distribution over such a variable is called a multinomial. In the case of a binary-valued distribution random variable X, where Val(X) = {false, true}, we often use x1 to denote the value true for X, and x0 to denote the value false. The distribution of such a random variable is called a Bernoulli Bernoulli distribution. distribution We also use boldface type to denote sets of random variables. Thus, X, Y , or Z are typically used to denote a set of random variables, while x, y, z denote assignments of values to the

variables in these sets. We extend the definition of Val(X) to refer to sets of variables in the obvious way. Thus, x is always a member of Val(X). For Y ⊆ X, we use xhY i to refer to the assignment within x to the variables in Y . For two assignments x (to X) and y (to Y ), we say that x ∼ y if they agree on the variables in their intersection, that is, xhX ∩ Y i = yhX ∩ Y i. In many cases, the notation P(X = x) is redundant, since the fact that x is a value of X is already reported by our choice of letter. Thus, in many texts on probability, the identity of a random variable is not explicitly mentioned, but can be inferred through the notation used for its value. Thus, we use P(x) as a shorthand for P(X = x) when the identity of the random variable is clear from the context. Another shorthand notation is that Px refers to a sum over all possible values that X can take. Thus, the preceding statement will often appear as Px P(x) = 1. Finally, another standard notation has to do with conjunction. Rather than write P((X = x) ∩ (Y = y)), we write P(X = x, Y = y), or just P(x, y).

3.3 Marginal and Joint Distributions
Once we define a random variable X, we can consider the distribution over events that can be marginal described using X. This distribution is often referred to as the marginal distribution over the distribution random variable X. We denote this distribution by P(X). Returning to our population example, consider the random variable Intelligence. The marginal distribution over Intelligence assigns probability to specific events such as P(Intelligence = high) and P(Intelligence = low), as well as to the trivial event P(Intelligence ∈ {high, low}). Note that these probabilities are defined by the probability distribution over the original space. For concreteness, suppose that P(Intelligence = high) = 0.3, P(Intelligence = low) = 0.7. If we consider the random variable Grade, we can also define a marginal distribution. This is a distribution over all events that can be described in terms of the Grade variable. In our example, we have that P(Grade = A) = 0.25, P(Grade = B) = 0.37, and P(Grade = C) = 0.38. It should be fairly obvious that the marginal distribution is a probability distribution satisfying the properties of definition 2.1. In fact, the only change is that we restrict our attention to the subsets of S that can be described with the random variable X. In many situations, we are interested in questions that involve the values of several random variables. For example, we might be interested in the event “Intelligence = high and Grade = A.” joint distribution To discuss such events, we need to consider the joint distribution over these two random variables. In general, the joint distribution over a set X = {X1, . . . , Xn} of random variables is denoted by P(X1, . . . , Xn) and is the distribution that assigns probabilities to events that are specified in terms of these random variables. We use ξ to refer to a full assignment to the variables in X , that is, ξ ∈ Val(X). The joint distribution of two random variables has to be consistent with the marginal distribution, in that P(x) = P y P(x, y). This relationship is shown in figure 2.1, where we compute the marginal distribution over Grade by summing the probabilities along each row. Similarly, we find the marginal distribution over Intelligence by summing out along each column. The resulting sums are typically written in the row or column margins, whence the term “marginal distribution.” Suppose we have a joint distribution over the variables X = {X1, . . . , Xn}. The most fine-grained events we can discuss using these variables are ones of the form “X1 = x1 and X2 = x2, . . ., and Xn = xn” for a choice of values x1, . . . , xn for all the variables. Moreover,

Intelligence low high A 0.07 0.18 0.25 Grade B 0.28 0.09 0.37 C 0.35 0.03 0.38 0.7 0.3 1 Figure 2.1 Example of a joint distribution P(Intelligence, Grade): Values of Intelligence (columns) and Grade (rows) with the associated marginal distribution on each variable. any two such events must be either identical or disjoint, since they both assign values to all the variables in X . In addition, any event defined using variables in X must be a union of a set of canonical such events. Thus, we are eectively working in a canonical outcome space: a space where each outcome space outcome corresponds to a joint assignment to X1, . . . , Xn. More precisely, all our probability computations remain the same whether we consider the original outcome space (for example, all students), or the canonical space (for example, all combinations of intelligence and grade). atomic outcome We use ξ to denote these atomic outcomes: those assigning a value to each variable in X . For example, if we let X = {Intelligence, Grade}, there are six atomic outcomes, shown in figure 2.1. The figure also shows one possible joint distribution over these six outcomes. Based on this discussion, from now on we will not explicitly specify the set of outcomes and measurable events, and instead implicitly assume the canonical outcome space.

3.4 Conditional Probability
The notion of conditional probability extends to induced distributions over random variables. For conditional example, we use the notation P (Intelligence | Grade = A) to denote the conditional distribution distribution over the events describable by Intelligence given the knowledge that the student’s grade is A. Note that the conditional distribution over a random variable given an observation of the value of another one is not the same as the marginal distribution. In our example, P (Intelligence = high) = 0.3, and P (Intelligence = high | Grade = A) = 0.18/0.25 = 0.72. Thus, clearly P (Intelligence | Grade = A) is dierent from the marginal distribution P (Intelligence). The latter distribution represents our prior knowledge about students before learning anything else about a particular student, while the conditional distribution represents our more informed distribution after learning her grade. We will often use the notation P (X | Y ) to represent a set of conditional probability distributions. Intuitively, for each value of Y , this object assigns a probability over values of X using the conditional probability. This notation allows us to write the shorthand version of the chain rule: P (X, Y ) = P (X)P (Y | X), which can be extended to multiple variables as P (X1, . . . , Xk) = P (X1)P (X2 | X1) · · · P (Xk | X1, . . . , Xk−1). (2.5) Similarly, we can state Bayes’ rule in terms of conditional probability distributions: P (X | Y ) = P (X)P (Y | X) P (Y ) . (2.6)

\section{Independence and Conditional Independence}

4.1 Independence
As we mentioned, we usually expect P(α|β)P(α|β) to be different from P(α)P(α). That is, learning that ββ is true changes our probability over αα. However, in some situations equality can occur, so that P(α|β)=P(α)P(α|β)=P(α). That is, learning that ββ occurs did not change our probability of αα.

Definition independent events

We say that an event αα is independent of event ββ in PP, denoted P⊨(α⊥β)P⊨(α⊥β), if P(α|β)=P(α)P(α|β)=P(α) or if P(β)=0P(β)=0.

We can also provide an alternative definition for the concept of independence:

Proposition 2.1

A distribution PP satisfies (α⊥β)(α⊥β) if and only if P(α∩β)=P(α)P(β)P(α∩β)=P(α)P(β).

PROOF Consider first the case where P(β)=0P(β)=0; here, we also have P(α∩β)=0P(α∩β)=0, and so the equivalence immediately holds. When P(β)≠0P(β)≠0, we can use the chain rule; we write P(α∩β)=P(α|β)P(β)P(α∩β)=P(α|β)P(β). Since αα is independent of ββ, we have that P(α|β)=P(α)P(α|β)=P(α). Thus, P(α∩β)=P(α)P(β)P(α∩β)=P(α)P(β). Conversely, suppose that P(α∩β)=P(α)P(β)P(α∩β)=P(α)P(β). Then, by definition, we have that

P(α|β)=P(α∩β)P(β)=P(α)P(β)P(β)=P(α).
P(α|β)=P(α∩β)P(β)=P(α)P(β)P(β)=P(α).
As an immediate consequence of this alternative definition, we see that independence is a symmetric notion. That is, (α ⊥ β) implies (β ⊥ α). Example 2.3 For example, suppose that we toss two coins, and let α be the event “the first toss results in a head” and β the event “the second toss results in a head.” It is not hard to convince ourselves that we expect that these two events to be independent. Learning that β is true would not change our probability of α. In this case, we see two dierent physical processes (that is, coin tosses) leading to the events, which makes it intuitive that the probabilities of the two are independent. In certain cases, the same process can lead to independent events. For example, consider the event α denoting “the die outcome is even” and the event β denoting “the die outcome is 1 or 2.” It is easy to check that if the die is fair (each of the six possible outcomes has probability 1 6 ), then these two events are independent.

4.2 Conditional Independence
 While independence is a useful property, it is not often that we encounter two indepen- dent events. A more common situation is when two events are independent given an additional event. For example, suppose we want to reason about the chance that our student is accepted to graduate studies at Stanford or MIT. Denote by Stanford the event “admitted to Stanford” and by MIT the event “admitted to MIT.” In most reasonable distributions, these two events are not independent. If we learn that a student was admitted to Stanford, then our estimate of her probability of being accepted at MIT is now higher, since it is a sign that she is a promising student

Now, suppose that both universities base their decisions only on the student’s grade point average (GPA), and we know that our student has a GPA of A. In this case, we might argue that learning that the student was admitted to Stanford should not change the probability that she will be admitted to MIT: Her GPA already tells us the information relevant to her chances of admission to MIT, and finding out about her admission to Stanford does not change that. Formally, the statement is P(MIT | Stanford, GradeA) = P(MIT | GradeA). In this case, we say that MIT is conditionally independent of Stanford given GradeA. Definition 2.3 We say that an event α is conditionally independent of event β given event γ in P, denoted conditional independence P |= (α ⊥ β | γ), if P(α | β ∩ γ) = P(α | γ) or if P(β ∩ γ) = 0. It is easy to extend the arguments we have seen in the case of (unconditional) independencies to give an alternative definition. Proposition 2.2 P satisfies (α ⊥ β | γ) if and only if P(α ∩ β | γ) = P(α | γ)P(β | γ).

4.3 Independence of Random Variables
Until now, we have focused on independence between events. Thus, we can say that two events, such as one toss landing heads and a second also landing heads, are independent. However, we would like to say that any pair of outcomes of the coin tosses is independent. To capture such statements, we can examine the generalization of independence to sets of random variables. Definition 2.4 Let X, Y , Z be sets of random variables. We say that X is conditionally independent of Y given conditional independence Z in a distribution P if P satisfies (X = x ⊥ Y = y | Z = z) for all values x ∈ Val(X), y ∈ Val(Y ), and z ∈ Val(Z). The variables in the set Z are often said to be observed. If the set observed variable Z is empty, then instead of writing (X ⊥ Y | ∅), we write (X ⊥ Y ) and say that X and Y are marginally independent. marginal independence Thus, an independence statement over random variables is a universal quantification over all possible values of the random variables. The alternative characterization of conditional independence follows immediately: Proposition 2.3 The distribution P satisfies (X ⊥ Y | Z) if and only if P(X, Y | Z) = P(X | Z)P(Y | Z). Suppose we learn about a conditional independence. Can we conclude other independence properties that must hold in the distribution? We have already seen one such example: symmetry • Symmetry: (X ⊥ Y | Z) =⇒ (Y ⊥ X | Z). (2.7) There are several other properties that hold for conditional independence, and that often provide a very clean method for proving important properties about distributions. Some key properties are:

• Decomposition: (X ⊥ Y , W | Z) =⇒ (X ⊥ Y | Z). (2.8) weak union • Weak union: (X ⊥ Y , W | Z) =⇒ (X ⊥ Y | Z, W). (2.9) contraction • Contraction: (X ⊥ W | Z, Y )&(X ⊥ Y | Z) =⇒ (X ⊥ Y , W | Z). (2.10) An additional important property does not hold in general, but it does hold in an important subclass of distributions. Definition 2.5 A distribution P is said to be positive if for all events α ∈ S such that α 6= ∅, we have that positive distribution P(α) > 0. For positive distributions, we also have the following property: intersection • Intersection: For positive distributions, and for mutually disjoint sets X, Y , Z, W : (X ⊥ Y | Z, W)&(X ⊥ W | Z, Y ) =⇒ (X ⊥ Y , W | Z). (2.11) The proof of these properties is not dicult. For example, to prove Decomposition, assume that (X ⊥ Y, W | Z) holds. Then, from the definition of conditional independence, we have that P(X, Y, W | Z) = P(X | Z)P(Y, W | Z). Now, using basic rules of probability and arithmetic, we can show P(X, Y | Z) = X w P(X, Y, w | Z) = X w P(X | Z)P(Y, w | Z) = P(X | Z) X w P(Y, w | Z) = P(X | Z)P(Y | Z). The only property we used here is called “reasoning by cases” (see exercise 2.6). We conclude that (X ⊥ Y | Z).

\section{Querying a Distribution}

Our focus throughout this book is on using a joint probability distribution over multiple random variables to answer queries of interest.

\subsection{5.1 Probability Queries}

probability query Perhaps the most common query type is the probability query. Such a query consists of two parts: evidence • The evidence: a subset E of random variables in the model, and an instantiation e to these variables; query variables • the query variables: a subset Y of random variables in the network. Our task is to compute P(Y | E = e), posterior that is, the posterior probability distribution over the values y of Y , conditioned on the fact that distribution E = e. This expression can also be viewed as the marginal over Y , in the distribution we obtain by conditioning on e.

\subsection{5.2 MAP Queries}

A second important type of task is that of finding a high-probability joint assignment to some subset of variables. The simplest variant of this type of task is the MAP query (also called MAP assignment most probable explanation (MPE)), whose aim is to find the MAP assignment — the most likely assignment to all of the (non-evidence) variables. More precisely, if we let W = X − E, our task is to find the most likely assignment to the variables in W given the evidence E = e: MAP(W | e) = argmax w P(w, e), (2.12) where, in general, argmaxx f(x) represents the value of x for which f(x) is maximal. Note that there might be more than one assignment that has the highest posterior probability. In this case, we can either decide that the MAP task is to return the set of possible assignments, or to return an arbitrary member of that set. It is important to understand the dierence between MAP queries and probability queries. In a MAP query, we are finding the most likely joint assignment to W . To find the most likely assignment to a single variable A, we could simply compute P(A | e) and then pick the most likely value. However, the assignment where each variable individually picks its most  likely value can be quite dierent from the most likely joint assignment to all variables simultaneously. This phenomenon can occur even in the simplest case, where we have no evidence. Example 2.4 Consider a two node chain A → B where A and B are both binary-valued. Assume that: a0 a1 0.4 0.6 A b0 b1 a0 0.1 0.9 a1 0.5 0.5 (2.13) We can see that P(a1) > P(a0), so that MAP(A) = a1. However, MAP(A, B) = (a0, b1): Both values of B have the same probability given a1. Thus, the most likely assignment containing a1 has probability 0.6 × 0.5 = 0.3. On the other hand, the distribution over values of B is more skewed given a0, and the most likely assignment (a0, b1) has the probability 0.4 × 0.9 = 0.36. Thus, we have that argmaxa,b P(a, b) 6= (argmaxa P(a),argmaxb P(b)).

\subsection{5.3 Marginal MAP Queries}

To motivate our second query type, let us return to the phenomenon demonstrated in example 2.4. Now, consider a medical diagnosis problem, where the most likely disease has multiple possible symptoms, each of which occurs with some probability, but not an overwhelming probability. On the other hand, a somewhat rarer disease might have only a few symptoms, each of which is very likely given the disease. As in our simple example, the MAP assignment to the data and the symptoms might be higher for the second disease than for the first one. The solution here is to look for the most likely assignment to the disease variable(s) only, rather than the most likely assignment to both the disease and symptom variables. This approach suggests marginal MAP the use of a more general query type. In the marginal MAP query, we have a subset of variables Y that forms our query. The task is to find the most likely assignment to the variables in Y given the evidence E = e: MAP(Y | e) = arg max y P(y | e). If we let Z = X − Y − E, the marginal MAP task is to compute: MAP(Y | e) = arg max Y X Z P(Y , Z | e). Thus, marginal MAP queries contain both summations and maximizations; in a way, it contains elements of both a conditional probability query and a MAP query. Note that example 2.4 shows that marginal MAP assignments are not monotonic: the most likely assignment MAP(Y1 | e) might be completely dierent from the assignment to Y1 in MAP({Y1, Y2} | e). Thus, in particular, we cannot use a MAP query to give us the correct answer to a marginal MAP query.

\section{Continuous Spaces}
In the previous section, we focused on random variables that have a finite set of possible values. In many situations, we also want to reason about continuous quantities such as weight, height, duration, or cost that take real numbers in IR. When dealing with probabilities over continuous random variables, we have to deal with some technical issues. For example, suppose that we want to reason about a random variable X that can take values in the range between 0 and 1. That is, Val(X) is the interval [0, 1]. Moreover, assume that we want to assign each number in this range equal probability. What would be the probability of a number x? Clearly, since each x has the same probability, and there are infinite number of values, we must have that P(X = x) = 0. This problem appears even if we do not require uniform probability.

6.1 Probability Density Functions
How do we define probability over a continuous random variable? We say that a function density function p : IR 7→ IR is a probability density function or (PDF) for X if it is a nonnegative integrable

function such that Z Val(X) p(x)dx = 1. That is, the integral over the set of possible values of X is 1. The PDF defines a distribution for X as follows: for any x in our event space: P(X ≤ a) = aZ −∞ p(x)dx. cumulative The function P is the cumulative distribution for X. We can easily employ the rules of distribution probability to see that by using the density function we can evaluate the probability of other events. For example, P(a ≤ X ≤ b) = bZa p(x)dx. Intuitively, the value of a PDF p(x) at a point x is the incremental amount that x adds to the cumulative distribution in the integration process. The higher the value of p at and around x, the more mass is added to the cumulative distribution as it passes x. The simplest PDF is the uniform distribution. Definition 2.6 A variable X has a uniform distribution over [a, b], denoted X ∼ Unif[a,b] if it has the PDF uniform distribution p(x) =  0 otherwise b− 1a b ≥ x ≥ a. Thus, the probability of any subinterval of [a, b] is proportional its size relative to the size of [a, b]. Note that, if b − a < 1, then the density can be greater than 1. Although this looks unintuitive, this situation can occur even in a legal PDF, if the interval over which the value is greater than 1 is not too large. We have only to satisfy the constraint that the total area under the PDF is 1. As a more complex example, consider the Gaussian distribution. Definition 2.7 A random variable X has a Gaussian distribution with mean µ and variance σ2, denoted X ∼ Gaussian distribution N µ; σ2, if it has the PDF p(x) = √21πσ e− (x2 − σµ 2)2 . standard A standard Gaussian is one with mean 0 and variance 1. Gaussian A Gaussian distribution has a bell-like curve, where the mean parameter µ controls the location of the peak, that is, the value for which the Gaussian gets its maximum value. The variance parameter σ2 determines how peaked the Gaussian is: the smaller the variance, the


more peaked the Gaussian. Figure 2.2 shows the probability density function of a few dierent Gaussian distributions. More technically, the probability density function is specified as an exponential, where the expression in the exponent corresponds to the square of the number of standard deviations σ that x is away from the mean µ. The probability of x decreases exponentially with the square of its deviation from the mean, as measured in units of its standard deviation.

6.2 Joint Density Functions

The discussion of density functions for a single variable naturally extends for joint distributions of continuous random variables. Definition 2.8 Let P be a joint distribution over continuous random variables X1, . . . , Xn. A function p(x1, . . . , xn) joint density is a joint density function of X1, . . . , Xn if • p(x1, . . . , xn) ≥ 0 for all values x1, . . . , xn of X1, . . . , Xn. • p is an integrable function. • For any choice of a1, . . . , an, and b1, . . . , bn, P(a1 ≤ X1 ≤ b1, . . . , an ≤ Xn ≤ bn) = b1 Za1 · · · b nZ a n p(x1, . . . , xn)dx1 . . . dxn. Thus, a joint density specifies the probability of any joint event over the variables of interest. Both the uniform distribution and the Gaussian distribution have natural extensions to the multivariate case. The definition of a multivariate uniform distribution is straightforward. We defer the definition of the multivariate Gaussian to section 7.1. From the joint density we can derive the marginal density of any random variable by integrating out the other variables. Thus, for example, if p(x, y) is the joint density of X and Y

then p(x) = ∞ Z −∞ p(x, y)dy. To see why this equality holds, note that the event a ≤ X ≤ b is, by definition, equal to the event “a ≤ X ≤ b and −∞ ≤ Y ≤ ∞.” This rule is the direct analogue of marginalization for discrete variables. Note that, as with discrete probability distributions, we abuse notation a bit and use p to denote both the joint density of X and Y and the marginal density of X. In cases where the distinction is not clear, we use subscripts, so that pX will be the marginal density, of X, and pX,Y the joint density.

6.3 Conditional Density Functions
As with discrete random variables, we want to be able to describe conditional distributions of continuous variables. Suppose, for example, we want to define P(Y | X = x). Applying the definition of conditional distribution (equation (2.1)), we run into a problem, since P(X = x) = 0. Thus, the ratio of P(Y, X = x) and P(X = x) is undefined. To avoid this problem, we might consider conditioning on the event x −  ≤ X ≤ x + , which can have a positive probability. Now, the conditional probability is well defined. Thus, we might consider the limit of this quantity when  → 0. We define P(Y | x) = lim →0 P(Y | x −  ≤ X ≤ x + ). When does this limit exist? If there is a continuous joint density function p(x, y), then we can derive the form for this term. To do so, consider some event on Y , say a ≤ Y ≤ b. Recall that P(a ≤ Y ≤ b | x −  ≤ X ≤ x + ) = P(a ≤ Y ≤ b, x −  ≤ X ≤ x + ) P(x −  ≤ X ≤ x + ) = Ra b Rx x− + p(x0, y)dydx0 Rx x− + p(x0)dx0 . When  is suciently small, we can approximate x+ Z x− p(x0)dx0 ≈ 2p(x). Using a similar approximation for p(x0, y), we get P(a ≤ Y ≤ b | x −  ≤ X ≤ x + ) ≈ Ra b 2p(x, y)dy 2p(x) = bZa p(x, y) p(x) dy. We conclude that p(x,y) p(x) is the density of P(Y | X = x).

Let p(x, y) be the joint density of X and Y . The conditional density function of Y given X is conditional density function defined as p(y | x) = p(x, y) p(x) When p(x) = 0, the conditional density is undefined. The conditional density p(y | x) characterizes the conditional distribution P(Y | X = x) we defined earlier. The properties of joint distributions and conditional distributions carry over to joint and conditional density functions. In particular, we have the chain rule p(x, y) = p(x)p(y | x) (2.14) and Bayes’ rule p(x | y) = p(x)p(y | x) p(y) . (2.15) As a general statement, whenever we discuss joint distributions of continuous random variables, we discuss properties with respect to the joint density function instead of the joint distribution, as we do in the case of discrete variables. Of particular interest is the notion of (conditional) independence of continuous random variables. Definition 2.10 Let X, Y , and Z be sets of continuous random variables with joint density p(X, Y , Z). We say conditional that X is conditionally independent of Y given Z if independence p(x | z) = p(x | y, z) for all x, y, z such that p(z) > 0.

\section{Kì vọng và phương sai}

\subsection{Kì vọng}

\definition{kì vọng}{
Cho $X$ là một biến rời rạc nhận các giá trị số, khi đó kì vọng của $X$ dưới phân phối $P$ là

$$\mathbb{E}_P[X] = \sum_{x} x \cdot P(x).$$

Nếu $X$ là một biến liên tục, khi đó sử dụng hàm mật độ

$$\mathbb{E}_P[X] = \int x \cdot p(x).$$

Ví dụ, nếu $X$ là số chấm hiện ra khi tung một xúc sắc với xác suất xuất hiện mỗi mặt là $\frac{1}{6}$. Khi đó, kì vọng của $\mathbb{E} [X]= 1 \cdot \frac{1}{6} + 2 \cdot \frac{1}{6} + \cdots + 6 \cdot \frac{1}{6} = 3.5.$ Nếu trong trường hợp xúc sắc không cân bằng $P(X=6) = 0.5$ và $P(X=x)=0.1)$ với $x < 6$, khi đó, $\mathbb{E} [X] = 1 \cdot 0.1 + 2 \cdot 0.1 + \cdots + 5 \cdot 0.1 + 6 \cdots 0.5 = 4.5.$
}


Often we are interested in expectations of a function of a random variable (or several random variables). Thus, we might consider extending the definition to consider the expectation of a functional term such as X2 + 0.5X. Note, however, that any function g of a set of random variables X1, . . . , Xk is essentially defining a new random variable Y : For any outcome ω ∈ Ω, we define the value of Y as g(fX1(ω), . . . , fXk(ω)). Based on this discussion, we often define new random variables by a functional term. For example Y = X2, or Y = eX. We can also consider functions that map values of one or more categorical random variables to numerical values. One such function that we use quite often is indicator function the indicator function, which we denote 11{X = x}. This function takes value 1 when X = x, and 0 otherwise. In addition, we often consider expectations of functions of random variables without bothering to name the random variables they define. For example IEP [X + Y ]. Nonetheless, we should keep in mind that such a term does refer to an expectation of a random variable. We now turn to examine properties of the expectation of a random variable. First, as can be easily seen, the expectation of a random variable is a linear function in that random variable. Thus, IE[a · X + b] = aIE[X] + b. A more complex situation is when we consider the expectation of a function of several random variables that have some joint behavior. An important property of expectation is that the expectation of a sum of two random variables is the sum of the expectations. Proposition 2.4 IE[X + Y ] = IE[X] + IE[Y ]. linearity of This property is called linearity of expectation. It is important to stress that this identity is true expectation even when the variables are not independent. As we will see, this property is key in simplifying many seemingly complex problems. Finally, what can we say about the expectation of a product of two random variables? In general, very little: Example 2.5 Consider two random variables X and Y , each of which takes the value +1 with probability 1/2, and the value −1 with probability 1/2. If X and Y are independent, then IE[X · Y ] = 0. On the other hand, if X and Y are correlated in that they always take the same value, then IE[X · Y ] = 1. However, when X and Y are independent, then, as in our example, we can compute the expectation simply as a product of their individual expectations: Proposition 2.5 If X and Y are independent, then IE[X · Y ] = IE[X] · IE[Y ]. conditional We often also use the expectation given some evidence. The conditional expectation of X expectation given y is IEP [X | y] = X x x · P(x | y).

\subsection{Phương sai}

\definition{phương sai}{
Kì vọng của $X$ chỉ giá trị trung bình của $X$. Tuy nhiên, nó không chỉ sự khác nhau giữa các giá trị mà $X$ có thể nhận.

$$\mathbb{V}ar_P[X] = \mathbb{E}_P \left[(X-\mathbb{E}_P[X])^2\right].$$
}



 A measure of this deviation is the variance of X. VVarP [X] = IEP h(X − IEP [X])2i. Thus, the variance is the expectation of the squared dierence between X and its expected value. It gives us an indication of the spread of values of X around the expected value. An alternative formulation of the variance is VVar[X] = IEX2 − (IE[X])2 . (2.16) (see exercise 2.11). Similar to the expectation, we can consider the expectation of a functions of random variables. Proposition 2.6 If X and Y are independent, then VVar[X + Y ] = VVar[X] + VVar[Y ]. It is straightforward to show that the variance scales as a quadratic function of X. In particular, we have: VVar[a · X + b] = a2VVar[X]. For this reason, we are often interested in the square root of the variance, which is called the standard standard deviation of the random variable. We define deviation σX = pVVar[X]. The intuition is that it is improbable to encounter values of X that are farther than several standard deviations from the expected value of X. Thus, σX is a normalized measure of “distance” from the expected value of X. As an example consider the Gaussian distribution of definition 2.7. Proposition 2.7 Let X be a random variable with Gaussian distribution N(µ, σ2), then IE[X] = µ and VVar[X] = σ2. Thus, the parameters of the Gaussian distribution specify the expectation and the variance of the distribution. As we can see from the form of the distribution, the density of values of X drops exponentially fast in the distance x−µ σ . Not all distributions show such a rapid decline in the probability of outcomes that are distant from the expectation. However, even for arbitrary distributions, one can show that there is a decline. Theorem 2.1 (Chebyshev inequality): Chebyshev’s inequality P (|X − IEP [X]| ≥ t) ≤ VVarP [X] t2 .

We can restate this inequality in terms of standard deviations: We write t = kσX to get P(|X − IEP [X]| ≥ kσX) ≤ 1 k2. Thus, for example, the probability of X being more than two standard deviations away from IE[X] is less than 1/4.

\section{Các hàm phân phối thông dụng}

Phần này có thêm khảo \cite{Goodfellow-et-al-2016} và giáo trình xác suất thống kê của thạc sỹ Trần Thiện Khải, đại học Trà Vinh  \footnote{\href{http://www.ctec.tvu.edu.vn/ttkhai/xacsuatthongke_dh.htm}{http://www.ctec.tvu.edu.vn/ttkhai/xacsuatthongke\_dh.htm}}

\diary{17/01/2018 Lòng vòng thế nào hôm nay lại tìm được  \href{https://dominhhai.github.io/vi/2017/10/prob-com-var}{blog của bạn Đỗ Minh Hải}, rất hay
}
\subsection{Biến rời rạc}

Tổng kết các phân phối rời rạc tham khảo slide 3, chapter 9 của khóa CS 109 \footnote{\href{https://web.stanford.edu/class/archive/cs/cs109/cs109.1166/ppt/9-ContinuousDist.pdf}{Slide 3, chapter 9, CS109 Stanford}}

\textbf{Phân phối Bernoulli:}

\begin{itemize}
  \item xác suất xuất hiện mặt ngửa $X \sim Ber(p)$
\end{itemize}

\textbf{Phân phối Bernoulli:}

\begin{itemize}
  \item xác suất xuất hiện mặt ngửa $X \sim Ber(p)$
\end{itemize}

\textbf{Phân phối Binomial:}

\begin{itemize}
  \item xác suất thành công n lần $X \sim B(n,p)$
\end{itemize}

\textbf{Phân phối Poisson:}

\begin{itemize}
  \item xác suất thành công n lần $X \sim Poi(\lambda)$
\end{itemize}

\textbf{Phân phối Geometric:}

\begin{itemize}
  \item số lần thử đến khi thành công $X \sim Geo(p)$
\end{itemize}

\textbf{Phân phối Negative Binomial:}

\begin{itemize}
  \item số lần thử đến khi $r$ thành công $X \sim NegBin(r,p)$
\end{itemize}

\textbf{Phân phối Hyper Geometric:}

\begin{itemize}
  \item số bóng trắng lấy được trong $N$ bóng chứa $m$ bóng trắng  $X \sim HypG(n, N, m)$
\end{itemize}

\subsection{Biến ngẫu nhiên Bernoulli}

\textbf{Phép thử Bernoulli} là một phép thử chỉ có hai khả năng xảy ra "thành công" hoặc "thất bại". Trong thực tế, các ví dụ của phép thử Bernoulli xuất hiện rất phổ biến như \textit{tung đồng xu}, \textit{sự kiện một ổ đĩa bị hỏng}, \textit{sự kiện một ai đó thích xem một bộ phim trên Netflix}.
iên
Định nghĩa X là một biến ngẫu nhiên, nhận giá trị bằng 1 nếu sự kiên thành công, nhận giá trị bằng 0 nếu sự kiện thất bại.

Xác suất sự kiện thành công (X nhận giá trị bằng 1) là $P(X = 1) = p(1) = p$, xác suất sự kiên thất bại $P(X = 0) = p(0) = 1 - p$

\textbf{X là biến ngẫu nhiên Bernoulli}

$$X \sim Ber(p)$$

Khi đó, kì vọng của X

$$E[X] = p$$

Phương sai của X

$$Var(X)=p(1-p)$$

\subsection{Phân phối Bernoulli}

Như đã đề cập về phép thử Béc-nu-li rằng mọi phép thử của nó chỉ cho 2 kết quả duy nhất là $A$ với xác suất $p$ và $\bar A$ với xác suất $q=1-p$
Biến ngẫu nhiên $X$ tuân theo phân phối Béc-nu-li
$$X \sim B(p)$$
với tham số $p \in \mathbb{R}, 0 \leq p \leq 1$ là xác suất xuất hiện của $A$ tại mỗi phép thử
\newline
\begin{tabular}{ | l | l | l | }
  \hline
  Định nghĩa & & Giá trị \\
  \hline
  PMF & p(x) & $p(x)$ | $p^x (1-p)^{1-x}, x \in \{0, 1\} $ \\
  \hline
  CDF & $F(x;p)$  &
  \begin{cases}
    0 & \text{for } x < 0 \\
    1-p & \text{for } 0 \leq x < 1 \\
    1 & \text{for } x \geq 1
  \end{cases} \\
  \hline
  Kỳ vọng & $E[X]$ & $p$ \\
  \hline
  Phương sai & $Var(X)$ & $p(1-p)$ \\
  \hline
\end{tabular}

\subsection{Phân phối Binomial}

\begin{itemize}
  \item xác suất thành công n lần $X \sim B(n,p)$
\end{itemize}

\subsection{Phân phối Poisson}

\begin{itemize}
  \item xác suất thành công n lần $X \sim Poi(\lambda)$
\end{itemize}

\subsection{Phân phối Geometric}

\begin{itemize}
  \item số lần thử đến khi thành công $X \sim Geo(p)$
\end{itemize}

\subsection{Phân phối Negative Binomial}

\begin{itemize}
  \item số lần thử đến khi $r$ thành công $X \sim NegBin(r,p)$
\end{itemize}

\subsection{Phân phối Hyper Geometric}

\begin{itemize}
  \item số bóng trắng lấy được trong $N$ bóng chứa $m$ bóng trắng  $X \sim HypG(n, N, m)$
\end{itemize}

\subsection{Biến liên tục}

\subsection{Phân phối đều}

\definition{phân phối đều}{
Là phân phối mà xác suất xuất hiện của các sự kiện là như nhau.
}

\newline

Biến ngẫu nhiên $X$ tuân theo phân phối đều rời rạc

$$X \sim Unif (a, b)$$

với tham số $a, b \in \mathbb Z; a < b$ là khoảng giá trị của $X$, đặt $n = b-a+1$
\newline
Ta sẽ có:
\newline
\begin{tabular}[c]{ | l | l | }
  \hline
  Định nghĩa & Giá trị \\
  \hline
  PMF & $p(x)$ | $\dfrac{1}{n}, \forall x \in [a,b]$ \\
  \hline
  CDF - $F(x;a,b)$ & $\dfrac{x-a+1}{n}, \forall x \in [a,b]$ \\
  \hline
  Kỳ vọng - $E[X]$ & $\dfrac{a+b}{2}$ \\
  \hline
  Phương sai - $Var(X)$ & $\dfrac{n^2-1}{12}$ \\
  \hline
\end{tabular}

\newline

\noindent Ví dụ 1:
\index{bài toán chờ xe buýt}

\definition{bài toán chờ xe buýt}{
Lịch chạy của xe buýt tại một trạm xe buýt như sau: chiếc xe buýt đầu tiên trong ngày sẽ khởi hành từ trạm này vào lúc 7 giờ, cứ sau mỗi 15 phút sẽ có một xe khác đến trạm. Giả sử một hành khách đến trạm trong khoảng thời gian từ 7 giờ đến 7 giờ 30. Tìm xác suất để hành khách này chờ:

a) Ít hơn 5 phút.

b) Ít nhất 12 phút.

}

\textbf{Giải}

Gọi X là số phút sau 7 giờ mà hành khách đến trạm.

Ta có: $X \sim R[0;30]$.

a) Hành khách sẽ chờ ít hơn 5 phút nếu đến trạm giữa 7 giờ 10 và 7 giờ 15 hoặc giữa 7 giờ 25 và 7 giờ 30. Do đó xác suất cần tìm là:

$$P(0<X<15) + P(25<X<30)=\frac{5}{30} + \frac{5}{30}=\frac{1}{3}$$

b) Hành khách chờ ít nhất 12 phút nếu đến trạm giữa 7 giờ và 7 giờ 3 phút hoặc giữa 7 giờ 15 phút và 7 giờ 18 phút. Xác suất cần tìm là:

$$P(0<X<3) + P(15<X<18)=\frac{3}{30} + \frac{3}{30}=\frac{1}{5}$$

%  \part{Lập trình}

\chapter{Lập trình là gì?}

\section{Các vấn đề lập trình}

Các vấn đề lập trình với từng ngôn ngữ

\subsection{Nhập môn}

Phần 1: Cơ bản

\begin{lstlisting}
├── 1. introduction
├── 2. syntax
├── 3. data structure
├── 4. oop
\end{lstlisting}

Phần 2: Xây dựng ứng dụng

\begin{lstlisting}
├── 16. database
├── 5. networking
├── 6. os
├── 14. ui
├── 15. web
\end{lstlisting}

Phần 3: Các chủ đề nâng cao

\begin{lstlisting}
├── 7. parallel
├── 8. event based
├── 9. error handling
├── 10. logging
\end{lstlisting}

Phần 4: Phát triển phần mềm chuyên nghiệp

\begin{lstlisting}
├── 11. configuration
├── 12. documentation
├── 13. test
├── 17. ide
├── 18. package manager
├── 19. build tool
├── 20. make module
└── 21. production (docker)
\end{lstlisting}

\section{Introduction}

Installation (environment, IDE)

Hello world

Courses

Resources


\section{Syntax}

variables and expressions

conditional

loops and Iteration

functions

define, use

parameters

scope of variables

anonymous functions

callbacks

self-invoking functions, inner functions

functions that return functions, functions that redefined themselves

closures

naming convention

comment convention

\section{Cấu trúc dữ liệu}

Number

String

Collection

DateTime

Boolean

Object

\section{Lập trình hướng đối tượng}

Classes & Objects

Inheritance

Encapsulation

Abstraction

Polymorphism

For OOP Example: see Python: OOP

\subsection{Bài tập}

\textbf{Quản lý tài khoản ngân hàng}

\section{Networking}

REST (example with chat app sender, receiver, message)

\subsection{Bài tập}

Guess My Number Game

\section{GUI - Giao diện}

Quản lý hot girl

Quản lý truyện tranh

Create Analog Clock

Chương trình lịch âm dương

Chương trình học từ tiếng Anh

\section{Game}

\begin{itemize}
  \item Create Pong Game
  \item Create flappy bird
  \item Create Bouncing Game
\end{itemize}


\section{Cơ sở dữ liệu}

\subsection{Thử thách}


\section{How to ask a question}

Focus on questions about an actual problem you have faced. Include details about what you have tried and exactly what you are trying to do.

Ask about...

✔ Specific programming problems

✔ Software algorithms

✔ Coding techniques

✔ Software development tools

Not all questions work well in our format. Avoid questions that are primarily opinion-based, or that are likely to generate discussion rather than answers.

Don't ask about...

✖ Questions you haven't tried to find an answer for (show your work!)

✖ Product or service recommendations or comparisons

✖ Requests for lists of things, polls, opinions, discussions, etc.

✖ Anything not directly related to writing computer programs

\section{Các nguyên tắc lập trình}

Generic

KISS (Keep It Simple Stupid)

YAGNI

Do The Simplest Thing That Could Possibly Work

Keep Things DRY

Code For The Maintainer

Avoid Premature Optimization

Inter-Module/Class

Minimise Coupling

Law of Demeter

Composition Over Inheritance

Orthogonality

Module/Class

Maximise Cohesion

Liskov Substitution Principle

Open/Closed Principle

Single Responsibility Principle

Hide Implementation Details

Curly's Law

Software Quality Laws

First Law of Software Quality


\section{Các mô hình lập trình}

Main paradigm approaches 1

1. Imperative


Description:

Computation as statements that directly change a program state (datafields)

Main Characteristics:

Direct assignments, common data structures, global variables

Critics: Edsger W. Dijkstra, Michael A. Jackson

Examples: Assembly, C, C++, Java, PHP, Python

2. Structured

Description:

A style of imperative programming with more logical program structure

Main Characteristics:

Structograms, indentation, either no, or limited use of, goto statements

Examples: C, C++, Java, Python

3. Procedural

Description:

Derived from structured programming, based on the concept of modular programming or the procedure call

Main Characteristics:

Local variables, sequence, selection, iteration, and modularization

Examples: C, C++, Lisp, PHP, Python

4. Functional


Description:

Treats computation as the evaluation of mathematical functions avoiding state and mutable data

Main Characteristics:

Lambda calculus, compositionality, formula, recursion, referential transparency, no side effects

Examples: Clojure, Coffeescript, Elixir, Erlang, F#, Haskell, Lisp, Python, Scala, SequenceL, SML

5. Event-driven including time driven

Description:

Program flow is determined mainly by events, such as mouse clicks or interrupts including timer

Main Characteristics:

Main loop, event handlers, asynchronous processes

Examples: Javascript, ActionScript, Visual Basic

6. Object-oriented

Description:

Treats datafields as objects manipulated through pre-defined methods only

Main Characteristics:

Objects, methods, message passing, information hiding, data abstraction, encapsulation, polymorphism, inheritance, serialization-marshalling

Examples: Common Lisp, C++, C#, Eiffel, Java, PHP, Python, Ruby, Scala

7. Declarative

Description:

Defines computation logic without defining its detailed control flow

Main Characteristics:

4GLs, spreadsheets, report program generators

Examples: SQL, regular expressions, CSS, Prolog

8. Automata-based programming

Description:

Treats programs as a model of a finite state machine or any other formal automata

Main Characteristics:

State enumeration, control variable, state changes, isomorphism, state transition table

Examples: AsmL

\section{Testing}

\includegraphics{programming/introduction/unit_test_tdd}

1. Definition 1 2

Test-driven development (TDD) is a software development process that relies on the repetition of a very short development cycle:

\includegraphics[width=\linewidth]{programming/introduction/tdd.jpg}

Step 1: First the developer writes an (initially failing) automated test case that defines a desired improvement or new function,

Step 2: Then produces the minimum amount of code to pass that test,

Step 3: Finally refactors the new code to acceptable standards.

Kent Beck, who is credited with having developed or 'rediscovered' the technique, stated in 2003 that TDD encourages simple designs and inspires confidence.

2. Principles 2

Kent Beck defines

Never with a single line of code unless you have a failing automated test.
Eliminate duplication
Red: (Automated test fail) Green (Automated test pass because dev code has been written) Refactor (Eliminate duplication, Clean the code)

3. Assertions & Assert Framework

\includegraphics[width=\linewidth]{programming/introduction/tdd_assertion.png}

Assert that the expected results have occurred.
[code lang="java"] @Test public void test() { assertEquals(2, 1 + 1); } [/code]


4. Test Runners 3

\includegraphics[width=\linewidth]{programming/introduction/tdd_test_runner.png}

When testing a large real-world web app there may be tens or hundreds of test cases, and we certainly don't want to run each one manually. In such as scenario we need to use a test runner to find and execute the tests for us, and in this article we'll explore just that.

A test runner provides the a good basis for a real testing framework. A test runner is designed to run tests, tag tests with attributes (annotations), and provide reporting and other features.

In general you break your tests up into 3 standard sections; setUp(), tests, and tearDown(), typical for a test runner setup.

The setUp() and tearDown() methods are run automatically for every test, and contain respectively:

The setup steps you need to take before running the test, such as unlocking the screen and killing open apps.
The cooldown steps you need to run after the test, such as closing the Marionette session.

5. Test Frameworks

Language	Test Frameworks
C++/VisualStudio	C++: Test
Web Service	rest-assured
Web UI	SeleniumHQ

\section{Logging}

Logging is the process of recording application actions and state to a secondary interface.

\includegraphics[width=\linewidth]{programming/introduction/logging}

Logging System

\includegraphics[width=\linewidth]{programming/introduction/logging_system}

Levels

Level	When it’s used
DEBUG	Detailed information, typically of interest only when diagnosing problems.
INFO	Confirmation that things are working as expected.
WARNING	An indication that something unexpected happened, or indicative of some problem in the near future (e.g. ‘disk space low’). The software is still working as expected.

ERROR

Due to a more serious problem, the software has not been able to perform some function.
CRITICAL	A serious error, indicating that the program itself may be unable to continue running.
Best Practices 2 4 5
Logging should always be considered when handling an exception but should never take the place of a real handler.
Keep all logging code in your production code. Have an ability to enable more/less detailed logging in production, preferably per subsystem and without restarting your program.
Make logs easy to parse by grep and by eye. Stick to several common fields at the beginning of each line. Identify time, severity, and subsystem in every line. Clearly formulate the message. Make every log message easy to map to its source code line.
If an error happens, try to collect and log as much information as possible. It may take long but it's OK because normal processing has failed anyway. Not having to wait when the same condition happens in production with a debugger attached is priceless.

\section{Lập trình hàm}

Functional
Without mutable variables, assignment, conditional

Advantages 1
Most functional languages provide a nice, protected environment, somewhat like JavaLanguage. It's good to be able to catch exceptions instead of having CoreDumps in stability-critical applications.
FP encourages safe ways of programming. I've never seen an OffByOne mistake in a functional program, for example... I've seen one. Adding two lengths to get an index but one of them was zero-indexed. Easy to discover though. -- AnonymousDonor
Functional programs tend to be much more terse than their ImperativeLanguage counterparts. Often this leads to enhanced programmer productivity.
FP encourages quick prototyping. As such, I think it is the best software design paradigm for ExtremeProgrammers... but what do I know.
FP is modular in the dimension of functionality, where ObjectOrientedProgramming is modular in the dimension of different components.
Generic routines (also provided by CeePlusPlus) with easy syntax. ParametricPolymorphism
The ability to have your cake and eat it. Imagine you have a complex OO system processing messages - every component might make state changes depending on the message and then forward the message to some objects it has links to. Wouldn't it be just too cool to be able to easily roll back every change if some object deep in the call hierarchy decided the message is flawed? How about having a history of different states?
Many housekeeping tasks made for you: deconstructing data structures (PatternMatching), storing variable bindings (LexicalScope with closures), strong typing (TypeInference), * GarbageCollection, storage allocation, whether to use boxed (pointer-to-value) or unboxed (value directly) representation...
Safe multithreading! Immutable data structures are not subject to data race conditions, and consequently don't have to be protected by locks. If you are always allocating new objects, rather than destructively manipulating existing ones, the locking can be hidden in the allocation and GarbageCollection system.

\section{Lập trình song song}

Paralell/Concurrency Programming
1. Callback Pattern 2
Callback functions are derived from a programming paradigm known as functional programming. At a fundamental level, functional programming specifies the use of functions as arguments. Functional programming was—and still is, though to a much lesser extent today—seen as an esoteric technique of specially trained, master programmers.

Fortunately, the techniques of functional programming have been elucidated so that mere mortals like you and me can understand and use them with ease. One of the chief techniques in functional programming happens to be callback functions. As you will read shortly, implementing callback functions is as easy as passing regular variables as arguments. This technique is so simple that I wonder why it is mostly covered in advanced JavaScript topics.

[code lang="javascript"] function getN(){ return 10; }

var n = getN();

function getAsyncN(callback){ setTimeout(function(){ callback(10); }, 1000); }

function afterGetAsyncN(result){ var n = 10; console.log(n); }

getAsyncN(afterGetAsyncN); [/code]

2. Promise Pattern 1 3
What is a promise?
The core idea behind promises is that a promise represents the result of an asynchronous operation.

A promise is in one of three different states:

pending - The initial state of a promise.
fulfilled - The state of a promise representing a successful operation.
rejected - The state of a promise representing a failed operation.
Once a promise is fulfilled or rejected, it is immutable (i.e. it can never change again).


\begin{lstlisting}[language=Javscript]
function aPromise(message){
  return new Promise(function(fulfill, reject){
    if(message == "success"){
      fulfill("it is a success Promise");
    } if(message == "fail"){
      reject("it is a fail Promise");
    }
  });
}
\end{lstlisting}

Usage:

\begin{lstlisting}[language=Javascript]
aPromise("success").then(function(successMessage){
  console.log(successMessage) }, function(failMessage){
  // it is a success Promise
  console.log(failMessage)
})
\end{lstlisting}

\begin{lstlisting}[language=Javascript]
aPromise("fail").then(function(successMessage){
  console.log(successMessage) }, function(failMessage){
  console.log(failMessage)
}) // it is a fail Promise
\end{lstlisting}

\section{IDE - Môi trường phát triển tích hợp}

An integrated development environment (IDE) is a software application that provides comprehensive facilities to computer programmers for software development. An IDE normally consists of a source code editor, build automation tools and a debugger. Most modern IDEs have intelligent code completion.

1. Navigation

Word Navigation Line Navigation File Navigation

2. Editing

Auto Complete Code Complete Multicursor Template (Snippets)

3. Formatting

Debugging
Custom Rendering for Object
%\chapter{Python}

\section{Giới thiệu}

\begin{item}
  \item `Python` is a widely used general-purpose, high-level programming language. Its design philosophy emphasizes code readability, and its syntax allows programmers to express concepts in fewer lines of code than would be possible in languages such as C++ or Java.
  \item The language provides constructs intended to enable clear programs on both a small and large scale.
\end{item}

Python Tutorial
Python is a general-purpose interpreted, interactive, object-oriented, and high-level programming language. It was created by Guido van Rossum during 1985- 1990. Like Perl, Python source code is also available under the GNU General Public License (GPL). This tutorial gives enough understanding on Python programming language.

Python is Interpreted

Python is processed at runtime by the interpreter. You do not need to compile your program before executing it. This is similar to PERL and PHP.

Python is Interactive

You can actually sit at a Python prompt and interact with the interpreter directly to write your programs.

Python is Object-Oriented

Python supports Object-Oriented style or technique of programming that encapsulates code within objects.

Python is Beginner Friendly

Python is a great language for the beginner-level programmers and supports the development of a wide range of applications from simple text processing to WWW browsers to games.

Audience
This tutorial is designed for software programmers who need to learn Python programming language from scratch.


\textbf{Sách}

\href{https://docs.google.com/document/d/1gQFMXZtynpuTenoOQNGCHttArT4NspTWcyJQr5ps9Mk/edit?usp=sharing}{Tập hợp các sách python}

\textbf{Khoá học}

\href{1frO9QYhgsXbMzcyXoA4czWkxTWF8RBTJVf9uoO1rElU}{Tập hợp các khóa học python}

\textbf{Tham khảo}

\href{http://blog.tryolabs.com/2015/12/15/top-10-python-libraries-of-2015/}{Top 10 Python Libraries Of 2015}

\section{Cài đặt}

Get Started
Welcome! This tutorial details how to get started with Python.

For Windows
Anaconda 4.3.0
Anaconda is BSD licensed which gives you permission to use Anaconda commercially and for redistribution.

1. Download the installer
2. Optional: Verify data integrity with MD5 or SHA-256
3. Double-click the .exe file to install Anaconda and follow the instructions on the screen
Python 3.6 version
64-BIT INSTALLER
Python 2.7 version
64-BIT INSTALLER
Step 2. Discover the Map

https://docs.python.org/2/library/index.html

For CentOS
Developer tools
The Development tools will allow you to build and compile software from source code. Tools for building RPMs are also included, as well as source code management tools like Git, SVN, and CVS.

\begin{lstlisting}[language=bash]
yum groupinstall "Development tools"
yum install zlib-devel
yum install bzip2-devel
yum install openssl-devel
yum install ncurses-devel
yum install sqlite-devel
\end{lstlisting}

Python & Anaconda
Anaconda is BSD licensed which gives you permission to use Anaconda commercially and for redistribution.

\begin{lstlisting}[language=bash]
cd /opt
wget --no-check-certificate https://www.python.org/ftp/python/2.7.6/Python-2.7.6.tar.xz
tar xf Python-2.7.6.tar.xz
cd Python-2.7.6
./configure --prefix=/usr/local
make && make altinstall
## link
ln -s /usr/local/bin/python2.7 /usr/local/bin/python
# final check
which python
python -V
# install Anaconda
cd ~/Downloads
wget https://repo.continuum.io/archive/Anaconda-2.3.0-Linux-x86_64.sh
bash ~/Downloads/Anaconda-2.3.0-Linux-x86_64.sh
\end{lstlisting}

\section{Cơ bản}

\section{Cú pháp cơ bản}

Print, print

\begin{lstlisting}[language=python]
print "Hello World"
\end{lstlisting}


Conditional

\begin{lstlisting}[language=Python]
if you_smart:
    print "learn python"
else:
    print "go away"
\end{lstlisting}

Loop

In general, statements are executed sequentially: The first statement in a function is executed first, followed by the second, and so on. There may be a situation when you need to execute a block of code several number of times.

Programming languages provide various control structures that allow for more complicated execution paths. A loop statement allows us to execute a statement or group of statements multiple times. The following diagram illustrates a loop statement


Python programming language provides following types of loops to handle looping requirements.

while loop	Repeats a statement or group of statements while a given condition is TRUE. It tests the condition before executing the loop body.
for loop	Executes a sequence of statements multiple times and abbreviates the code that manages the loop variable.
nested loops	You can use one or more loop inside any another while, for or do..while loop.
While Loop
A while loop statement in Python programming language repeatedly executes a target statement as long as a given condition is true.

Syntax

The syntax of a while loop in Python programming language is

\begin{lstlisting}[language=Python]
while expression:
   statement(s)
\end{lstlisting}

Example

\begin{lstlisting}[language=Python]
count = 0
while count < 9:
   print 'The count is:', count
   count += 1
print "Good bye!"
\end{lstlisting}


For Loop

It has the ability to iterate over the items of any sequence, such as a list or a string.

Syntax

\begin{lstlisting}[language=Python]
for iterating_var in sequence:
   statements(s)
\end{lstlisting}

If a sequence contains an expression list, it is evaluated first. Then, the first item in the sequence is assigned to the iterating variable iterating_var. Next, the statements block is executed. Each item in the list is assigned to iterating_var, and the statement(s) block is executed until the entire sequence is exhausted.

Example

\begin{lstlisting}[language=Python]
for i in range(10):
    print "hello", i

for letter in 'Python':
   print 'Current letter :', letter

fruits = ['banana', 'apple',  'mango']
for fruit in fruits:
   print 'Current fruit :', fruit

print "Good bye!"
\end{lstlisting}

Yield and Generator

Yield is a keyword that is used like return, except the function will return a generator.

\begin{lstlisting}[language=Python]
def createGenerator():
    yield 1
    yield 2
    yield 3
mygenerator = createGenerator() # create a generator
print(mygenerator) # mygenerator is an object!
# <generator object createGenerator at 0xb7555c34>
for i in mygenerator:
    print(i)
# 1
# 2
# 3
\end{lstlisting}


Visit Yield and Generator explained for more information

Functions

Variable-length arguments

\begin{lstlisting}[language=Python]
def functionname([formal_args,] *var_args_tuple ):
   "function_docstring"
   function_suite
   return [expression]
\end{lstlisting}

Example

\begin{lstlisting}[language=Python]
#!/usr/bin/python

# Function definition is here
def printinfo( arg1, *vartuple ):
   "This prints a variable passed arguments"
   print "Output is: "
   print arg1
   for var in vartuple:
      print var
   return;

# Now you can call printinfo function
printinfo( 10 )
printinfo( 70, 60, 50 )
\end{lstlisting}

Coding Convention
Code layout
Indentation: 4 spaces

Suggest Readings

"Python Functions". www.tutorialspoint.com
"Python Loops". www.tutorialspoint.com
"What does the “yield” keyword do?". stackoverflow.com
"Improve Your Python: 'yield' and Generators Explained". jeffknupp.com

\textbf{Vấn đề với mảng}

\begin{item}
  \item Random Sampling \footnote{tham khảo [pytorch](http://pytorch.org/docs/master/torch.html?highlight=randn#torch.randn), [numpy](https://docs.scipy.org/doc/numpy-1.13.0/reference/routines.random.html))} - sinh ra một mảng ngẫu nhiên trong khoảng (0, 1), mảng ngẫu nhiên số nguyên trong khoảng (x, y), mảng ngẫu nhiên là permutation của số từ 1 đến n
\end{item}

\section{Yield and Generators}

Coroutines and Subroutines
When we call a normal Python function, execution starts at function's first line and continues until a return statement, exception, or the end of the function (which is seen as an implicit return None) is encountered. Once a function returns control to its caller, that's it. Any work done by the function and stored in local variables is lost. A new call to the function creates everything from scratch.

This is all very standard when discussing functions (more generally referred to as subroutines) in computer programming. There are times, though, when it's beneficial to have the ability to create a "function" which, instead of simply returning a single value, is able to yield a series of values. To do so, such a function would need to be able to "save its work," so to speak.

I said, "yield a series of values" because our hypothetical function doesn't "return" in the normal sense. return implies that the function is returning control of execution to the point where the function was called. "Yield," however, implies that the transfer of control is temporary and voluntary, and our function expects to regain it in the future.

In Python, "functions" with these capabilities are called generators, and they're incredibly useful. generators (and the yield statement) were initially introduced to give programmers a more straightforward way to write code responsible for producing a series of values. Previously, creating something like a random number generator required a class or module that both generated values and kept track of state between calls. With the introduction of generators, this became much simpler.

To better understand the problem generators solve, let's take a look at an example. Throughout the example, keep in mind the core problem being solved: generating a series of values.

Note: Outside of Python, all but the simplest generators would be referred to as coroutines. I'll use the latter term later in the post. The important thing to remember is, in Python, everything described here as a coroutine is still a generator. Python formally defines the term generator; coroutine is used in discussion but has no formal definition in the language.

Example: Fun With Prime Numbers
Suppose our boss asks us to write a function that takes a list of ints and returns some Iterable containing the elements which are prime1 numbers.

Remember, an Iterable is just an object capable of returning its members one at a time.

"Simple," we say, and we write the following:

\begin{lstlisting}[language=Python]
def get_primes(input_list):
    result_list = list()
    for element in input_list:
        if is_prime(element):
            result_list.append()

    return result_list
\end{lstlisting}

or better yet...

\begin{lstlisting}[language=Python]
def get_primes(input_list):
    return (element for element in input_list if is_prime(element))

# not germane to the example, but here's a possible implementation of
# is_prime...

def is_prime(number):
    if number > 1:
        if number == 2:
            return True
        if number % 2 == 0:
            return False
        for current in range(3, int(math.sqrt(number) + 1), 2):
            if number % current == 0:
                return False
        return True
    return False
\end{lstlisting}

Either get_primes implementation above fulfills the requirements, so we tell our boss we're done. She reports our function works and is exactly what she wanted.

Dealing With Infinite Sequences
Well, not quite exactly. A few days later, our boss comes back and tells us she's run into a small problem: she wants to use our get_primes function on a very large list of numbers. In fact, the list is so large that merely creating it would consume all of the system's memory. To work around this, she wants to be able to call get_primes with a start value and get all the primes larger than start (perhaps she's solving Project Euler problem 10).

Once we think about this new requirement, it becomes clear that it requires more than a simple change to get_primes. Clearly, we can't return a list of all the prime numbers from start to infinity (operating on infinite sequences, though, has a wide range of useful applications). The chances of solving this problem using a normal function seem bleak.

Before we give up, let's determine the core obstacle preventing us from writing a function that satisfies our boss's new requirements. Thinking about it, we arrive at the following: functions only get one chance to return results, and thus must return all results at once. It seems pointless to make such an obvious statement; "functions just work that way," we think. The real value lies in asking, "but what if they didn't?"

Imagine what we could do if get_primes could simply return the next value instead of all the values at once. It wouldn't need to create a list at all. No list, no memory issues. Since our boss told us she's just iterating over the results, she wouldn't know the difference.

Unfortunately, this doesn't seem possible. Even if we had a magical function that allowed us to iterate from n to infinity, we'd get stuck after returning the first value:

def get_primes(start):
    for element in magical_infinite_range(start):
        if is_prime(element):
            return element
Imagine get_primes is called like so:

def solve_number_10():
    # She *is* working on Project Euler #10, I knew it!
    total = 2
    for next_prime in get_primes(3):
        if next_prime < 2000000:
            total += next_prime
        else:
            print(total)
            return
Clearly, in get_primes, we would immediately hit the case where number = 3 and return at line 4. Instead of return, we need a way to generate a value and, when asked for the next one, pick up where we left off.

Functions, though, can't do this. When they return, they're done for good. Even if we could guarantee a function would be called again, we have no way of saying, "OK, now, instead of starting at the first line like we normally do, start up where we left off at line 4." Functions have a single entry point: the first line.

Enter the Generator
This sort of problem is so common that a new construct was added to Python to solve it: the generator. A generator "generates" values. Creating generators was made as straightforward as possible through the concept of generator functions, introduced simultaneously.

A generator function is defined like a normal function, but whenever it needs to generate a value, it does so with the yield keyword rather than return. If the body of a def contains yield, the function automatically becomes a generator function (even if it also contains a return statement). There's nothing else we need to do to create one.

generator functions create generator iterators. That's the last time you'll see the term generator iterator, though, since they're almost always referred to as "generators". Just remember that a generator is a special type of iterator. To be considered an iterator, generators must define a few methods, one of which is next(). To get the next value from a generator, we use the same built-in function as for iterators: next().

This point bears repeating: to get the next value from a generator, we use the same built-in function as for iterators: next().

(next() takes care of calling the generator's next() method). Since a generator is a type of iterator, it can be used in a for loop.

So whenever next() is called on a generator, the generator is responsible for passing back a value to whomever called next(). It does so by calling yield along with the value to be passed back (e.g. yield 7). The easiest way to remember what yield does is to think of it as return (plus a little magic) for generator functions.**

Again, this bears repeating: yield is just return (plus a little magic) for generator functions.

Here's a simple generator function:

>>> def simple_generator_function():
>>>    yield 1
>>>    yield 2
>>>    yield 3
And here are two simple ways to use it:

>>> for value in simple_generator_function():
>>>     print(value)
1
2
3
>>> our_generator = simple_generator_function()
>>> next(our_generator)
1
>>> next(our_generator)
2
>>> next(our_generator)
3
Magic?
What's the magic part? Glad you asked! When a generator function calls yield, the "state" of the generator function is frozen; the values of all variables are saved and the next line of code to be executed is recorded until next() is called again. Once it is, the generator function simply resumes where it left off. If next() is never called again, the state recorded during the yield call is (eventually) discarded.

Let's rewrite get_primes as a generator function. Notice that we no longer need the magical_infinite_range function. Using a simple while loop, we can create our own infinite sequence:

def get_primes(number):
    while True:
        if is_prime(number):
            yield number
        number += 1
If a generator function calls return or reaches the end its definition, a StopIteration exception is raised. This signals to whoever was calling next() that the generator is exhausted (this is normal iterator behavior). It is also the reason the while True: loop is present in get_primes. If it weren't, the first time next() was called we would check if the number is prime and possibly yield it. If next() were called again, we would uselessly add 1 to number and hit the end of the generator function (causing StopIteration to be raised). Once a generator has been exhausted, calling next() on it will result in an error, so you can only consume all the values of a generator once. The following will not work:

>>> our_generator = simple_generator_function()
>>> for value in our_generator:
>>>     print(value)

>>> # our_generator has been exhausted...
>>> print(next(our_generator))
Traceback (most recent call last):
  File "<ipython-input-13-7e48a609051a>", line 1, in <module>
    next(our_generator)
StopIteration

>>> # however, we can always create a new generator
>>> # by calling the generator function again...

>>> new_generator = simple_generator_function()
>>> print(next(new_generator)) # perfectly valid
1
Thus, the while loop is there to make sure we never reach the end of get_primes. It allows us to generate a value for as long as next() is called on the generator. This is a common idiom when dealing with infinite series (and generators in general).

Visualizing the flow
Let's go back to the code that was calling get_primes: solve_number_10.

def solve_number_10():
    # She *is* working on Project Euler #10, I knew it!
    total = 2
    for next_prime in get_primes(3):
        if next_prime < 2000000:
            total += next_prime
        else:
            print(total)
            return
It's helpful to visualize how the first few elements are created when we call get_primes in solve_number_10's for loop. When the for loop requests the first value from get_primes, we enter get_primes as we would in a normal function.

We enter the while loop on line 3
The if condition holds (3 is prime)
We yield the value 3 and control to solve_number_10.
Then, back in solve_number_10:

The value 3 is passed back to the for loop
The for loop assigns next_prime to this value
next_prime is added to total
The for loop requests the next element from get_primes
This time, though, instead of entering get_primes back at the top, we resume at line 5, where we left off.

def get_primes(number):
    while True:
        if is_prime(number):
            yield number
        number += 1 # <<<<<<<<<<
Most importantly, number still has the same value it did when we called yield (i.e. 3). Remember, yield both passes a value to whoever called next(), and saves the "state" of the generator function. Clearly, then, number is incremented to 4, we hit the top of the while loop, and keep incrementing number until we hit the next prime number (5). Again we yield the value of number to the for loop in solve_number_10. This cycle continues until the for loop stops (at the first prime greater than 2,000,000).

Moar Power
In PEP 342, support was added for passing values into generators. PEP 342 gave generators the power to yield a value (as before), receive a value, or both yield a value and receive a (possibly different) value in a single statement.

To illustrate how values are sent to a generator, let's return to our prime number example. This time, instead of simply printing every prime number greater than number, we'll find the smallest prime number greater than successive powers of a number (i.e. for 10, we want the smallest prime greater than 10, then 100, then 1000, etc.). We start in the same way as get_primes:

def print_successive_primes(iterations, base=10):
    # like normal functions, a generator function
    # can be assigned to a variable

    prime_generator = get_primes(base)
    # missing code...
    for power in range(iterations):
        # missing code...

def get_primes(number):
    while True:
        if is_prime(number):
        # ... what goes here?
The next line of get_primes takes a bit of explanation. While yield number would yield the value of number, a statement of the form other = yield foo means, "yield foo and, when a value is sent to me, set other to that value." You can "send" values to a generator using the generator's send method.

def get_primes(number):
    while True:
        if is_prime(number):
            number = yield number
        number += 1
In this way, we can set number to a different value each time the generator yields. We can now fill in the missing code in print_successive_primes:

def print_successive_primes(iterations, base=10):
    prime_generator = get_primes(base)
    prime_generator.send(None)
    for power in range(iterations):
        print(prime_generator.send(base ** power))
Two things to note here: First, we're printing the result of generator.send, which is possible because send both sends a value to the generator and returns the value yielded by the generator (mirroring how yield works from within the generator function).

Second, notice the prime_generator.send(None) line. When you're using send to "start" a generator (that is, execute the code from the first line of the generator function up to the first yield statement), you must send None. This makes sense, since by definition the generator hasn't gotten to the first yield statement yet, so if we sent a real value there would be nothing to "receive" it. Once the generator is started, we can send values as we do above.

Round-up
In the second half of this series, we'll discuss the various ways in which generators have been enhanced and the power they gained as a result. yield has become one of the most powerful keywords in Python. Now that we've built a solid understanding of how yield works, we have the knowledge necessary to understand some of the more "mind-bending" things that yield can be used for.

Believe it or not, we've barely scratched the surface of the power of yield. For example, while send does work as described above, it's almost never used when generating simple sequences like our example. Below, I've pasted a small demonstration of one common way send is used. I'll not say any more about it as figuring out how and why it works will be a good warm-up for part two.

\begin{lstlisting}[language=Python]
import random

def get_data():
    """Return 3 random integers between 0 and 9"""
    return random.sample(range(10), 3)

def consume():
    """Displays a running average across lists of integers sent to it"""
    running_sum = 0
    data_items_seen = 0

    while True:
        data = yield
        data_items_seen += len(data)
        running_sum += sum(data)
        print('The running average is {}'.format(running_sum / float(data_items_seen)))

def produce(consumer):
    """Produces a set of values and forwards them to the pre-defined consumer
    function"""
    while True:
        data = get_data()
        print('Produced {}'.format(data))
        consumer.send(data)
        yield

if __name__ == '__main__':
    consumer = consume()
    consumer.send(None)
    producer = produce(consumer)

    for _ in range(10):
        print('Producing...')
        next(producer)
\end{lstlisting}

Remember...
There are a few key ideas I hope you take away from this discussion:

generators are used to generate a series of values
yield is like the return of generator functions
The only other thing yield does is save the "state" of a generator function
A generator is just a special type of iterator
Like iterators, we can get the next value from a generator using next()
for gets values by calling next() implicitly

\section{Cấu trúc dữ liệu}

Number
Basic Operation

\begin{lstlisting}[language=Python]
1
1.2
1 + 2
abs(-5)
\end{lstlisting}


\section{Quản lý gói với Anaconda}

\noindent Cài đặt package tại một branch của một project trên github

\begin{lstlisting}[language=Python]
$ pip install git+https://github.com/tangentlabs/django-oscar-paypal.git@issue/34/oscar-0.6#egg=django-oscar-paypal
\end{lstlisting}

\noindent Trích xuất danh sách package

\begin{lstlisting}
$ pip freeze > requirements.txt
\end{lstlisting}

\noindent \textbf{Chạy ipython trong environment anaconda}

\noindent Chạy đống lệnh này

\begin{lstlisting}[language=bash]
  conda install nb_conda
  source activate my_env
  python -m IPython kernelspec install-self --user
  ipython notebook
\end{lstlisting}

\noindent \textbf{Interactive programming với ipython}

\noindent Trích xuất ipython ra slide (không hiểu sao default `--to slides` không work nữa, lại phải thêm tham số `--reveal-prefix` [^1]

\begin{lstlisting}[language=bash]
jupyter nbconvert "file.ipynb"
  --to slides
  --reveal-prefix "https://cdnjs.cloudflare.com/ajax/libs/reveal.js/3.1.0"
\end{lstlisting}

**Tham khảo thêm**

* https://stackoverflow.com/questions/37085665/in-which-conda-environment-is-jupyter-executing
* https://github.com/jupyter/notebook/issues/541#issuecomment-146387578
* https://stackoverflow.com/a/20101940/772391

\noindent \textbf{python 3.4 hay 3.5}

Có lẽ 3.5 là lựa chọn tốt hơn (phải có của tensorflow, pytorch, hỗ trợ mock)

### Quản lý môi trường phát triển với conda

Chạy lệnh `remove` để xóa một môi trường

\begin{lstlisting}[language=bash]
conda remove --name flowers --all
\end{lstlisting}

\section{Test với python}

\textbf{Sử dụng những loại test nào?}

Hiện tại mình đang viết unittest với default class của python là Unittest. Thực ra toàn sử dụng `assertEqual` là chính!

Ngoài ra mình cũng đang sử dụng tox để chạy test trên nhiều phiên bản python (python 2.7, 3.5). Điều hay của tox là mình có thể thiết kế toàn bộ cài đặt project và các dependencies package trong file `tox.ini`

\textbf{Chạy test trên nhiều phiên bản python với tox}

Pycharm hỗ trợ debug tox (quá tuyệt!), chỉ với thao tác đơn giản là nhấn chuột phải vào file tox.ini của project.

\section{Xây dựng docs với readthedocs và sphinx}

\noindent \textbf{20/12/2017}: Tự nhiên hôm nay tất cả các class có khai báo kế thừa ở project languageflow không thể index được. Vãi thật. Làm thằng đệ không biết đâu mà build model.

Thử build lại chục lần, thay đổi file conf.py và package\_reference.rst chán chê không được. Giả thiết đầu tiên là do hai nguyên nhân (1) docstring ghi sai, (2) nội dung trong package\_reference.rst bị sai. Sửa chán chê cũng vẫn thể, thử checkout các commit của git. Không hoạt động!

Mất khoảng vài tiếng mới để ý thằng readthedocs có phần log cho từng build một. Lần mò vào build gần nhất và build (mình nhớ là) thành công cách đây 2 ngày

\noindent Log build gần nhất

\begin{lstlisting}
Running Sphinx v1.6.5
making output directory...
loading translations [en]... done
loading intersphinx inventory from https://docs.python.org/objects.inv...
intersphinx inventory has moved: https://docs.python.org/objects.inv -> https://docs.python.org/2/objects.inv
loading intersphinx inventory from http://docs.scipy.org/doc/numpy/objects.inv...
intersphinx inventory has moved: http://docs.scipy.org/doc/numpy/objects.inv -> https://docs.scipy.org/doc/numpy/objects.inv
building [mo]: targets for 0 po files that are out of date
building [readthedocsdirhtml]: targets for 8 source files that are out of date
updating environment: 8 added, 0 changed, 0 removed
reading sources... [ 12%] authors
reading sources... [ 25%] contributing
reading sources... [ 37%] history
reading sources... [ 50%] index
reading sources... [ 62%] installation
reading sources... [ 75%] package_reference
reading sources... [ 87%] readme
reading sources... [100%] usage

looking for now-outdated files... none found
pickling environment... done
checking consistency... done
preparing documents... done
writing output... [ 12%] authors
writing output... [ 25%] contributing
writing output... [ 37%] history
writing output... [ 50%] index
writing output... [ 62%] installation
writing output... [ 75%] package_reference
writing output... [ 87%] readme
writing output... [100%] usage
\end{lstlisting}

Log build hồi trước

\begin{lstlisting}[language=bash]
Running Sphinx v1.5.6
making output directory...
loading translations [en]... done
loading intersphinx inventory from https://docs.python.org/objects.inv...
intersphinx inventory has moved: https://docs.python.org/objects.inv -> https://docs.python.org/2/objects.inv
loading intersphinx inventory from http://docs.scipy.org/doc/numpy/objects.inv...
intersphinx inventory has moved: http://docs.scipy.org/doc/numpy/objects.inv -> https://docs.scipy.org/doc/numpy/objects.inv
building [mo]: targets for 0 po files that are out of date
building [readthedocs]: targets for 8 source files that are out of date
updating environment: 8 added, 0 changed, 0 removed
reading sources... [ 12%] authors
reading sources... [ 25%] contributing
reading sources... [ 37%] history
reading sources... [ 50%] index
reading sources... [ 62%] installation
reading sources... [ 75%] package_reference
reading sources... [ 87%] readme
reading sources... [100%] usage

/home/docs/checkouts/readthedocs.org/user_builds/languageflow/checkouts/develop/languageflow/transformer/count.py:docstring of languageflow.transformer.count.CountVectorizer:106: WARNING: Definition list ends without a blank line; unexpected unindent.
/home/docs/checkouts/readthedocs.org/user_builds/languageflow/checkouts/develop/languageflow/transformer/tfidf.py:docstring of languageflow.transformer.tfidf.TfidfVectorizer:113: WARNING: Definition list ends without a blank line; unexpected unindent.
../README.rst:7: WARNING: nonlocal image URI found: https://img.shields.io/badge/latest-1.1.6-brightgreen.svg
looking for now-outdated files... none found
pickling environment... done
checking consistency... done
preparing documents... done
writing output... [ 12%] authors
writing output... [ 25%] contributing
writing output... [ 37%] history
writing output... [ 50%] index
writing output... [ 62%] installation
writing output... [ 75%] package_reference
writing output... [ 87%] readme
writing output... [100%] usage
\end{lstlisting}

Đập vào mắt là sự khác biệt giữa documentation type

Lỗi

\begin{lstlisting}[language=bash]
building [readthedocsdirhtml]: targets for 8 source files that are out of date
\end{lstlisting}

Chạy

\begin{lstlisting}[language=bash]
building [readthedocs]: targets for 8 source files that are out of date
\end{lstlisting}

Hí ha hí hửng. Chắc trong cơn bất loạn sửa lại settings đây mà. Sửa lại nó trong phần Settings (Admin &gt; Settings &gt; Documentation type)

![](https://magizbox.files.wordpress.com/2017/10/screenshot-from-2017-12-20-09-54-23.png)

Khi chạy nó đã cho ra log đúng

\begin{lstlisting}[language=bash]
building [readthedocsdirhtml]: targets for 8 source files that are out of date
\end{lstlisting}

Nhưng vẫn lỗi. Vãi!!! Sau khoảng 20 phút tiếp tục bấn loạn, chửi bới readthedocs các kiểu. Thì để ý dòng này

Lỗi

\begin{lstlisting}[language=bash]
Running Sphinx v1.6.5
\end{lstlisting}


Chạy

\begin{lstlisting}[language=bash]
Running Sphinx v1.5.6
\end{lstlisting}

Ngay dòng đầu tiên mà không để ý, ngu thật. Aha, Hóa ra là thằng readthedocs nó tự động update phiên bản sphinx lên 1.6.5. Mình là mình chúa ghét thay đổi phiên bản (code đã mệt rồi, lại còn phải tương thích với nhiều phiên bản nữa thì ăn c** à). Đầu tiên search với Pycharm thấy dòng này trong `conf.py`

\begin{lstlisting}[language=bash]
# If your documentation needs a minimal Sphinx version, state it here.
# needs_sphinx = '1.0'
\end{lstlisting}

Đổi thành

\begin{lstlisting}[language=bash]
# If your documentation needs a minimal Sphinx version, state it here.
needs_sphinx = '1.5.6'
\end{lstlisting}

Vẫn vậy (holy sh*t). Thử sâu một tẹo (thực sự là rất nhiều tẹo). Thấy cái này trong trang Settings

![](https://magizbox.files.wordpress.com/2017/10/screenshot-from-2017-12-20-10-01-39.png)

Ờ há. Thằng đần này cho phép trỏ đường dẫn tới một file trong project để cấu hình dependency. Haha.
Tạo thêm một file `requirements` trong thư mục `docs` với nội dung

\begin{lstlisting}
sphinx==1.5.6
\end{lstlisting}


Sau đó cấu hình nó trên giao diện web của readthedocs

![](https://magizbox.files.wordpress.com/2017/10/screenshot-from-2017-12-20-10-04-49.png)

Build thử. Build thử thôi. Cảm giác đúng lắm rồi đấy. Và... nó chạy. Ahihi

![](https://magizbox.files.wordpress.com/2017/10/screenshot-from-2017-12-20-10-06-32.png)

\textbf{Kinh nghiệm}

* Khi không biết làm gì, hãy làm 3 việc. Đọc LOG. Phân tích LOG. Và cố gắng để LOG thay đổi theo ý mình.

PS: Trong quá trình này, cũng không thèm build thằng PDF với Epub nữa. Tiết kiệm được bao nhiêu thời gian.

\section{Pycharm Pycharm}

01/2018: Pycharm là trình duyệt ưa thích của mình trong suốt 3 năm vừa rồi.

Hôm nay tự nhiên lại gặp lỗi không tự nhận unittest, không resolve được package import bởi relative path. Vụ không tự nhận unittest sửa bằng cách xóa file .idea là xong. Còn vụ không resolve được package import bởi relative path thì vẫn chịu rồi. Nhìn code cứ đỏ lòm khó chịu thật.

\section{Vì sao lại code python?}

\textbf{01/11/2017}
Thích python vì nó quá đơn giản (và quá đẹp).

[^1]: https://github.com/jupyter/nbconvert/issues/91#issuecomment-283736634
%\chapter{C++}


C++ is a general-purpose programming language. It has imperative, object-oriented and generic programming features, while also providing facilities for low-level memory manipulation. It was designed with a bias toward system programming and embedded, resource-constrained and large systems, with performance, efficiency and flexibility of use as its design highlights. C++ has also been found useful in many other contexts, with key strengths being software infrastructure and resource-constrained applications, including desktop applications, servers (e.g. e-commerce, web search or SQL servers), and performance-critical applications (e.g. telephone switches or space probes). C++ is a compiled language, with implementations of it available on many platforms and provided by various organizations, including the Free Software Foundation (FSF's GCC), LLVM, Microsoft, Intel and IBM.

View online \href{http://magizbox.com/training/cpp/site/}{http://magizbox.com/training/cpp/site/}

\section{Get Started}

What do I need to start with CLion?
In general to develop in C/C++ with CLion you need:

CMake, 2.8.11+ (Check JetBrains guide for updates)
GCC/G++/Clang (Linux) or
MinGW 3. or MinGW — w64 3.-4. or Cygwin 1.7.32 (minimum required) up to 2.0. (Windows)
Downloading and Installing CMake
Downloading and installing CMake is pretty simple, just go to the website, download and install by following the recommended guide there or the on Desktop Wizard.

Download and install file cmake-3.9.0-win64-x65.msi
> cmake
Usage

  cmake [options] <path-to-source>
  cmake [options] <path-to-existing-build>

Specify a source directory to (re-)generate a build system for it in the
current working directory.  Specify an existing build directory to
re-generate its build system.

Run 'cmake --help' for more information.
Downloading and Getting Cygwin
Cygwin is a large collection of GNU and Open Source tools which provide functionality similar to a Linux distribution on Windows

Download file setup-x86_64.exe from the website https://cygwin.com/install.html

Install setup-x86_64.exe file



This is the root directory where Cygwin will be located, usually the recommended C:\ works



Choose where to install LOCAL DOWNLOAD PACKAGES: This is not the same as root directory, but rather where packages (ie. extra C libraries and tools) you download using Cygwin will be located



Follow the recommended instructions until you get to packages screen:



Once you get to the packages screen, this is where you customize what libraries or tools you will install. From here on I followed the above guide but here’s the gist:

From this window, choose the Cygwin applications to install. For our purposes, you will select certain GNU C/C++ packages.

Click the + sign next to the Devel category to expand it.

You will see a long list of possible packages that can be downloaded. Scroll the list to see more packages.

Pick each of the following packages by clicking its corresponding “Skip” marker.

gcc-core: C compiler subpackage
gcc-g++: C++ subpackage
libgcc1: C runtime library
gdb: The GNU Debugger
make: The GNU version of the ‘make’ utility
libmpfr4 : A library for multiple-precision floating-point arithmetic with exact rounding
Download and install CLion
Download file CLion-2017.2.exe from website https://www.jetbrains.com/clion/download/#section=windows



Config environment File > Settings... > Build, Execution, Deployment

Choose Cygwin home: C:\cygwin64
Choose CMake executable: Bundled CMake 3.8.2
Run your first C++ program with CLion

\section{Basic Syntax}

C/C++
Hello World
#include <iostream>
using namespace std;

int main() {
    cout << "hello world";
}
Convention
Naming
variable_name_like_this
class_data_memeber_name_like_this_
kConstantNamesLikeThis
ClassNameLikeThis
filenamelikethis_myusefulclass_test.cc
Comment
Class Comment
// Iterates over the contents of a GargantuanTable.
// Example:
//    GargantuanTableIterator* iter = table->NewIterator();
//    for (iter->Seek("foo"); !iter->done(); iter->Next()) {
//      process(iter->key(), iter->value());
//    }
//    delete iter;
class GargantuanTableIterator {
  ...
};
Todo Comment
// TODO(kl@gmail.com): Use a "*" here for concatenation operator.
// TODO(Zeke) change this to use relations.

\section{Cấu trúc dữ liệu}

Data Structure
Number
C++ offer the programmer a rich assortment of built-in as well as user defined data types. Following table lists down seven basic C++ data types:

Boolean - bool
Character - char
Integer - int
Floating point - float
Double floating point - double
Valueless - void
Wide character - wchar_t
Several of the basic types can be modified using one or more of these type modifiers: signed, unsigned, short, long

Following is the example, which will produce correct size of various data types on your computer.

#include <iostream>
using namespace std;

int main() {
   cout << "Size of char : " << sizeof(char) << endl;
   cout << "Size of int : " << sizeof(int) << endl;
   cout << "Size of short int : " << sizeof(short int) << endl;
   cout << "Size of long int : " << sizeof(long int) << endl;
   cout << "Size of float : " << sizeof(float) << endl;
   cout << "Size of double : " << sizeof(double) << endl;
   cout << "Size of wchar_t : " << sizeof(wchar_t) << endl;
   return 0;
}
String
String Basic

#include <iostream>
#include <string>
using namespace std ;

// assign a string
string s1 = "www.java2s.com\n";
cout << s1;

// input a string
string s2;
cin >> s2;

// concatenate two strings
string s_c = s1 + s2;

// compare strings
s1 == s2;
Collection
Pointer
A pointer is a variable whose value is the address of another variable. Like any variable or constant, you must declare a pointer before you can work with it.

The general form of a pointer variable declaration is:

type *variable_name;
// example
int    *ip;    // pointer to an integer
double *dp;    // pointer to a double
float  *fp;    // pointer to a float
char   *ch;    // pointer to character
Pointer Lab



#include <iostream>
using namespace std;

/*
 * Look at these lines
 */
int* a;
a = new int[3];
a[0] = 10;
a[1] = 2;
cout << "Address of pointer a: &a = " << &a << endl;
cout << "Value   of pointer a:  a = " << a << endl << endl;
cout << "Address of a[0]: &a[0] = " << &a[0] << endl;
cout << "Value   of a[0]: a[0]  = " << a[0]  << endl;
cout << "Value   of a[0]: *a    = " << *a    << endl << endl;
cout << "Address of a[1]: &a[1] = " << &a[1] << endl;
cout << "Value   of a[1]: a[1]  = " << a[1]  << endl;
cout << "Value   of a[1]: *(a+1)= " << *(a+1)<< endl << endl;
cout << "Address of a[2]: &a[2] = " << &a[2] << endl;
cout << "Value   of a[2]: a[2]  = " << a[2]  << endl;
cout << "Value   of a[2]: *(a+2)= " << *(a+2)<< endl << endl;
Result:

Address of pointer a: &a = 008FF770
Value   of pointer a:  a = 00C66ED0

Address of a[0]: &a[0] = 00C66ED0
Value   of a[0]: a[0]  = 10
Value   of a[0]: *a    = 10

Address of a[1]: &a[1] = 00C66ED4
Value   of a[1]: a[1]  = 2
Value   of a[1]: *(a+1)= 2

Address of a[2]: &a[2] = 00C66ED8
Value   of a[2]: a[2]  = -842150451
Value   of a[2]: *(a+2)= -842150451
Stack, Queue, Linked List, Array, Deque, List, Map, Set

Datetime
The C++ standard library does not provide a proper date type. C++ inherits the structs and functions for date and time manipulation from C. To access date and time related functions and structures, you would need to include header file in your C++ program.

There are four time-related types: clock_t, time_t, size_t, and tm. The types clock_t, size_t and time_t are capable of representing the system time and date as some sort of integer.

The structure type tm holds the date and time in the form of a C structure having the following elements:

struct tm {
   int tm_sec;   // seconds of minutes from 0 to 61
   int tm_min;   // minutes of hour from 0 to 59
   int tm_hour;  // hours of day from 0 to 24
   int tm_mday;  // day of month from 1 to 31
   int tm_mon;   // month of year from 0 to 11
   int tm_year;  // year since 1900
   int tm_wday;  // days since sunday
   int tm_yday;  // days since January 1st
   int tm_isdst; // hours of daylight savings time
}
Current date and time

Consider you want to retrieve the current system date and time, either as a local time or as a Coordinated Universal Time (UTC). Following is the example to achieve the same:

#include <iostream>
#include <ctime>

using namespace std;

int main( ) {
   // current date/time based on current system
   time_t now = time(0);

   // convert now to string form
   char* dt = ctime(&now);

   cout << "The local date and time is: " << dt << endl;

   // convert now to tm struct for UTC
   tm *gmtm = gmtime(&now);
   dt = asctime(gmtm);
   cout << "The UTC date and time is:"<< dt << endl;
}
When the above code is compiled and executed, it produces the following result:

The local date and time is: Sat Jan  8 20:07:41 2011

The UTC date and time is:Sun Jan  9 03:07:41 2011

\section{Lập trình hướng đối tượng}

Object Oriented Programming
Classes and Objects
#include <iostream>
using namespace std;

class Pacman {

    private:
      int x;
      int y;
    public:
    Pacman(int x, int y);
    void show();
};

Pacman::Pacman(int x, int y){
    this->x = x;
    this->y = y;
}

void Pacman::show(){
    std::cout << "(" << this->x << ", " << this->y << ")";
}

int main() {
    // your code goes here
    Pacman p = Pacman(2, 3);
    p.show();
    return 0;
}
Template
Function Template

#include <iostream>
#include <string>

using namespace std;

template <typename T>

T Max(T a, T b)
{
    return a < b ? b : a;
}

int main()
{

    int i = 39;
    int j = 20;
    cout << Max(i, j) << endl;

    double f1 = 13.5;
    double f2 = 20.7;
    cout << Max(f1, f2) << endl;

    string s1 = "Hello";
    string s2 = "World";
    cout << Max(s1, s2) << endl;

    double n1 = 20.3;
    float n2 = 20.4;
    // it will show an error
    // Error: no instance of function template "Max" matches the argument list
    //        arguments types are: (double, float)
    cout << Max(n1, n2) << endl;
    return 0;
}

\section{Cơ sở dữ liệu}


Database
Sqlite with Visual Studio 2013
Step 1: Create new project 1.1 Create a new C++ Win32 Console application.

Step 2: Download Sqlite DLL

2.1. Download the native SQLite DLL from: http://sqlite.org/sqlite-dll-win32-x86-3070400.zip 2.2. Unzip the DLL and DEF files and place the contents in your project’s source folder (an easy way to find this is to right click on the tab and click the “Open Containing Folder” menu item.

Step 3: Build LIB file

3.1. Open a “Developer Command Prompt” and navigate to your source folder. (If you can't find this tool, follow this post in stackoverflow Where is Developer Command Prompt for VS2013? to create it) 3.2. Create an import library using the following command line: LIB /DEF:sqlite3.def

Step 4: Add Dependencies

4.1. Add the library (i.e. sqlite3.lib) to your Project Properties -> Configuration Properties -> Linker -> Input -> Additional Dependencies. 4.2. Download http://sqlite.org/sqlite-amalgamation-3070400.zip 4.3. Unzip the sqlite3.h header file and place into your source directory. 4.4. Include the the sqlite3.h header file in your source code. 4.5. You will need to include the sqlite3.dll in the same directory as your program (or in a System Folder).

Step 5: Run test code

#include "stdafx.h"
#include <ios>
#include <iostream>
#include "sqlite3.h"

using namespace std;

int _tmain(int argc, _TCHAR* argv[])
{
   int rc;
   char *error;

   // Open Database
   cout << "Opening MyDb.db ..." << endl;
   sqlite3 *db;
   rc = sqlite3_open("MyDb.db", &db);
   if (rc)
   {
      cerr << "Error opening SQLite3 database: " << sqlite3_errmsg(db) << endl << endl;
      sqlite3_close(db);
      return 1;
   }
   else
   {
      cout << "Opened MyDb.db." << endl << endl;
   }

   // Execute SQL
   cout << "Creating MyTable ..." << endl;
   const char *sqlCreateTable = "CREATE TABLE MyTable (id INTEGER PRIMARY KEY, value STRING);";
   rc = sqlite3_exec(db, sqlCreateTable, NULL, NULL, &error);
   if (rc)
   {
      cerr << "Error executing SQLite3 statement: " << sqlite3_errmsg(db) << endl << endl;
      sqlite3_free(error);
   }
   else
   {
      cout << "Created MyTable." << endl << endl;
   }

   // Execute SQL
   cout << "Inserting a value into MyTable ..." << endl;
   const char *sqlInsert = "INSERT INTO MyTable VALUES(NULL, 'A Value');";
   rc = sqlite3_exec(db, sqlInsert, NULL, NULL, &error);
   if (rc)
   {
      cerr << "Error executing SQLite3 statement: " << sqlite3_errmsg(db) << endl << endl;
      sqlite3_free(error);
   }
   else
   {
      cout << "Inserted a value into MyTable." << endl << endl;
   }

   // Display MyTable
   cout << "Retrieving values in MyTable ..." << endl;
   const char *sqlSelect = "SELECT * FROM MyTable;";
   char **results = NULL;
   int rows, columns;
   sqlite3_get_table(db, sqlSelect, &results, &rows, &columns, &error);
   if (rc)
   {
      cerr << "Error executing SQLite3 query: " << sqlite3_errmsg(db) << endl << endl;
      sqlite3_free(error);
   }
   else
   {
      // Display Table
      for (int rowCtr = 0; rowCtr <= rows; ++rowCtr)
      {
         for (int colCtr = 0; colCtr < columns; ++colCtr)
         {
            // Determine Cell Position
            int cellPosition = (rowCtr * columns) + colCtr;

            // Display Cell Value
            cout.width(12);
            cout.setf(ios::left);
            cout << results[cellPosition] << " ";
         }

         // End Line
         cout << endl;

         // Display Separator For Header
         if (0 == rowCtr)
         {
            for (int colCtr = 0; colCtr < columns; ++colCtr)
            {
               cout.width(12);
               cout.setf(ios::left);
               cout << "~~~~~~~~~~~~ ";
            }
            cout << endl;
         }
      }
   }
   sqlite3_free_table(results);

   // Close Database
   cout << "Closing MyDb.db ..." << endl;
   sqlite3_close(db);
   cout << "Closed MyDb.db" << endl << endl;

   // Wait For User To Close Program
   cout << "Please press any key to exit the program ..." << endl;
   cin.get();

   return 0;
}

\section{Testing}

Create Unit Test in Visual Studio 2013
Step 1. Create TDDLab Solution
1.1 Open Visual Studio 2013

1.2 File ->  New Project... ->

Click Visual C++ -> Win32

Choose Win32 Console Application

Fill to Name input text: TDDLab

Click OK -> Next

1.3 In project settings, remove options:

Precompiled Header
Securirty Develoment Lifecyde(SQL) check
1.4 Click Finish

Step 2. Create Counter Class
2.1 Right-click TDDLab -> Add -> Class...

2.2 Choose Visual C++ -> C++ Class -> Add

2.3 Fill in Class name box Counter -> Finish

2.4 In Counter.h file, add this below function

int add(int a, int b);
2.5 In Counter.cpp, add this below function

int Counter::add(int a, int b) {
  return a+b;
}
Your Counter class should look like this



Step 3. Create TDDLabTest Project
3.1 Right-click Solution 'TDDLab' -> Add -> New Project...

3.2 Choose Visual C++ -> Test

3.3 Choose Native Unit Test Project

3.4 Fill to Name input text: TDDLabTest

Step 4. Write unit test
4.1 In unittest1.cpp, add header of Counter class

#include "../TDDLab/Counter.h"
4.2 In TEST_METHOD function

{
  Counter counter;
  Assert::AreEqual(2, counter.add(1, 1));
}
4.3 Click TEST in menu bar -> Run -> `All Test (Ctrl + R, A)

Step 5. Fix error LNK 2019: unresolved external symbol
5.1 Change Configuration Type of TDDLab project

Right click  TDDLab project -> Properties
General -> Configuration Type -> Static library (.lib) -> OK
5.2 Add Reference to TDDLabTest project

Right click TDDLabTest solution -> Properties -> Common Properties -> Add New Reference
Choose TDDLab -> OK -> OK
Step 6. Run Tests
Click TEST in menu bar -> Run -> `All Test (Ctrl + R, A)

Test should be passed.

\section{IDE & Debugging}

Visual Studio 2013
Install Extension

VsVim

googletest guide

Folder Structure with VS 2013

solution
│   README.md
│
|───project1
|   │   file011.txt
|   │   file012.txt
|   │
|───project2
|   │   file011.txt
|   │   file012.txt
|   │
Auto Format

Ctrl + K, Ctrl + D
Git in Visual Studio

https://git-scm.com/book/en/v2/Git-in-Other-Environments-Git-in-Visual-Studio

Online IDE
codechef ide


%\chapter{Javascript}

View online \href{http://magizbox.com/training/java/site/}{http://magizbox.com/training/java/site/}

What is Javascript?
JavaScript is a high-level, dynamic, untyped, and interpreted programming language. It has been standardized in the ECMAScript language specification. Alongside HTML and CSS, it is one of the three core technologies of World Wide Web content production; the majority of websites employ it and it is supported by all modern Web browsers without plug-ins. JavaScript is prototype-based with first-class functions, making it a multi-paradigm language, supporting object-oriented, imperative, and functional programming styles. It has an API for working with text, arrays, dates and regular expressions, but does not include any I/O, such as networking, storage, or graphics facilities, relying for these upon the host environment in which it is embedded.


\section{Installation}

Google Chrome
Pycharm

\section{IDE}

Google Chrome Developer Tools

The Chrome Developer Tools (DevTools for short), are a set of web authoring and debugging tools built into Google Chrome. The DevTools provide web developers deep access into the internals of the browser and their web application. Use the DevTools to efficiently track down layout issues, set JavaScript breakpoints, and get insights for code optimization.

\section{Basic Syntax}

1. Code Formatting
Indent with 2 spaces

// Object initializer.
var inset = {
  top: 10,
  right: 20,
  bottom: 15,
  left: 12
};

// Array initializer.
this.rows_ = [
  '"Slartibartfast" <fjordmaster@magrathea.com>',
  '"Zaphod Beeblebrox" <theprez@universe.gov>',
  '"Ford Prefect" <ford@theguide.com>',
  '"Arthur Dent" <has.no.tea@gmail.com>',
  '"Marvin the Paranoid Android" <marv@googlemail.com>',
  'the.mice@magrathea.com'
];

// Used in a method call.
goog.dom.createDom(goog.dom.TagName.DIV, {
  id: 'foo',
  className: 'some-css-class',
  style: 'display:none'
}, 'Hello, world!');
2. Naming
functionNamesLikeThis
variableNamesLikeThis
ClassNamesLikeThis
EnumNamesLikeThis
methodNamesLikeThis
CONSTANT_VALUES_LIKE_THIS
foo.namespaceNamesLikeThis.bar
filenameslikethis.js.
3. Comment
Use JSDoc

3.1 Class Comment
/**
 * Class making something fun and easy.
 * @param {string} arg1 An argument that makes this more interesting.
 * @param {Array.<number>} arg2 List of numbers to be processed.
 * @constructor
 * @extends {goog.Disposable}
 */
project.MyClass = function(arg1, arg2) {
  // ...
};
goog.inherits(project.MyClass, goog.Disposable);
3.2 Method Comment
/**
 * Operates on an instance of MyClass and returns something.
 * @param {project.MyClass} obj Instance of MyClass which leads to a long
 *     comment that needs to be wrapped to two lines.
 * @return {boolean} Whether something occurred.
 */
function PR_someMethod(obj) {
  // ...
}
4. Expression and Statements
Expression
A fragment of code that produces a value is called an Expression

22
"this is an epression"
(5 > 6) ? false : true
Statements
The Simplest kind of stagement is an expression with a semi colon

!false;
5 + 6;
5. Loop and iteration
while
var number = 0;
while (number <= 12) {
  console.log(number);
  number = number + 2;
}
do..while
do {
  var yourName = prompt("Who are you?");
} while (!yourName);
console.log(yourName);
for
for (var i = 0; i < 10; i++) {
  console.log(i);
}
6. Function
6.1 Defining a Function
var square = function(x) {
  return x * x;
};
square(5);
6.2 Scope
Scope is the area where contains all variable or function are living.
Scope has some rules:
Child Scope can access all variable and function in parent Scope. (E.g: Local Scope can access Global Scope)
function saveName(firstName) {
    var temp = "temp";
    function capitalizeName() {
        temp = temp + " here";
        return firstName.toUpperCase();
    }
    var capitalized = capitalizeName();
    return capitalized;
}
alert(saveName("Robert"));
But parent Scope can access variable and function inside children scope (E.g: Global Scope cannot acces to local Scope)
function talkDirty () {
    var saying = "Oh, you little VB lover, you";
    return alert(saying);
}
alert(saying); //->Error
6.3 Call Stack
The storage where computer stores context is called CALL STACK.

// CALL STACK
function greet(who) {
    console.log("Hello " + who);
    ask("How are you?");
    console.log("I'm fine");
};

function ask(question) {
    console.log("well, " + question);
};

greet("Harry");
console.log("Bye");
Out of Call Stack

function chicken() {
    return egg();
}

function egg() {
    return chicken();
}
console.log(chicken() + " came first");
6.4. Optional Argument
We can pass too many or too few arguments to the function without any SyntaxError.
If we pass too much arguments, the extra ones are ignored
If we pass to few arguments, the missing ones get value undefined
function power(base, exponent) {
    if (exponent == undefined) {
        exponent = 2;
    }
    var result = 1;
    for (var count = 0; count < exponent; count++) {
        result = result * base;
    }
    return result;
}
console.log(power(4));
console.log(power(4,3));
upside: flexible
downside: hard to control the error

6.5 Closure
Look at this example:

function sayHello(name){
    var text = 'Hello' + name;
    var say = function(){
        console.log(text);
    }
    return say;
}
var say2 = sayHello("ahaha");
say2();
if in C program, does say2() work?
The answer is nope! Because in C program, when a function returns, the Stack-flame will be destroyed, and all the local variable such as text will undefinded. So, when say2() is called, there is no text anymore, and the error, will be shown!
But, in JavaScript, This code works!! Because, it provides for us an Object called Closure! Closure is borned when we define a function in another function, it keep all the live local variable. So, when say2() is called, the closure will give all the value of local variable outside it, and text will be identity.!

var globalVariable = 10;
function func(){
    var name = "xxx";
    function getName(){
        return name;
    }
    function speak(){
        var sound = "alo";
        function scream(){
            console.log(globalVariable);
            console.log(name);
            return "aaaaaaaaaa!";
        }
        function talk(){
            var voice = getName() + " speak " + sound;
            console.log(voice);
            return voice;
        }
        scream();
        talk();
    }
    speak();
}
func();
6.6. Recursion
Recursion is function can call itself, as long as it is not overflow

function power(base, exponent){
    if (exponent == 0){
        return 1;
    }
    else{
        return base * power(base, exponent -1);
    }
}
console.log(power(2,3));

function FindSolution(target){
    function Find(start, history){
        if (start == target){
            return history;
        }
        else if (start > target){
            return null;
        }
        else{
            return Find(start + 5, "(" + history + " + 5 ") ||
            Find(start * 3, "(" + history + " * 3)");
        }
    }
    return Find(1, "1");
}
console.log(FindSolution(25));
6.7. Arguments object
The arguments object contains all parameters you pass to a function.

function argumentCounter() {
    console.log("you gave me", arguments.length, "argument.");
}
argumentCounter("Straw man", "Tautology", "Ad hominem");
6.8. Higher-Order Function
###Filter array
var ancestry = JSON.parse(ANCESTRY_FILE);
console.log(ancestry.length);

function filter(array, test) {
    var passed = [];
    for (var i = 0; i < array.length; i++){
        if (test(array[i])){
            passed.push(array[i]);
        }
    }
    return passed;
}
console.log(filter(ancestry, function(person){
    return person.born > 1900 && person.born < 1925;
}));

### TRANSFORMING WITH A MAP
function map(array, transform) {
  var mapped = [];
  for (var i = 0; i < array.length; i++)
    mapped.push(transform(array[i]));
  return mapped;
}

var overNinety = ancestry.filter(function(person) {
  return person.died - person.born > 90;
});
console.log(map(overNinety, function(person) {
  return person.name;
}));


### REDUCE
function reduce(array, combine, start) {
  var current = start;
  for (var i = 0; i < array.length; i++)
    current = combine(current, array[i]);
  return current;
}
console.log(reduce([1, 2, 3, 4], function(a, b) {
  return a + b;
}, 0));
#Problem: using map and reduce, transform [1,2,3,4] to [1,2],[3,4]

var a = [1, 2, 3, 4]
a = _.map(a, function(i){
    if(i % 2 == 0){
        return [[],[i]]
    } else {
        return [[i], []]
    }
});
a = _.reduce(a, function(x, y){
   return [x[0].concat(y[0]), x[1].concat(y[1])]
})


### BINDING FUNCTION
var theSet = ["Carel Haverbeke", "Maria van Brussel",
              "Donald Duck"];
function isInSet(set, person) {
  return set.indexOf(person.name) > -1;
}

console.log(ancestry.filter(function(person) {
  return isInSet(theSet, person);
}));
console.log(ancestry.filter(isInSet.bind(null, theSet)));
What's the cleanest way to write a multiline string in JavaScript? [duplicate] ↩

Google JavaScript Style Guide ↩

\section{Data Structure}

\subsection{Number}

Some example of number: 10, 1.234, 1.99e9, NaN, Infinity, -Infinity

console.log(2.99e9);
console.log(0 /0);
console.log(1 /0);
console.log(-1 /0);
Automatic Conversion

console.log(8 * null); // -> 0
console.log("5" - 1); // -> 4
console.log("5" + 1); //-> 51
console.log(false == 0) //-> true

\subsection{String}

sprintf
In index.html

<script src="cdnjs.cloudflare.com/ajax/libs/sprintf/1.0.3/sprintf.js"/>

In script.js

// arguments
sprintf("%1$s %2$s", "hello", "sprintf")
# hello sprintf

// object
var user = {
    name: "Dolly"
}
sprintf("Hello %(name)s", user)
# Hello Dolly

// array of object
var users = [
    {name: "Dolly"},
    {name: "Molly"}
]
sprintf("Hello %(users[0].name)s and %(users[1].name)s", {users: users})
# Hello Dolly and Molly
Multiline String
str = "\
line 1\
line 2\
line 3";
Regular Expression in JavaScript
This lab is based on Chapter9: EloquentJavaScript

Creating a regular expression
There are 2 ways:

var re1 = new RegExp("abc");
var re2 = /abc/
there are some special characters such as question mark, or plus sign. If you want to use them, you have to use backslash. Like this:

var eighteen = /eighteen\+/;
var question = /question\?/;
Testing for match
Regular Express has a number of method. Simplest is test

console.log(/abc/.test("abcd"));
console.log(/abc/.test("abxde"));
Matching a set of character []: Put a set of characters between 2 square bracket

console.log(/[0123456789]/.test("1245"));
console.log(/[0-9]/.test("1");
console.log(/[0-9]/.test("acd");
console.log(/[0-9]/.test("aaascacas1"));
There are some special character:
\d Any digit character (Like [0-9])

var datetime = /\d\d-\d\d-\d\d\d\d\s\d\d:\d\d/;
console.log(datetime.test("16-06-2016 14:09"));
console.log(dateTime.test("30-jan-2003 15:20"));
\w An alphanumeric character (“word character”)

var word = /\w/;
console.log(word.test("@#@#"));
\s Any whitespace character (space, tab, newline, and similar)

var space = /\d\.\s+abc/;
console.log(space.test("1. abd"));
console.log(space.test("1.     abd"));
console.log(space.test("1.abd"));
\D A character that is not a digit

var notDigit = /\D/;
console.log(notDigit.test("ww"));
console.log(notDigit.test("1a"));
console.log(notDigit.test("1124"));
\W A nonalphanumeric character

var nonAlphanumbericChar = /\W/;
console.log(nonAlphanumbericChar.test("abc12231"));
console.log(nonAlphanumbericChar.test("!@#%{}_"));
\S A nonwhitespace character

var nonWhiteSpace = /\S/;
console.log(nonWhiteSpace.test("abc123"));
console.log(nonWhiteSpace.test("1.  abcd"));
console.log(nonWhiteSpace.test("  "));
"." Any character except for newline

var anyThing = /...\./;
console.log(anyThing.test("abc."));
console.log(anyThing.test("acbacd."));
console.log(anyThing.test("acba"));
"^" Using caret character to match any except the ones

var notBinary = /[^01]/;
console.log(notBinary.test("01101011100"));
console.log(notBinary.test("01021010010"));
Repeating parts of Pattern
The square bracket [] above only match 1 digit. How can regex match more than 1 digit?
"+" Match one or more
"*" Match zero or more

console.log(/\d+/.test(1234));
console.log(/\d+/.test());

console.log(/\d*/.test(1234));
console.log(/\d*/.test())
"?" Question mark test a character exist or not is still oke

var ball = /bal?l/;
console.log(ball.test("ball"));
console.log(ball.test("bal"));
{a,b} the character before exist from a to b times. Check datetime:

var datetime = /\d{1,2}-\d{1,2}-\d{4} \d{1,2}:\d{1,2}/;
console.log(datetime.test("20-12-2015 14:09"));
var checkTimes = /waz{3,5}up/;
console.log(checkTimes.test("wazzzzzup"));
console.log(checkTimes.test("wazzzup"));
console.log(checkTimes.test("wazup"));
Grouping Subexpressions
() using prentheses to make whole group like one character

var cartoonCrying = /boo+(hoo+)+/i; //i to match all Captalize or normal text
console.log(cartoonCrying.test("Boohoooohoohooo"));
console.log(cartoonCrying.test("boohoooohooOOO"));
Matches and group
Test is a simplest method, and it only return true or false.
exec (execute) is anther method in regex. It returns null if no match, and object if match.

var match  = /\d+/.exec("one two 100");
console.log(match);
console.log(match.input);
console.log(match.index);
if in the expression has a group subexpression, then it will return the text contain this subexpress, and the text match this subexpress:

var quotedText = /'([^']*)'/;
console.log(quotedText.exec("she said 'hello'"));
and if the subexpression appears one more times, then the result will be displayed the last match one.

console.log(/bad(ly)?/.exec("bad"));
console.log(/(\d)+/.exec("123"));
The date type
create new Date(). return the current time

var date =  new Date();
console.log(new Date(2009, 11, 9);
console.log(new Date(2009, 11, 9, 23, 59, 61));
<!--TimeStamp-->
console.log(new Date(2009, 11, 9, 23, 59, 61).getTime());
console.log(new Date(1260378001000));
<!--getFullYear, getMonth,...-->
var date = new Date();
console.log(date.getFullYear());
console.log(date.getMonth());
console.log(date.getDate());
console.log(date.getHours());
console.log(date.getMinutes());
console.log(date.getSeconds());
Word and string boundaries
console.log(/cat/.test("concatenate"));
console.log(/cat/.test("con123cat-129e0enate"));
console.log(/\bcat\b/.test("concatenate"));
console.log(/\bcat\b/.test("con123cat-129e0enate"));
Choice patterm
Only one in the list beween the "|" match

var animalCount = /\b\d+ (pig|cow|chicken)s?\b/;
console.log(animalCount.test("15 pigs"));
console.log(animalCount.test("15 pigchickens"));
Replace
Replace will find the first match and replace.if we want to replace all matches, using "g" behind the expresssion

console.log("papa".replace("p", "m"));
console.log("Borobudur".replace(/[ou]/, "a"));
console.log("Borobudur".replace(/[ou]/g, "a"));
Replace can refer back to the matched, and using them

console.log("Le, Khanh\nNguyen, Hung\nDuong, Bach".replace(/([\w]+), ([\w]+)/g, "$1 $2"));
Greed
function stripComments(code) {
  return code.replace(/\/\/.*|\/\*[^]*\*\//g, "");
}
console.log(stripComments("1 + /* 2 */3"));
// → 1 + 3
console.log(stripComments("x = 10;// ten!"));
// → x = 10;
console.log(stripComments("1 /* a */+/* b */ 1"));
// → 1  1
Search method
Search method return the first index if the regular expression match.
And return -1 if not found

console.log("  word".search(/\S/));
// → 2
console.log("    ".search(/\S/));
// → -1
The last index property
In the regular expression has a property is lastIndex. And when this Regex do some method, it will start from the lastIndex. And after doing something, the lastIndex will update to the behind the index of the match exec.

var pattern = /y/g;
pattern.lastIndex = 3; //lastIndex update to 3
var match = pattern.exec("xyzzy"); //lastIndex update to 5
console.log(pattern.lastIndex);

match = pattern.exec("xyzzyxxx"); //Not match any "y" from index 5
console.log(match.index);
console.log(pattern.lastIndex);
Looping Over the Line
Applying the hepoloris of lastIndex, we can using while to do something like this:

var input = "A string with 3 numbers in it... 42 and 88.";
var number = /\b(\d+)\b/g;
var match;
while (match = number.exec(input))
  console.log("Found", match[1], "at", match.index);

\subsection{Collection}

Some useful methods with array
push and pop
var a = [1,2,3,4];
console.log(a.pop(), a);
console.log(a.push(3), a);
shift and unshift
console.log(a.shift(), a);
console.log(a.unshift(1), a);
indexOf and lastIndexOf
var b = [1,2,3,4,2,3,1];
console.log(b.indexOf(1));
console.log(b.lastIndexOf(1));
slice
console.log([0,1,2,3,4].slice(2,4));
console.log([0,1,2,3,4].slice(2));
concat
var a = [1,2,3];
var b = [4,5,6];
a.concat(b);
console.log(a);

\subsection{Datetime}

Current Time
moment().format('MMMM Do YYYY, h:mm:ss a');
Moment.js ↩

\subsection{Boolean}

Boolean has only 2 values: true and false

console.log("Abc" < "Abcd") // -> true
console.log("abc" < "Abcd") // -> false
console.log("123" == "123") // -> true
console.log(NaN == NaN) // -> false
what is the different?

console.log("5" == 5);
console.log("5" === 5);

\subsection{Object}

Object
Define an object
var object = {
  number: 10,
  string: "string",
  array: [1,2,3],
  object: {a: 1, b: 2}
}
Add new property to object
object.newProperty = "value";
object['key'] = 'value';
delete property
delete object.newProperty;
Window object (global object)
The Global scope is stored in an object which called window

function test(){
    var local = 10;
    console.log("local" in window);
    console.log(window.local);
}
test();
var global = 10;
console.log("global" in window);
console.log(window.global);

\section{OOP}


1. Classes and Objects
Constructor
function Ball(position){
    this.position = position;
    this.display = function(){
        console.log(this.position[0], ", ", this.position[1]);
    }
}

ball = new Ball([2, 3]);
ball.display();
2. Inheritance
Person = function (name, birthday, job) {
  this.name = name;
  this.birthday = birthday;
  this.job = job;
};

Person.prototype.display = function () {
  console.log(this.name, "\n");
  console.log(this.birthday, "\n");
  console.log(this.job, "\n");
};

Politician = function (name, birthday) {
  Person.call(this, name, birthday, "Politician");
};
Politician.prototype = Object.create(Person.prototype);
Politician.prototype.constructor = Politician;

var person1 = new Person("Barack Obama", "04/08/1961", "Politician");
var person2 = new Politician("David Cameron", "09/10/1966");
person1.display();
person2.display();

Object-Oriented Programming
var rabbit = {};
rabbit.speak = function(line){
console.log("The rabit says:'" + line + "'");
 };
rabbit.speak("I'm alive");

function speak(line){
   console.log("The "+ this.type + " rabbit says '" + line + "'");
 }

var whiteRabbit = {type: "white", speak: speak};
var fatRabbit = {type: "fat", speak: speak};
whiteRabbit.speak("Oh my ears and whiskers, " + "how late it's getting!");
fatRabbit.speak("I could sure use a carrot right now");

// Prototype
// Prototype is another object that is used as a fallback source of properties
// When object request a property that it does not have, its prototype will be searched for the property
var empty = {};
console.log(empty.toString);
console.log(empty.toString);

// Get prototype of an object 2 ways:
console.log(Object.getPrototypeOf({}) == Object.prototype);
console.log(Object.getPrototypeOf(Object.prototype));

// Using Object.create to create an object with an specific prototype
var protoRabbit = {
  speak: function(line){
    console.log("The " + this.type + " rabbit says '" + line + "'");
  }
};

var killerRabbit = Object.create(protoRabbit);
killerRabbit.type = "Killer";
killerRabbit.speak("Skreeee!");

// Constructor
function Rabbit(type){
   this.type = type;
}
var killerRabbit = new Rabbit("Killer");
var blackRabbit = new Rabbit("black");
console.log(blackRabbit.type);

// using prototype to add a new method
Rabbit.prototype.speak = function(line) {
  console.log("The " + this.type + " rabit says '" + line + "'");
};
blackRabbit.speak("Doom...");


// OVERRIDING DERIVED PROPERTIES
Rabbit.prototype.teeth = "small";
console.log(killerRabbit.teeth);

killerRabbit.teeth = "Long, sharp, and bloody";
console.log(killerRabbit.teeth);
console.log(blackRabbit.teeth);
console.log(Rabbit.prototype.teeth);

// PROTOTYPE INTERFERENCE
// A prototype can be used at any time to add methods, properties
// to all objects based on it
Rabbit.prototype.dance = function (){
  console.log("The " + this.type + " rabbit dances a jig");
};
killerRabbit.dance();
// but there is a problem:
var map = {};
function storePhi(event, phi){
   map[event] = phi;
}

storePhi("pizza", 0.069);
storePhi("touched tree", -0.081);
console.log(map);

Object.prototype.nonsense = "hi";
for (var name in map) {
  console.log(name);
}
console.log("nonsense" in map);
console.log("toString" in map);
delete Object.prototype.nonsense;
//  we can use Object.defineProperty to solve it
Object.defineProperty(Object.prototype, "hiddenNonsense", {
   enumerable: false,
   value: "hi"
});

for (var name in map) {
  console.log(name);
}
console.log(map.hiddenNonsense);
// but there still has a problem
console.log("toString" in map);
console.log(map.hasOwnProperty("toString"));

// PROTOTYPE-LESS OBJECTS
// if we only want to create an fresh object, without prototype then we tranform null to create
var map = Object.create(null);
map["pizza"] = 0.09;
console.log("toString" in map);
console.log("pizza" in map);

// POLYMORPHISM
// laying out a table: example for polymorphism
function rowHeights(rows) {
    return rows.map(function(row){
        return row.reduce(function(max, cell) {
            return Math.max(max, cell.minHeight());
        }, 0);
    });
}

function colWidths(rows) {
    return rows[0].map(function(_, i) {
        return rows.reduce(function(max, row){
            return Math.max(max, row[i].minWidth());
        }, 0);
    });
}

function drawTable(rows) {
    var heights = rowHeights(rows);
    var widths = colWidths(rows);

    function drawLine(blocks, lineNo) {
        return blocks.map(function(block) {
            return block[lineNo];
        }).join(" ");
    }

    function drawRow(row, rowNum){
        var blocks = row.map(function(cell, colNum) {
           return cell.draw(widths[colNum], heights[rowNum]);
        });
        return blocks[0].map(function(_, lineNo) {
            return drawLine(blocks, lineNo);
        }).join("\n");
    }

    return rows.map(drawRow).join("\n");
}

function repeat(string, times){
    var result = "";
    for (var i = 0; i < times; i++){
        result += string;
    }
    return result;
}

function TextCell(text){
    this.text = text.split("\n");
}
TextCell.prototype.minWidth = function(){
    return this.text.reduce(function(width, line){
        return Math.max(width, line.lenght);
    }, 0);
};
TextCell.prototype.minHeight = function(){
    return this.text.length;
}
TextCell.prototype.minHeight = function(){
    return this.text.lenght;
}
TextCell.prototype.minHeight = function(){
    return this.text.length;
}
TextCell.prototype.draw = function(width, height){
    var result = [];
    for (var i = 0; i < height; i++){
        var line = this.text[i] || "";
        result.push(line + repeat(" ", width - line.length));
    }
    return result;
}

var rows = [];
for (var i = 0; i < 5; i++){
    var row = [];
    for (var j = 0; j < 5; j++){
        if ((i + j) % 2 == 0){
            row.push(new TextCell("1234"));
        } else{
            row.push(new TextCell("5"));
        }
    }
    rows.push(row);
}
console.log(drawTable(rows));

// // GETTERS AND SETTERS
// var pile = {
//     elements: ["eggshell", "orange peel", "worm"],
//     get height(){
//         return this.elements.length;
//     },
//     set height(value) {
//         console.log("Ignoring attemp to set high to ", value);
//     }
// };

// console.log(pile.height);
// pile.height = 100;
// console.log(pile.height);

[1]: Introduction to Object-Oriented JavaScript [2]: How to call parent constructor?

\section{Networking}

POST
$.ajax({
  type: "POST",
  url: "http://service.com/items",
  data: JSON.stringify({"name": "new item"}),
  contentType: 'application/json'
}).done(function (data) {
  console.log(data)
}).fail(function () {
});

\section{Logging}


Javascript Logging
Having a fancy JavaScript debugger is great, but sometimes the fastest way to find bugs is just to dump as much information to the console as you can.

console.log
console.assert
console.error

\section{Documentation}

Components
jsdoc (with docdash template)

JSDoc is an API documentation generator for JavaScript, similar to JavaDoc or PHPDoc. You add documentation comments directly to your source code, right along side the code itself. The JSDoc Tool will scan your source code, and generate a complete HTML documentation website for you.

gulp, PyCharm

Usage
Step 1. Install gulp-jsdoc
npm install --save-dev gulp gulp-jsdoc docdash
Step 2. Create documentation task
Create documentation task in gulpfile.js file

var template = {
  "path": "./node_modules/docdash"
};

gulp.task('docs', function(){
  return gulp.src("./src/*.js")
    .pipe(jsdoc('./docs', template));
});
Step 3. Refresh Gulp tasks
In pycharm, click to refresh button in gulp window.

Step 4. Add comment to your code
Add comment to your code, You can see an example: should.js

/**
 * Simple utility function for a bit more easier should assertion
 * extension
 * @param {Function} f So called plugin function. It should accept
 * 2 arguments: `should` function and `Assertion` constructor
 * @memberOf should
 * @returns {Function} Returns `should` function
 * @static
 * @example
 *
 * should.use(function(should, Assertion) {
 *   Assertion.add('asset', function() {
 *      this.params = { operator: 'to be asset' };
 *
 *      this.obj.should.have.property('id').which.is.a.Number();
 *      this.obj.should.have.property('path');
 *  })
 * })
 */
should.use = function(f) {
  f(should, should.Assertion);
  return this;
};

Types: boolean, string, number, Array (see more)

Step 5. Run docs task
In pycharm, click to docs task in gulp window.

\section{Error Handling}

In javascript bugs may be displayed is NaN or underfined and program still run but after that, the wrong value can cause some mistake when we use it So, finding bugs and fix them is the quiet hard work in javascript But we can do, and this job is called debugging

STRICT MODE
This is the way to find errors that javascript ignores. Example is using an undefined variable. if we dont use strick mode, then everything will be ok, but if using, the error will be shown

function SpotProblem(){
//     "use strict";
    for (counter = 0; counter < 10; counter++){
        console.log("Good!");
    }
}
SpotProblem();
console.log(counter);
strick mode can find error when using this in local, but it is still in global. Example: When we forget to declare the key word "new" when create an new Object

"use strict";
function Person(name){
    this.name = name;
}
var john = Person("John");
console.log(name);
And there are another cases, that trick mode is not allowed: Delete an object is not allowed

"use strict";
var x = 3.14;
delete x;

"use strict";
var obj = {v1: 3, v2: 4};
delete pbj;

"use strict";
var func = function(){};
delete func;
Duplicate parameter is not allowed

"use strict";
var func = function(a1, a1){
    console.log(a1);
}
Reserve Word is not allowed to name variable

"use strict";
var arguments = 5;
var eval = 6;
console.log(arguments);
console.log(eval);
TESTING
Testing makes sure that the program working well, and if there are any changes, testing will automatic show us the error, thus, we know where need to fix

function Vector(x, y){
    this.x = x;
    this.y = y;
}
Vector.prototype.plus = function(other){
    return new Vector(this.x + other.x, this.y + other.y);
}

function TestVector(){
    var p1 = new Vector(10, 20);
    var p2 = new Vector(-10, 5);
    var p3 = p1.plus(p2);

    if (p1.x !== 10) return "fail: x property";
    if (p1.y !== 20) return "fail: y property";
    if (p2.x !== -10) return "fail: nagative x property";
    if (p2.y !== 5) return "fail: y property";
    if (p3.x !== 0) return "fail: x property from plus";
    if (p3.y !== 25) return "fail: y property from plus";
    return "Vector is Oke";
}
TestVector();
DEBUGGING
when the testing is fail, we have to debug to find the bugs.
The first we should guess the bug. And then we put break point in the line, we assume it make bug
If that is the exactly bug we want to find, then we fix it, and write more test for this case
In this example code below, the function convert the number in the decima to another. we run and see the result is wrong, so we guess that the error may be caused by the result variable, then we put break point in the line contains result variable.

function ConvertNumber(n, base) {
  var result = "", sign = "";
  if (n < 0) {
    sign = "-";
    n = -n;
  }
  do {
    result = String(n % base) + result;
    n /= base; //-> n = Math.floor(n / base);
  } while (n > 0);
  return sign + result;
}
console.log(ConvertNumber(13, 10));
console.log(ConvertNumber(14, 2));
ERROR PROPAGATION
Sometime our code is working well with normal input. But with special one, they can cause error. So, we have to consider all situation can make Flaws, and handling them.
This example code below has an if..else to handle the wrong input if user types not a number in the prompt input

function promptNumber(question) {
  var result = Number(prompt(question, ""));
  if (isNaN(result)) return null;
  else return result;
}
console.log(promptNumber("How many trees do you see?"));
EXCEPTION
In the Error Propagation, we can control the errors if we know them. But what will happen if we don't know the error? For solving this problem, javascript provides for us an try...catch.. to control error we dont know or not sure

try {
    throw new Error("Invalid defination");
} catch (error){
    console.log(error);
}

function promtDirection(question){
    var result = prompt(question, "");
    if (result.toLowerCase() == "left") return "L";
    if (result.toLowerCase() == "right") return "R";
    throw new Error("Invalid direction: " + result);
}

function look(){
    if (promtDirection("Which way?") == "L") {
        return "a house";
    }
    else{
        return "two angry bears";
    }
}

try {
    console.log("you see", look());
} catch (error) {
    console.log("Something went wrong: " + error);
}
CLEAN UP AFTER EXCEPTIONS
We have a block of code below:

var context = null;
function withContext(newContext, body){
  var oldContext = context;
  context = newContext;
  var result = body();
  context = oldContext;
  return result;
}
withContext("new", function(){
  var a = b/0;
  return a;
});
What would happend with context? It cannot be excute the last line code, because in withContext function, it will throw off the stack by an exception. So javascript provides a try...finally...

var context = null;
function withContext(newContext, body){
  var oldContext = context;
  context = newContext;
  try{
    return body();
  } finally {
    context = oldContext;
  }
}
withContext("new", function(){
  var a = b/0;
  return a;
});
SELECTIVE CATCHING
There are some errors cannot handle by environment. So, if we let the error go through, it can cause broken program.
For examnple, the Error() in environment cannot catch the infinitive loop in the try block, if we dont catch this problem, the programm will crash soon

for (;;) {
  try {
    var dir = promtDirection("Where?");
    console.log("You chose ", dir);
    break;
  } catch (e) {
    console.log("Not a valid direction. Try again.");
  }
}
The loop will break out if the promptDirection() can excute.
But it doesn't. Because it is not defined before, so the environment catch it and go through the catch to show error
The circle again and again will make the program crash.
So we will create a special Exception.

function InputError(message){
  this.message = message;
  this.stack = (new Error()).stack;
}
InputError.prototype = Object.create(Error.prototype);
InputError.prototype.name = "InputError";
Error: has an property is stack. it contains all exception, which environment can catch. Then, we have the promptDirection function to return the result if Enter valid format, or an exception if invalid

function promptDirection(question){
  var result = prompt(question, "");
  if (result.toLowerCase() == "left") return "L";
  if (result.toLowerCase() == "right") return "R";
  throw new InputError("Invalid direction: " + result);
}
Finally, we can catch all exception we want

for (;;){
  try {
    var dir = promptDirection("Where?");
    console.log("You choose ", dir);
    break;
  } catch(e) {
    if (e instanceof InputError){
      console.log("Not a valid direction. Try again. ");
    }
    else {
      throw e;
    }
  }
}
ASSERTIONS
function AssertionFailed(message) {
  this.message = message;
}
AssertionFailed.prototype = Object.create(Error.prototype);

function assert(test, message) {
  if (!test)
    throw new AssertionFailed(message);
}

function lastElement(array) {
  assert(array.length > 0, "empty array in lastElement");
  return array[array.length - 1];
}

\section{Testing}

Mocha
Mocha is a feature-rich JavaScript test framework running on Node.js and the browser, making asynchronous testing simple and fun. Mocha tests run serially, allowing for flexible and accurate reporting, while mapping uncaught exceptions to the correct test cases.

Installation
bower install -D mocha chai
Usage
Step 1. Make index.html

<!DOCTYPE html>
<html>
<head>
  <meta charset="utf-8">
  <title>Tests</title>
  <link rel="stylesheet" media="all" href="mocha.css">
</head>
<body>
  <div id="mocha"></div>
  <script src="mocha.js"></script>
  <script src="chai.js"></script>
  <script src="functions.js"></script>
  <script>mocha.setup('bdd'); chai.should();</script>
  <script src="tests.js"></script>
  <script>mocha.run();</script>
</body>
</html>
Step 2. Edit functions.js

function sum(a, b){
  return a + b;
}

function asynchronusSum(a, b){
  return new Promise(function(fulfill, reject){
    fulfill(a + b);
  });
}
Step 3. Edit tests.js

describe('Calculator', function() {
  this.timeout(5000);
  describe('#sum()', function() {
    it('should return sum of two number', function() {
      sum(2, 3).should.equal(5)
    });
  });

  describe('#asynchronusSum()', function() {
    it('should return sum of two number', function(done) {
      asynchronusSum(2, 3).then(function(output){
          output.should.equal(5);
          done();
      })
    });
  });
});

\section{Package Manager}

Bower
A package manager for the web

Web sites are made of lots of things — frameworks, libraries, assets, utilities, and rainbows. Bower manages all these things for you.

Bower works by fetching and installing packages from all over, taking care of hunting, finding, downloading, and saving the stuff you’re looking for. Bower keeps track of these packages in a manifest file, bower.json. How you use packages is up to you. Bower provides hooks to facilitate using packages in your tools and workflows.

Bower is optimized for the front-end. Bower uses a flat dependency tree, requiring only one version for each package, reducing page load to a minimum.

http://bower.io/

[code] bower install jquery underscore moment sprintf -S [/code]

HTML <bower based>

<script src="./bower_components/jquery/dist/jquery.js"></script>
<script src="./bower_components/moment/moment.js"></script>
<script src="./bower_components/underscore/underscore.js"></script>
<script src="./bower_components/sprintf/src/sprintf.js"></script>
HTML <cdn based>

<script src="//cdnjs.cloudflare.com/ajax/libs/jquery/3.0.0-beta1/jquery.js"></script>
<script src="//cdnjs.cloudflare.com/ajax/libs/underscore.js/1.8.3/underscore.js"></script>
<script src="//cdnjs.cloudflare.com/ajax/libs/sprintf/1.0.3/sprintf.js"></script>

\section{Build Tool}

Gulp

Automate and enhance your workflow

Here's some of the sweet stuff you try out with this repo.

Compile CoffeeScript (with source maps!)
Compile Handlebars Templates
Compile SASS with Compass
LiveReload
require non-CommonJS code, with dependencies
Set up module aliases
Run a static Node server (with logging)
Pop open your app in a Browser
Report Errors through Notification Center
Image processing
Installation
npm install -S gulp gulp-concat
Usage
Watch

var gulp = require('gulp');
var concat = require('gulp-concat');
var uglify = require('gulp-uglify');
var jsdoc = require("gulp-jsdoc");

var third_parties = [
  "bower_components/jquery/dist/jquery.js",
  "bower_components/bootstrap/dist/js/bootstrap.js",
  "bower_components/underscore/underscore.js",
  "bower_components/ring/ring.js",
  "bower_components/moment/moment.js",
  "bower_components/sprintf/src/sprintf.js",
  "bower_components/uri.js/src/URI.js",
  "bower_components/run/run.js"
];

var modules = [
  "modules/your_script.js"
];

gulp.watch(third_parties, ['js_thirdparty']);
gulp.watch(modules, ['js_modules']);

gulp.task('js_thirdparty', function () {
  return gulp
    .src(third_parties)
    .pipe(concat('third_party.uglify.js'))
    .pipe(uglify())
    .pipe(gulp.dest('./scripts'));
});

gulp.task('js_modules', function () {
  return gulp
    .src(modules)
    .pipe(concat('modules.uglify.js'))
    //.pipe(uglify())
    .pipe(gulp.dest('./scripts'));
});

gulp.task('documentation', function () {
  return gulp
    .src("./modules/*/*.js")
    .pipe(jsdoc('./documentation'));
});

gulp.task('default', ['js_thirdparty', 'js_modules']);
http://gulpjs.com/

Deprecated
grunt

\section{Make Module}

Make Module
sample modules: underscore, momentjs

Folder Structure
|- docs
|- test
|- src
|   |-- your_module.js
|- .gitignore
|- bower.json




%\chapter{Java}

01/11/2017: Java đơn giản là gay nhé. Không chơi. Viết java chỉ viết thế này thôi. Không viết hơn. Thề!
\chapter{PHP}

PHP là ngôn ngữ lập trình web dominate tất cả các anh tài khác mà (chắc là) chỉ dịu đi khi mô hình REST xuất hiện. Nhớ lần đầu gặp bạn Laravel mà cảm giác cuộc đời sang trang.

Cuối tuần này lại phải xem làm sao cài được xdebug vào PHPStorm cho thằng em tập tành lập trình. Haizzz

### Tương tác với cơ sở dữ liệu

Liệt kê danh sách các bản ghi trong bảng groups

```
$sql = "SELECT * FROM `groups`";
$groups = mysqli_query($conn, $sql);
```

Xóa một bản ghi trong bảng groups

```
$sql = "DELETE FROM `groups` WHERE id = `5`";
mysqli_query($conn, $sql);
```

### Cài đặt debug trong PHPStorm

https://www.youtube.com/watch?v=mEJ21RB0F14

(1) XAMPP

- Download XAMPP (cho PHP 7.1.x - do XDebug chưa chính thức hỗ trợ 7.2.0)
https://www.apachefriends.org/xampp-files/7.1.12/xampp-win32-7.1.12-0-VC14-installer.exe
- Install XAMPP xampp-win32-7.1.12-0-VC14-installer.exe
- Truy cập vào địa chỉ http://localhost/dashboard/phpinfo.php để kiểm tra cài đặt đã thành công chưa

(2) Tải và cài đặt PHPStorm

- Download PHPStorm https://download-cf.jetbrains.com/webide/PhpStorm-2017.3.2.exe
- Install PHPStorm

(3) Tạo một web project trong PHPStorm
- Chọn interpreter trỏ đến PHP trong xampp

(4) Viết một chương trình add.php

```php
$a = 2;
$b = 3;
$c = $a + $b;

echo $c;
```

Click vào `add.php`, chọn Debug, PHPStorm sẽ báo chưa cài XDebug

(5) Cài đặt XDebug theo hướng dẫn tại https://gist.github.com/odan/1abe76d373a9cbb15bed

Click vào add.php, chọn Debug

(6) Cài đặt XDebug với PHPStorm Marklets
Vào trang https://www.jetbrains.com/phpstorm/marklets/
Trong phần Zend Debugger
- chọn cổng 9000
- IP: 127.0.0.1
Nhấn nút Generate

Bookmark các link &quot;Start debugger&quot;, &quot;Stop debugger&quot; lên trình duyệt

(7) Debug PHP từ trình duyệt

* Vào trang http://localhost/untitled/add.php
* Click vào bookmark Start debugger
* Trong PHPStorm, nhấn vào biểu tượng &quot;Start Listening for PHP Debug Connections&quot;
* Đặt breakpoint tại dòng thứ 5
* Refresh lại trang http://localhost/untitled/add.php, lúc này, breakpoint sẽ dừng ở dòng 5

%\chapter{R}

View online \href{http://magizbox.com/training/r/site/}{http://magizbox.com/training/r/site/}

R
R is a programming language and software environment for statistical computing and graphics supported by the R Foundation for Statistical Computing. The R language is widely used among statisticians and data miners for developing statistical software and data analysis. Polls, surveys of data miners, and studies of scholarly literature databases show that R's popularity has increased substantially in recent years.

R is a GNU package. The source code for the R software environment is written primarily in C, Fortran, and R. R is freely available under the GNU General Public License, and pre-compiled binary versions are provided for various operating systems. While R has a command line interface, there are several graphical front-ends available.[

R was created by Ross Ihaka and Robert Gentleman at the University of Auckland, New Zealand, and is currently developed by the R Development Core Team, of which Chambers is a member. R is named partly after the first names of the first two R authors and partly as a play on the name of S. The project was conceived in 1992, with an initial version released in 1995 and a stable beta version in 2000.

\section{R Courses}

I'm going to give a course about R, but it's take a lot of time to finish. I will give at least one lesson a week. You can track it here

(next) Data visualization with R
Everything you need to know about R
Read and Write Data
Importing data from JSON into R
Manipulate Data
Manipulate String and Datetime
Actually, beside my works, there are a lot of excellent and free courses in the internet for you

Beginner

tryr from codeschool

tryr is a course for beginners created by codeschool. This course contains R Syntax, Vectors, Matrices, Summary Statistics, Factors, Data Frames and Working With Real-World Data sections.

Introduction to R from datacamp

This course created by datacamp - a "online learning platform that focuses on building the best learning experience for Data Science in specific". Here is the introduction about this course quoted from authors "In this introduction to R, you will master the basics of this beautiful open source language such as factors, lists and data frames. With the knowledge gained in this course, you will be ready to undertake your first very own data analysis." It contains 6 chapters: Intro to basics, Vectors, Matrices, Factors, Data frames and Lists.

Intermediate and Advanced

R Programming of Johns Hopkins University in coursera Learn how to program in R and how to use R for effective data analysis. This is the second course in the Johns Hopkins Data Science Specialization. It's a 4-weeks course, contains: Overview of R, R data types and objects, reading and writing data (week 1),  Control structures, functions, scoping rules, dates and times (week 2), Loop functions, debugging tools (week 3) and Simulation, code profiling (week 4)

An Introduction to Statistical Learning with Applications in R of two experts Trevor Hastie and Rob Tibshirani from Standfor Unitiversity

This course was introduced by Kevin Markham in r-blogger in september 2014. "I found it to be an excellent course in statistical learning (also known as “machine learning”), largely due to the high quality of both the textbook and the video lectures. And as an R user, it was extremely helpful that they included R code to demonstrate most of the techniques described in the book." In this course you will learn about Statistical Learning, Linear Regression, Classification, Resampling Methods, Linear Model Selection and Regularization, Moving Beyond Linearity, Tree-Based Methods, Support Vector Machines and Unsupervised Learning

Cheatsheet – Python & R codes for common Machine Learning Algorithms

\section{Everything you need to know about R}

In this post I maintain all useful references for someone want to write nice R code.

Google’s R Style Guide at google
R is a high-level programming language used primarily for statistical computing and graphics. The goal of the R Programming Style Guide is to make our R code easier to read, share, and verify. The rules below were designed in collaboration with the entire R user community at Google.

Installing R packages at r-bloggers
https://www.r-bloggers.com/installing-r-packages/

This is a short post giving steps on how to actually install R packages.

Managing your projects in a reproducible fashion at nicercode
https://nicercode.github.io/blog/2013-04-05-projects/

Managing your projects in a reproducible fashion doesn’t just make your science reproducible, it makes your life easier.

Creating R Packages
http://cran.r-project.org/doc/contrib/Leisch-CreatingPackages.pdf

This tutorial gives a practical introduction to creating R packages. We discuss how object oriented programming and S formulas can be used to give R code the usual look and feel, how to start a package from a collection of R functions, and how to test the code once the package has been created. As running example we use functions for standard linear regression analysis which are developed from scratch

How to write trycatch in R
http://stackoverflow.com/questions/12193779/how-to-write-trycatch-in-r

Welcome to the R world 😉

Debugging with RStudio
https://support.rstudio.com/hc/en-us/articles/200713843-Debugging-with-RStudio

RStudio includes a visual debugger that can help you understand code and find bugs.

Optimising code
http://adv-r.had.co.nz/Profiling.html#performance-profiling

Optimising code to make it run faster is an iterative process:

Find the biggest bottleneck (the slowest part of your code). Try to eliminate it (you may not succeed but that’s ok). Repeat until your code is “fast enough.” This sounds easy, but it’s not.
%\chapter{Scala}

View online \href{http://magizbox.com/training/scala/site/}{http://magizbox.com/training/scala/site/}

Scala is a programming language for general software applications. Scala has full support for functional programming and a very strong static type system. This allows programs written in Scala to be very concise and thus smaller in size than other general-purpose programming languages. Many of Scala's design decisions were inspired by criticism of the shortcomings of Java.

\section{Installation}

Windows
Step 1. Download scala from http://www.scala-lang.org/downloads

Step 2. Run installer

Step 3. Verify

Open terminal and check which version of scala

$ scala -version

Scala code runner version 2.11.5 -- Copyright 2002-2013, LAMP/EPFL

\section{IDE}

I use IntelliJ IDEA 2016.2 as scala IDE

IntelliJ IDEA Installation Guide
Online IDE
You can use tryscala as an online IDE

http://www.tryscala.com/

\section{Basic Syntax}

Print print
> println("Hello, Scala!");

Hello, Scala!
Conditional
if Statement

if statement consists of a Boolean expression followed by one or more statements.

var x = 10;
if( x < 20 ){
 println("This is if statement");
}
if-else Statement

var x = 30
if( x < 20 ){
  println("This is if statement");
} else {
  println("This is else statement");
}
if-else if-else Statement

 var x = 30;
if( x == 10 ){
 println("Value of X is 10");
} else if( x == 20 ){
 println("Value of X is 20");
} else if( x == 30 ){
 println("Value of X is 30");
} else{
 println("This is else statement");
}
Coding Convention 1
Keep It Simple
Don't pack two much in one expression
/*
 * It's amazing what you can get done in a single statement
 * But that does not mean you have to do it.
 */
jp.getRawClasspath.filter(
  _.getEntryKind == IClasspathEntry.CPE_SOURCE).
  iterator.flatMap(entry =>
    flatten(ResourcesPlugin.getWorkspace.
      getRoot.findMember(entry.getPath)))
Refactor
There's a lot of value in meaningfull names.
Easy to add them using inline vals and defs
val sources = jp.getRawClasspath.filter(
  _.getEntryKind == IClasspathEntry.CPE_SOURCE)
def workspaceRoot =
  ResourcesPlugin.getWorkspace.getRoot
def filesOfEntry(entry: Set[File]) =
  flatten(worspaceRoot.findMember(entry.getPath)
sources.iterator flatMap filesOfEntry
Prefer Functional
By default

use vals, not vars
use recursions or combinators, not loops
use immutable collections
concentrate on transformations, not CRUD
When to deviate from the default - sometimes, mutable gives better performance. - sometimes (but not that often!) it adds convenience

But don't diablolize local state
Why does mutable state lead to complexity?

It interacts with different program parts in ways that are hard to track.

=> Local state is less harmful than global state.

"Var" Shortcuts
var interfaces = parseClassHeader()...
if (isAnnotation) interfaces += ClassFileAnnotation
Refactor

val parsedIfaces = parseClassHeader()
val interfaces =
  if (isAnnotation) parsedIfaces + ClassFileAnnotation
  else parsedIfaces
Martin Odersky - Scala with Style ↩
%\chapter{NodeJS}

View online \href{http://magizbox.com/training/nodejs/site/}{http://magizbox.com/training/nodejs/site/}

Node.js is an open-source, cross-platform JavaScript runtime environment for developing a diverse variety of tools and applications. Although Node.js is not a JavaScript framework, many of its basic modules are written in JavaScript, and developers can write new modules in JavaScript. The runtime environment interprets JavaScript using Google's V8 JavaScript engine. Node.js has an event-driven architecture capable of asynchronous I/O. These design choices aim to optimize throughput and scalability in Web applications with many input/output operations, as well as for real-time Web applications (e.g., real-time communication programs and browser games). Node.js was originally written in 2009 by Ryan Dahl. The initial release supported only Linux. Its development and maintenance was led by Dahl and later sponsored by Joyent.

\section{Get Started}

Installation
Windows
In this section I will show you how to Install Node.js® and NPM on Windows

Prerequisites
Node isn’t a program that you simply launch like Word or Photoshop: you won’t find it pinned to the taskbar or in your list of Apps. To use Node you must type command-line instructions, so you need to be comfortable with (or at least know how to start) a command-line tool like the Windows Command Prompt, PowerShell, Cygwin, or the Git shell (which is installed along with Github for Windows).

Installation Overview
Installing Node and NPM is pretty straightforward using the installer package available from the Node.js® web site.

Installation Steps
1. Download the Windows installer from the Nodes.js® web site.

2. Run the installer (the .msi file you downloaded in the previous step.)

3. Follow the prompts in the installer (Accept the license agreement, click the NEXT button a bunch of times and accept the default installation settings).



4. Restart your computer. You won’t be able to run Node.js® until you restart your computer.

Ubuntu
In this section I will show you how to Install Node.js® and NPM on Ubuntu

# update os
sudo apt-get update
# install node with apt-get
sudo apt-get install nodejs
# install npm with apt-get
sudo apt-get install npm
Test
Make sure you have Node and NPM installed by running simple commands to see what version of each is installed and to run a simple test program:

> node -v
v6.9.5

> npm -v
3.10.10
Suggested Readings
How To Install Node.js on an Ubuntu 14.04 server
How to Install Node.js® and NPM on Windows

\section{Basic Syntax}

Print
console.log("Hello World");
Conditional
if(you_smart){
    console.log("learn nodejs");
} else {
    console.log("go away");
}
Loop
for(var count = 0; count < 10; count++){
    console.log(count);
}
Function
function print_info(arg1, arg2){
    console.log(arg1);
    console.log(arg2);
}

\section{File System & IO}

File System & IO
Node implements File I/O using simple wrappers around standard POSIX functions. The Node File System (fs) module can be imported using the following syntax −

var fs = require("fs")
Synchronous vs Asynchronous
Every method in the fs module has synchronous as well as asynchronous forms. Asynchronous methods take the last parameter as the completion function callback and the first parameter of the callback function as error. It is better to use an asynchronous method instead of a synchronous method, as the former never blocks a program during its execution, whereas the second one does.

Example

Create a text file named input.txt with the following content −

Tutorials Point is giving self learning content
to teach the world in simple and easy way!!!!!
Let us create a js file named main.js with the following code −

var fs = require("fs");

// Asynchronous read
fs.readFile('input.txt', function (err, data) {
   if (err) {
      return console.error(err);
   }
   console.log("Asynchronous read: " + data.toString());
});

// Synchronous read
var data = fs.readFileSync('input.txt');
console.log("Synchronous read: " + data.toString());

console.log("Program Ended");
Now run the main.js to see the result −

$ node main.js
Verify the Output.

Synchronous read: Tutorials Point is giving self learning content
to teach the world in simple and easy way!!!!!

Program Ended
Asynchronous read: Tutorials Point is giving self learning content
to teach the world in simple and easy way!!!!!
The following sections in this chapter provide a set of good examples on major File I/O methods.
Open a File
Syntax

Following is the syntax of the method to open a file in asynchronous mode −

fs.open(path, flags[, mode], callback)
Parameters

Here is the description of the parameters used −

path − This is the string having file name including path.
flags − Flags indicate the behavior of the file to be opened. All possible values have been mentioned below.
mode − It sets the file mode (permission and sticky bits), but only if the file was created. It defaults to 0666, readable and writeable.
callback − This is the callback function which gets two arguments (err, fd).
Flags

Flags for read/write operations are −

r - Open file for reading. An exception occurs if the file does not exist.
r+ - Open file for reading and writing. An exception occurs if the file does not exist.
rs - Open file for reading in synchronous mode.
rs+ - Open file for reading and writing, asking the OS to open it synchronously. See notes for 'rs' about using this with caution.
w - Open file for writing. The file is created (if it does not exist) or truncated (if it exists).
wx - Like 'w' but fails if the path exists.
w+ - Open file for reading and writing. The file is created (if it does not exist) or truncated (if it exists).
wx+ - Like 'w+' but fails if path exists.
a - Open file for appending. The file is created if it does not exist.
ax - Like 'a' but fails if the path exists.
a+ - Open file for reading and appending. The file is created if it does not exist.
ax+ - Like 'a+' but fails if the the path exists.
Example

Let us create a js file named main.js having the following code to open a file input.txt for reading and writing.

var fs = require("fs");

// Asynchronous - Opening File
console.log("Going to open file!");
fs.open('input.txt', 'r+', function(err, fd) {
   if (err) {
      return console.error(err);
   }
  console.log("File opened successfully!");
});
Now run the main.js to see the result −

$ node main.js
Verify the Output.

Going to open file!
File opened successfully!
Get File Information
Syntax

Following is the syntax of the method to get the information about a file −

fs.stat(path, callback)
Parameters

Here is the description of the parameters used −

path − This is the string having file name including path.
callback − This is the callback function which gets two arguments (err, stats) where stats is an object of fs.Stats type which is printed below in the example.
Apart from the important attributes which are printed below in the example, there are several useful methods available in fs.Stats class which can be used to check file type. These methods are given in the following table.

Method Description

stats.isFile() - Returns true if file type of a simple file.
stats.isDirectory() - Returns true if file type of a directory.
stats.isBlockDevice() - Returns true if file type of a block device.
stats.isCharacterDevice() - Returns true if file type of a character device.
stats.isSymbolicLink() - Returns true if file type of a symbolic link.
stats.isFIFO() - Returns true if file type of a FIFO.
stats.isSocket() - Returns true if file type of asocket.
Example

Let us create a js file named main.js with the following code −

var fs = require("fs");

console.log("Going to get file info!");
fs.stat('input.txt', function (err, stats) {
   if (err) {
       return console.error(err);
   }
   console.log(stats);
   console.log("Got file info successfully!");

   // Check file type
   console.log("isFile ? " + stats.isFile());
   console.log("isDirectory ? " + stats.isDirectory());
});
Now run the main.js to see the result −

$ node main.js
Verify the Output.

Going to get file info!
{
   dev: 1792,
   mode: 33188,
   nlink: 1,
   uid: 48,
   gid: 48,
   rdev: 0,
   blksize: 4096,
   ino: 4318127,
   size: 97,
   blocks: 8,
   atime: Sun Mar 22 2015 13:40:00 GMT-0500 (CDT),
   mtime: Sun Mar 22 2015 13:40:57 GMT-0500 (CDT),
   ctime: Sun Mar 22 2015 13:40:57 GMT-0500 (CDT)
}
Got file info successfully!
isFile ? true
isDirectory ? false
Writing a File
Syntax

Following is the syntax of one of the methods to write into a file −

fs.writeFile(filename, data[, options], callback)
This method will over-write the file if the file already exists. If you want to write into an existing file then you should use another method available.

Parameters

Here is the description of the parameters used −

path − This is the string having the file name including path.
data − This is the String or Buffer to be written into the file.
options − The third parameter is an object which will hold {encoding, mode, flag}. By default. encoding is utf8, mode is octal value 0666. and flag is 'w'
callback − This is the callback function which gets a single parameter err that returns an error in case of any writing error.
Example

Let us create a js file named main.js having the following code −

var fs = require("fs");

console.log("Going to write into existing file");
fs.writeFile('input.txt', 'Simply Easy Learning!',  function(err) {
   if (err) {
      return console.error(err);
   }

   console.log("Data written successfully!");
   console.log("Let's read newly written data");
   fs.readFile('input.txt', function (err, data) {
      if (err) {
         return console.error(err);
      }
      console.log("Asynchronous read: " + data.toString());
   });
});
Now run the main.js to see the result −

$ node main.js
Verify the Output.

Going to write into existing file
Data written successfully!
Let's read newly written data
Asynchronous read: Simply Easy Learning!
Reading a File
Syntax

Following is the syntax of one of the methods to read from a file −

fs.read(fd, buffer, offset, length, position, callback)
This method will use file descriptor to read the file. If you want to read the file directly using the file name, then you should use another method available.

Parameters

Here is the description of the parameters used −

fd − This is the file descriptor returned by fs.open().
buffer − This is the buffer that the data will be written to.
offset − This is the offset in the buffer to start writing at.
length − This is an integer specifying the number of bytes to read.
position − This is an integer specifying where to begin reading from in the file. * If position is null, data will be read from the current file position. callback − This is the callback function which gets the three arguments, (err, bytesRead, buffer).
Example

Let us create a js file named main.js with the following code −

var fs = require("fs");
var buf = new Buffer(1024);

console.log("Going to open an existing file");
fs.open('input.txt', 'r+', function(err, fd) {
   if (err) {
      return console.error(err);
   }
   console.log("File opened successfully!");
   console.log("Going to read the file");
   fs.read(fd, buf, 0, buf.length, 0, function(err, bytes){
      if (err){
         console.log(err);
      }
      console.log(bytes + " bytes read");

      // Print only read bytes to avoid junk.
      if(bytes > 0){
         console.log(buf.slice(0, bytes).toString());
      }
   });
});
Now run the main.js to see the result −

$ node main.js
Verify the Output.

Going to open an existing file
File opened successfully!
Going to read the file
97 bytes read
Tutorials Point is giving self learning content
to teach the world in simple and easy way!!!!!
Closing a File
Syntax

Following is the syntax to close an opened file −

fs.close(fd, callback)
Parameters

Here is the description of the parameters used −

fd − This is the file descriptor returned by file fs.open() method.
callback − This is the callback function No arguments other than a possible exception are given to the completion callback.
Example Let us create a js file named main.js having the following code −

var fs = require("fs");
var buf = new Buffer(1024);

console.log("Going to open an existing file");
fs.open('input.txt', 'r+', function(err, fd) {
   if (err) {
      return console.error(err);
   }
   console.log("File opened successfully!");
   console.log("Going to read the file");

   fs.read(fd, buf, 0, buf.length, 0, function(err, bytes){
      if (err){
         console.log(err);
      }

      // Print only read bytes to avoid junk.
      if(bytes > 0){
         console.log(buf.slice(0, bytes).toString());
      }

      // Close the opened file.
      fs.close(fd, function(err){
         if (err){
            console.log(err);
         }
         console.log("File closed successfully.");
      });
   });
});
Now run the main.js to see the result −

$ node main.js
Verify the Output.

Going to open an existing file
File opened successfully!
Going to read the file
Tutorials Point is giving self learning content
to teach the world in simple and easy way!!!!!

File closed successfully.
Truncate a File
Syntax

Following is the syntax of the method to truncate an opened file −

fs.ftruncate(fd, len, callback)
Parameters

Here is the description of the parameters used −

fd − This is the file descriptor returned by fs.open().
len − This is the length of the file after which the file will be truncated.
callback − This is the callback function No arguments other than a possible ekxception are given to the completion callback.
Example

Let us create a js file named main.js having the following code −

var fs = require("fs");
var buf = new Buffer(1024);

console.log("Going to open an existing file");
fs.open('input.txt', 'r+', function(err, fd) {
   if (err) {
      return console.error(err);
   }
   console.log("File opened successfully!");
   console.log("Going to truncate the file after 10 bytes");

   // Truncate the opened file.
   fs.ftruncate(fd, 10, function(err){
      if (err){
         console.log(err);
      }
      console.log("File truncated successfully.");
      console.log("Going to read the same file");

      fs.read(fd, buf, 0, buf.length, 0, function(err, bytes){
         if (err){
            console.log(err);
         }

         // Print only read bytes to avoid junk.
         if(bytes > 0){
            console.log(buf.slice(0, bytes).toString());
         }

         // Close the opened file.
         fs.close(fd, function(err){
            if (err){
               console.log(err);
            }
            console.log("File closed successfully.");
         });
      });
   });
});
Now run the main.js to see the result −

$ node main.js
Verify the Output.

Going to open an existing file
File opened successfully!
Going to truncate the file after 10 bytes
File truncated successfully.
Going to read the same file
Tutorials
File closed successfully.
Delete a File
Syntax Following is the syntax of the method to delete a file −

fs.unlink(path, callback)
Parameters

Here is the description of the parameters used −

path − This is the file name including path.
callback − This is the callback function No arguments other than a possible exception are given to the completion callback.
Example

Let us create a js file named main.js having the following code −

var fs = require("fs");

console.log("Going to delete an existing file");
fs.unlink('input.txt', function(err) {
   if (err) {
      return console.error(err);
   }
   console.log("File deleted successfully!");
});
Now run the main.js to see the result −

$ node main.js
Verify the Output.

Going to delete an existing file
File deleted successfully!
Create a Directory
Syntax

Following is the syntax of the method to create a directory −

fs.mkdir(path[, mode], callback)
Parameters

Here is the description of the parameters used −

path − This is the directory name including path.
mode − This is the directory permission to be set. Defaults to 0777.
callback − This is the callback function No arguments other than a possible exception are given to the completion callback.
Example

Let us create a js file named main.js having the following code −

var fs = require("fs");

console.log("Going to create directory /tmp/test");
fs.mkdir('/tmp/test',function(err){
   if (err) {
      return console.error(err);
   }
   console.log("Directory created successfully!");
});
Now run the main.js to see the result −

$ node main.js
Verify the Output.

Going to create directory /tmp/test
Directory created successfully!
Read a Directory
Syntax

Following is the syntax of the method to read a directory −

fs.readdir(path, callback)
Parameters

Here is the description of the parameters used −

path − This is the directory name including path.
callback − This is the callback function which gets two arguments (err, files) where files is an array of the names of the files in the directory excluding '.' and '..'.
Example

Let us create a js file named main.js having the following code −

var fs = require("fs");

console.log("Going to read directory /tmp");
fs.readdir("/tmp/",function(err, files){
   if (err) {
      return console.error(err);
   }
   files.forEach( function (file){
      console.log( file );
   });
});
Now run the main.js to see the result −

$ node main.js
Verify the Output.

Going to read directory /tmp
ccmzx99o.out
ccyCSbkF.out
employee.ser
hsperfdata_apache
test
test.txt
Remove a Directory
Syntax

Following is the syntax of the method to remove a directory −

fs.rmdir(path, callback)
Parameters

Here is the description of the parameters used −

path − This is the directory name including path.
callback − This is the callback function No argume nts other than a possible exception are given to the completion callback.
Example

Let us create a js file named main.js having the following code −

var fs = require("fs");

console.log("Going to delete directory /tmp/test");
fs.rmdir("/tmp/test",function(err){
   if (err) {
      return console.error(err);
   }
   console.log("Going to read directory /tmp");

   fs.readdir("/tmp/",function(err, files){
      if (err) {
         return console.error(err);
      }
      files.forEach( function (file){
         console.log( file );
      });
   });
});
Now run the main.js to see the result −

$ node main.js
Verify the Output.

Going to read directory /tmp
ccmzx99o.out
ccyCSbkF.out
employee.ser
hsperfdata_apache
test.txt

\section{Package Manager}

Package Manager: NPM
Node Package Manager (NPM) provides two main functionalities −

Online repositories for node.js packages/modules which are searchable on search.nodejs.org
Command line utility to install Node.js packages, do version management and dependency management of Node.js packages.
NPM comes bundled with Node.js installables after v0.6.3 version. To verify the same, open console and type the following command and see the result −

$ npm --version
2.7.1
If you are running an old version of NPM then it is quite easy to update it to the latest version. Just use the following command from root −

$ sudo npm install npm -g
/usr/bin/npm -> /usr/lib/node_modules/npm/bin/npm-cli.js
npm@2.7.1 /usr/lib/node_modules/npm
Installing Modules
There is a simple syntax to install any Node.js module −

$ npm install <Module Name>
For example, following is the command to install a famous Node.js web framework module called express −

$ npm install express
Now you can use this module in your js file as following −

var express = require('express');
Global vs Local Installation
By default, NPM installs any dependency in the local mode. Here local mode refers to the package installation in node_modules directory lying in the folder where Node application is present. Locally deployed packages are accessible via require() method. For example, when we installed express module, it created node_modules directory in the current directory where it installed the express module.

$ ls -l
total 0
drwxr-xr-x 3 root root 20 Mar 17 02:23 node_modules
Alternatively, you can use npm ls command to list down all the locally installed modules.

Globally installed packages/dependencies are stored in system directory. Such dependencies can be used in CLI (Command Line Interface) function of any node.js but cannot be imported using require() in Node application directly. Now let's try installing the express module using global installation.

$ npm install express -g
This will produce a similar result but the module will be installed globally. Here, the first line shows the module version and the location where it is getting installed.

express@4.12.2 /usr/lib/node_modules/express
├── merge-descriptors@1.0.0
├── utils-merge@1.0.0
├── cookie-signature@1.0.6
├── methods@1.1.1
├── fresh@0.2.4
├── cookie@0.1.2
├── escape-html@1.0.1
├── range-parser@1.0.2
├── content-type@1.0.1
├── finalhandler@0.3.3
├── vary@1.0.0
├── parseurl@1.3.0
├── content-disposition@0.5.0
├── path-to-regexp@0.1.3
├── depd@1.0.0
├── qs@2.3.3
├── on-finished@2.2.0 (ee-first@1.1.0)
├── etag@1.5.1 (crc@3.2.1)
├── debug@2.1.3 (ms@0.7.0)
├── proxy-addr@1.0.7 (forwarded@0.1.0, ipaddr.js@0.1.9)
├── send@0.12.1 (destroy@1.0.3, ms@0.7.0, mime@1.3.4)
├── serve-static@1.9.2 (send@0.12.2)
├── accepts@1.2.5 (negotiator@0.5.1, mime-types@2.0.10)
└── type-is@1.6.1 (media-typer@0.3.0, mime-types@2.0.10)
You can use the following command to check all the modules installed globally −

$ npm ls -g
Using package.json
package.json is present in the root directory of any Node application/module and is used to define the properties of a package. Let's open package.json of express package present in node_modules/express/

{
   "name": "express",
      "description": "Fast, unopinionated, minimalist web framework",
      "version": "4.11.2",
      "author": {

         "name": "TJ Holowaychuk",
         "email": "tj@vision-media.ca"
      },

   "contributors": [{
      "name": "Aaron Heckmann",
      "email": "aaron.heckmann+github@gmail.com"
   },

   {
      "name": "Ciaran Jessup",
      "email": "ciaranj@gmail.com"
   },

   {
      "name": "Douglas Christopher Wilson",
      "email": "doug@somethingdoug.com"
   },

   {
      "name": "Guillermo Rauch",
      "email": "rauchg@gmail.com"
   },

   {
      "name": "Jonathan Ong",
      "email": "me@jongleberry.com"
   },

   {
      "name": "Roman Shtylman",
      "email": "shtylman+expressjs@gmail.com"
   },

   {
      "name": "Young Jae Sim",
      "email": "hanul@hanul.me"
   } ],
   "license": "MIT", "repository": {
      "type": "git",
      "url": "https://github.com/strongloop/express"
   },
   "homepage": "https://expressjs.com/", "keywords": [
      "express",
      "framework",
      "sinatra",
      "web",
      "rest",
      "restful",
      "router",
      "app",
      "api"
   ],
   "dependencies": {
      "accepts": "~1.2.3",
      "content-disposition": "0.5.0",
      "cookie-signature": "1.0.5",
      "debug": "~2.1.1",
      "depd": "~1.0.0",
      "escape-html": "1.0.1",
      "etag": "~1.5.1",
      "finalhandler": "0.3.3",
      "fresh": "0.2.4",
      "media-typer": "0.3.0",
      "methods": "~1.1.1",
      "on-finished": "~2.2.0",
      "parseurl": "~1.3.0",
      "path-to-regexp": "0.1.3",
      "proxy-addr": "~1.0.6",
      "qs": "2.3.3",
      "range-parser": "~1.0.2",
      "send": "0.11.1",
      "serve-static": "~1.8.1",
      "type-is": "~1.5.6",
      "vary": "~1.0.0",
      "cookie": "0.1.2",
      "merge-descriptors": "0.0.2",
      "utils-merge": "1.0.0"
   },
   "devDependencies": {
      "after": "0.8.1",
      "ejs": "2.1.4",
      "istanbul": "0.3.5",
      "marked": "0.3.3",
      "mocha": "~2.1.0",
      "should": "~4.6.2",
      "supertest": "~0.15.0",
      "hjs": "~0.0.6",
      "body-parser": "~1.11.0",
      "connect-redis": "~2.2.0",
      "cookie-parser": "~1.3.3",
      "express-session": "~1.10.2",
      "jade": "~1.9.1",
      "method-override": "~2.3.1",
      "morgan": "~1.5.1",
      "multiparty": "~4.1.1",
      "vhost": "~3.0.0"
   },
   "engines": {
      "node": ">= 0.10.0"
   },
   "files": [
      "LICENSE",
      "History.md",
      "Readme.md",
      "index.js",
      "lib/"
   ],
   "scripts": {
      "test": "mocha --require test/support/env
         --reporter spec --bail --check-leaks test/ test/acceptance/",
      "test-cov": "istanbul cover node_modules/mocha/bin/_mocha
         -- --require test/support/env --reporter dot --check-leaks test/ test/acceptance/",
      "test-tap": "mocha --require test/support/env
         --reporter tap --check-leaks test/ test/acceptance/",
      "test-travis": "istanbul cover node_modules/mocha/bin/_mocha
         --report lcovonly -- --require test/support/env
         --reporter spec --check-leaks test/ test/acceptance/"
   },
   "gitHead": "63ab25579bda70b4927a179b580a9c580b6c7ada",
   "bugs": {
      "url": "https://github.com/strongloop/express/issues"
   },
   "_id": "express@4.11.2",
   "_shasum": "8df3d5a9ac848585f00a0777601823faecd3b148",
   "_from": "express@*",
   "_npmVersion": "1.4.28",
   "_npmUser": {
      "name": "dougwilson",
      "email": "doug@somethingdoug.com"
   },
   "maintainers": [
      {
         "name": "tjholowaychuk",
         "email": "tj@vision-media.ca"
      },
      {
         "name": "jongleberry",
         "email": "jonathanrichardong@gmail.com"
      },
      {
         "name": "shtylman",
         "email": "shtylman@gmail.com"
      },
      {
         "name": "dougwilson",
         "email": "doug@somethingdoug.com"
      },
      {
         "name": "aredridel",
         "email": "aredridel@nbtsc.org"
      },
      {
         "name": "strongloop",
         "email": "callback@strongloop.com"
      },
      {
         "name": "rfeng",
         "email": "enjoyjava@gmail.com"
      }
   ],
   "dist": {
      "shasum": "8df3d5a9ac848585f00a0777601823faecd3b148",
      "tarball": "https://registry.npmjs.org/express/-/express-4.11.2.tgz"
   },
   "directories": {},
      "_resolved": "https://registry.npmjs.org/express/-/express-4.11.2.tgz",
      "readme": "ERROR: No README data found!"
}
Attributes of Package.json
name − name of the package
version − version of the package
description − description of the package
homepage − homepage of the package
author − author of the package
contributors − name of the contributors to the package
dependencies − list of dependencies. NPM automatically installs all the dependencies mentioned here in the node_module folder of the package. repository − repository type and URL of the package
main − entry point of the package
keywords − keywords
Uninstalling a Module
Use the following command to uninstall a Node.js module.

$ npm uninstall express
Once NPM uninstalls the package, you can verify it by looking at the content of /node_modules/ directory or type the following command −

$ npm ls
Updating a Module
Update package.json and change the version of the dependency to be updated and run the following command.

$ npm update express
Search a Module
Search a package name using NPM.

$ npm search express
Create a Module
Creating a module requires package.json to be generated. Let's generate package.json using NPM, which will generate the basic skeleton of the package.json.

$ npm init

This utility will walk you through creating a package.json file.
It only covers the most common items, and tries to guess sane defaults.

See 'npm help json' for definitive documentation on these fields
and exactly what they do.

Use 'npm install <pkg> --save' afterwards to install a package and
save it as a dependency in the package.json file.

Press ^C at any time to quit.
name: (webmaster)
You will need to provide all the required information about your module. You can take help from the above-mentioned package.json file to understand the meanings of various information demanded. Once package.json is generated, use the following command to register yourself with NPM repository site using a valid email address.

$ npm adduser
Username: mcmohd
Password:
Email: (this IS public) mcmohd@gmail.com
It is time now to publish your module −

$ npm publish
If everything is fine with your module, then it will be published in the repository and will be accessible to install using NPM like any other Node.js module.

\section{Command Line}

Pass command line arguments
The arguments are stored in process.argv

Here are the node docs on handling command line args:

 process.argv is an array containing the command line arguments. The first element will be 'node', the second element will be the name of the JavaScript file. The next elements will be any additional command line arguments.

// print process.argv
process.argv.forEach(function (val, index, array) {
  console.log(index + ': ' + val);
});
This will generate:

$ node process-2.js one two=three four
0: node
1: /Users/mjr/work/node/process-2.js
2: one
3: two=three
4: four





%\input{programming/octave.tex}
%\chapter{Toolbox}

View online \href{http://magizbox.com/training/toolbox/site/}{http://magizbox.com/training/toolbox/site/}

Toolbox by MG
The Toolbox contains all the little tools you never know where to find.

Text Editor
Vim : Vim is a clone of Bill Joy's vi text editor program for Unix. It was written by Bram Moolenaar based on source for a port of the Stevie editor to the Amiga and first released publicly in 1991. Vim is designed for use both from a command-line interface and as a standalone application in a graphical user interface. Vim is free and open source software and is released under a license that includes some charityware clauses, encouraging users who enjoy the software to consider donating to children in Uganda. The license is compatible with the GNU General Public License. Although it was originally released for the Amiga, Vim has since been developed to be cross-platform, supporting many other platforms. In 2006, it was voted the most popular editor amongst Linux Journal readers; in 2015 the Stack Overflow developer survey found it to be the third most popular text editor; and in 2016 the Stack Overflow developer survey found it to be the fourth most popular development environment.

Virtual Machine
VirtualBox : Oracle VM VirtualBox (formerly Sun VirtualBox, Sun xVM VirtualBox and Innotek VirtualBox) is a free and open-source hypervisor for x86 computers currently being developed by Oracle Corporation. Developed initially by Innotek GmbH, it was acquired by Sun Microsystems in 2008 which was in turn acquired by Oracle in 2010. VirtualBox may be installed on a number of host operating systems, including: Linux, macOS, Windows, Solaris, and OpenSolaris. There are also ports to FreeBSD and Genode. It supports the creation and management of guest virtual machines running versions and derivations of Windows, Linux, BSD, OS/2, Solaris, Haiku, OSx86 and others, and limited virtualization of macOS guests on Apple hardware. For some guest operating systems, a "Guest Additions" package of device drivers and system applications is available which typically improves performance, especially of graphics.

VMWare : VMware, Inc. is a subsidiary of Dell Technologies that provides cloud computing and platform virtualization software and services. It was the first commercially successful company to virtualize the x86 architecture. VMware's desktop software runs on Microsoft Windows, Linux, and macOS, while its enterprise software hypervisor for servers, VMware ESXi, is a bare-metal hypervisor that runs directly on server hardware without requiring an additional underlying operating system.

\section{Vim}

Vim
Running Vim for the First Time
To start Vim, enter this command:

gvim file.txt
In UNIX you can type this at any command prompt. If you are running Microsoft Windows, open an MS-DOS prompt window and enter the command. In either case, Vim starts editing a file called file.txt. Because this is a new file, you get a blank window. This is what your screen will look like:

+---------------------------------------+
|#                  |
|~                  |
|~                  |
|~                  |
|~                  |
|"file.txt" [New file]          |
+---------------------------------------+
    ('#" is the cursor position.)
The tilde (~) lines indicate lines not in the file. In other words, when Vim runs out of file to display, it displays tilde lines. At the bottom of the screen, a message line indicates the file is named file.txt and shows that you are creating a new file. The message information is temporary and other information overwrites it.

THE VIM COMMAND

The gvim command causes the editor to create a new window for editing. If you use this command:

vim file.txt
the editing occurs inside your command window. In other words, if you are running inside an xterm, the editor uses your xterm window. If you are using an MS-DOS command prompt window under Microsoft Windows, the editing occurs inside this window. The text in the window will look the same for both versions, but with gvim you have extra features, like a menu bar. More about that later.

Inserting text
The Vim editor is a modal editor. That means that the editor behaves differently, depending on which mode you are in. The two basic modes are called Normal mode and Insert mode. In Normal mode the characters you type are commands. In Insert mode the characters are inserted as text. Since you have just started Vim it will be in Normal mode. To start Insert mode you type the "i" command (i for Insert). Then you can enter the text. It will be inserted into the file. Do not worry if you make mistakes; you can correct them later. To enter the following programmer's limerick, this is what you type:

iA very intelligent turtle
Found programming UNIX a hurdle
After typing "turtle" you press the key to start a new line. Finally you press the key to stop Insert mode and go back to Normal mode. You now have two lines of text in your Vim window:

+---------------------------------------+
|A very intelligent turtle      |
|Found programming UNIX a hurdle    |
|~                  |
|~                  |
|                   |
+---------------------------------------+
WHAT IS THE MODE?

To be able to see what mode you are in, type this command:

:set showmode
You will notice that when typing the colon Vim moves the cursor to the last line of the window. That's where you type colon commands (commands that start with a colon). Finish this command by pressing the <Enter> key (all commands that start with a colon are finished this way). Now, if you type the "i" command Vim will display --INSERT-- at the bottom of the window. This indicates you are in Insert mode.

+---------------------------------------+
|A very intelligent turtle      |
|Found programming UNIX a hurdle    |
|~                  |
|~                  |
|-- INSERT --               |
+---------------------------------------+
If you press <Esc> to go back to Normal mode the last line will be made blank.

GETTING OUT OF TROUBLE

One of the problems for Vim novices is mode confusion, which is caused by forgetting which mode you are in or by accidentally typing a command that switches modes. To get back to Normal mode, no matter what mode you are in, press the key. Sometimes you have to press it twice. If Vim beeps back at you, you already are in Normal mode.

==============================================================================

Moving around
After you return to Normal mode, you can move around by using these keys:

h   left                        *hjkl*
j   down
k   up
l   right
At first, it may appear that these commands were chosen at random. After all, who ever heard of using l for right? But actually, there is a very good reason for these choices: Moving the cursor is the most common thing you do in an editor, and these keys are on the home row of your right hand. In other words, these commands are placed where you can type them the fastest (especially when you type with ten fingers).

Note:
You can also move the cursor by using the arrow keys.  If you do,
however, you greatly slow down your editing because to press the arrow
keys, you must move your hand from the text keys to the arrow keys.
Considering that you might be doing it hundreds of times an hour, this
can take a significant amount of time.
   Also, there are keyboards which do not have arrow keys, or which
locate them in unusual places; therefore, knowing the use of the hjkl
keys helps in those situations.
One way to remember these commands is that h is on the left, l is on the right and j points down. In a picture:

           k
       h     l
         j
The best way to learn these commands is by using them. Use the "i" command to insert some more lines of text. Then use the hjkl keys to move around and insert a word somewhere. Don't forget to press to go back to Normal mode. The |vimtutor| is also a nice way to learn by doing.

For Japanese users, Hiroshi Iwatani suggested using this:

        Komsomolsk
            ^
            |
   Huan Ho  <--- --->  Los Angeles
(Yellow river)      |
            v
          Java (the island, not the programming language)
==============================================================================

Deleting characters
To delete a character, move the cursor over it and type "x". (This is a throwback to the old days of the typewriter, when you deleted things by typing xxxx over them.) Move the cursor to the beginning of the first line, for example, and type xxxxxxx (seven x's) to delete "A very ". The result should look like this:

+---------------------------------------+
|intelligent turtle         |
|Found programming UNIX a hurdle    |
|~                  |
|~                  |
|                   |
+---------------------------------------+
Now you can insert new text, for example by typing:

iA young <Esc>
This begins an insert (the i), inserts the words "A young", and then exits insert mode (the final ). The result:

+---------------------------------------+
|A young intelligent turtle     |
|Found programming UNIX a hurdle    |
|~                  |
|~                  |
|                   |
+---------------------------------------+
DELETING A LINE

To delete a whole line use the "dd" command. The following line will then move up to fill the gap:

+---------------------------------------+
|Found programming UNIX a hurdle    |
|~                  |
|~                  |
|~                  |
|                   |
+---------------------------------------+
DELETING A LINE BREAK

In Vim you can join two lines together, which means that the line break between them is deleted. The "J" command does this. Take these two lines:

A young intelligent
turtle
Move the cursor to the first line and press "J":

A young intelligent turtle
==============================================================================

Undo and Redo
Suppose you delete too much. Well, you can type it in again, but an easier way exists. The "u" command undoes the last edit. Take a look at this in action: After using "dd" to delete the first line, "u" brings it back. Another one: Move the cursor to the A in the first line:

A young intelligent turtle
Now type xxxxxxx to delete "A young". The result is as follows:

 intelligent turtle
Type "u" to undo the last delete. That delete removed the g, so the undo restores the character.

g intelligent turtle
The next u command restores the next-to-last character deleted:

ng intelligent turtle
The next u command gives you the u, and so on:

ung intelligent turtle
oung intelligent turtle
young intelligent turtle
 young intelligent turtle
A young intelligent turtle

Note:
If you type "u" twice, and the result is that you get the same text
back, you have Vim configured to work Vi compatible.  Look here to fix
this: |not-compatible|.
   This text assumes you work "The Vim Way".  You might prefer to use
the good old Vi way, but you will have to watch out for small
differences in the text then.
REDO

If you undo too many times, you can press CTRL-R (redo) to reverse the preceding command. In other words, it undoes the undo. To see this in action, press CTRL-R twice. The character A and the space after it disappear:

young intelligent turtle
There's a special version of the undo command, the "U" (undo line) command. The undo line command undoes all the changes made on the last line that was edited. Typing this command twice cancels the preceding "U".

A very intelligent turtle
  xxxx              Delete very

A intelligent turtle
          xxxxxx        Delete turtle

A intelligent
                Restore line with "U"
A very intelligent turtle
                Undo "U" with "u"
A intelligent
The "U" command is a change by itself, which the "u" command undoes and CTRL-R redoes. This might be a bit confusing. Don't worry, with "u" and CTRL-R you can go to any of the situations you had. More about that in section |32.2|.

Reference: http://vimdoc.sourceforge.net/htmldoc/usr_02.html

\section{Virtual Box}

Virtual Box
Export and Import VirtualBox VM images?
Export
Open VirtualBox and enter into the File option to choice Export Appliance...



You will then get an assistance window to help you generating the image.

Select the VM to export
Enter the output file path and name


You can choice a format, which I always leave the default OVF 1.

Finally you can write metadata like Version and Description
Now you have an OVA file that you can carry to whatever machine to use it.

Import
Open VirtualBox and enter into the File option to choice Import

You will then get an assistance window to help you loading the image.

Enter the path to the file that you have previously exported


Then you can modify the settings of the VM like RAM size, CPU, etc.


My recommendation on this is to enable the Reinitialize the MAC address of all the network cards option

Press Import and done!
Now you have cloned the VM from the host machine into another one

Reference: https://askubuntu.com/questions/588426/how-to-export-and-import-virtualbox-vm-images

Install Guest Additions
Guest Additions installs on the guest system and includes device drivers and system applications that optimize performance of the machine. Launch the guest OS in VirtualBox and click on Devices and Install Guest Additions.



The AutoPlay window opens on the guest OS and click on the Run VBox Windows Additions executable.



Click yes when the UAC screen comes up.



Now simply follow through the installation wizard.



During the installation wizard you can choose the Direct3D acceleration if you would like it. Remember this is going to take up more of your Host OS’s resources and is still experimental possibly making the guest unstable.



When the installation starts you will need to allow the Sun display adapters to be installed.



After everything has completed a reboot is required.

\section{VMWare}

VMWare
VMware Workstation is a program that allows you to run a virtual computer within your physical computer. The virtual computer runs as if it was its own machine. A virtual machine is great for trying out new operating systems such as Linux, visiting websites you don't trust, creating a computing environment specifically for children, testing the effects of computer viruses, and much more. You can even print and plug in USB drives. Read this guide to get the most out of VMware Workstation.

Installing VMware Workstation


1. Make sure your computer meets the system requirements. Because you will be running an operating system from within your own operating system, VMware Workstation has fairly high system requirements. If you don’t meet these, you may not be able to run VMware effectively. You must have a 64-bit processor. VMware supports Windows and Linux operating systems. You must have enough memory to run your operating system, the virtual operating system, and any programs inside that operating system. 1 GB is the minimum, but 3 or more is recommended. You must have a 16-bit or 32-bit display adapter. 3D effects will most likely not work well inside the virtual operating system, so gaming is not always efficient. You need at least 1.5 GB of free space to install VMware Workstation, along with at least 1 GB per operating system that you install.



2. Download the VMware software. You can download the VMware installer from the Download Center on the VMware website. Select the newest version and click the link for the installer. You will need to login with your VMware username. You will be asked to read and review the license agreement before you can download the file. You can only have one version of VMware Workstation installed at a time.



3. Install VMware Workstation. Once you have downloaded the file, right-click on the file and select “Run as administrator”. You will be asked to review the license again. Most users can use the Typical installation option. At the end of the installation, you will be prompted for your license key. Once the installation is finished, restart the computer. Part

Installing an Operating System


1. Open VMware. Installing a virtual operating system is much like installing it on a regular PC. You will need to have the installation disc or ISO image as well as any necessary licenses for the operating system that you want to install.

You can install most distributions of Linux as well as any version of Windows.


2. Click File. Select New Virtual Machine and then choose Typical. VMware will prompt you for the installation media. If it recognizes the operating system, it will enable Easy Installation:

Physical disc – Insert the installation disc for the operating system you want to install and then select the drive in VMware.
ISO image – Browse to the location of the ISO file on your computer.
Install operating system later. This will create a blank virtual disk. You will need to manually install the operating system later.


3. Enter in the details for the operating system. For Windows and other licensed operating systems, you will need to enter your product key. You will also need to enter your preferred username and a password if you want one. * If you are not using Easy Install, you will need to browse the list for the operating system you are installing.



4. Name your virtual machine. The name will help you identify it on your physical computer. It will also help distinguish between multiple virtual computers running different operating systems.



5. Set the disk size. You can allocate any amount of free space on your computer to the virtual machine to act as the installed operating system’s hard drive. Make sure to set enough to install any programs that you want to run in the virtual machine.



6. Customize your virtual machine’s virtual hardware. You can set the virtual machine to emulate specific hardware by clicking the “Customize Hardware” button. This can be useful if you are trying to run an older program that only supports certain hardware. Setting this is optional.



7. Set the virtual machine to start. Check the box labeled “Power on this virtual machine after creation” if you want the virtual machine to start up as soon as you finish making it. If you don’t check this box, you can select your virtual machine from the list in VMware and click the Power On button.



8. Wait for your installation to complete. Once you’ve powered on the virtual machine for the first time, the operating system will begin to install automatically. If you provided all of the correct information during the setup of the virtual machine, then you should not have to do anything. If you didn’t enter your product key or create a username during the virtual machine setup, you will most likely be prompted during the installation of the operating system.



9. Check that VMware Tools is installed. Once the operating system is installed, the program VMware Tools should be automatically installed. Check that it appears on the desktop or in the program files for the newly installed operating system.

VMware tools are configuration options for your virtual machine, and keeps your virtual machine up to date with any software changes.

Navigating VMware


1. Start a virtual machine. To start a virtual machine, click the VM menu and select the virtual machine that you want to turn on. You can choose to start the virtual machine normally, or boot directly to the virtual BIOS.



2. Stop a virtual machine. To stop a virtual machine, select it and then click the VM menu. Select the Power option.

Power Off – The virtual machine turns off as if the power was cut out.
Shut Down Guest – This sends a shutdown signal to the virtual machine which causes the virtual machine to shut down as if you had selected the shutdown option.
You can also turn off the virtual machine by using the shutdown option in the virtual operating system.


3. Move files between the virtual machine and your physical computer. Moving files between your computer and the virtual machine is as simple as dragging and dropping. Files can be moved in both directions between the computer and the virtual machine, and can also be dragged from one virtual machine to another.

When you drag and drop, the original will stay in the original location and a copy will be created in the new location.
You can also move files by copying and pasting.
Virtual machines can connect to shared folders as well.


4. Add a printer to your virtual machine. You can add any printer to your virtual machine without having to install any extra drivers, as long as it is already installed on your host computer.

Select the virtual machine that you want to add the printer to.
Click the VM menu and select Settings.
Click the Hardware tab, and then click Add. This will start the Add Hardware wizard.
Select Printer and then click Finish. Your virtual printer will be enabled the next time you turn the virtual machine on.


5. Connect a USB drive to the virtual machine. Virtual machines can interact with a USB drive the same way that your normal operating system does. The USB drive cannot be accessed on both the host computer and the virtual machine at the same time.

If the virtual machine is the active window, the USB drive will be automatically connected to the virtual machine when it is plugged in.
If the virtual machine is not the active window or is not running, select the virtual machine and click the VM menu. Select Removable Devices and then click Connect. The USB drive will automatically connect to your virtual machine.


6. Take a snapshot of a virtual machine. A snapshot is a saved state and will allow you to load the virtual machine to that precise moment as many times as you need.

Select your virtual machine, click the VM menu, hover over Snapshot and select Take Snapshot.
Give your Snapshot a name. You can also give it a description, though this is optional.
Click OK to save the Snapshot.
Load a saved Snapshot by clicking the VM menu and then selecting Snapshot. Choose the Snapshot you wish to load from the list and click Go To.


7. Become familiar with keyboard shortcuts. A combination of the "Ctrl" and other keys are used to navigate virtual machines. For example, "Ctrl," "Alt" and "Enter" puts the current virtual machine in full screen mode or moves through multiple machines. "Ctrl," "Alt" and "Tab" will move between virtual machines when the mouse is being used by 1 machine.


  \part{Khoa học dữ liệu}

\chapter{Trí tuệ nhân tạo}

View online \href{http://magizbox.com/training/ai/site/}{http://magizbox.com/training/ai/site/}

Artificial intelligence (AI) is the intelligence exhibited by machines or software. It is also the name of the academic field of study which studies how to create computers and computer software that are capable of intelligent behavior. Major AI researchers and textbooks define this field as "the study and design of intelligent agents", in which an intelligent agent is a system that perceives its environment and takes actions that maximize its chances of success. John McCarthy, who coined the term in 1955, defines it as "the science and engineering of making intelligent machines".

\section{Autonomous Agents}

limited ability to perceive its environment
process the environment and calculate an action
no global plan / leader
Vehicles

Action / Selection
Steering
Locomotion
Steering Behavior 1 2


Steering = Desired - Velocity

Seek
Flow Filed Following
Path Following
Group Steering
https://github.com/shiffman/The-Nature-of-Code-Examples/tree/master/chp06_agents\

Massive Battle: Coordinated Movement of Autonomous Agents ↩

Craig Reynolds, Steering Behaviors For Autonomous Characters ↩

\section{Cellular Automator}

https://www.youtube.com/watch?v=DKGodqDs9sA&index=1&list=PLRqwX-V7Uu6YrWXvEQFOGbCt6cX83Xunm

Cellular Automata

Grid of cell
Each cell has state, neighborhood
cell state at time t defined by a function of neighborhood states at time t-1
Elementary Cellular Automata

\section{Fractal}

L-System

\section{The Pac-Man project}

Today I found an interesting AI project - The Pac-Man

http://ai.berkeley.edu/images/pacman_game.gif

Here is the project overview

The Pac-Man projects were developed for UC Berkeley's introductory artificial intelligence course, CS 188. They apply an array of AI techniques to playing Pac-Man. However, these projects don't focus on building AI for video games. Instead, they teach foundational AI concepts, such as informed state-space search, probabilistic inference, and reinforcement learning. These concepts underly real-world application areas such as natural language processing, computer vision, and robotics. We designed these projects with three goals in mind. The projects allow students to visualize the results of the techniques they implement. They also contain code examples and clear directions, but do not force students to wade through undue amounts of scaffolding. Finally, Pac-Man provides a challenging problem environment that demands creative solutions; real-world AI problems are challenging, and Pac-Man is too. In our course, these projects have boosted enrollment, teaching reviews, and student engagement. The projects have been field-tested, refined, and debugged over multiple semesters at Berkeley. We are now happy to release them to other universities for educational use.
In the next part of this post, I will show my works on this project

Project 1: Search in Pacman

[caption id="" align="alignleft" width="231"]DFS[/caption]

[caption id="" align="alignleft" width="233"]BFS[/caption]
\chapter{Học máy}

View online \href{http://magizbox.com/training/machinelearning/site/}{http://magizbox.com/training/machinelearning/site/}

\begin{itemize}
  \item Vấn đề với HMM và CRF?
  \item Học MLE và MAP?
\end{itemize}



Machine learning is a branch of science that deals with programming the systems in such a way that they automatically learn and improve with experience. Here, learning means recognizing and understanding the input data and making wise decisions based on the supplied data.

We can think of machine learning as approach to automate tasks like predictions or modelling. For example, consider an email spam filter system, instead of having programmers manually looking at the emails and coming up with spam rules. We can use a machine learning algorithm and feed it input data (emails) and it will automatically discover rules that are powerful enough to distinguish spam emails.

Machine learning is used in many application nowadays like spam detection in emails or movie recommendation systems that tells you movies that you might like based on your viewing history. The nice and powerful thing about machine learning is: It learns when it gets more data and hence it gets more and more powerful the more data we give them.

**Có bao nhiêu thuật toán Machine Learning?**

Có rất nhiều thuật toán Machine Learning, bài viết [Điểm qua các thuật toán Machine Learning hiện đại](https://ongxuanhong.wordpress.com/2015/10/22/diem-qua-cac-thuat-toan-machine-learning-hien-dai/) của Ông Xuân Hồng tổng hợp khá nhiều thuật toán. Theo đó, các thuật toán Machine Learning được chia thành các nhánh lớn như `regression`, `bayesian`, `regularization`, `decision tree`, `instance based`, `dimesionality reduction`, `clustering`, `deep learning`, `neural networks`, `associated rule`, `ensemble`... Ngoài ra thì còn có các cheatsheet của [sklearn](http://scikit-learn.org/stable/tutorial/machine_learning_map/index.html).

Việc biết nhiều thuật toán cũng giống như ra đường mà có nhiều lựa chọn về xe cộ. Tuy nhiên, quan trọng là có task để làm, sau đó thì cập nhật SOTA của task đó để biết các công cụ mới.

**Xây dựng model cần chú ý điều gì?**

Khi xây dựng một model cần chú ý đến vấn đề tối ưu hóa tham số (có thể sử dụng [GridSearchCV](sklearn.model_selection.GridSearchCV))

Bài phát biểu này có vẻ cũng rất hữu ích [PYCON UK 2017: Machine learning libraries you'd wish you'd known about](https://www.youtube.com/watch?v=nDF7_8FOhpI). Có đề cập đến

* [DistrictDataLabs/yellowbrick](https://github.com/DistrictDataLabs/yellowbrick) (giúp visualize model được train bởi sklearn)
* [marcotcr/lime](https://github.com/marcotcr/lime) (giúp inspect classifier)
* [TeamHG-Memex/eli5](https://github.com/TeamHG-Memex/eli5) (cũng giúp inspect classifier, hỗ trợ nhiều model như xgboost, crfsuite, đặc biệt có TextExplainer sử dụng thuật toán từ eli5)
* [rhiever/tpot](https://github.com/rhiever/tpot) (giúp tối ưu hóa pipeline)
* [dask/dask](https://github.com/dask/dask) (tính toán song song và lập lịch)

Ghi chú về các thuật toán trong xử lý ngôn ngữ tự nhiên tại [underthesea.flow/wiki](https://github.com/magizbox/underthesea.flow/wiki/Develop)

Framework để train, test hiện tại vẫn rất thoải mái sklearn. tensorboard cung cấp phần log cũng khá hay.

[Câu trả lời hay](https://www.quora.com/What-are-the-most-important-machine-learning-techniques-to-master-at-this-time/answer/Sean-McClure-3?srid=5O2u) cho câu hỏi [Những kỹ thuật machine learning nào quan trọng nhất để master?](https://www.quora.com/What-are-the-most-important-machine-learning-techniques-to-master-at-this-time), đặc biệt là dẫn đến bài [The State of ML and Data Science 2017](https://www.kaggle.com/surveys/2017) của Kaggle.

**Tài liệu học PGM**

[Playlist youtube](https://www.youtube.com/watch?v=WPSQfOkb1M8&amp;list=PL50E6E80E8525B59C) khóa học Probabilistic Graphical Models của cô Daphne Koller. Ngoài ra còn có một [tutorial](http://mensxmachina.org/files/software/demos/bayesnetdemo.html) dở hơi ở đâu về tạo Bayesian network

**[Chưa biết] Tại sao Logistic Regression lại là Linear Model?**

Trong quyển Deep Learning, chương 6, trang 165, tác giả có viết

```
Linear models, such as logistic regression and linear
regression, are appealing because they can be fit
efficiently and reliably, either in closed form or
with convex optimization
```

Mình tự hỏi tại sao logistic regression lại là linear, trong khi nó có sử dụng hàm logit (nonlinear)? Tìm hiểu hóa ra cũng có bạn hỏi giống mình trên [stats.stackexchange.com](https://stats.stackexchange.com/questions/93569/why-is-logistic-regression-a-linear-classifier). Ngoài câu trả lời trên stats.stackexchange, đọc một số cái khác [Generalized Linear Models, SPSS Statistics 22.0.0](https://www.ibm.com/support/knowledgecenter/en/SSLVMB_22.0.0/com.ibm.spss.statistics.help/spss/advanced/idh_idd_genlin_typeofmodel.htm)
 và [6.1 - Introduction to Generalized Linear Models, Analysis of Discrete Data, Pennsylvania State University](https://onlinecourses.science.psu.edu/stat504/node/216) cũng vẫn chưa hiểu lắm.

Hiện tại chỉ hiểu là các lớp model này chỉ có thể hoạt động trên các tập linear separable, có lẽ do việc map input x, luôn có một liên kết linear $latex wx$, trước khi đưa vào hàm non-linear.

**Các tập dữ liệu thú vị**

*Iris dataset*: dữ liệu về hoa iris

Là một ví dụ cho bài toán phân loại

*Weather problem*: dữ liệu thời tiết. Có thể tìm được ở trong quyển Data Mining: Practical Machine Learning Tools and Techniques

Là một ví dụ cho bài toán cây quyết định

## Deep Learning

**Tài liệu Deep Learning**

Lang thang thế nào lại thấy trang này [My Reading List for Deep Learning!](https://www.microsoft.com/en-us/research/wp-content/uploads/2017/02/DL_Reading_List.pdf) của một anh ở Microsoft. Trong đó, (đương nhiên) có Deep Learning của thánh Yoshua Bengio, có một vụ hay nữa là bài review "Deep Learning" của mấy thánh Yann Lecun, Yoshua Bengio, Geoffrey Hinton trên tạp chí Nature. Ngoài ra còn có nhiều tài liệu hữu ích khác.

### Các layer trong deep learning [^2]

#### Sparse Layers

[**nn.Embedding**](http://pytorch.org/docs/master/nn.html#embedding) ([hướng dẫn](http://pytorch.org/tutorials/beginner/nlp/word_embeddings_tutorial.html))
★ grep code: [Shawn1993/cnn-text-classification-pytorch](https://github.com/Shawn1993/cnn-text-classification-pytorch/blob/master/model.py#L18)
Đóng vai trò như một lookup table, map một word với dense vector tương ứng

#### Convolution Layers

[**nn.Conv1d**](http://pytorch.org/docs/master/nn.html#conv1d), [**nn.Conv2d**](http://pytorch.org/docs/master/nn.html#conv2d), [**nn.Conv3d**](http://pytorch.org/docs/master/nn.html#conv3d) [^1]
★ grep code: [Shawn1993/cnn-text-classification-pytorch](https://github.com/Shawn1993/cnn-text-classification-pytorch/blob/master/model.py#L20-L24), [galsang/CNN-sentence-classification-pytorch](https://github.com/galsang/CNN-sentence-classification-pytorch/blob/master/model.py#L36-L38)

Các tham số trong Convolution Layer

* `kernel_size` (hay là filter size)

Đối với NLP, kernel_size thường bằng region_size * word_dim (đối với conv1d) hay (region_size, word_dim) đối với conv2d

<small>Quá trình tạo feature map đối với region size bằng 2</small>
![](https://media.giphy.com/media/l2QE2y1UQP7vIgiti/giphy.gif)

* `in_channels`, `out_channels` (là số lượng `feature maps`)

Kênh (channels) là các cách nhìn (view) khác nhau đối với dữ liệu. Ví dụ, trong ảnh thường có 3 kênh RGB (red, green, blue), có thể áp dụng convolution giữa các kênh. Với văn bản cũng có thể có các kênh khác nhau, như khi có các kênh sử dụng các word embedding khác nhau (word2vec, GloVe), hoặc cùng một câu nhưng biểu diễn ở các ngôn ngữ khác nhau.

* `stride`

Định nghĩa bước nhảy của filter.

![](http://d3kbpzbmcynnmx.cloudfront.net/wp-content/uploads/2015/11/Screen-Shot-2015-11-05-at-10.18.08-AM-1024x251.png)

Hình minh họa sự khác biệt giữa các feature map đối với stride=1 và stride=2. Feature map đối với stride = 1 có kích thước là 5, feature map đối với stride = 3 có kích thước là 3. Stride càng lớn thì kích thước của feature map càng nhỏ.

Trong bài báo của Kim 2014, `stride = 1` đối với `nn.conv2d` và `stride = word_dim` đối với `nn.conv1d`

Toàn bộ tham số của mạng CNN trong bài báo Kim 2014,

![](http://d3kbpzbmcynnmx.cloudfront.net/wp-content/uploads/2015/11/Screen-Shot-2015-11-06-at-8.03.47-AM.png)

| Description         | Values          |
|---------------------|-----------------|
| input word vectors  | Google word2vec |
| filter region size  | (3, 4, 5)       |
| feature maps        | 100             |
| activation function | ReLU            |
| pooling             | 1-max pooling   |
| dropout rate        | 0.5             |
| $latex l&amp;s=2$2 norm constraint  | 3               |

Đọc thêm:

* [Lecture 13: Convolutional Neural Networks (for NLP). CS224n-2017](http://web.stanford.edu/class/cs224n/lectures/cs224n-2017-lecture13-CNNs.pdf)
* [DeepNLP-models-Pytorch - 8. Convolutional Neural Networks](https://nbviewer.jupyter.org/github/DSKSD/DeepNLP-models-Pytorch/blob/master/notebooks/08.CNN-for-Text-Classification.ipynb)
* [A Sensitivity Analysis of (and Practitioners’ Guide to) Convolutional Neural Networks for Sentence Classification. Zhang 2015](https://arxiv.org/pdf/1510.03820.pdf)

**BTS**

22/11/2017 - Phải nói quyển này hơi nặng so với mình. Nhưng thôi cứ cố gắng vậy.
24/11/2017 - Từ hôm nay, mỗi ngày sẽ ghi chú một phần (rất rất nhỏ) về Deep Learning [tại đây](https://docs.google.com/document/d/1KxDrw5s6uYHNLda7t0rhp0RM_TlUGxydQ-Qi1JOPFr8/edit?usp=sharing)

[^1]: [Understanding Convolutional Neural Networks for NLP](http://www.wildml.com/2015/11/understanding-convolutional-neural-networks-for-nlp)
[^2]: [http://pytorch.org/docs/master/nn.html](http://pytorch.org/docs/master/nn.html)

\section{Machine Learning Process}

The good life is a process, not a state of being. It is a direction not a destination.

Carl Rogers



I searched a framework fit for every data mining task, I found a good one from an article of Oracle.

And here is my summary. The data mining process has 4 steps:

Step 1. Problem Definition

This initial phase of a data mining project focuses on understanding the project objectives and requirements. Once you have specified the project from a business perspective, you can formulate it as a data mining problem and develop a preliminary implementation plan.

Step 2. Data Gathering & Preparation

The data understanding phase involves data collection and exploration. As you take a closer look at the data, you can determine how well it addresses the business problem. You might decide to remove some of the data or add additional data. This is also the time to identify data quality problems and to scan for patterns in the data.

 Data Access
 Data Sampling

 Data Transformation

Data in the real world is dirty [3]. They are often incomplete (lacking attribute values, lacking certain attributes of interest, or containing only aggregate data), noisy (containing errors or outliers),‰ inconsistent (containing discrepancies in codes or names). Step 3. Model Building In this phase, you select and apply various modeling techniques and calibrate the parameters to optimal values. If the algorithm requires data transformations, you will need to step back to the previous phase to implement them

 Create Model
 Test Model

  Evaluate & Interpret Model

Some important questions [2]:

Is at least one of predictors useful in predicting the response? (F-statistics)
Do all the predictors help to explain Y, or is only a subset of the predictors useful? (all subsets or best subsets)
How well does the model fit the data?
Given a set of predictor values, what response value should we predict, and how accurate is our prediction?
Step 4. Knowledge Deployment Knowledge deployment is the use of data mining within a target environment. In the deployment phase, insight and actionable information can be derived from data.
Model Apply
Custom Reports
External Applications
References
The Data Mining Process, Oracle
Trevor Hastie and Rob Tibshirani, Model Selection and Qualitative Predictors, URL:https://www.youtube.com/watch?v=3T6RXmIHbJ4
Nguyen Hung Son, Data cleaning and Data preprocessing, URL:http://www.mimuw.edu.pl/~son/datamining/DM/4-preprocess.pdf

\subsection{Problem Definition}

This initial phase of a data mining project focuses on understanding the project objectives and requirements. Once you have specified the project from a business perspective, you can formulate it as a data mining problem and develop a preliminary implementation plan.

For example, your business problem might be: "How can I sell more of my product to customers?" You might translate this into a data mining problem such as: "Which customers are most likely to purchase the product?" A model that predicts who is most likely to purchase the product must be built on data that describes the customers who have purchased the product in the past. Before building the model, you must assemble the data that is likely to contain relationships between customers who have purchased the product and customers who have not purchased the product. Customer attributes might include age, number of children, years of residence, owners/renters, and so on.

\subsection{Data Gathering}

The data understanding phase involves data collection and exploration. As you take a closer look at the data, you can determine how well it addresses the business problem. You might decide to remove some of the data or add additional data. This is also the time to identify data quality problems and to scan for patterns in the data.

The data preparation phase covers all the tasks involved in creating the case table you will use to build the model. Data preparation tasks are likely to be performed multiple times, and not in any prescribed order. Tasks include table, case, and attribute selection as well as data cleansing and transformation. For example, you might transform a DATE_OF_BIRTH column to AGE; you might insert the average income in cases where the INCOME column is null.

Additionally you might add new computed attributes in an effort to tease information closer to the surface of the data. For example, rather than using the purchase amount, you might create a new attribute: "Number of Times Amount Purchase Exceeds $500 in a 12 month time period." Customers who frequently make large purchases may also be related to customers who respond or don't respond to an offer.

Thoughtful data preparation can significantly improve the information that can be discovered through data mining.

Data Sources
Open Data

wikipedia dumps: https://dumps.wikimedia.org/other/pagecounts-raw/

\subsection{Data Preprocessing}

The quality of the data and the amount of useful information it contains affect greatly how well an algorithm can learn. Hence, it is important to preprocess the dataset before using it. The most common preprocessing steps are: removing missing values, converting categorical data into shape suitable for machine learning algorithm and feature scaling.

Missing Data
Sometimes the samples in the dataset are missing some values and we want to deal with these missing values before passing it to the machine learning algorithm. There are a number of strategies we can follow

Remove samples with missing values: This approach is by far the most convenient but we may end up removing too many samples and by that we would be losing valuable information that can help the machine learning algorithm.
Imputing missing values: Instead of removing the entire sample we use interpolation to estimate the missing values. For example, we could substitute a missing value by the mean of the entire column.
Categorical Data
In general, features can be numerical (e.g. price, length, width, etc…) or categorical (e.g. color, size, etc..). Categorical features are further split into nominal and ordinal features.

Ordinal features can be sorted and ordered. For example, size (small, medium, large), we can order these sizes large > medium > small. While nominal features do not have an order for example, color, it doesn’t make any sense to say that red is larger than blue.

Most machine learning algorithm require that you convert categorical features into numerical values. One solution would to assign each value a different number starting from zero. (e.g. small à 0 ,medium à 1 ,large à 2)

This works well for ordinal features but might cause problems with nominal features (e.g. blue à 0, white à 1, yellow à 2) because even though colors are not ordered the learning algorithm will assume that white is larger than blue and yellow is larger than white and this is not correct.

To get around this problem is to use one-hot encoding, the idea is to create a new feature for each unique value of the nominal feature.



In the above example, we converted the color feature into three new features Red, Green, Blue and we used binary values to indicate the color. For example, a sample with “Red” color is now encoded as (Red=1, Green=0, Blue=0)

Feature Scaling
Why have we do Feature Scaling?

We have to predict the house prices base on 2 features:

House sizes (feet2)
Number of bedrooms in the house
And we relized that house sizes are about 1000 times  the number of bedrooms. When features differ by orders of magnitude, first performing feature scaling can make gradient descent converge much more quickly.

Perform Feature Scaling

Subtract the mean value (the average value) of each feature from the dataset.
After subtracting the mean, additionally scale (divide) the feature values by their respective "standard deviations."
Function: x′=x−x¯σx′=x−x¯σ where xx is the original feature vector, x¯x¯ is the mean of that feature vector, and σσ is its standard deviation.
Feature Scaling Function implementation in Octave

function [X_norm, mu, sigma] = featureNormalize(X)
X_norm = X;
mu = zeros(1, size(X, 2)); % storing the mean value in mu
sigma = zeros(1, size(X, 2)); % storing the standard deviation in sigma

for i = 1:length(mu),
mu(i) = mean(X(:,i));
end;

for i = 1:length(sigma),
sigma(i) = std(X(:,i));
end;

X_norm = (X .- mu)./sigma;
end
Related Reading
Introduction to Machine Learning

\subsection{Model Building}

In this phase, you select and apply various modeling techniques and calibrate the parameters to optimal values. If the algorithm requires data transformations, you will need to step back to the previous phase to implement them

Create Model
Test Model
Evaluate & Interpret Model
Some important questions

Is at least one of predictors useful in predicting the response? (F-statistics)
Do all the predictors help to explain Y, or is only a subset of the predictors useful? (all subsets or best subsets)
How well does the model fit the data?
Given a set of predictor values, what response value should we predict, and how accurate is our prediction?
Create Model
First thing first, start with simple and fast model, then you known how difficult the problem is.

One import thing is create a well pipeline for your experiments, it is very helpful in turning features, model selection, save your experiment and write reports.

Feature Selections
After train model, some model will give active features (such as CRF), it is clue for you to feature selection. If amount active features is too small compared to amount features, it is the problem. In this case the better way to enhance is try reduce amount of features and see how well this set fit data. Keep in mind the more number of features is, the complex model is, and it will make your model over fitting.
Storing the model
Number of active features: 5566 (35383)
Number of active attributes: 4343 (20722)
example after training crf model with python-crfsuite
Test Model
This phase determines how well the model fit data. See Evaluation for details.

What to do next
In an interview Andrew Ng said about building machine learning model

"I often make an analogy to building a rocket ship. A rocket ship is a giant engine together with a ton of fuel. Both need to be really big. If you have a lot of fuel and a tiny engine, you won’t get off the ground. If you have a huge engine and a tiny amount of fuel, you can lift up, but you probably won’t make it to orbit. So you need a big engine and a lot of fuel.

The reason that machine learning is really taking off now is that we finally have the tools to build the big rocket engine — that is giant computers, that’s our rocket engine. And the fuel is the data. We finally are getting the data that we need."

We need both big rocket engine and data to make our model works.

Related Reading
Inside The Mind That Built Google Brain: On Life, Creativity, And Failure, huffingtonpost.com

\subsection{Evaluation}

Training vs Test Data
We typically split the input data into learning and testing datasets. The then run the machine learning algorithm on the learning dataset to generate the prediction model. Later, we use the test dataset to evaluate our model.



It is important that the test data is separate from the one used in training otherwise we will be kind of cheating because may for example the generated model memorizes the data and hence if the test data is also part of the training data then our evaluation scores of the model will be higher than they actually are.

The data is usually split 75% training and 25% data or 2/3 training and 1/3 testing. It is important to note that: the smaller the training set the more challenging it is for the algorithm to discover the rules.

In addition, when splitting the dataset, you need to maintaining class proportions and population statistics otherwise we will have some classes that are under represented in the training dataset and over represented in the test dataset.

For example, you may have 100 sample and a total of 80 samples are labeled with Class-A and the remaining 20 instances are labeled with Class-B. you want to make sure when splitting the data that you maintain this representation.

One way to avoid this problem and to make sure that all classes are represented in both training and testing datasets is stratification. It is the process of rearranging the data as to ensure each set is a good representative of the whole. In our previous example, (80/20 samples), it is best to arrange the data such that in every set, each class comprises around 80:20 ratios of the two classes.

Cross Validation
A crucial step when building our machine learning model is to estimate its performance on that that the model hadn't seen before. We want to make sure that the model generalizes well to new unseen data.

One case, the machine learning algorithm has different parameters and we want to tune these parameters to achieve the best performance. (Note: the parameters of the machine learning algorithm are called hyperparameters). Another case, sometimes we want to try out different algorithms and choose the best performing one. Below are some of the techniques used.

Holdout Method
We simply split the data into training and testing datasets. We train the algorithm on the training dataset to generate a model. In order, to evaluate different algorithms we use the testing data to evaluate each algorithm.

However, if we reuse the same test dataset over and over again during algorithm selection, the test data has now come part of the training data. Hence, when we use the test data for the final evaluation the generated model is biased towards the test data and the performance score is optimistic.

Holdout Validation
As before, we split the data into training and testing dataset. Then, the training data is further split into training and validation sets.

The training data is used to train different models. Then the validation data is used to compute performance of each of them and we select the best one. Finally, the model is then used for the test set to evaluate performance. The next figure illustrates this idea.



However, because we use the validation set multiple times, Holdout validation is sensitive to how we partition the data and that is what K-fold cross validation tries to solve.

K-fold cross validation
Initially, we split the data into training and testing dataset. Furthermore, the training dataset is split into K chunks.

Suppose we will use 5-fold cross validation, the training data set is split into 5 chunks and the training phase will take place over 5 iterations. In each iteration we use one chunk as the validation dataset while the rest of the chunk are grouped together to form the training dataset.

This is very similar to Holdout validation except in each iteration the validation data is different and this removed the bias. Each iteration generates a score and the final score is the average score of all iteration. As before we select the best model and use the test data for the final performance evaluation.

Related Readings

Introduction to Machine Learning

\section{Types of Machine Learning}

There are three different types of machine learning: supervised, unsupervised and reinforcement learning. 4

Supervised Learning
The goal of supervised learning is to learn a model from labelled training data that allows us to make predictions about future data. For supervised machine learning to work we need to feed the algorithm two things: the input data and our knowledge about it labels).

The spam filter example mentioned earlier is a good example of supervised learning; we have a bunch of emails (data) and we know whether each email is spam or not (labels).



Supervised learning can be divided into two subcategories:

Classification: It is used to predict categories or class labels based on past observations i.e. we have discrete variable you want to distinguish into discrete categorical outcome. For example, in the email spam filter system the output is discrete "spam" or "not spam".
Regression: It is used to predict a continuous outcome. For example, to determine the price of houses and how it is affected by the number of rooms in that house. The input data is the house features (no. of rooms, location, size in square feet,) and the output is the price (the continuous outcome).
Unsupervised Learning
The goal of unsupervised learning is to discover hidden structure or patterns in unlabeled data and it can be divided into two subcategories

Clustering: It is used to organize information into meaningful clusters (subgroups) without having prior knowledge of their meaning. For example, the figure below shows how we can use clustering to organize unlabeled data into groups based on their features.



Dimensionality Reduction (Compression): It is used to reduce a higher dimension data into a lower dimension ones. To put it more clearly consider this example. A telescope has terabytes of data and not all of these data can be stored and so we can use dimensionality reduction to extract the most informative features of these data to be stored. Dimensionality reduction is also a good candidate to visualize data because if you have data in higher dimensions you can compress it to 2D or 3D to easily plot and visualize it.

Reinforcement Learning
The goal of reinforcement learning is to develop a system that improves its performance based on the interaction with a dynamic environment and there is a delayed feedback that act as a reward. i.e. reinforcement learning is learning by doing with a delayed reward. A classic example of reinforcement learning is a chess game, the computer decided a series of moves and the reward is the "win" or "lose" at the end the game.

You might think that this is similar to supervised learning where the reward is basically a label for the data but the core difference is this feedback/reward is not the truth but it is a measure of how well the action to achieving a certain goal.

Microsoft Azure Machine Learning 1


Machine Learning Cheat Sheet for scikit-learn 2


DLib C++ Library - Machine Learning Guide 3


Challenges
Very much features (> 100)
Very much data (> 1e9 items)
Text Data, Images, Videos
Training Times
Accuracy, Over Fitting
Machine learning algorithm cheat sheet for Microsoft Azure Machine Learning Studio ↩

Machine Learning Cheat Sheet (for scikit-learn) ↩

DLib C++ Library - Machine Learning Guide ↩

Introduction to Machine Learning ↩

\section{How to learn a ML Algorithm?}

1. Motivation

Each algorithm have its own motivation. It may a simple example to see how it work

2. Problem Definition

Where can we apply this algorithm? How did it work in real world applications

3. Mathematics Representation

Problem Equations, notations

We will discuss about mathematics representation of algorithm, notations we use for problem

4. Algorithm

We will discuss how to solve this mathematics problems

5. Examples

We will apply algorithm with a few examples (1-2 dimension is highly recommended, because we will plot these data and model easily)

In this section, we can see how well (bad) algorithm works with these data

6. Implementation Notice

We will give some notes about implement this algorithm to real world problems. What case we want to apply this algorithm? What case we don't?

7. Quiz

One way to rethink about problem is doing quiz.

8. Exercise

\section{Machine Learning Algorithms}

\subsection{Linear Regression}

Linear Regression
In-Out


Input: Continuous Output: Continuous

When to use 1
Econometric Modeling
Marketing Mix Model
Customer Lifetime Value
Examples
Ex1. Linear Regression with Boston Dataset

__author__ = 'rain'

from sklearn.datasets import load_boston
from sklearn.cross_validation import train_test_split
from sklearn.linear_model import LinearRegression, Ridge
boston = load_boston()
data = boston['data']
X, y = data[:, :-1], data[:, -1]
X_train, X_test, y_train, y_test =
  train_test_split(X, y, test_size=0.3)
print boston['DESCR']
clf_linear = LinearRegression()
clf_linear.fit(X_train, y_train)
linear_score = clf_linear.score(X_test, y_test)
#-> 0.671
print(clf_linear.coef_)
print(clf_linear.intercept_)

clf_ridge = Ridge(alpha=1.0)
clf_ridge.fit(X_train, y_train)
# 0.674
ridge_score = clf_ridge.score(X_test, y_test)

print y_test
print clf_linear.predict(X_test)
print clf_ridge.predict(X_test)
Ex2. Linear Regression with market data set (coursera) ([python language="notebook"]/python)

Logistic Regression


In-Out 1
In: continuos
Out: True/False
1. Hyposthesis Representation
hθ(x)=g(θTx) where g(z)=11+e−z
hθ(x)=g(θTx) where g(z)=11+e−z
g(z)g(z) is sigmoid function or logistic function

hθ(x)hθ(x) estimated probability of y=1y=1 given xx

In spam detection problem, hθ(x)=0.7hθ(x)=0.7 means it's 70% chance this email is spam.

2. Decision Boundary
Logistic Regression

3. Cost Function
cost(hθ(x),y)=−ylog(hθ(x))−(1−y)log(1−hθ(x))
cost(hθ(x),y)=−ylog(hθ(x))−(1−y)log(1−hθ(x))
Loss Function

J(θ)=1m∑i=1mcost(hθ(x(i)),y(i))=−1m∑i=1my(i)loghθ(x(i))+(1−y(i))log(1−hθ(x(i)))
J(θ)=1m∑i=1mcost(hθ(x(i)),y(i))=−1m∑i=1my(i)loghθ(x(i))+(1−y(i))log(1−hθ(x(i)))
4. Gradient Descent
Gradient

∂J(θ)∂θj=1m∑i=1m(hθ(x(i))−y(i))x(i)j
∂J(θ)∂θj=1m∑i=1m(hθ(x(i))−y(i))xj(i)
5. Predict
p(θ,X)=hθ(X)≥0.5
p(θ,X)=hθ(X)≥0.5
6. Regularization
6.1 Feature Mapping
Cost Function

mapFeature(x)=⎡⎣⎢⎢⎢⎢⎢⎢⎢⎢⎢⎢⎢⎢⎢⎢⎢⎢⎢⎢⎢⎢1x1x2x21x1x2x22x31⋯x1x52x62⎤⎦⎥⎥⎥⎥⎥⎥⎥⎥⎥⎥⎥⎥⎥⎥⎥⎥⎥⎥⎥⎥
mapFeature(x)=[1x1x2x12x1x2x22x13⋯x1x25x26]
6.2 Cost Function and Gradient
Cost Function
J(θ)=1m∑i=1m[−y(i)log(hθ(x(i)))−(1−y(i))log(1−hθ(x(i)))]+λ2m∑j=1nθ2j
J(θ)=1m∑i=1m[−y(i)log(hθ(x(i)))−(1−y(i))log(1−hθ(x(i)))]+λ2m∑j=1nθj2
Gradient

∂J(θ)∂θj=1m∑mi=1(hθ(x(i))−y(i))x(i)j∂J(θ)∂θj=1m∑i=1m(hθ(x(i))−y(i))xj(i) for j=0j=0

∂J(θ)∂θj=(1m∑mi=1(hθ(x(i))−y(i))x(i)j)+λmθj∂J(θ)∂θj=(1m∑i=1m(hθ(x(i))−y(i))xj(i))+λmθj for j≥1j≥1

Code
Bank Marketing Data Set

import statsmodels.api as sm
import pandas as pd
from statsmodels.tools.tools import categorical
from sklearn.preprocessing import LabelEncoder
from sklearn.linear_model import LogisticRegression
from sklearn.cross_validation import train_test_split
from sklearn.metrics import confusion_matrix
import numpy
from sklearn.tree import DecisionTreeClassifier


def get_data():
    return pd.read_csv(&quot;./bank/bank-full.csv&quot;, header=0, sep=&quot;;&quot;)

data = get_data()

data.job = LabelEncoder().fit_transform(data.job)
data.marital = LabelEncoder().fit_transform(data.marital)
data.education = LabelEncoder().fit_transform(data.education)
data.default = LabelEncoder().fit_transform(data.default)
data.housing = LabelEncoder().fit_transform(data.housing)
data.loan = LabelEncoder().fit_transform(data.loan)
data.month = LabelEncoder().fit_transform(data.month)
data.contact = LabelEncoder().fit_transform(data.contact)
data.poutcome = LabelEncoder().fit_transform(data.poutcome)

X = data.iloc[:, :-1]
y = data.iloc[:, -1]

X_train, X_test, y_train, y_test = train_test_split(X, y, test_size=0.3)

clf = LogisticRegression()
clf.fit(X_train, y_train)
score = clf.score(X_test, y_test)

print confusion_matrix(y_test, clf.predict(X_test))
# [[11807   203]
#  [ 1243   311]]
# it's too bad

Examples
Affair Dataset, Logistic Regression with scikit-learn
Linear Regression vs Logistic Regression vs Poisson Regression ↩

\subsection{Classification}

Classification


Classification 1
A very familiar example is the email spam-catching system: given a set of emails marked as spam and not-spam, it learns the characteristics of spam emails and is then able to process future email messages to mark them as spam or not-spam.

The technique used in the above example of email spam-catching system is one of the most common machine learning techniques: classification (actually, statistical classification). More precisely it is a supervised statistical classification. Supervised because the system needs to be first trained using already classified training data as opposed to an unsupervised system where such training is not done.

A supervised learning system that performs classification is known as a learner or, more commonly, a classifier.

The classifier is first fed training data in which each item is already labeled with the correct label or class. This data is used to train the learning algorithm, which creates models that can then be used to label/classify similar data.

Formally, given a set of input items,  and a set of labels/classes,  and training data is the label/class for $latex x_i$, a classifier is a mapping from X to Y $latex f(T, x) = y$.

Binary Classification
Algorithms 1
Two-class SVM
100 features, linear model

Two-class Logistic Regression
Fast training, linear model
Two-class Bayes point machine
Fast training, linear model
Two-class random forest
Accuracy, fast training
Two-class boosted decision tree
Accuracy, fast training
Two-class neural network
Accuracy, long training times
Multiclass Classification


Introduction 2
In machine learning, multiclass or multinomial classification is the problem of classifying instances into one of the more than two classes (classifying instances into one of the two classes is called binary classification).

While some classification algorithms naturally permit the use of more than two classes, others are by nature binary algorithms; these can, however, be turned into multinomial classifiers by a variety of strategies.

Multiclass classification should not be confused with multi-label classification, where multiple labels are to be predicted for each instance.

Algorithms 1
Multiclass Logistic Regression
Multiclass SVM
Multiclass Neural Network
Multiclass Decision Forest
Multiclass Decision Jungle
Confusion Matrix
sklearn plot confusion matrix with labels 3

import matplotlib.pyplot as plt
def plot_confusion_matrix(cm, title='Confusion matrix', cmap=plt.cm.Blues, labels=None):
    fig = plt.figure()
    ax = fig.add_subplot(111)
    cax = ax.matshow(cm)
    plt.title(title)
    fig.colorbar(cax)
    if labels:
        ax.set_xticklabels([''] + labels)
        ax.set_yticklabels([''] + labels)
    plt.xlabel('Predicted')
    plt.ylabel('True')
    plt.show()


Multilabel Classification


Introduction 1
In machine learning, multi-label classification and the strongly related problem of multi-output classification are variants of the classification problem where multiple target labels must be assigned to each instance. Multi-label classification should not be confused with multiclass classification, which is the problem of categorizing instances into one of more than two classes. Formally, multi-label learning can be phrased as the problem of finding a model that maps inputs x to binary vectors y, rather than scalar outputs as in the ordinary classification problem.

There are two main methods for tackling the multi-label classification problem:[1] problem transformation methods and algorithm adaptation methods. Problem transformation methods transform the multi-label problem into a set of binary classification problems, which can then be handled using single-class classifiers. Algorithm adaptation methods adapt the algorithms to directly perform multi-label classification. In other words, rather than trying to convert the problem to a simpler problem, they try to address the problem in its full form.

Implements
Multiclass and multilabel algorithms
SVM
Multi-label classification ↩

Multiclass classification ↩

sklearn plot confusion matrix with labels ↩

\subsection{Clustering}
Using K-Means to cluster wine dataset
Recently, I joined Cluster Analysis course in coursera. The content of first week is about Partitioning-Based Clustering Methods where I learned about some cluster algorithms based on distance such as K-Means, K-Medians and K-Modes. I would like to turn what I learn into practice so I write this post as an excercise of this course.

In this post, I will use K-Means for clustering wine data set which I found in one of excellent posts about K-Mean in r-statistics website.

Meet the data


The wine data set contains the results of a chemical analysis of wines grown in a specific area of Italy. Three types of wine are represented in the 178 samples, with the results of 13 chemical analyses recorded for each sample. The Type variable has been transformed into a categoric variable.

 data(wine, package=&quot;rattle&quot;)
head(wine)

#&gt;   Type Alcohol Malic  Ash Alcalinity Magnesium Phenols
#&gt; 1    1   14.23  1.71 2.43       15.6       127    2.80
#&gt; 2    1   13.20  1.78 2.14       11.2       100    2.65
#&gt; 3    1   13.16  2.36 2.67       18.6       101    2.80
#&gt; 4    1   14.37  1.95 2.50       16.8       113    3.85
#&gt; 5    1   13.24  2.59 2.87       21.0       118    2.80
#&gt; 6    1   14.20  1.76 2.45       15.2       112    3.27
#&gt;   Flavanoids Nonflavanoids Proanthocyanins Color  Hue
#&gt; 1       3.06          0.28            2.29  5.64 1.04
#&gt; 2       2.76          0.26            1.28  4.38 1.05
#&gt; 3       3.24          0.30            2.81  5.68 1.03
#&gt; 4       3.49          0.24            2.18  7.80 0.86
#&gt; 5       2.69          0.39            1.82  4.32 1.04
#&gt; 6       3.39          0.34            1.97  6.75 1.05
#&gt;   Dilution Proline
#&gt; 1     3.92    1065
#&gt; 2     3.40    1050
#&gt; 3     3.17    1185
#&gt; 4     3.45    1480
#&gt; 5     2.93     735
#&gt; 6     2.85    1450
Explore and Preprocessing Data
Let's see structure of wine data set

 str(wine)

#&gt; &apos;data.frame&apos;:  178 obs. of  14 variables:
#&gt; $ Type           : Factor w/ 3 levels &quot;1&quot;,&quot;2&quot;,&quot;3&quot;: 1 1 1 1 1 1 1 1 1 1 ...
#&gt; $ Alcohol        : num  14.2 13.2 13.2 14.4 13.2 ...
#&gt; $ Malic          : num  1.71 1.78 2.36 1.95 2.59 1.76 1.87 2.15 1.64 1.35 ...
#&gt; $ Ash            : num  2.43 2.14 2.67 2.5 2.87 2.45 2.45 2.61 2.17 2.27 ...
#&gt; $ Alcalinity     : num  15.6 11.2 18.6 16.8 21 15.2 14.6 17.6 14 16 ...
#&gt; $ Magnesium      : int  127 100 101 113 118 112 96 121 97 98 ...
#&gt; $ Phenols        : num  2.8 2.65 2.8 3.85 2.8 3.27 2.5 2.6 2.8 2.98 ...
#&gt; $ Flavanoids     : num  3.06 2.76 3.24 3.49 2.69 3.39 2.52 2.51 2.98 3.15 ...
#&gt; $ Nonflavanoids  : num  0.28 0.26 0.3 0.24 0.39 0.34 0.3 0.31 0.29 0.22 ...
#&gt; $ Proanthocyanins: num  2.29 1.28 2.81 2.18 1.82 1.97 1.98 1.25 1.98 1.85 ...
#&gt; $ Color          : num  5.64 4.38 5.68 7.8 4.32 6.75 5.25 5.05 5.2 7.22 ...
#&gt; $ Hue            : num  1.04 1.05 1.03 0.86 1.04 1.05 1.02 1.06 1.08 1.01 ...
#&gt; $ Dilution       : num  3.92 3.4 3.17 3.45 2.93 2.85 3.58 3.58 2.85 3.55 ...
#&gt; $ Proline        : int  1065 1050 1185 1480 735 1450 1290 1295 1045 1045 ...
Wine data set contains 1 categorical variables (label) and 13 numerical variables. But these numerical variables is not scaled, I use scale function for scaling and centering data and then assign it as training data.

 data.train &lt;- scale(wine[-1])
Data is already centered and scaled.

 summary(data.train)
#&gt;   Alcohol             Malic
#&gt; Min.   :-2.42739   Min.   :-1.4290
#&gt; 1st Qu.:-0.78603   1st Qu.:-0.6569
#&gt; Median : 0.06083   Median :-0.4219
#&gt; Mean   : 0.00000   Mean   : 0.0000
#&gt; 3rd Qu.: 0.83378   3rd Qu.: 0.6679
#&gt; Max.   : 2.25341   Max.   : 3.1004
#&gt;      Ash             Alcalinity
#&gt; Min.   :-3.66881   Min.   :-2.663505
#&gt; 1st Qu.:-0.57051   1st Qu.:-0.687199
#&gt; Median :-0.02375   Median : 0.001514
#&gt; Mean   : 0.00000   Mean   : 0.000000
#&gt; 3rd Qu.: 0.69615   3rd Qu.: 0.600395
#&gt; Max.   : 3.14745   Max.   : 3.145637
#&gt;   Magnesium          Phenols
#&gt; Min.   :-2.0824   Min.   :-2.10132
#&gt; 1st Qu.:-0.8221   1st Qu.:-0.88298
#&gt; Median :-0.1219   Median : 0.09569
#&gt; Mean   : 0.0000   Mean   : 0.00000
#&gt; 3rd Qu.: 0.5082   3rd Qu.: 0.80672
#&gt; Max.   : 4.3591   Max.   : 2.53237
#&gt;   Flavanoids      Nonflavanoids
#&gt; Min.   :-1.6912   Min.   :-1.8630
#&gt; 1st Qu.:-0.8252   1st Qu.:-0.7381
#&gt; Median : 0.1059   Median :-0.1756
#&gt; Mean   : 0.0000   Mean   : 0.0000
#&gt; 3rd Qu.: 0.8467   3rd Qu.: 0.6078
#&gt; Max.   : 3.0542   Max.   : 2.3956
#&gt; Proanthocyanins        Color
#&gt; Min.   :-2.06321   Min.   :-1.6297
#&gt; 1st Qu.:-0.59560   1st Qu.:-0.7929
#&gt; Median :-0.06272   Median :-0.1588
#&gt; Mean   : 0.00000   Mean   : 0.0000
#&gt; 3rd Qu.: 0.62741   3rd Qu.: 0.4926
#&gt; Max.   : 3.47527   Max.   : 3.4258
#&gt;      Hue              Dilution
#&gt; Min.   :-2.08884   Min.   :-1.8897
#&gt; 1st Qu.:-0.76540   1st Qu.:-0.9496
#&gt; Median : 0.03303   Median : 0.2371
#&gt; Mean   : 0.00000   Mean   : 0.0000
#&gt; 3rd Qu.: 0.71116   3rd Qu.: 0.7864
#&gt; Max.   : 3.29241   Max.   : 1.9554
#&gt;    Proline
#&gt; Min.   :-1.4890
#&gt; 1st Qu.:-0.7824
#&gt; Median :-0.2331
#&gt; Mean   : 0.0000
#&gt; 3rd Qu.: 0.7561
# &gt; Max.   : 2.963
Model Fitting
Now the fun part begins. I use NbClust function to determine what is the best number of clusteres k for K-Means

 nc &lt;- NbClust(data.train,
              min.nc=2, max.nc=15,
              method=&quot;kmeans&quot;)
barplot(table(nc$Best.n[1,]),
        xlab=&quot;Numer of Clusters&quot;,
        ylab=&quot;Number of Criteria&quot;,
        main=&quot;Number of Clusters Chosen by 26 Criteria&quot;)


According to the graph, we can find the best number of clusters is 3. Beside NbClust function which provides 30 indices for determing the number of clusters and proposes the best clustering scheme, we can draw the sum of square error (SSE) scree plot and look for a bend or elbow in this graph to determine appropriate k

 wss &lt;- 0
for (i in 1:15){
  wss[i] &lt;-
    sum(kmeans(data.train, centers=i)$withinss)
}
plot(1:15,
  wss,
  type=&quot;b&quot;,
  xlab=&quot;Number of Clusters&quot;,
  ylab=&quot;Within groups sum of squares&quot;)


Both two methods suggest k=3 is best choice for us. It's reasonsable if we take notice that the original data set also contains 3 classes.

Fit the model
We now fit wine data to K-Means with k = 3

 fit.km &lt;- kmeans(data.train, 3)
Then interpret the result

 fit.km

#&gt; K-means clustering with 3 clusters of sizes 51, 65, 62
#&gt;
#&gt; Cluster means:
#&gt;      Alcohol      Malic        Ash Alcalinity
#&gt; 1  0.1644436  0.8690954  0.1863726  0.5228924
#&gt; 2 -0.9234669 -0.3929331 -0.4931257  0.1701220
#&gt; 3  0.8328826 -0.3029551  0.3636801 -0.6084749
#&gt;     Magnesium     Phenols  Flavanoids Nonflavanoids
#&gt; 1 -0.07526047 -0.97657548 -1.21182921    0.72402116
#&gt; 2 -0.49032869 -0.07576891  0.02075402   -0.03343924
#&gt; 3  0.57596208  0.88274724  0.97506900   -0.56050853
#&gt;   Proanthocyanins      Color        Hue   Dilution
#&gt; 1     -0.77751312  0.9388902 -1.1615122 -1.2887761
#&gt; 2      0.05810161 -0.8993770  0.4605046  0.2700025
#&gt; 3      0.57865427  0.1705823  0.4726504  0.7770551
#&gt;      Proline
#&gt; 1 -0.4059428
#&gt; 2 -0.7517257
#&gt; 3  1.1220202
#&gt;
#&gt; Clustering vector:
#&gt;   [1] 3 3 3 3 3 3 3 3 3 3 3 3 3 3 3 3 3 3 3 3 3 3 3 3 3
#&gt;  [26] 3 3 3 3 3 3 3 3 3 3 3 3 3 3 3 3 3 3 3 3 3 3 3 3 3
#&gt;  [51] 3 3 3 3 3 3 3 3 3 2 2 1 2 2 2 2 2 2 2 2 2 2 2 3 2
#&gt;  [76] 2 2 2 2 2 2 2 2 1 2 2 2 2 2 2 2 2 2 2 2 3 2 2 2 2
#&gt; [101] 2 2 2 2 2 2 2 2 2 2 2 2 2 2 2 2 2 2 1 2 2 3 2 2 2
#&gt; [126] 2 2 2 2 2 1 1 1 1 1 1 1 1 1 1 1 1 1 1 1 1 1 1 1 1
#&gt; [151] 1 1 1 1 1 1 1 1 1 1 1 1 1 1 1 1 1 1 1 1 1 1 1 1 1
#&gt; [176] 1 1 1
#&gt;
#&gt; Within cluster sum of squares by cluster:
#&gt; [1] 326.3537 558.6971 385.6983
#&gt;  (between_SS / total_SS =  44.8 %)
#&gt;
#&gt; Available components:
#&gt;
#&gt; [1] &quot;cluster&quot;      &quot;centers&quot;      &quot;totss&quot;
#&gt; [4] &quot;withinss&quot;     &quot;tot.withinss&quot; &quot;betweenss&quot;
# &gt; [7] &quot;size&quot;         &quot;iter&quot;         &quot;ifault&quot
The result shows information about cluster means, clustering vector, sum of square by cluster and available components. Let's do some visualizations to see how data set is clustered.

First, I use plotcluster function from fpc package to draw discriminant projection plot

 library(fpc)
plotcluster(data.train, fit.km$cluster)


We can see the data is clustered very well, there are no collapse between clusters. Next, we draw parallel coordinates plot to see how variables contributed in each cluster

 library(MASS)
parcoord(data.train, fit.km$cluster)


We can extract some insights from above graph suc as black cluster contains wine with low flavanoids value, low proanthocyanins value, low hue value. Or green cluster contains wine which has dilution value higher than wine in red cluster.

Evaluation
Because the original data set wine also has 3 classes, it is reasonable if we compare these classes with 3 clusters fited by K-Means

 confuseTable.km &lt;- table(wine$Type, fit.km$cluster)
confuseTable.km
#&gt;    1  2  3
#&gt; 1  0  0 59
#&gt; 2  3 65  3
# &gt; 3 48  0
We can see only 6 sample is missed. Let's use randIndex from flexclust to compare these two parititions - one from data set and one from result of clustering method.

 library(flexclust)
randIndex(ct.km)
#&gt;      ARI
#&gt; 0.897495
It's quite close to 1 so K-Means is good model for clustering wine data set.

References
Choosing number of cluster in K-Means, http://stackoverflow.com/a/15376462/1036500
K-means Clustering (from “R in Action”), http://www.r-statistics.com/2013/08/k-means-clustering-from-r-in-action/
Color the cluster output in r,  http://stackoverflow.com/questions/15386960/color-the-cluster-output-in-r

\subsection{Ensemble}

Ensemble Algorithms 1
Ensemble methods are models composed of multiple weaker models that are independently trained and whose predictions are combined in some way to make the overall prediction.

Much effort is put into what types of weak learners to combine and the ways in which to combine them. This is a very powerful class of techniques and as such is very popular.

Boosting
Bootstrapped Aggregation (Bagging)
AdaBoost
Stacked Generalization (blending)
Gradient Boosting Machines (GBM)
Gradient Boosted Regression Trees (GBRT)
Random Forest
XGBoost
XGBoost is short for eXtreme gradient boosting.

Features 1
Easy to use
Easy to install
Highly developed R/python for users
Efficiency
Automatic parallel computation on a single machine
Can be run on a cluster.
Accuracy
Good results for most data sets
Feasibility
Customized object and evaluation
Turnable parameters
Xgboost Optimization 2
You can use xgb.plot_important to decide how many features in your model.
Use xgb.cv (example) instead of xgb.train with watchlist (example)
https://www.kaggle.com/c/otto-group-product-classification-challenge/forums/t/12947/achieve-0-50776-on-the-leaderboard-in-a-minute-with-xgboost?page=5

Installation
Installation in Windows 64bit, Python 2.7, Anaconda

git clone https://github.com/dmlc/xgboost
git checkout 9bc3d16
Open project in xgboost/windows with Visual Studio 2013
In Visual Studio 2013, open Configuration Manager...,
choose Release in Active solution configuration
choose x64 in Active solution platform
Rebuild xgboost, xgboost_wrapper
Copy all file in xgboost/windows/x64/Release folder to xgboost/wrapper
Go to xgboost/python-package, run command python setup.py install
Check xgboost by running command python -c "import xgboost"
Examples
Multi class classification:

Understanding XGBoost Model on Otto Dataset

Resources
http://www.slideshare.net/ShangxuanZhang/xgboost
youtube, Kaggle Winning Solution Xgboost algorithm -- Let us learn from its author ↩

Notes on Parameter Tuning ↩

\subsection{Dimensionality Reduction}


Dimensionality Reduction Algorithms
Like clustering methods, dimensionality reduction seek and exploit the inherent structure in the data, but in this case in an unsupervised manner or order to summarise or describe data using less information.

This can be useful to visualize dimensional data or to simplify data which can then be used in a supervized learning method. Many of these methods can be adapted for use in classification and regression.

Principal Component Analysis (PCA)
Principal Component Regression (PCR)
Partial Least Squares Regression (PLSR)
Sammon Mapping
Multidimensional Scaling (MDS)
Projection Pursuit
Linear Discriminant Analysis (LDA)
Mixture Discriminant Analysis (MDA)
Quadratic Discriminant Analysis (QDA)
Flexible Discriminant Analysis (FDA)
t-SNE


t-Distributed Stochastic Neighbor Embedding (t-SNE) 1 is a (prize-winning) technique for dimensionality reduction that is particularly well suited for the visualization of high-dimensional datasets. The technique can be implemented via Barnes-Hut approximations, allowing it to be applied on large real-world datasets. We applied it on data sets with up to 30 million examples. The technique and its variants are introduced in the following papers:

L.J.P. van der Maaten. Accelerating t-SNE using Tree-Based Algorithms. Journal of Machine Learning Research 15(Oct):3221-3245, 2014. PDF [Supplemental material]
L.J.P. van der Maaten and G.E. Hinton. Visualizing Non-Metric Similarities in Multiple Maps. Machine Learning 87(1):33-55, 2012. PDF
L.J.P. van der Maaten. Learning a Parametric Embedding by Preserving Local Structure. In Proceedings of the Twelfth International Conference on Artificial Intelligence & Statistics (AI-STATS), JMLR W&CP 5:384-391, 2009. PDF
L.J.P. van der Maaten and G.E. Hinton. Visualizing High-Dimensional Data Using t-SNE. Journal of Machine Learning Research 9(Nov):2579-2605, 2008. PDF [Supplemental material] [Talk]

\subsection{Anomaly Detection}


Motivation and Examples
Algorithms
Evaluation
AD: Examples
Problem motivation 1
Anomaly detection is a reasonably commonly used type of machine learning application
Can be thought of as a solution to an unsupervised learning problem
But, has aspects of supervised learning
What is anomaly detection?
Imagine you're an aircraft engine manufacturer
As engines roll off your assembly line you're doing QA
Measure some features from engines (e.g. heat generated and vibration)
You now have a dataset of x1 to xm (i.e. m engines were tested)
Say we plot that dataset
Next day you have a new engine
An anomaly detection method is used to see if the new engine is anomalous (when compared to the previous engines)
If the new engine looks like this;
Probably OK - looks like the ones we've seen before
But if the engine looks like this
Uh oh! - this looks like an anomalous data-point
More formally
We have a dataset which contains normal (data)
How we ensure they're normal is up to us
In reality it's OK if there are a few which aren't actually normal
Using that dataset as a reference point we can see if other examples are anomalous
How do we do this?
First, using our training dataset we build a model
We can access this model using p(x)
This asks, "What is the probability that example x is normal"
Having built a model
if $latex p(x_{test}) < \epsilon$ --> flag this as an anomaly
if $latex p(x_{test}) \ge \epsilon$ --> this is OK
ε is some threshold probability value which we define, depending on how sure we need/want to be
We expect our model to (graphically) look something like this;
i.e. this would be our model if we had 2D data
Examples 1
Fraud detection
Users have activity associated with them, such as
Length on time on-line
Location of login
Spending frequency
Using this data we can build a model of what normal users' activity is like
What is the probability of "normal" behavior?
Identify unusual users by sending their data through the model
Flag up anything that looks a bit weird
Automatically block cards/transactions
Manufacturing
Already spoke about aircraft engine example
Monitoring computers in data center
If you have many machines in a cluster
Computer features of machine
$latex x_1$ = memory use
$latex x_2$ = number of disk accesses/sec
$latex x_3$ = CPU load
In addition to the measurable features you can also define your own complex features
$latex x_4$ = CPU load/network traffic
If you see an anomalous machine
Maybe about to fail
Look at replacing bits from it

\section{Recomendation System}
ntroduction 2
Two motivations for talking about recommender systems

Important application of ML systems
Many technology companies find recommender systems to be absolutely key
Think about websites (amazon, Ebay, iTunes genius)
Try and recommend new content for you based on passed purchase
Substantial part of Amazon's revenue generation
Improvement in recommender system performance can bring in more income
Kind of a funny problem
In academic learning, recommender systems receives a small amount of attention
But in industry it's an absolutely crucial tool
Talk about the big ideas in machine learning
Not so much a technique, but an idea
As soon, features are really important
There's a big idea in machine learning that for some problems you can learn what a good set of features are
So not select those features but learn them
Recommender systems do this - try and identify the crucial and relevant features
Example - predict movie ratings
You're a company who sells movies
You let users rate movies using a 1-5 star rating
To make the example nicer, allow 0-5 (makes math easier)
You have five movies
And you have four users
Admittedly, business isn't going well, but you're optimistic about the future as a result of your truly outstanding (if limited) inventory

To introduce some notation

$latex n_u$ - Number of users (called $?^{nu}$ occasionally as we can't subscript in superscript)
$latex n_m$ - Number of movies
$latex r(i, j)$ - 1 if user j has rated movie i (i.e. bitmap)
$latex y(i,j)$ - rating given by user j to move i (defined only if $latex r(i,j) = 1$)
So for this example
$latex n_u = 4$
$latex n_m = 5$
Summary of scoring
Alice and Bob gave good ratings to rom coms, but low scores to action films
Carol and Dave game good ratings for action films but low ratings for rom coms
We have the data given above
The problem is as follows
Given $latex r(i,j)$ and $latex y^{(i,j)}$ - go through and try and predict missing values (?s)
Come up with a learning algorithm that can fill in these missing values
KDD 2015 Tutorial: Shlomo Berkovsky and Jill Freyne, Web Personalisation and Recommender Systems

1. Approaches 1


Attribute-based Recommendations

You like action movies, starring Clint Eastwood, you might like "Good, Bad and the Ugly" (Netflix)

Item Hierachy

You bought Printer you will also need ink (Bestbuy)

Association Rules

Content-Based Recommender Collaborative Filtering - Item-Item Similarity

You like Godfather so you will like Scarface (Netflix)

Collaborative Filtering - User-User Similarity

People like you who bought beer also bought diapers (Target)

Social+Interest Graph Based

Your friends like Lady Gaga so you will like Lady Gaga (Facebook, Linkedin)

Model Based

Training SVM, LDA, SVD for implicit features.

2. Challenges
Kaggle Challenge: Million Song Dataset Challenge

3. Articles
How Big Data is used in Recommendation Systems to change our lives
4. Recommendation Interface
4.1 Type of Input
predictions
recommendations
filtering
organic vs explicit presentation
4.2 Type of Output
explicit
implicit
Apriori
https://en.wikipedia.org/wiki/Apriori_algorithm

https://github.com/asaini/Apriori

Item item collaborative filtering
Works when |U| >> |I|

items dont change much
RS: Examples
Google News

http://1.bp.blogspot.com/_7ZYqYi4xigk/TCuWLmXhdjI/AAAAAAAAGVI/umfi5tHpBr0/s1600/Google+News+Redesign+June+30+2010+AM+PT.jpg

RS: Association Rules


Content Based Recommendation
User–User Collaborative Filtering
User - User 1
User user look simular in row space

$p_{u, i} = \overline{r_u} + \frac{\sum_{u' \in N} s(u, u') (r_{u', i} - \overline{r_u'})}{\sum_{u' \in N}|s(u, u')|}&s=2$

http://files.grouplens.org/papers/FnT%20CF%20Recsys%20Survey.pdf ↩

mlclass lecture notes, Recommender Systems ↩





\chapter{Học sâu}

\section{Tài liệu Deep Learning}

Lang thang thế nào lại thấy trang này \href{https://www.microsoft.com/en-us/research/wp-content/uploads/2017/02/DL_Reading_List.pdf}{My Reading List for Deep Learning!} của một anh ở Microsoft. Trong đó, (đương nhiên) có Deep Learning của thánh Yoshua Bengio, có một vụ hay nữa là bài review "Deep Learning" của mấy thánh Yann Lecun, Yoshua Bengio, Geoffrey Hinton trên tạp chí Nature. Ngoài ra còn có nhiều tài liệu hữu ích khác.

\section{Các layer trong deep learning}

\subsection{Sparse Layers}

\href{http://pytorch.org/docs/master/nn.html#embedding}{nn.Embedding} (\href{http://pytorch.org/tutorials/beginner/nlp/word_embeddings_tutorial.html}{hướng dẫn})

★ grep code: \href{https://github.com/Shawn1993/cnn-text-classification-pytorch/blob/master/model.py#L18}{Shawn1993/cnn-text-classification-pytorch}

Đóng vai trò như một lookup table, map một word với dense vector tương ứng

\subsection{Convolution Layers}

\index{convolution}

\href{http://pytorch.org/docs/master/nn.html#conv1d}{nn.Conv1d}, \href{http://pytorch.org/docs/master/nn.html#conv2d}{nn.Conv2d}, \href{http://pytorch.org/docs/master/nn.html#conv3d}{nn.Conv3d})

★ grep code: \href{https://github.com/Shawn1993/cnn-text-classification-pytorch/blob/master/model.py#L20-L24}{Shawn1993/cnn-text-classification-pytorch}, \href{https://github.com/galsang/CNN-sentence-classification-pytorch/blob/master/model.py#L36-L38}{galsang/CNN-sentence-classification-pytorch}

Các tham số trong Convolution Layer

* \textit{kernel\_size} (hay là filter size)

Đối với NLP, $kernel\_size$ thường bằng $region\_size * word\_dim$ (đối với conv1d) hay ($region\_size$, $word\_dim$) đối với conv2d

<small>Quá trình tạo feature map đối với region size bằng 2</small>
![](https://media.giphy.com/media/l2QE2y1UQP7vIgiti/giphy.gif)

* `in_channels`, `out_channels` (là số lượng `feature maps`)

Kênh (channels) là các cách nhìn (view) khác nhau đối với dữ liệu. Ví dụ, trong ảnh thường có 3 kênh RGB (red, green, blue), có thể áp dụng convolution giữa các kênh. Với văn bản cũng có thể có các kênh khác nhau, như khi có các kênh sử dụng các word embedding khác nhau (word2vec, GloVe), hoặc cùng một câu nhưng biểu diễn ở các ngôn ngữ khác nhau.

* `stride`

Định nghĩa bước nhảy của filter.

![](http://d3kbpzbmcynnmx.cloudfront.net/wp-content/uploads/2015/11/Screen-Shot-2015-11-05-at-10.18.08-AM-1024x251.png)

Hình minh họa sự khác biệt giữa các feature map đối với stride=1 và stride=2. Feature map đối với stride = 1 có kích thước là 5, feature map đối với stride = 3 có kích thước là 3. Stride càng lớn thì kích thước của feature map càng nhỏ.

Trong bài báo của Kim 2014, `stride = 1` đối với `nn.conv2d` và `stride = word_dim` đối với `nn.conv1d`

Toàn bộ tham số của mạng CNN trong bài báo Kim 2014,

![](http://d3kbpzbmcynnmx.cloudfront.net/wp-content/uploads/2015/11/Screen-Shot-2015-11-06-at-8.03.47-AM.png)

| Description         | Values          |
|---------------------|-----------------|
| input word vectors  | Google word2vec |
| filter region size  | (3, 4, 5)       |
| feature maps        | 100             |
| activation function | ReLU            |
| pooling             | 1-max pooling   |
| dropout rate        | 0.5             |
| $latex l&amp;s=2$2 norm constraint  | 3               |

Đọc thêm:

* [Lecture 13: Convolutional Neural Networks (for NLP). CS224n-2017](http://web.stanford.edu/class/cs224n/lectures/cs224n-2017-lecture13-CNNs.pdf)
* [DeepNLP-models-Pytorch - 8. Convolutional Neural Networks](https://nbviewer.jupyter.org/github/DSKSD/DeepNLP-models-Pytorch/blob/master/notebooks/08.CNN-for-Text-Classification.ipynb)
* [A Sensitivity Analysis of (and Practitioners’ Guide to) Convolutional Neural Networks for Sentence Classification. Zhang 2015](https://arxiv.org/pdf/1510.03820.pdf)

**BTS**

22/11/2017 - Phải nói quyển này hơi nặng so với mình. Nhưng thôi cứ cố gắng vậy.
24/11/2017 - Từ hôm nay, mỗi ngày sẽ ghi chú một phần (rất rất nhỏ) về Deep Learning [tại đây](https://docs.google.com/document/d/1KxDrw5s6uYHNLda7t0rhp0RM_TlUGxydQ-Qi1JOPFr8/edit?usp=sharing)

[^1]: [Understanding Convolutional Neural Networks for NLP](http://www.wildml.com/2015/11/understanding-convolutional-neural-networks-for-nlp)
[^2]: [http://pytorch.org/docs/master/nn.html](http://pytorch.org/docs/master/nn.html)
\chapter{Xử lý ngôn ngữ tự nhiên}

**05/01/2018**: "điên đầu" với Sphinx và HTK

HTK thì đã bỏ rồi vì quá lằng nhằng.

Sphinx thì setup được đối với dữ liệu nhỏ rồi. Nhưng không thể làm nó hoạt động với dữ liệu của VIVOS. Chắc hôm nay sẽ switch sang Kaldi vậy.

**26/12/2017**: Automatic Speech Recognition 100

Sau mấy ngày "vật lộn" với code base của Truong Do, thì cuối cùng cũng produce voice được. Cảm giác rất thú vị. Quyết định làm luôn ASR. Tìm mãi chẳng thấy code base đâu (chắc do lĩnh vực mới nên không có kinh nghiệm). May quá lại có bạn frankydotid có project về nhận diện tiếng Indonesia ở [github](https://github.com/frankydotid/Indonesian-Speech-Recognition). Trong README.md bạn đấy bảo là phải cần đọc HTK Book. Tốt quá đang cần cơ bản.

**20/12/2017**: Text to speech 100

Cảm ơn project rất hay của [bạn Truong Do ở vais](https://vais.vn/vi/tai-ve/hts_for_vietnamese/), nếu không có project này chắc mình phải mất rất nhiều thời gian mới có được phiên bản text to speech đầu tiên.

Tóm lại thì việc sinh ra tiếng nói từ text gồm 4 giai đoạn

1. Sinh ra features từ file wav sử dụng tool sptk
2. Tạo một lab, trong đó có dữ liệu huấn luyện (những đặc trưng của âm thanh được trích xuất từ bước 1), text đầu vào
3. Sử dụng htk để train dữ liệu từ thư mục lab, đầu ra là một model
4. Sử dụng model để sinh ra output với text đầu vào, dùng hts_engine để decode, kết quả được wav files.

Phù. 4 bước đơn giản thế này thôi mà không biết. Lục cả internet ra mãi chẳng hiểu, cuối cùng file phân tích file `train.sh` của bạn Truong Do mới hiểu. Ahihi

**24/11/2017**: Nhánh của Trí tuệ nhân tạo mà hiện tại mình đang theo đuổi. Project hiện tại là [underthesea](https://github.com/magizbox/underthesea). Với mục đích là xây dựng một toolkit cho xử lý ngôn ngữ tự nhiên tiếng Việt.


\chapter{Nhận dạng tiếng nói}

\diary{26/12/2017 - Automatic Speech Recognition 100. Sau mấy ngày "vật lộn" với code base của Truong Do, thì cuối cùng cũng produce voice được. Cảm giác rất thú vị. Quyết định làm luôn ASR. Tìm mãi chẳng thấy code base đâu (chắc do lĩnh vực mới nên không có kinh nghiệm). May quá lại có bạn frankydotid có project về nhận diện tiếng Indonesia ở \href{https://github.com/frankydotid/Indonesian-Speech-Recognition}{github}. Trong README.md bạn đấy bảo là phải cần đọc HTK Book. Tốt quá đang cần cơ bản.}

\diary{19/01/2018: Hôm nay thực sự quá mệt với bạn Kaldi. Mãi không thể decode với các thuộc tính LDA-MLLT được? Hỏi mãi ở kaldi-help mà không có reply}

\dinary{05/01/2018 - "điên đầu" với Sphinx và HTK. HTK thì đã bỏ rồi vì quá lằng nhằng. Sphinx thì setup được đối với dữ liệu nhỏ rồi. Nhưng không thể làm nó hoạt động với dữ liệu của VIVOS. Chắc hôm nay sẽ switch sang Kaldi vậy.}


Trong hệ thống nhận dạng tiếng nói, tín hiệu âm thanh được thu thập như những mẫu phù hợp cho quá trình xử lý của máy tính và được đưa vào quá trình nhận diện. Đầu ra của hệ thống là một câu phụ đề của câu nói.

Nhận dạng tiếng nói là một nhiệm vụ phức tạp và hệ thống tốt nhất trong nhận dạng tiếng nói rất phức tạp. Có rất nhiều cách tiếp cận cho mỗi thành phần. Trong phần này, người viết chỉ muốn đưa ra một cái nhìn tổng thể về nhận dạng tiếng nói, các khó khăn chính, các thành phần cơ bản, chức năng và tương tác của chúng trong một hệ thống nhận dạng tiếng nói.

\noindent Các thành phần của hệ thống nhận dạng tiếng nói

\includegraphics[width=10cm]{data_science/speech_recognition/asr_components}

Trong bước thứ nhất, trích rút thông tin \textit{Feature Extraction}, các mẫu tín hiệu được tham số hóa. Mục tiêu là trích xuất ra một tập các tham số (đặc trưng) từ tín hiệu có nhiều thông tin hữu ích nhất cho quá trình phân loại.  Các đặc trưng chính được trích xuất với điều kiện \textit{thích nghi} với các sự thay đổi của âm thanh và \textit{nhạy cảm} với các nội dung ngôn ngữ.

\definition{mô hình âm học}{Trong module phân loại, các vector đặc trưng được ánh xạ với các pattern, được gọi là \textbf{mô hình âm học} (acoustic model). Mô hình học thường là HMM được train với toàn bộ từ, hay âm như là một đơn vị ngôn ngữ.}

\definition{Từ điển phát âm}{
Từ điển phát âm (pronunciation dictionary) định nghĩa cách kết hợp âm cho các ký tự. Nó có thể chứa cách phát âm khác nhau cho cùng một từ. Bảng 1 hiển thị chính xác một từ điển. Từ (graphme) ở cột bên trái ứng với cách phát âm (các âm) ở cột bên phải (các ký tự âm trong bảng được dùng phổ biến đối với tiếng Anh)
}

\begin{tabular}{ | l | l | }
  \hline
  word & pronunciation \\ \hline
  INCREASE & ih n k r iy s \\ \hline
  INCREASED & ih n k r iy s t \\ \hline
  INCREASES & ih n k r iy s ah z \\ \hline
  INCREASING & ih n k r iy s ih ng  \\ \hline
  INCREASINGLY & ih n k r iy s ih ng l iy \\ \hline
  INCREDIBLE & ih n k r eh d ah b ah l \\ \hline
\end{tabular}

\textit{Mô hình ngôn ngữ} (language model) chứa các thông tin về cú pháp. Mục tiêu để dự đoán khả năng một từ xuất hiện sau các từ khác trong một ngôn ngữ. Nói cách khác, xác xuất để một từ k xảy ra sau khi k-1 từ sau đó được định nghĩa bởi $P(w_k | w_{k-1}, w_{k-2}, ..., w_1)$

\textbf{Mô hình hóa sub-word với HMMs}

Trong các hệ thống ASR, HMMs được dùng để biểu diễn các đơn vị dưới từ (ví dụ như âm). Với ngôn ngữ, thông thường có 40 âm. Số lượng âm phụ thuộc vào từ điển được sử dụng. Số lượng âm phụ thuộc vào từ điển được sử dụng. Mô hình từ có thể được xây dựng bằng cách kết hợp các mô hình dưới từ.

Trong thực tế, khi nhận dạng một âm phụ thuộc rất nhiều vào các âm bên cạnh. Do đó, mô hình âm phụ thuộc ngữ cảnh (*context dependence*) được sử dụng rất phổ biến. Mô hình *biphone* chú ý đến âm bên trái hoặc âm bên phải, mô hình *triphone* chú ý đến cả hai phía, với một âm, các mô hình khác nhau được sử dụng trong ngữ cảnh khác nhau. Hình dưới thể hiện các mô hình monophone, biphone và triphone của từ *bat* (b ae t)

![](http://www.igi.tugraz.at/lehre/CI/SS08/tutorials/ASR/img10.gif)

### Quá trình huấn luyện

**Huấn luyện các mô hình monophone**

Một mô hình monophone là một mô hình âm học, trong đó không chứa thông tin ngữ cảnh về các âm trước và sau. Nó được sử dụng như thành phần cơ bản cho các mô hình triphone - mô hình sử dụng những thông tin về ngữ cảnh.

Việc huấn luyện sử dụng framework Gaussian Mixture Model/Hidden Markov Model.

**Dóng hàng âm thanh trong mô hình âm học**

Các tham số trong mô hình âm học được tính toán trong quá trình huấn luyện; tuy nhiên, quá trình này có thể được tối ưu hóa bởi việc lặp lại quá trình huấn luyện và dòng hàng. Còn lại là huấn luyện Viterbi (liên quan đến phương pháp này, nhưng dùng nhiều khối lượng tính toán hơn là thuật toán Forward-Backward và Expectation Maximization). Bằng cách dóng hàng âm thanh - phụ đề với mô hình âm học hiện tại, các thuật toán huấn luyện có thể sử dụng kết quả này để cải thiện và hiệu chỉnh tham số của mô hình. Do đó, mỗi quá trình huấn luyện sẽ theo bởi một bước dóng hàng trong đó âm thanh và văn bản được dóng hàng lại.

**Huấn luyện các mô hình triphone**

Trong khi các mô hình monophone đơn giản biểu diễn các đặc trưng âm thanh như một đơn âm, trong khi các âm vị sẽ thay đổi đáng kể phụ thuộc vào ngữ cảnh. Mô hình triphone thể hiện một âm trong ngữ cảnh với hai âm bên cạnh.

Đến đây, một vấn đề là không phải tất cả các đơn vị triphone được thể hiện trong dữ liệu huấn luyên. Có tất cả (# of phonemes)^3 triphone, nhưng chỉ có một tập thực sự tồn tại trong dữ liệu. Hơn nữa, các đơn vị xảy ra nhiều lần trong dữ liệu đưa ra kết quả thống kê tốt hơn trong dữ liệu. Một nhóm cây quyết định phân chia các triphones vào các nhóm, mục đích giảm thiểu tham số và đưa ra quyết định tốt hơn.

**Dóng hàng các mô hình âm học và huấn luyện lại các mô hình triphone**

Lặp lại các bước dòng hàng âm thanh và huấn luyện các mô hình triphone với các thuật toán huấn luyện để hiệu chỉnh mô hình. Các phương pháp phổ biến là delta+delta-delta, LDA-MLLT và SAT. Các giải thuật dóng hàng bao gồm dóng hàng cho từng người nói và FMLLR.

\textbf{Các thuật toán huấn luyện}

Huấn luyện delta+delta-delta tính các đặc trưng delta và double-delta, hay các hệ số động, để thêm vào các đặc trưng MFCC. Delta và delta-delta là các đặc trưng số học, tính các đạo hàm bậc 1 và 2 của tín hiệu. Do đó, phép tính toán này thường được thực hiện trên một window của các đặc trưng vector. Trong khi một window của hai đặc trưng vector có thể hiệu quả, nó là các xấp xỉ thô (giống như delta-diffrence là một xấp xỉ thô của đạo hàm). Đặc trưng delta được tính toán trong các window của các đặc trưng cơ bản, trong khi delta-delta được tính toán trong các window của đặc trựng delta.

LDA-MLLT viết tắt của Linear Discriminant Analysis - Maximum Likelihood Linear Transform. Linear Discriminant Analysis lấy các đặc trưng vector và xây dựng các trạng thái HMM, nhưng giảm thiểu không gian vector. Maximum Likelihood Linear Transfrom lấy các đặc trưng được giảm từ LDA, và thực hiện các biến đổi đối với từng người nói. MLLT sau đó thực hiện một bước chuẩn hóa, để giảm sự khác biệt giữa các người nói.

SAT viết tắt của Speaker Adaptive Training. SAT cũng thực hiện các chuẩn hóa đối với người nói bằng cách thực hiện biến đổi trên mỗi người nói. Kết quả của quá trình này đồng nhất và chuẩn hóa hơn, cho phép mô hình có thể sử dụng những tham số này để giảm thiểu sự biến đổi của âm, đối với từng người nói hoặc môi trường thu.

\textbf{Các thuật toán dóng hàng}

Thuật toán dòng hàng luôn luôn cố định, trong đó các kịch bản chấp nhận các loại đầu vào âm học khác nhau. Dòng hàng đối với từng người nói, sẽ tách biệt thông tin giữa các người nói trong quá trình dóng hàng.

fMLLR viết tắt của Feature Space Maximum Likelihood Linear Regression. Sau quá trình huấn luyện SAT, các mô hình âm học không huấn luyện trên các đặc trưng ban đầu, mà đối với các đặc trưng chuẩn hóa theo người nói. Với quá trình dóng hàng, xóa bỏ sự khác biệt giữa người nói (bằng cách nghịch đạo ma trận fMLLR), sau đó loại bỏ nó khỏi mô hình *bằng cách nhân ma trận nghịch đảo với đặc trưng vector). Mô hình âm học quasi-speaker-independent có thể sử dụng trong quá trình dóng hàng.

\textbf{Dóng hàng (Forced Alignment)}

Hệ thống nhận dạng tiếng nói sử dụng một máy tìm kiếm bên cạnh mô hình âm học và ngôn ngữ trong đó chứa tập các từ, âm và tập dữ liệu để đối chiếu với dữ liệu âm thanh cho câu nói. Máy tìm kiếm này sử dụng các đặc trưng được trích xuất bởi dữ liệu âm thanh để xác định sự xuất hiện của từ, âm và đưa ra kết quả.

![](https://www.isip.piconepress.com/projects/speech/software/tutorials/production/fundamentals/v1.0/section_04/images/srec_s04_04_p01.jpg)

Quá trình dòng hàng cũng tương tự như vậy, nhưng khác ở một điểm quan trong. Thay vì đưa vào tập các từ có thể để tìm kiếm, máy tìm kiếm đưa vào đoạn phụ đề tương ứng với câu nói. Hệ thống sau đó dóng hàng dữ liệu văn bản với dữ liệu âm thanh, xác định đoạn nào trong âm thanh tương ứng với từ cụ thể nào trong dữ liệu văn bản.

![](https://www.isip.piconepress.com/projects/speech/software/tutorials/production/fundamentals/v1.0/section_04/images/fallign_s04_04_p01.jpg)

Dóng hàng có thể sử dụng để dóng âm trong dữ liệu với bản với dữ liệu âm thanh, giống như hình dưới đây, các âm được xác định trong từng đoạn của âm thanh.

![](https://www.isip.piconepress.com/projects/speech/software/tutorials/production/fundamentals/v1.0/section_04/images/fallign2_s04_04_p01.jpg)

\section{Hidden Markov Model}

Hidden Markov Model (HMM) là mô hình trọng số với các trọng số ở cung, chỉ khả năng xuất hiện của cung.

*Một trong những ứng dụng của HMM, là phán đoán chuỗi các trạng thái thay đổi, dựa vào chuỗi các quan sát*

Các trọng số trong trạng thái gọi là observation likelihood, các trọng số ở cung gọi là transition likelihood.

Sau đây là một ví dụ:

\begin{itemize}
  \item Thời tiết trong một ngày có thể là NÓNG hoặc LẠNH
  \item Khi trời NÓNG, 20\% bạn dùng 1 viên đá, 40\% bạn dùng 2 viên, 40\% cho 3 viên.
  \item Khi trời LẠNH, 50\% bạn dùng 1 viên, 40\% bạn dùng 2, 10\% bạn dùng 3. Đây là các khả năng trong quá trình quan sát (observation likelihood)
  \item Khi trời NÓNG, 30\% nó sẽ chuyển sang LẠNH, 70\% giữ nguyên. Khi trời LẠNH, 40\% nó sẽ chuyển sang NÓNG, 60\% giữ nguyên. Đây là khả năng dịch chuyển (trainsition likelihood)
\end{itemize}

* ![https://qph.ec.quoracdn.net/main-qimg-a6744f9e17e59f3729d6fef02d54391b.webp](https://qph.ec.quoracdn.net/main-qimg-a6744f9e17e59f3729d6fef02d54391b.webp)

Giờ, giả sử chung ta quan sát trong 3 ngày, bạn dùng 1,2,3 viên đá. Thời tiết có khả năng diễn ra như thế nào?

Đến đây chúng ta dùng thuật toán Viterbi. Về cơ bản, nó là dynamic programming với hai chiều $[state, position\_in\_sequence]$

Gọi S là trạng thái hiện tại {HOT, COLD} trong quan sát i, S' là trạng thái trước đó, và A là lượng đá tiêu thụ {1, 2, 3} trong quan sát i

$$Viterbi[S,i] = Viterbi[S', i-1] * p(S|S') * p(A|S)$$
$$V[S,i] = V[S',i-1] * transition\_likelihood * observation\_likelihood$$

HMM được sử dụng trong các hệ thống thỏa mãn

\begin{enumerate}
  \item Có hữu hạn các trạng thái nội tại (internal state), là nguyên nhân của các sự kiện (external events) (các quan sát)
  \item Trạng thái nội tại không quan sát được (hidden)
  \item Trạng thái hiện tại chỉ phụ thuộc vào trạng thái trước đó (qúa trình Markov)
\end{enumerate}

Wow! George nhanh chóng liên hệ vụ của anh đấy với mô hình HMM. George nhận ra rằng CCTV footage từ các cập có thể coi như là chuỗi quan sát được, anh đấy có thể dùng mô hình và sử dụng nó để phát hiện hành vị ẩn mà Bob và William hoạt động.

\textbf{3 vấn đề cơ bản} được Jack Ferguson giới thiệu trong những năm 1960

\begin{enumerate}
  \item (Likelihood): Cho một HMM $\lambda = (A, B)$ và một chuỗi quan sát $O$, xác định likelihood $P(O|\lambda)$
  \item (Decoding): Cho một chuỗi quan sát $O$, và một HMM $\lambda = (A,B)$, xác định chuỗi ẩn $Q$ tốt nhất
  \item (Learning): Cho một chuỗi quan sát $O$, một tập các trạng thái trong HMM, học các tham số $A$ và $B$
\end{enumerate}

\section{Likelihood Computation}

Vấn đề đầu tiên là tính xác suất xảy ra của một chuỗi quan sát. Ví dụ, trong bài toán ăn đá ở hình 9.3, xác suất xảy ra chuỗi *3 1 3* là bao nhiêu?

**Tính toán Likelihood**: Chuỗi một HMM $\lambda = (A, B)$, và mỗi chuỗi quan sát $O$, xác định likelihood $P(O|\lambda)$

Thuật toán Forward, nếu sử dụng Bayes rule, để tính likelihood, cần khối lượng tính toán $N^T$ với N là số trạng thái có thể có và T là chiều dài chuỗi quan sát. Ví dụ trong bài toán gán nhãn có N=10 nhãn, chiều dài của chuỗi trung bình là 28, thì cần $10^{28}$ bước tính toán. Một giải thuật với hiệu quả $O(N^2T)$ được đề xuất với tên gọi \textbf{forward algorithm}

Tài liệu tham khảo

* http://www.igi.tugraz.at/lehre/CI/SS08/tutorials/ASR/node1.html
* https://www.isip.piconepress.com/projects/speech/software/tutorials/production/fundamentals/v1.0/section_04/s04_04_p01.html
* http://www.igi.tugraz.at/lehre/CI/SS08/tutorials/ASR/node1.html
* https://www.isip.piconepress.com/projects/speech/software/tutorials/production/fundamentals/v1.0/section_04/s04_04_p01.html
* https://www.quora.com/What-is-a-simple-explanation-of-the-Hidden-Markov-Model-algorithm

\chapter{Tổng hợp tiếng nói}

\diary{20/12/2017: Text to speech 100. Cảm ơn project rất hay của \href{https://vais.vn/vi/tai-ve/hts_for_vietnamese/}{bạn Truong Do ở vais}, nếu không có project này chắc mình phải mất rất nhiều thời gian mới có được phiên bản text to speech đầu tiên.}

Tóm lại thì việc sinh ra tiếng nói từ text gồm 4 giai đoạn

\begin{enumerate}
  \item Sinh ra features từ file wav sử dụng tool sptk
  \item Tạo một lab, trong đó có dữ liệu huấn luyện (những đặc trưng của âm thanh được trích xuất từ bước 1), text đầu vào
  \item Sử dụng htk để train dữ liệu từ thư mục lab, đầu ra là một model
  \item Sử dụng model để sinh ra output với text đầu vào, dùng hts\_engine để decode, kết quả được wav files.
\end{enumerate}

Phù. 4 bước đơn giản thế này thôi mà không biết. Lục cả internet ra mãi chẳng hiểu, cuối cùng file phân tích file `train.sh` của bạn Truong Do mới hiểu. Ahihi
\chapter{Phân loại văn bản}

<h3>Naive Bayes Classifier</h3>
Tham khảo thư viện <a href="http://scikit-learn.org/stable/modules/naive_bayes.html" target="_blank" rel="noopener">Scikit-learn</a>

Xét bài toán classification với C classes 1,2,…,C. Tính xác suất để 1 điểm dữ liệu rơi vào class C ta có công thức: $latex P(\frac{c}{x})$. Tức tính xác suất để đầu ra là class C biết rằng đầu vào là vector x. Việc xác định class của điểm dữ liệu đó bằng cách chọn ra class có xác suất cao nhất:<p style="text-align:center;">
c = argmax($latex P(\frac{c}{x})$) với c ∈ {1,…,C}</p>
Sử dụng quy tắc Bayes:<p style="text-align:center;">
c = argmax($latex P(\frac{c}{x})$) = argmax($latex P(\frac{P(\frac{c}{x})P(x)}{P(x)}$) = argmax($latex P(\frac{P(\frac{c}{x})}{P(c)}$))</p>

<h4>Các phân phối thường dùng</h4>
<strong>Gaussian Naive Bayes</strong>
Mô hình này được sử dụng chủ yếu trong loại dữ liệu mà các thành phần là các biến liên tục.
<strong>Multinomial Naive Bayes</strong>
Mô hình này chủ yếu được sử dụng trong phân loại văn bản mà feature vectors được tính bằng Bags of Words. Lúc này, mỗi văn bản được biểu diễn bởi một vector có độ dài d chính là số từ trong từ điển. Giá trị của thành phần thứ i trong mỗi vector chính là số lần từ thứ i xuất hiện trong văn bản đó.
Khi đó, $latex P(\frac{x_i}{c})$ tỉ lệ với tần suất từ thứ i xuất hiện trong các văn bản của class c:<p style="text-align:center;"> $latex P(\frac{x_i}{c})$ = $latex \frac{Nx_i}{Nc}$</p>
<p style="padding-left:70px;">Trong đó:</p>
<p style="padding-left:90px;">$latex Nx_i$ là tổng số lần từ thứ i xuất hiện trong các văn bản của class c, nó được tính là tổng của tất cả các thành phần thứ i của các feature vectors ứng với class c.</p>
<p style="padding-left:90px;">$latex Nc$ là tổng số từ (kể cả lặp) xuất hiện trong class c. Hay bằng tổng độ dài của toàn bộ các văn bản thuộc vào class c.</p>
Nếu có một từ mới chưa bao giờ xuất hiện trong class c thì biểu thức trên sẽ bằng 0, điều này dẫn đến vế phải của c bằng 0.
<strong>Bernoulli Naive Bayes</strong>
Mô hình này được áp dụng cho các loại dữ liệu mà mỗi thành phần là một giá trị binary. Ví dụ: cũng với loại văn bản nhưng thay vì đếm tổng số lần xuất hiện của 1 từ trong văn bản, ta chỉ cần quan tâm từ đó có xuất hiện hay không.
Khi đó: $latex P(\frac{x_i}{c})$ = $latex P(\frac{i}{c})x_i$ + (1 − $latex P(\frac{i}{c})$(1 − $latex x_i $))
Với $latex P(\frac{i}{c})$ là xác suất từ thứ i xuất hiện trong các văn bản của class c.
\chapter{Pytorch}

**Bí kíp luyện công**

(cập nhật 08/12/2017): cảm giác [talk](http://videolectures.net/deeplearning2017_chintala_torch/) của anh Soumith Chintala (Facebook, research editor) ở Deep Learning Summer School 2017 khá thú vị.

Sau khi nghe bài này thì hâm mộ luôn anh Soumith Chintala, tìm loạt bài anh trình bày luôn

* [PyTorch: Fast Differentiable Dynamic Graphs in Python with a Tensor JIT](https://www.youtube.com/watch?v=DBVLcgq2Eg0&amp;t=2s), Strange Loop Sep 2017
* [Keynote: PyTorch: Framework for fast, dynamic deep learning and scientific computing](https://www.youtube.com/watch?v=LAMwEJZqesU&amp;t=66s), EuroSciPy Aug 2017


## So sánh giữa Tensorflow và Pytorch?

Có 2 điều cần phải nói khi mọi người luôn luôn so sánh giữa Tensorflow và Pytorch. (1) Tensorflow khiến mọi người "không thoải mái" (2) Pytorch thực sự là một đối thủ trên bàn cân. Một trong những câu trả lời hay nhất mình tìm được là của anh Hieu Pham (Google Brain) [trả lời trên quora (25/11/2017)](https://www.quora.com/What-are-your-reviews-between-PyTorch-and-TensorFlow/answer/Hieu-Pham-20?srid=5O2u). Điều quan trọng nhất trong câu trả lời này là *"Dùng Pytorch rất sướng cho nghiên cứu, nhưng scale lên mức business thì Tensorflow là lựa chọn tốt hơn"*

## Behind The Scene

(15/11/2017) Hôm nay bắt đầu thử nghiệm pytorch với project thần thánh classification sử dụng cnn https://github.com/Shawn1993/cnn-text-classification-pytorch

Cảm giác đầu tiên là make it run khá đơn giản

```
conda create -n test-torch python=3.5
pip install http://download.pytorch.org/whl/cu80/torch-0.2.0.post3-cp35-cp35m-manylinux1_x86_64.whl
pip install torchvision
pip install torchtext
```

Thế là `main.py` chạy! Hay thật. Còn phải vọc để bạn này chạy với CUDA nữa.

**Cài đặt CUDA trong ubuntu 16.04**

Kiểm tra VGA

```
$ lspci | grep VGA
01:00.0 VGA compatible controller: NVIDIA Corporation GM204 [GeForce GTX 980] (rev a1)
```

Kiểm tra CUDA đã cài đặt trong Ubuntu [^1]

```
$ nvcc --version
nvcc: NVIDIA (R) Cuda compiler driver
Copyright (c) 2005-2016 NVIDIA Corporation
Built on Sun_Sep__4_22:14:01_CDT_2016
Cuda compilation tools, release 8.0, V8.0.44
```

Kiểm tra pytorch chạy với cuda `test_cuda.py`

```python
import torch
print("Cuda:", torch.cuda.is_available())
```

```
$ python test_cuda.py
CUDA: True
```

Chỉ cần cài đặt thành công CUDA là pytorch tự work luôn. Ngon thật!

*Ngày X*

Chẳng hiểu sao update system kiểu nào mà hôm nay lại không sử dụng được CUDA `torch.cuda.is_available() = False`. Sau khi dùng lệnh `torch.Tensor().cuda()` thì gặp lỗi

```
AssertionError:
The NVIDIA driver on your system is too old (found version 8000).
Please update your GPU driver by downloading and installing a new
version from the URL: http://www.nvidia.com/Download/index.aspx
Alternatively, go to: https://pytorch.org/binaries to install
a PyTorch version that has been compiled with your version
of the CUDA driver.
```

Kiểm tra lại thì mình đang dùng nvidia-361, làm thử theo [link này](http://www.linuxandubuntu.com/home/how-to-install-latest-nvidia-drivers-in-linux) để update NVIDIA, chưa biết kết quả ra sao?

May quá, sau khi update lên nvida-387 là ok. Haha

**Ngày 2**

Hôm qua đã bắt đầu implement một nn với pytorch rồi. Hướng dẫn ở [Deep Learning with PyTorch: A 60 Minute Blitz](http://pytorch.org/tutorials/beginner/deep_learning_60min_blitz.html) hữu ích phết.

Hướng dẫn implement các mạng neural với pytorch rất hay tại [PyTorch-Tutorial](https://github.com/MorvanZhou/PyTorch-Tutorial)

(lượm lặt) Trang này [Awesome-pytorch-list](https://github.com/bharathgs/Awesome-pytorch-list) chứa rất nhiều link hay về pytorch như tập hợp các thư viện liên quan, các hướng dẫn và ví dụ sau đó là các cài đặt của các paper sử dụng pytorch.

(lượm lặt) Loạt video hướng dẫn pytorch [PyTorchZeroToAll](https://www.youtube.com/watch?v=SKq-pmkekTk&amp;list=PLlMkM4tgfjnJ3I-dbhO9JTw7gNty6o_2m) của tác giả Sung Kim trên youtube.

Bước tiếp theo là visualize loss và graph trong tensorboard, sử dụng [tensorboard_logger](https://github.com/TeamHG-Memex/tensorboard_logger) khá hay.

```
pip install tensorboard_logger
pip install tensorboard
```

Chạy tensorboard server

```
tensorboard --log-dir=runs
```

**Ngày 3**: Vấn đề kỹ thuật

Hôm qua cố gắng implement một phần thuật toán CNN cho bài toán phân lớp văn bản. Vấn đề đầu tiên là biểu diển sentence thế nào. Cảm giác load word vector vào khá chậm. Mà thằng tách từ của underthesea cũng chậm kinh khủng.

Một vài link tham khảo về bài toán CNN: [Implementing a CNN for Text Classification in TensorFlow](http://www.wildml.com/2015/12/implementing-a-cnn-for-text-classification-in-tensorflow/), [Text classification using CNN : Example](https://agarnitin86.github.io/blog/2016/12/23/text-classification-cnn)

[^1]: https://askubuntu.com/questions/799184/how-can-i-install-cuda-on-ubuntu-16-04


\chapter{Big Data}

View online \href{http://magizbox.com/training/bigdata/site/}{http://magizbox.com/training/bigdata/site/}

Big Data Q&A
1. What is "Big Data"? 1
https://www.youtube.com/watch?v=TzxmjbL-i4Y

2. How big is big data? 2


3. How much data is "Big Data"? 3


4. What are characteristics of "Big Data"? 4


5. What is big data ecosystem? 5


6. What is big data landscape 6


7. What are benefits of big data? 7


https://www.youtube.com/watch?v=TzxmjbL-i4Y ↩ ↩

http://scoop.intel.com/what-happens-in-an-internet-minute/ ↩ ↩

http://www.quora.com/How-much-data-is-Big-Data ↩ ↩

https://en.wikipedia.org/wiki/Big_data#Characteristics ↩ ↩

http://www.clearpeaks.com/blog/big-data/big-data-ecosystem-spark-and-tableau ↩ ↩

https://vladimerbotsvadze.wordpress.com/2015/01/28/the-big-data-landscape-technology-businessintelligence-analytics/ ↩ ↩

http://blog.galaxyweblinks.com/big-data-with-bigger-benefits/ ↩ ↩

\section{Distribution Storage}

\subsection{HDFS}

The Hadoop Distributed File System (HDFS) — a subproject of the Apache Hadoop project—is a distributed, highly fault-tolerant file system designed to run on low-cost commodity hardware. HDFS provides high-throughput access to application data and is suitable for applications with large data sets. This article explores the primary features of HDFS and provides a high-level view of the HDFS architecture.
: sequenceiq/hadoop-docker

Big Data Stack: HDFS, Kibana, ElasticSearch, Neo4J, Apache Spark

\subsection{HBase}

Apache HBase™ is the Hadoop database, a distributed, scalable, big data store. Download Apache HBase™ Click here to download Apache HBase™.


1. When Would I Use Apache HBase? 1
HBase isn’t suitable for every problem.

First, make sure you have enough data. If you have hundreds of millions or billions of rows, then HBase is a good candidate. If you only have a few thousand/million rows, then using a traditional RDBMS might be a better choice due to the fact that all of your data might wind up on a single node (or two) and the rest of the cluster may be sitting idle.

Second, make sure you can live without all the extra features that an RDBMS provides (e.g., typed columns, secondary indexes, transactions, advanced query languages, etc.) An application built against an RDBMS cannot be "ported" to HBase by simply changing a JDBC driver, for example. Consider moving from an RDBMS to HBase as a complete redesign as opposed to a port.

Third, make sure you have enough hardware. Even HDFS doesn’t do well with anything less than 5 DataNodes (due to things such as HDFS block replication which has a default of 3), plus a NameNode.

HBase can run quite well stand-alone on a laptop - but this should be considered a development configuration only.

2. Features 2
Linear and modular scalability.
Strictly consistent reads and writes.
Automatic and configurable sharding of tables
Automatic failover support between RegionServers.
Convenient base classes for backing Hadoop MapReduce jobs with Apache HBase tables.
Easy to use Java API for client access.
Block cache and Bloom Filters for real-time queries.
Query predicate push down via server side Filters
Thrift gateway and a REST-ful Web service that supports XML, Protobuf, and binary data encoding options
Extensible jruby-based (JIRB) shell
Support for exporting metrics via the Hadoop metrics subsystem to files or Ganglia; or via JMX
3. Architecture


HBase Shell
[code lang="shell"]

list all table
list [/code]

Up & Running
1. Download
HBase 0.94.27 (HBase 0.98 won't work)

[code lang="shell"] wget https://www.apache.org/dist/hbase/hbase-0.94.27/hbase-0.94.27.tar.gz tar -xzf hbase-0.94.27.tar.gz [/code]

2. Setup
1. edit $HBASE_ROOT/conf/hbase-site.xml and add

[code lang="xml"] hbase.rootdir file:///full/path/to/where/the/data/should/be/stored hbase.cluster.distributed false [/code]

3. Verify
Go to http://localhost:60010 to see if HBase is running.

When Should I Use HBase? ↩
HBase ↩
Config HBase Remote
1. Change /etc/hosts
[code] 127.0.0.1 [username] [server_ip] hbase.io [/code]

Example

[code] 127.0.0.1 crawler 192.168.0.151 hbase.io [/code]

2. Change hostname
[code] hostname hbase.io [/code]

3. Change region servers
Edit $HBASE_ROOT/conf/regionservers

[code] hbase.io [/code]

4. Change $HABSE_ROOT/conf/hbase-site.xml
[code lang="xml" title="hbase-site.xml"] <?xml-stylesheet type="text/xsl" href="configuration.xsl"?> hbase.rootdir file:///home/username/Downloads/hbase/data hbase.cluster.distributed false hbase.zookeeper.quorum hbase.io zookeeper.znode.parent /hbase-unsecure hbase.rpc.timeout 2592000000 [/code]

Docker
HBase 0.94

Image: https://github.com/Banno/docker-hbase-standalone

[code] docker run -d -p 2181:2181 -p 60000:60000 -p 60010:60010 -p 60020:60020 -p 60030:60030 banno/hbase-standalone [/code]

Compose

[code] hbase.vmware: build: ./docker-hbase-standalone/. command: "/opt/hbase/hbase-0.94.15-cdh4.7.0/bin/hbase master start" hostname: hbase.vmware ports: - 2181:2181 - 60000:60000 - 60010:60010 - 60020:60020 - 60030:60030 volumes: - ./docker-hbase-standalone/hbase-0.94.15-cdh4.7.0:/opt/hbase/hbase-0.94.15-cdh4.7.0 - ./data/hbase:/tmp/hbase-root/hbase /code]

\section{Distribution Computing}

\subsection{Apache Spark}




Apache Spark is an open-source cluster computing framework originally developed in the AMPLab at UC Berkeley. In contrast to Hadoop's two-stage disk-based MapReduce paradigm, Spark's in-memory primitives provide performance up to 100 times faster for certain applications. By allowing user programs to load data into a cluster's memory and query it repeatedly, Spark is well-suited to machine learning algorithms.
Installation
Requirements: Hadoop, YARN

Install Hadoop

Insatll YARN

Install Java

Verification
Tutorial
From Pandas to Apache Spark’s DataFrame

Big Data Stack: HDFS, Kibana, ElasticSearch, Neo4J, Apache Spark

Apache Spark: Tutorials
Beginners Guide: Apache Spark Machine Learning with Large Data


Spark and Spark Streaming Unit Testing Recipes for Running Spark Streaming Applications in Production- Databricks

Spark Streaming


Spark and Spark Streaming Unit Testing Recipes for Running Spark Streaming Applications in Production- Databricks

\section{Components}

\subsection{Ambari}

The Apache Ambari project is aimed at making Hadoop management simpler by developing software for provisioning, managing, and monitoring Apache Hadoop clusters. Ambari provides an intuitive, easy-to-use Hadoop management web UI backed by its RESTful APIs.

Ambari enables System Administrators to:

Provision a Hadoop Cluster

Ambari provides a step-by-step wizard for installing Hadoop services across any number of hosts.
Ambari handles configuration of Hadoop services for the cluster.
Manage a Hadoop Cluster

Ambari provides central management for starting, stopping, and reconfiguring Hadoop services across the entire cluster.
Monitor a Hadoop Cluster

Ambari provides a dashboard for monitoring health and status of the Hadoop cluster.
Ambari leverages Ambari Metrics System for metrics collection.
Ambari leverages Ambari Alert Framework for system alerting and will notify you when your attention is needed (e.g., a node goes down, remaining disk space is low, etc).
Ambari enables Application Developers and System Integrators to:

Easily integrate Hadoop provisioning, management, and monitoring capabilities to their own applications with the Ambari REST APIs.
Docker


Receipts:

Image: sequenceiq/ambari (git)
Multinode cluster with Ambari 1.7.0 1
Get the docker images

[code] docker pull sequenceiq/ambari:1.7.0 [/code]

Get ambari-functions [code] curl -Lo .amb j.mp/docker-ambari-170 && . .amb [/code]

Create your cluster – automated

[code] amb-deploy-cluster 3 [/code]

Multinode cluster with Ambari 1.7.0 ↩

\subsection{Kibana}

Kibana is an open source data visualization plugin for Elasticsearch. It provides visualization capabilities on top of the content indexed on an Elasticsearch cluster. Users can create bar, line and scatter plots, or pie charts and maps on top of large volumes of data.

\subsection{Logstash}

https://www.digitalocean.com/community/tutorials/how-to-use-logstash-and-kibana-to-centralize-logs-on-centos-6

\subsection{Elasticsearch}


Elasticsearch is a search server based on Lucene. It provides a distributed, multitenant-capable full-text search engine with a RESTful web interface and schema-free JSON documents. Elasticsearch is developed in Java and is released as open source under the terms of the Apache License. Elasticsearch is the second most popular enterprise search engine
1. Basic Concenpts
Relational Database	Elasticsearch
Database	Index
Table	Type
Row	Document
Column	Field
Schema	Mapping
2. Index & Query
Get all indices
/_stats
Search API 1
Search All
/bank/_search?q=*
hits.hits – actual array of search results (defaults to first 10 documents)

Query Language
elasticsearch provides a full Query DSL based on JSON to define queries.

curl -XPOST /bank/_search
// match all, limit 10 offset 10
{
  "query": { "match_all": {} },
  "from": 10,
  "size": 10
}

// select fields
{
  "query": { "match_all": {} },
  _source: ["account_number", "balance"]
  "size": 10
}

// where account equals 20
{
  "query": { "match": { "account_number": 20 } }
}
Filter

curl -XPOST elastic:9200/index/type/_search -d '
{
  "query" : {
    "filtered" :
    {
      "query" : { "term" : { "feature" : 1 } } ,
      "filter" : {
        "and" : [
          {
            "range": {
              "_timestamp": {
                "from": 1441964671000,
                "to": 1441964672000
              }
            }
          }
        ]
      }
    }
  }
}
Sort

curl -XPOST elastic:9200/index/type/_search -d '
{
  "query" : {
    "filtered" :
    {
      "query" : { "term" : { "feature" : 1 } } ,
      "filter" : {
        "and" : [
          {
            "range": {
              "_timestamp": {
                "from": 1441964671000,
                "to": 1441964672000
              }
            }
          }
        ]
      }
    }
  }
}
3. Mapping
Timestamp 2
Enable and store timestamp

curl -XPOST localhost:9200/test
{
"mappings" : {
    "_default_":{
        "_timestamp" : {
            "enabled" : true,
            "store" : true
        }
    }
  }
}'
Relationships Management 3 4
Inner Object

👍 Easy, fast, performant
👎 No need for special queries
☛ Only applicable when one-to-one relationships are maintained
Nested

👍 Nested docs are stored in the same Lucene block as each other, which helps read/query performance. Reading a nested doc is faster than the equivalent parent/child.
👎 Updating a single field in a nested document (parent or nested children) forces ES to reindex the entire nested document. This can be very expensive for large nested docs
👎 “Cross referencing” nested documents is impossible
☛ Best suited for data that does not change frequently
Parent/Child

👍 Updating a child doc does not affect the parent or any other children, which can potentially save a lot of indexing on large docs
👎 Children are stored separately from the parent, but are routed to the same shard. So parent/children are slightly less performance on read/query than nested
👎 Parent/child mappings have a bit extra memory overhead, since ES maintains a “join” list in memory
👎 Sorting/scoring can be difficult with Parent/Child since the Has Child/Has Parent operations can be opaque at times
Denormalization

👍 You get to manage all the relations yourself!
👎 Most flexible, most administrative overhead
☛ May be more or less performant depending on your setup
4. Backup
Elastic Dump 5
Tools for moving and saving indicies.

bin/elasticdump
  --input=http://localhost:9200/index_1
  --output=http://localhost:9200/index_1_backup
  --type=data
  --scrollTime=100
Alias 6
curl -XPOST 'http://localhost:9200/_aliases' -d '
{
    &quot;actions&quot; : [
        { &quot;remove&quot; : { &quot;index&quot; : &quot;test1&quot;, &quot;alias&quot; : &quot;alias1&quot; } },
        { &quot;add&quot; : { &quot;index&quot; : &quot;test1&quot;, &quot;alias&quot; : &quot;alias2&quot; } }
    ]
}'
5. Module Scripting 7
Ranking
Rank #2 from DB-Engines Ranking of Search Engines

The Search API ↩
http://stackoverflow.com/a/17146144/772391 ↩
http://stackoverflow.com/a/23407367/772391 ↩
https://www.elastic.co/guide/en/elasticsearch/guide/current/modeling-your-data.html ↩
https://github.com/taskrabbit/elasticsearch-dump ↩
https://www.elastic.co/guide/en/elasticsearch/reference/current/indices-aliases.html ↩
https://www.elastic.co/guide/en/elasticsearch/reference/current/modules-scripting.html ↩
Elasticsearch tutorial series 1: Metric Aggregations with Social Network Data
Table of content

Avg, Max, Min, Sum Aggregation
Cardinality Aggregation
Stats Aggregation
Extended Stats Aggregation
Percentile Aggregation
Percentile Ranks Aggregation
Top hits Aggregation
Avg, Max, Min, Sum, Count Aggregation
Doc: Avg Aggregation, Doc: Max Aggregation, Doc: Min Aggregation

Get max, min, avg, sum, count about number of likes, shares, comments

Request

POST /facebook_crawler/post/_search
{"aggs":{"sum_like":{"sum":{"field":"num_like"}},"min_like":{"min":{"field":"num_like"}},"avg_like":{"avg":{"field":"num_like"}},"max_like":{"max":{"field":"num_like"}},"sum_share":{"sum":{"field":"num_share"}},"min_share":{"min":{"field":"num_share"}},"avg_share":{"avg":{"field":"num_share"}},"max_share":{"max":{"field":"num_share"}},"sum_comment":{"sum":{"field":"num_comment"}},"min_comment":{"min":{"field":"num_comment"}},"avg_comment":{"avg":{"field":"num_comment"}},"max_comment":{"max":{"field":"num_comment"}}}}
Request

{
"aggregations": {
      "avg_comment": {
         "value": 75.23860589812332
      },
      "min_like": {
         "value": 0
      },
      "avg_like": {
         "value": 1761974365266098.2
      },
      "sum_like": {
         "value": 3238508883359088600
      },
      "max_share": {
         "value": 30407
      },
      "max_comment": {
         "value": 11000
      },
      "sum_share": {
         "value": 117844
      },
      "max_like": {
         "value": 2751488761761411000
      },
      "avg_share": {
         "value": 250.19957537154988
      },
      "sum_comment": {
         "value": 28064
      },
      "min_comment": {
         "value": 2
      },
      "min_share": {
         "value": 1
      }
   }
}
Cardinality Aggregation
Cardinality Aggregation

Get total of users

Request

POST /facebook_crawler/post/_search
{
    "aggs" : {
        "num_authors" : { "cardinality" : { "field" : "from.fb_id" } }
    }
}
Response

{
   "aggregations": {
      "num_authors": {
         "value": 7385
      }
   }
}
Stats Aggregation
Doc: Stats Aggregation

Basic Stats of like, share & comment

Request

POST /facebook_crawler/post/_search
{
    "aggs" : {
        "shares" : { "stats" : { "field" : "num_share" } },
        "likes" : { "stats" : { "field" : "num_like" } },
        "comments" : { "stats" : { "field" : "num_comment" } }
    }
}
Response

{
   "aggregations": {
      "shares": {
         "count": 471,
         "min": 1,
         "max": 30407,
         "avg": 250.19957537154988,
         "sum": 117844
      },
      "comments": {
         "count": 373,
         "min": 2,
         "max": 11000,
         "avg": 75.23860589812332,
         "sum": 28064
      },
      "likes": {
         "count": 1838,
         "min": 0,
         "max": 2751488761761411000,
         "avg": 1761974365266098.2,
         "sum": 3238508883359088600
      }
   }
}
Extended Stats Aggregation
Extended Stats Aggregation

Stats of like, share & comment with more metrics, such as sum, std_deviation, std_deviation_bounds, variance

Request

POST /facebook_crawler/post/_search
{
    "aggs" : {
        "like_stats" : { "extended_stats" : { "field" : "num_like" } },
        "share_stats" : { "extended_stats" : { "field" : "num_share" } },
        "comment_stats" : { "extended_stats" : { "field" : "num_comment" } }
    }
}
Response

{
   "aggregations": {
      "like_stats": {
         "count": 1838,
         "min": 0,
         "max": 2751488761761411000,
         "avg": 1761974365266098.2,
         "sum": 3238508883359088600,
         "sum_of_squares": 7.667542671405507e+36,
         "variance": 4.168572634260795e+33,
         "std_deviation": 64564484310345070,
         "std_deviation_bounds": {
            "upper": 130890942985956240,
            "lower": -127366994255424050
         }
      },
      "share_stats": {
         "count": 471,
         "min": 1,
         "max": 30407,
         "avg": 250.19957537154988,
         "sum": 117844,
         "sum_of_squares": 1769467022,
         "variance": 3694230.367812983,
         "std_deviation": 1922.0380765773043,
         "std_deviation_bounds": {
            "upper": 4094.2757285261587,
            "lower": -3593.8765777830586
         }
      },
      "comment_stats": {
         "count": 373,
         "min": 2,
         "max": 11000,
         "avg": 75.23860589812332,
         "sum": 28064,
         "sum_of_squares": 131531392,
         "variance": 346970.2299304962,
         "std_deviation": 589.0417896299856,
         "std_deviation_bounds": {
            "upper": 1253.3221851580945,
            "lower": -1102.844973361848
         }
      }
   }
}
Percentiles Aggregation
Doc: Percentiles Aggregation

Comment, Like, Share Percentiles

Request

POST /facebook_crawler/post/_search
{"aggs":{"like_percentiles":{"percentiles":{"field":"num_like"}},"share_percentiles":{"percentiles":{"field":"num_share"}},"comment_percentiles":{"percentiles":{"field":"num_comment"}}}}
Response

{
"aggregations": {
      "like_percentiles": {
         "values": {
            "1.0": 0,
            "5.0": 0,
            "25.0": 4,
            "50.0": 18.35,
            "75.0": 72.53579545454545,
            "95.0": 71343.74999999999,
            "99.0": 4338260523723.276
         }
      },
      "comment_percentiles": {
         "values": {
            "1.0": 2,
            "5.0": 2,
            "25.0": 5,
            "50.0": 10,
            "75.0": 26,
            "95.0": 139.39999999999998,
            "99.0": 1000
         }
      },
      "share_percentiles": {
         "values": {
            "1.0": 1,
            "5.0": 1,
            "25.0": 1,
            "50.0": 4,
            "75.0": 25,
            "95.0": 251.5,
            "99.0": 5560.3
         }
      }
   }
}
Like Percentiles with custom percents

Request

POST /facebook_crawler/post/_search
{
   "aggs": {
      "share_percentiles": {
         "percentiles": {
            "field": "num_share",
            "percents": [0, 10, 80, 90, 95]
         }
      }
   }
}
Response

{
   "aggregations": {
      "share_percentiles": {
         "values": {
            "0.0": 1,
            "10.0": 1,
            "80.0": 37.33333333333333,
            "90.0": 97,
            "95.0": 251.5
         }
      }
   }
}
Percentile Ranks Aggregation
Doc: Percentile Ranks Aggregation

How like, share, comment distribute

Request

POST /facebook_crawler/post/_search
{
   "aggs": {
      "like_percentile_ranks": {
         "percentile_ranks": {
            "field": "num_like",
            "values": [10, 100, 1000, 10000, 1000000, 10000000]
         }
      },
      "share_percentile_ranks": {
         "percentile_ranks": {
            "field": "num_share",
            "values": [10, 100, 1000, 10000, 1000000, 10000000]
         }
      },
      "comment_percentile_ranks": {
         "percentile_ranks": {
            "field": "num_comment",
            "values": [10, 100, 1000, 10000, 1000000, 10000000]
         }
      }
   }
}
Response

{
   "aggregations": {
      "share_percentile_ranks": {
         "values": {
            "10.0": 60.438782731776364,
            "100.0": 89.91507430997878,
            "1000.0": 97.37406386327386,
            "10000.0": 99.31579836222765,
            "1000000.0": 100,
            "1.0E7": 100
         }
      },
      "like_percentile_ranks": {
         "values": {
            "10.0": 39.281828073993466,
            "100.0": 79.39530545624125,
            "1000.0": 90.98349676683587,
            "10000.0": 94.14527905373414,
            "1000000.0": 95.9014681663581,
            "1.0E7": 96.57661015941164
         }
      },
      "comment_percentile_ranks": {
         "values": {
            "10.0": 49.865951742627345,
            "100.0": 92.18395545473294,
            "1000.0": 98.92761394101876,
            "10000.0": 99.56773202397807,
            "1000000.0": 100,
            "1.0E7": 100
         }
      }
   }
}
As we can see, only 0.7% posts have more than 10k shares, onley 0.04% posts have more than 10k comment, but there is an odd here. 4.1% posts have more than 1M like (WHAT!!!). We can spot some strange here.

Top hits Aggregation
Doc: Top hits Aggregation

Example

Request

Response

{

An Aggregation
Doc: Link

Config
elasticsearch.yml

discovery.zen.minimum_master_nodes: 1
discovery.zen.ping.multicast.enabled: false
discovery.zen.ping.unicast.hosts: ["localhost"]

network.host: 0.0.0.0
http.cors.enabled: true
http.cors.allow-origin: '*'
script.inline: on
script.indexed: on
Docker
Image

https://hub.docker.com/r/_/elasticsearch/

Run

docker run -d -v "$PWD/esdata":/usr/share/elasticsearch/data elasticsearch
Docker Folder

elasticsearch/
├── config
│   └── elasticsearch.yml
└── Dockerfile
Dockerfile

FROM elasticsearch:2.2.0

ADD config/elasticsearch.yml /elasticsearch/config/elasticsearch.yml
Compose

elasticsearch:
    build: ./elasticsearch/.
    ports:
       - 9200:9200
       - 9300:9300
    volumes:
       - ./data/elasticsearch:/usr/share/elasticsearch/data
Elasticsearch: Search Ignore Accents
The ICU 1 2 analysis plug-in for Elasticsearch uses the International Components for Unicode (ICU) libraries to provide a rich set of tools for dealing with Unicode. These include the icu_tokenizer, which is particularly useful for Asian languages, and a number of token filters that are essential for correct matching and sorting in all languages other than English.

Step 1: Install ICU-Plugin 3
cd /usr/share/elasticsearch
sudo bin/plugin install analysis-icu
Step 2: Create an analyzer setting:
"settings": {
      "analysis": {
         "analyzer": {
            "vnanalysis": {
               "tokenizer": "icu_tokenizer",
               "filter": [
                  "icu_folding",
                  "icu_normalizer"
               ]
            }
         }
      }
   }
Step 3: Create your index, create a field with type string and analyzer is vnanalysis you have created
"key": {
     "type": "string",
     "analyzer": "vnanalysis"
}
Step 4: Search with sense
POST /your_index/your_doc_type/_search
{
   "query": {
      "match": {
         "key": "kiem tra"
      }
   }
}
ES: Import CSV to Elasticsearch
https://gist.github.com/clemsos/8668698

Install lastest Elasticdump with NVM
As a matter of best practice we’ll update our packages:

apt-get update
The build-essential package should already be installed, however, we’re going still going to include it in our command for installation:

apt-get install build-essential libssl-dev
To install or update nvm, you can use the install script using cURL:

curl -o- https://raw.githubusercontent.com/creationix/nvm/v0.31.0/install.sh | bash
if you have below problem or after you type nvm ls-remote command it result N/A: curl: (77) error setting certificate verify locations: CAfile: /etc/pki/tls/certs/ca-bundle.crt CApath: none

head to this 1:

or Wget:

wget -qO- https://raw.githubusercontent.com/creationix/nvm/v0.31.0/install.sh | bash
Don't forget to restart your terminal

Then you use the following command to list available versions of nodejs

nvm ls-remote
To download, compile, and install the latest v5.0.x release of node, do this:

nvm install 5.0
And then in any new shell just use the installed version:

nvm use 5.0
Or you can just run it:

nvm run 5.0 --version
Or, you can run any arbitrary command in a subshell with the desired version of node:

nvm exec 4.2 node --version
You can also get the path to the executable to where it was installed:

nvm which 5.0
Node Version Manager

how to solve https problem ↩

ICU plug-in Github ↩

Installing the ICU plug-in ↩

\subsection{Neo4J}

version: 2.3.1

Neo4j is an open-source graph database, implemented in Java. The developers describe Neo4j as "embedded, disk-based, fully transactional Java persistence engine that stores data structured in graphs rather than in tables". Neo4j is the most popular graph database.

Installation

Docker
Docker Image: https://hub.docker.com/r/library/neo4j/

Run these below command to open neo4j

# clone datahub project
git clone https://github.com/magizbox/datahub.git

# change folder to datahub directory
cd datahub

# set your config in docker-compose.yml

# run docker
docker-compose up

Cypher

Schema Discovery
List all nodes label, list all relation type

> START n=node(*) RETURN distinct labels(n)

> match n-[r]-() return distinct type(r)
UI Way: Click to Overtab in Neo4j Browser

Sample 10 entities
> MATCH (n:Entity) RETURN n, rand() as random ORDER BY random LIMIT 10
Group By
http://www.markhneedham.com/blog/2013/02/17/neo4jcypher-sql-style-group-by-functionality/

Graph Algorithms

shortestPath, dijkstra

POST http://localhost:7474/db/data/node/72/paths

Headers
Accept: application/json
Authorization: Basic bmVvNGo6cGFzc3dk

Body
{
  "to" : "http://localhost:7474/db/data/node/77",
  "max_depth" : 5,
  "relationships" : {
    "type" : "FRIEND",
    "direction" : "out"
  },
  "algorithm" : "shortestPath"
}
Graph Analystic

pagerank, closeness_centrality, betweenness_centrality, triangle_count, connected_components, strongly_connected_components

Client

In this article you will know how to connect to neo4j database from python.

Python Client
We can use Py2neo to connect to neo4j from python.

Py2neo is a client library and comprehensive toolkit for working with Neo4j from within Python applications and from the command line. The core library has no external dependencies and has been carefully designed to be easy and intuitive to use.

Snippets to connect, create, add nodes, add relationship and update property

from py2neo import authenticate, Graph, Node, Relationship
# connect to graph
authenticate("localhost:7474", "neo4j", "passwd")
graph = Graph("http://localhost:7474/db/data/")

# create unique
graph.schema.create_uniqueness_constraint('Person', 'name')

# add nodes
graph.create(Node.cast('Person', {"name": "Alice"}))
graph.create(Node.cast('Person', {"name": "Bob"}))

# add relationship
source = graph.merge_one("Person", "name", "Alice")
target = graph.merge_one("Person", "name", "Bob")
graph.create_unique(Relationship(source, "FRIEND", target))

# update property
alice = graph.merge_one("Person", "name", "Alice")
alice["age"] = 30
alice.push()

\section{Web Crawling}

\subsection{Introduction}

Web Crawler
Static Crawler

Apache Nutch
Dynamic Crawler

nutch-selenium
Intelligent Extractor

boilerpipe
Web Content Extraction Through Machine Learning
Priority Crawler, Social Crawler

Features a crawler must provide
We list the desiderata for web crawlers in two categories: features that web crawlers must provide, followed by features they should provide.

Robustness:

The Web contains servers that create spider traps, which are generators of web pages that mislead crawlers into getting stuck fetching an infinite number of pages in a particular domain. Crawlers must be designed to be resilient to such traps. Not all such traps are malicious; some are the inadvertent side-effect of faulty website development.

Politeness:

Web servers have both implicit and explicit policies regulating the rate at which a crawler can visit them. These politeness policies must be respected.

Features a crawler should provide
Distributed The crawler should have the ability to execute in a distributed fashion across multiple machines.

Scalable

The crawler architecture should permit scaling up the crawl rate by adding extra machines and bandwidth.

Performance and efficiency

The crawl system should make efficient use of various system resources including processor, storage and network bandwidth.

Quality

Given that a significant fraction of all web pages are of poor utility for serving user query needs, the crawler should be biased towards fetching ``useful'' pages first.

Freshness

In many applications, the crawler should operate in continuous mode: it should obtain fresh copies of previously fetched pages. A search engine crawler, for instance, can thus ensure that the search engine's index contains a fairly current representation of each indexed web page. For such continuous crawling, a crawler should be able to crawl a page with a frequency that approximates the rate of change of that page.

Extensible

Crawlers should be designed to be extensible in many ways - to cope with new data formats, new fetch protocols, and so on. This demands that the crawler architecture be modular.

Crawling
The basic operation of any hypertext crawler (whether for the Web, an intranet or other hypertext document collection) is as follows.

The crawler begins with one or more URLs that constitute a seed set. It picks a URL from this seed set, then fetches the web page at that URL.
The fetched page is then parsed, to extract both the text and the links from the page (each of which points to another URL).
The extracted text is fed to a text indexer.
The extracted links (URLs) are then added to a URL frontier, which at all times consists of URLs whose corresponding pages have yet to be fetched by the crawler.
Initially, the URL frontier contains the seed set; as pages are fetched, the corresponding URLs are deleted from the URL frontier. The entire process may be viewed as traversing the web graph. In continuous crawling, the URL of a fetched page is added back to the frontier for fetching again in the future.
This seemingly simple recursive traversal of the web graph is complicated by the many demands on a practical web crawling system: the crawler has to be distributed, scalable, efficient, polite, robust and extensible while fetching pages of high quality. We examine the effects of each of these issues. Our treatment follows the design of the Mercator crawler that has formed the basis of a number of research and commercial crawlers. As a reference point, fetching a billion pages (a small fraction of the static Web at present) in a month-long crawl requires fetching several hundred pages each second. We will see how to use a multi-threaded design to address several bottlenecks in the overall crawler system in order to attain this fetch rate.

Before proceeding to this detailed description, we reiterate for readers who may attempt to build crawlers of some basic properties any non-professional crawler should satisfy:

Only one connection should be open to any given host at a time.
A waiting time of a few seconds should occur between successive requests to a host.
Politeness restrictions should be obeyed.

A New Approach to Dynamic Crawler
Build a crawler system for dynamic websites is not easy task. While you can use a web browser automator (like selenium), or event when you can integrate selenium with nutch (by using nutch-selenium). These solutions are still hard to develop, hard to test and hard to manage sessions because we still "translate" our process to languages (such as java or python)

I suppose a new approach for this problem. Instead of using a web browser automator, we can inject native javascript codes into browser (via extension or add-on).The advantages of this approach is we can easily inject third party libraries (like jquery (for dom selector), Run.js (for complicated process) and APIs that supported by browsers). And we can take advance of debugging tool and testing framework in javascript world.

If you want to know about more details, feel free to contact me.

\subsection{Scrapy}

Scrapy
An open source and collaborative framework for extracting the data you need from websites. In a fast, simple, yet extensible way.

Build and run your web spiders
$ pip install scrapy
$ cat > myspider.py <<EOF
import scrapy

class BlogSpider(scrapy.Spider):
    name = 'blogspider'
    start_urls = ['https://blog.scrapinghub.com']

    def parse(self, response):
        for title in response.css('h2.entry-title'):
            yield {'title': title.css('a ::text').extract_first()}

        next_page = response.css('div.prev-post > a ::attr(href)').extract_first()
        if next_page:
            yield scrapy.Request(response.urljoin(next_page), callback=self.parse)
EOF
$ scrapy runspider myspider.py
Deploy them to Scrapy Cloud
$ shub login
Insert your Scrapinghub API Key: <API_KEY>

# Deploy the spider to Scrapy Cloud

$ shub deploy

# Schedule the spider for execution
 shub schedule blogspider
Spider blogspider scheduled, watch it running here:
https://app.scrapinghub.com/p/26731/job/1/8

# Retrieve the scraped data
$ shub items 26731/1/8
{"title": "Improved Frontera: Web Crawling at Scale with Python 3 Support"}
{"title": "How to Crawl the Web Politely with Scrapy"}
...

\subsection{Apache Nutch}

Highly extensible, highly scalable Web crawler 1 Nutch is a well matured, production ready Web crawler. Nutch 1.x enables fine grained configuration, relying on Apache Hadoop™ data structures, which are great for batch processing.

History


Usecases


1. Features 1
1. Transparency Nutch is open source, so anyone can see how the ranking algorithms work. With commercial search engines, the precise details of the algorithms are secret so you can never know why a particular search result is ranked as it is. Furthermore, some search engines allow rankings to be based on payments, rather than on the relevance of the site's contents. Nutch is a good fit for academic and government organizations, where the perception of fairness of rankings may be more important.

2. Understanding We don't have the source code to Google, so Nutch is probably the best we have. It's interesting to see how a large search engine works. Nutch has been built using ideas from academia and industry: for instance, core parts of Nutch are currently being re-implemented to use the MapReduce.

Map Reduce distributed processing model, which emerged from Google Labs last year. And Nutch is attractive for researchers who want to try out new search algorithms, since it is so easy to extend.

3. Extensibility Don't like the way other search engines display their results? Write your own search engine--using Nutch! Nutch is very flexible: it can be customized and incorporated into your application. For developers, Nutch is a great platform for adding search to heterogeneous collections of information, and being able to customize the search interface, or extend the out-of-the-box functionality through the plugin mechanism. For example, you can integrate it into your site to add a search capability.

Process 5
0. initialize CrawlDb, inject seed URLs Repeat generate-fetch-update cycle n times:

1. The Injector takes all the URLs of the nutch.txt file and adds them to the CrawlDB. As a central part of Nutch, the CrawlDB maintains information on all known URLs (fetch schedule, fetch status, metadata, …).

2. Based on the data of CrawlDB, the Generator creates a fetchlist and places it in a newly created Segment directory.

3. Next, the Fetcher gets the content of the URLs on the fetchlist and writes it back to the Segment directory. This step usually is the most time-consuming one.

4. Now the Parser processes the content of each web page and for example omits all html tags. If the crawl functions as an update or an extension to an already existing one (e.g. depth of 3), the Updater would add the new data to the CrawlDB as a next step.

5. Before indexing, all the links need to be inverted by Link Inverter, which takes into account that not the number of outgoing links of a web page is of interest, but rather the number of inbound links. This is quite similar to how Google PageRank works and is important for the scoring function. The inverted links are saved in the Linkdb.

6-7. Using data from all possible sources (CrawlDB, LinkDB and Segments), the Indexer creates an index and saves it within the Solr directory. For indexing, the popular Lucene library is used. Now, the user can search for information regarding the crawled web pages via Solr.

Installation
Requirements

1. OpenJDK 7

2. Nutch 2.3 RC (yes, you need 2.3, 2.2 will not work)

wget https://archive.apache.org/dist/nutch/2.3/apache-nutch-2.3-src.tar.gz
tar -xzf apache-nutch-2.3-src.tar.gz
3. HBase 0.94.27 (HBase 0.98 won't work)

wget https://www.apache.org/dist/hbase/hbase-0.94.27/hbase-0.94.27.tar.gz
tar -xzf hbase-0.94.27.tar.gz
4. ElasticSearch 1.7

wget https://download.elastic.co/elasticsearch/elasticsearch/elasticsearch-1.7.0.tar.gz
tar -xzf elasticsearch-1.7.0.tar.gz
Other Options: nutch-2.3, hbase-0.94.26, ElasticSearch 1.4

Setup HBase
1. edit $HBASE_ROOT/conf/hbase-site.xml and add

<configuration>
    <property>
        <name>hbase.rootdir</name>
        <value>file:///full/path/to/where/the/data/should/be/stored</value>
    </property>
    <property>
        <name>hbase.cluster.distributed</name>
        <value>false</value>
    </property>
</configuration>
2. edit $HBASE_ROOT/conf/hbase-env.sh and enable JAVA_HOME and set it to the proper path:

-# export JAVA_HOME=/usr/java/jdk1.6.0/
+export JAVA_HOME=/usr/lib/jvm/java-7-openjdk-amd64/
This step might seem redundant, but even with JAVA_HOME being set in my shell, HBase just didn't recognize it.

3. kick off HBase:

$ $HBASE_ROOT/bin/start-hbase.sh
Configure Nutch
1. Enable the HBase dependency in $NUTCH_ROOT/ivy/ivy.xml by uncommenting the line

<dependency org="org.apache.gora" name="gora-hbase" rev="0.5" conf="*->default" />
2. Configure the HBase adapter by editing the $NUTCH_ROOT/conf/gora.properties

-#gora.datastore.default=org.apache.gora.mock.store.MockDataStore
+gora.datastore.default=org.apache.gora.hbase.store.HBaseStore
3. Build Nutch

$ cd $NUTCH_ROOT && ant clean && ant runtime
This can take a while and creates $NUTCH_ROOT/runtime/local.

4. configure Nutch by editing $NUTCH_ROOT/runtime/local/conf/nutch-site.xml

<configuration>
    <property>
        <name>http.agent.name</name>
        <value>mycrawlername</value>
        <!-- this can be changed to something more sane if you like -->
    </property>
    <property>
        <name>http.robots.agents</name>
        <value>mycrawlername</value>
        <!-- this is the robot name we're looking for in robots.txt files -->
    </property>
    <property>
        <name>storage.data.store.class</name>
        <value>org.apache.gora.hbase.store.HBaseStore</value>
    </property>
    <property>
        <name>plugin.includes</name>
        <!-- do \*\*NOT\*\* enable the parse-html plugin, if you want proper HTML parsing. Use something like parse-tika! -->
        <value>
            protocol-httpclient|urlfilter-regex|parse-(text|tika|js)|index-(basic|anchor)|query-(basic|site|url)|response-(json|xml)|summary-basic|scoring-opic|urlnormalizer-(pass|regex|basic)|indexer-elastic
        </value>
    </property>
    <property>
        <name>db.ignore.external.links</name>
        <value>true</value>
        <!-- do not leave the seeded domains (optional) -->
    </property>
    <property>
        <name>elastic.host</name>
        <value>localhost</value>
        <!-- where is ElasticSearch listening -->
    </property>
</configuration>
or you configure Nutch by editing $NUTCH_ROOT/runtime/local/conf/nutch-site.xml

<configuration>
    <property>
        <name>plugin.includes</name>
        <!-- do \*\*NOT\*\* enable the parse-html plugin, if you want proper HTML parsing. Use something like parse-tika! -->
        <value>
            protocol-http|protocol-httpclient|urlfilter-regex|
parse-(text|tika|js)|index-(basic|anchor)|query-(basic|site|url)|response-(json|xml)|
summary-basic|scoring-opic|urlnormalizer-(pass|regex|basic)|indexer-elastic|
index-metadata|index-more
        </value>
    </property>
    <property>
        <name>db.ignore.external.links</name>
        <value>true</value>
        <!-- do not leave the seeded domains (optional) -->
    </property>


<!-- elasticsearch index properties -->
<property>
  <name>elastic.host</name>
  <value>localhost</value>
  <description>The hostname to send documents to using TransportClient.
  Either host and port must be defined or cluster.
  </description>
</property>

<property>
  <name>elastic.port</name>
  <value>9300</value>
  <description>
  The port to connect to using TransportClient.
  </description>
</property>
<property>
  <name>elastic.index</name>
  <value>nutch</value>
  <description>
  The name of the elasticsearch index. Will normally be autocreated if it
  doesn't exist.
  </description>
</property>
<!-- end index -->
</configuration>
5. configure HBase integration by editing $NUTCH_ROOT/runtime/local/conf/hbase-site.xml

<?xml version="1.0" encoding="UTF-8"?>
<configuration>
   <property>
      <name>hbase.rootdir</name>
      <value>file:///full/path/to/where/the/data/should/be/stored</value>
      <!-- same path as you've given for HBase above -->
   </property>
   <property>
      <name>hbase.cluster.distributed</name>
      <value>false</value>
   </property>
</configuration>
or you configure HBase integration by editing $NUTCH_ROOT/runtime/local/conf/hbase-site.xml:

<configuration>
  <property>
    <name>hbase.rootdir</name>
    <value>file:///$PATH/database</value>
  </property>
  <property>
    <name>hbase.cluster.distributed</name>
    <value>false</value>
  </property>
  <property>
    <name>hbase.zookeeper.quorum</name>
    <value>hbase.io</value>
  </property>
  <property>
    <name>zookeeper.znode.parent</name>
    <value>/hbase-unsecure</value>
  </property>
  <property>
    <name>hbase.rpc.timeout</name>
    <value>2592000000</value>
  </property>
</configuration>
That's it. Everything is now setup to crawl websites.

Run Nutch
1. Create an empty directory. Add a textfile containing a list of seed URLs

$ mkdir seed
$ echo "https://www.website.com" >> seed/urls.txt
$ echo "https://www.another.com" >> seed/urls.txt
$ echo "https://www.example.com" >> seed/urls.txt
Inject them into Nutch by giving a file URL (!)

$ $NUTCH_ROOT/runtime/local/bin/nutch inject file:///path/to/seed/
2. Generate a new set of URLs to fetch.

This is is based on both the injected URLs as well as outdated URLs in the Nutch crawl db.

$ $NUTCH_ROOT/runtime/local/bin/nutch generate -topN 10
The above command will create job batches for 10 URLs.

3. Fetch the URLs. We are not clustering, so we can simply fetch all batches:

$ $NUTCH_ROOT/runtime/local/bin/nutch fetch -all
4. Now we parse all fetched pages:

$ $NUTCH_ROOT/runtime/local/bin/nutch parse -all
5. Last step: Update Nutch's internal database:

$ $NUTCH_ROOT/runtime/local/bin/nutch updatedb -all
On the first run, this will only crawl the injected URLs. The procedure above is supposed to be repeated regulargy to keep the index up to date.

6. Putting Documents into ElasticSearch

$ $NUTCH_ROOT/runtime/local/bin/nutch index -all
Configuration
Crawl nutch via proxy

Change $NUTCH_ROOT/runtime/local/conf/nutch-site.xml

<configuration>
    <property>
        <name>http.proxy.host</name>
        <value>192.168.80.1</value>
        <description>The proxy hostname. If empty, no proxy is used.</description>
    </property>
    <property>
        <name>http.proxy.port</name>
        <value>port</value>
        <description>The proxy port.</description>
    </property>
    <property>
        <name>http.proxy.username</name>
        <value>username</value>
        <description>Username for proxy. This will be used by 'protocol-httpclient', if the proxy server requests basic,
            digest
            and/or NTLM authentication. To use this, 'protocol-httpclient' must be present in the value of
            'plugin.includes'
            property. NOTE: For NTLM authentication, do not prefix the username with the domain, i.e. 'susam' is correct
            whereas
            'DOMAINsusam' is incorrect.
        </description>
    </property>
    <property>
        <name>http.proxy.password</name>
        <value>password</value>
        <description>Password for proxy. This will be used by 'protocol-httpclient', if the proxy server requests basic,
            digest
            and/or NTLM authentication. To use this, 'protocol-httpclient' must be present in the value of
            'plugin.includes'
            property.
        </description>
    </property>
</configuration>
Nutch Plugins
Extension Points
In writing a plugin, you're actually providing one or more extensions of the existing extension-points . The core Nutch extension-points are themselves defined in a plugin, the NutchExtensionPoints plugin (they are listed in the NutchExtensionPoints plugin.xml file). Each extension-point defines an interface that must be implemented by the extension. The core extension points are:

Point	Description	Example
IndexWriter	Writes crawled data to a specific indexing backends (Solr, ElasticSearch, a CVS file, etc.).
IndexingFilter	Permits one to add metadata to the indexed fields. All plugins found which implement this extension point are run sequentially on the parse (from javadoc).
Parser	Parser implementations read through fetched documents in order to extract data to be indexed. This is what you need to implement if you want Nutch to be able to parse a new type of content, or extract more data from currently parseable content.
HtmlParseFilter	Permits one to add additional metadata to HTML parses (from javadoc).
Protocol	Protocol implementations allow Nutch to use different protocols (ftp, http, etc.) to fetch documents.
URLFilter	URLFilter implementations limit the URLs that Nutch attempts to fetch. The RegexURLFilter distributed with Nutch provides a great deal of control over what URLs Nutch crawls, however if you have very complicated rules about what URLs you want to crawl, you can write your own implementation.
URLNormalizer	Interface used to convert URLs to normal form and optionally perform substitutions.
ScoringFilter	A contract defining behavior of scoring plugins. A scoring filter will manipulate scoring variables in CrawlDatum and in resulting search indexes. Filters can be chained in a specific order, to provide multi-stage scoring adjustments.
SegmentMergeFilter	Interface used to filter segments during segment merge. It allows filtering on more sophisticated criteria than just URLs. In particular it allows filtering based on metadata collected while parsing page.
Getting Nutch to Use a Plugin
In order to get Nutch to use a given plugin, you need to edit your conf/nutch-site.xml file and add the name of the plugin to the list of plugin.includes. Additionally we are required to add the various build configurations to build.xml in the plugin directory.

Develop nutch plugins
Project structure of a plugin
plugin-name
  plugin.xml
  build.xml
  ivy.xml
  src
    org
      apache
        nutch
          indexer
            uml-meta # source folder
              URLMetaIndexingFilter.java
          scoring
            uml-meta # source folder
              URLMetaScoringFilter.java
  test
    org
      apache
        nutch
          indexer
            uml-meta # test folder
              URLMetaIndexingFilterTest.java
          scoring
            uml-meta # test folder
              URLMetaScoringFilterTest.java
Follow this link to read develop nutch plugins

\subsubsection{Architecture}

Architectures


Data Structure
The web database is a specialized persistent data structure for mirroring the structure and properties of the web graph being crawled. It persists as long as the web graph that is being crawled (and re-crawled) exists, which may be months or years. The WebDB is used only by the crawler and does not play any role during searching. The WebDB stores two types of entities: pages and links.

A page represents a page on the Web, and is indexed by its URL and the MD5 hash of its contents. Other pertinent information is stored, too, including

the number of links in the page (also called outlinks);
fetch information (such as when the page is due to be refetched);
the page's score, which is a measure of how important the page is (for example, one measure of importance awards high scores to pages that are linked to from many other pages).
A link represents a link from one web page (the source) to another (the target). In the WebDB web graph, the nodes are pages and the edges are links.

A segment is a collection of pages fetched and indexed by the crawler in a single run. The fetchlist for a segment is a list of URLs for the crawler to fetch, and is generated from the WebDB. The fetcher output is the data retrieved from the pages in the fetchlist. The fetcher output for the segment is indexed and the index is stored in the segment. Any given segment has a limited lifespan, since it is obsolete as soon as all of its pages have been re-crawled. The default re-fetch interval is 30 days, so it is usually a good idea to delete segments older than this, particularly as they take up so much disk space. Segments are named by the date and time they were created, so it's easy to tell how old they are.

The index is the inverted index of all of the pages the system has retrieved, and is created by merging all of the individual segment indexes. Nutch uses Lucene for its indexing, so all of the Lucene tools and APIs are available to interact with the generated index. Since this has the potential to cause confusion, it is worth mentioning that the Lucene index format has a concept of segments, too, and these are different from Nutch segments. A Lucene segment is a portion of a Lucene index, whereas a Nutch segment is a fetched and indexed portion of the WebDB.

View gora-hbase-mapping.xml for more details

\subsubsection{Config}

Config nutch run intellij
Copy file

copy all the files in the runtime/conf on out/test/apache-Nutch-2.3 and out/production/apache-Nutch-2.3

add these lines to file $NUTCH_SRC/out/test/nutch-site.xml

<property>
   <name>plugin.folders</name>
   <value><nutch_src>/build/plugins</value>
 </property>
Run nutch in intellij
Run->Edit Configurations...->add path agrs:path to file list links crawler
Dev Nutch in Intellij
Receipts: IntellIJ 14, Apache Nutch 2.3

1. Get Nutch source

wget http://www.eu.apache.org/dist/nutch/2.3/apache-nutch-2.3-src.tar.gz
tar -xzf apache-nutch-2.3-src.tar.gz
2. Import Nutch source in IntellIJ

[wonderplugin_slider id="1"]

3. Get Dependencies by Ant

[wonderplugin_slider id="3"]

4. Import Dependencies to IntellIJ

[wonderplugin_slider id="4"]

Nutch Dev
1.Intasll java in ubuntu

-Downloads java version .zip

 http://www.oracle.com/technetwork/java/javase/downloads/jdk7-downloads-1880260.html
-Create folder jvm

 sudo mkdir /usr/lib/jvm/
-Cd to folder downloads java version .zip

 sudo mv jdk1.7.0_x/ /usr/lib/jvm/jdk1.7.0_x
-Run command line

  sudo update-alternatives --install /usr/bin/java java /usr/lib/jvm/jdk1.7.0_x/jre/bin/java 0
-Tets version java

  java -version
2.Intasll ant in ubuntu

-Downloads ant

http://ant.apache.org/manualdownload.cgi
-Add path ant vao file environment

 sudo nano /etc/environment
 $ANT_ROOT/bin
-Run command line

source /etc/environment
ant -version
3.Intasll hbase in ubuntu

-Downloads and extract hbase 0.94.27

  https://archive.apache.org/dist/hbase/hbase-0.94.27/
-Edit file $HABSE_ROOT/conf/hbase-site.xml

 <configuration>
  <property>
    <name>hbase.rootdir</name>
    <value>file:///$PATH_DATA_BASE/database</value>
  </property>
  <property>
    <name>hbase.cluster.distributed</name>
    <value>false</value>
  </property>
  <property>
    <name>hbase.zookeeper.quorum</name>
    <value>hbase.io</value>
  </property>
  <property>
    <name>zookeeper.znode.parent</name>
    <value>/hbase-unsecure</value>
  </property>
  <property>
    <name>hbase.rpc.timeout</name>
    <value>2592000000</value>
  </property>
</configuration>
-Edit file $HBASE_ROOT/conf/hbase-env.sh

  export JAVA_HOME=$PATH_JAVA_HOME
-Edit file $HBASE_ROOT/conf/regionservers

hbase.io.nutch
-Edit file hosts in ubuntu

  sudo nano /etc/hosts
  {ip} hbase.io.nutch
-Edit file hostname in ubuntu

 sudo nano /etc/hostname
 hbase.io.nutch
-Run and stop hbase in ubuntu

 Run hbase : cd $HBASE_ROOT/bin ./start-hbase.sh
 Stop hbase: cd $HBASE_ROOT/bin ./stop-hbase.sh
*Error in intasll hbase

- Error regionserver localhost(Edit file hosts and file host name)
- Error client no remote server intasll hbase(Turn off file firewall)
4.Build nutch in ant

-Downloads and extract nutch

  http://nutch.apache.org/
-Edit file $NUTCH_ROOT/ivy/ivy.xml

 <dependency org="org.apache.gora" name="gora-hbase" rev="0.5"
conf="*->default" />
-Edit file $NUTCH_ROOT/ivy/ivysettings.xml

 #<property name="repo.maven.org"
 #   value="http://repo1.maven.org/maven2/"
 #  override="false"/>

<property name = "repo.maven.org"
   value = "http://maven.oschina.net/content/groups/public/"
   override = "false" />
-Edit file $NUTCH_ROOT/conf/nutch-site.xml

<configuration>
<property>
   <name>plugin.folders</name>
   <value>$NUTCH_ROOT/build/plugins</value>
 </property>
<property>
        <name>http.agent.name</name>
        <value>mycrawlername</value>
        <!-- this can be changed to something more sane if you like -->
    </property>
    <property>
        <name>http.robots.agents</name>
        <value>mycrawlername</value>
        <!-- this is the robot name we're looking for in robots.txt files -->
    </property>
    <property>
        <name>storage.data.store.class</name>
        <value>org.apache.gora.hbase.store.HBaseStore</value>
    </property>
    <property>
        <name>plugin.includes</name>
        <!-- do \*\*NOT\*\* enable the parse-html plugin, if you want proper HTML parsing. Use something like parse-tika! -->
        <value>
            protocol-http|protocol-httpclient|urlfilter-regex|parse-(text|tika|js)|index-(basic|anchor)|query-(basic|site|url)|response-(json|xml)|summary-basic|scoring-opic|urlnormalizer-(pass|regex|basic)|indexer-elastic|index-metadata|index-more
        </value>
    </property>
    <property>
        <name>db.ignore.external.links</name>
        <value>true</value>
        <!-- do not leave the seeded domains (optional) -->
    </property>


<!-- elasticsearch index properties -->
<property>
  <name>elastic.host</name>
  <value>localhost</value>
  <description>The hostname to send documents to using TransportClient.
  Either host and port must be defined or cluster.
  </description>
</property>

<property>
  <name>elastic.port</name>
  <value>9300</value>
  <description>
  The port to connect to using TransportClient.
  </description>
</property>
<property>
  <name>elastic.index</name>
  <value>nutch</value>
  <description>
  The name of the elasticsearch index. Will normally be autocreated if it
  doesn't exist.
  </description>
</property>
<!-- end index -->

<property>
        <name>http.proxy.host</name>
        <value>192.168.80.1</value>
    </property>
    <property>
        <name>http.proxy.port</name>
        <value>8080</value>
    </property>
    <property>
        <name>http.proxy.username</name>
        <value>user1</value>
    </property>
    <property>
        <name>http.proxy.password</name>
        <value>user1</value>
    </property>
</configuration>
-Edit file file $NUTCH_ROOT/conf/gora.property

 gora.datastore.default=org.apache.gora.hbase.store.HBaseStore
-Build nucth

 ant runtime
 or
 ant eclipse -verbose
-Create file links

-Run nutch

 cd $NUTCH_ROOT/runtime/local/bin
 run inject : ./nutch inject file:///$PATH_LIKNS
 run generate : ./nutch generate -topN 10
 run fetch : ./nutch fetch -all
 run parse : ./nutch parse -all
 run updatedb : ./nutch updatedb -all
-Downloads and extract elastic

 https://www.elastic.co/downloads/elasticsearch
-Run elastic

cd $ELASTIC/bin
./elasticsearch
-Index data in elastic

 cd $NUTCH_ROOT/runtime/bin
 run index : ./nutch index -all
5.Run nutch intellij

Change $NUTCH_ROOT/runtime/local/conf/hbase-site.xml

<configuration>
<property>
<name>hbase.rootdir</name>
<value>file:///home/hainv/Downloads/crawler/data</value>
</property>
<property>
<name>hbase.cluster.distributed</name>
<value>false</value>
</property>
<property>
<name>hbase.zookeeper.quorum</name>
<value>hbase.io</value>
</property>
<property>
<name>zookeeper.znode.parent</name>
<value>/hbase-unsecure</value>
</property>
<property>
<name>hbase.rpc.timeout</name>
<value>2592000000</value>
</property>
</configuration>
Nutch plugin intellij
1.Structure nutch :[1]
2.Run nutch intellij
Downloads nucth2.3:http://nutch.apache.org/downloads.html Editing file $NUTCH_ROOT/ivy/ivysettings.xml

<ivysettings>
  <property name="oss.sonatype.org"
    value="http://oss.sonatype.org/content/repositories/releases/"
    override="false"/>
  <property name = "repo.maven.org"
      value = "http://maven.oschina.net/content/groups/public/"
      override = "false" />
  <property name="repository.apache.org"
    value="https://repository.apache.org/content/repositories/snapshots/"
    override="false"/>
  <property name="maven2.pattern"
    value="[organisation]/[module]/[revision]/[module]-[revision]"/>
  <property name="maven2.pattern.ext"
    value="${maven2.pattern}.[ext]"/>
  <!-- pull in the local repository -->
  <include url="${ivy.default.conf.dir}/ivyconf-local.xml"/>
  <settings defaultResolver="default"/>
  <resolvers>
    <ibiblio name="maven2"
      root="${repo.maven.org}"
      pattern="${maven2.pattern.ext}"
      m2compatible="true"
      />
    <ibiblio name="apache-snapshot"
      root="${repository.apache.org}"
      changingPattern=".*-SNAPSHOT"
      m2compatible="true"
      />
    <ibiblio name="restlet"
      root="http://maven.restlet.org"
      pattern="${maven2.pattern.ext}"
      m2compatible="true"
      />
     <ibiblio name="sonatype"
      root="${oss.sonatype.org}"
      pattern="${maven2.pattern.ext}"
      m2compatible="true"
      />

    <chain name="default" dual="true">
      <resolver ref="local"/>
      <resolver ref="maven2"/>
      <resolver ref="sonatype"/>
      <resolver ref="apache-snapshot"/>
    </chain>
    <chain name="internal">
      <resolver ref="local"/>
    </chain>
    <chain name="external">
      <resolver ref="maven2"/>
      <resolver ref="sonatype"/>
    </chain>
    <chain name="external-and-snapshots">
      <resolver ref="maven2"/>
      <resolver ref="apache-snapshot"/>
      <resolver ref="sonatype"/>
    </chain>
    <chain name="restletchain">
      <resolver ref="restlet"/>
    </chain>
  </resolvers>
  <modules>
    <module organisation="org.apache.nutch" name=".*" resolver="internal"/>
    <module organisation="org.restlet" name=".*" resolver="restletchain"/>
    <module organisation="org.restlet.jse" name=".*" resolver="restletchain"/>
  </modules>
</ivysettings>
Editing file $NUTCH_ROOT/ivy/ivy.xml

<dependency org="org.apache.gora" name="gora-hbase" rev="0.5" conf="*->default" />
Editing file $NUCTH_ROOT/conf/gora.properties

gora.datastore.default=org.apache.gora.hbase.store.HBaseStore
Editing file $NUTCH_ROOT/conf/nutch_site.xml

<configuration>
<property>
   <name>plugin.folders</name>
   <value>$NUTCH_ROOT/build/plugins</value>
 </property>
<property>
        <name>http.agent.name</name>
        <value>mycrawlername</value>
        <!-- this can be changed to something more sane if you like -->
    </property>
    <property>
        <name>http.robots.agents</name>
        <value>mycrawlername</value>
        <!-- this is the robot name we're looking for in robots.txt files -->
    </property>
    <property>
        <name>storage.data.store.class</name>
        <value>org.apache.gora.hbase.store.HBaseStore</value>
    </property>
    <property>
        <name>plugin.includes</name>
        <!-- do \*\*NOT\*\* enable the parse-html plugin, if you want proper HTML parsing. Use something like parse-tika! -->
        <value>
            protocol-httpclient|urlfilter-regex|parse-(text|tika|js)|index-(basic|anchor)|query-(basic|site|url)|response-(json|xml)|summary-basic|scoring-opic|urlnormalizer-(pass|regex|basic)|indexer-elastic
        </value>
    </property>
    <property>
        <name>db.ignore.external.links</name>
        <value>true</value>
        <!-- do not leave the seeded domains (optional) -->
    </property>
    <property>
        <name>elastic.host</name>
        <value>localhost</value>
        <!-- where is ElasticSearch listening -->
    </property>

<property>
        <name>http.proxy.host</name>
        <value>192.168.80.1</value>
        <description>The proxy hostname. If empty, no proxy is used.</description>
    </property>
    <property>
        <name>http.proxy.port</name>
        <value>8080</value>
        <description>The proxy port.</description>
    </property>
    <property>
        <name>http.proxy.username</name>
        <value>user1</value>
        <description>Username for proxy. This will be used by 'protocol-httpclient', if the proxy server requests basic,
            digest
            and/or NTLM authentication. To use this, 'protocol-httpclient' must be present in the value of
            'plugin.includes'
            property. NOTE: For NTLM authentication, do not prefix the username with the domain, i.e. 'susam' is correct
            whereas
            'DOMAINsusam' is incorrect.
        </description>
    </property>
    <property>
        <name>http.proxy.password</name>
        <value>user1</value>
        <description>Password for proxy. This will be used by 'protocol-httpclient', if the proxy server requests basic,
            digest
            and/or NTLM authentication. To use this, 'protocol-httpclient' must be present in the value of
            'plugin.includes'
            property.
        </description>
    </property>
</configuration>
Editing file $NUCTH_ROOT/conf/hbase-site.xml

<configuration>
    <property>
        <name>hbase.rootdir</name>
        <value>file:///home/rombk/Downloads/database</value>
    </property>
    <property>
        <name>hbase.cluster.distributed</name>
        <value>false</value>
    </property>
    <property>
        <name>hbase.zookeeper.quorum</name>
        <value>hbase.io</value>
    </property>
    <property>
        <name>zookeeper.znode.parent</name>
        <value>/hbase-unsecure</value>
    </property>
    <property>
        <name>hbase.rpc.timeout</name>
        <value>2592000000</value>
    </property>
</configuration>
Run terminal

 ant eclipse -verbose
Import nucth intellij


3.Run plugin creativecommons
Sample plugins that parse and index Creative Commons medadata.1 Step 1. Create folder creativecommons in path $NUTCH_HOME/out/test/

Step 2. Create file nutch-site.xml in folder $NUTCH_HOME/out/test/creativecommons and add content

<?xml version="1.0"?>
<?xml-stylesheet type="text/xsl" href="configuration.xsl"?>
<!-- Put site-specific property overrides in this file. -->
<configuration>
<property>
   <name>plugin.folders</name>
   <value>$NUTCH_HOME/build/plugins</value>
 </property>
<property>
   <name>http.agent.name</name>
   <value>mycrawlername</value>
<!-- this can be changed to something more sane if you like -->
</property>
<property>
   <name>http.robots.agents</name>
   <value>mycrawlername</value>
<!-- this is the robot name we're looking for in robots.txt files -->
</property>
<property>
   <name>storage.data.store.class</name>
   <value>org.apache.gora.hbase.store.HBaseStore</value>
</property>
<property>
   <name>plugin.includes</name>
  <!-- do \*\*NOT\*\* enable the parse-html plugin, if you want proper HTML parsing. Use something like parse-tika! -->
  <value>indexer-elastic|creativecommons|parse-html</value>
</property>
<property>
   <name>db.ignore.external.links</name>
   <value>true</value>
<!-- do not leave the seeded domains (optional) -->
</property>
<property>
   <name>elastic.host</name>
   <value>localhost</value>
<!-- where is ElasticSearch listening -->
</property>
<!-- config proxy-->
<property>
   <name>http.proxy.host</name>
   <value><hosts></value>
   <description>The proxy hostname. If empty, no proxy is used.</description>
</property>
<property>
   <name>http.proxy.port</name>
   <value><port></value>
   <description>The proxy port.</description>
</property>
<property>
   <name>http.proxy.username</name>
   <value><user1></value>
   <description>Username for proxy. This will be used by 'protocol-httpclient', if the proxy server requests basic,
digest
and/or NTLM authentication. To use this, 'protocol-httpclient' must be present in the value of
'plugin.includes'
property. NOTE: For NTLM authentication, do not prefix the username with the domain, i.e. 'susam' is correct
whereas
'DOMAINsusam' is incorrect.
     </description>
</property>
<property>
   <name>http.proxy.password</name>
   <value><user1></value>
   <description>Password for proxy. This will be used by 'protocol-httpclient', if the proxy server requests basic,
digest
and/or NTLM authentication. To use this, 'protocol-httpclient' must be present in the value of
'plugin.includes'
property.
    </description>
</property>
</configuration>
2.Run plugin feed
Plugin feed parsing of rss Error : Parsing of RSS feeds fails (tejasp) [2] and read file $NUTCH_ROOT/CHANFES.txt


%  \part{Lập trình}

\chapter{Lập trình là gì?}

\section{Các vấn đề lập trình}

Các vấn đề lập trình với từng ngôn ngữ

\subsection{Nhập môn}

Phần 1: Cơ bản

\begin{lstlisting}
├── 1. introduction
├── 2. syntax
├── 3. data structure
├── 4. oop
\end{lstlisting}

Phần 2: Xây dựng ứng dụng

\begin{lstlisting}
├── 16. database
├── 5. networking
├── 6. os
├── 14. ui
├── 15. web
\end{lstlisting}

Phần 3: Các chủ đề nâng cao

\begin{lstlisting}
├── 7. parallel
├── 8. event based
├── 9. error handling
├── 10. logging
\end{lstlisting}

Phần 4: Phát triển phần mềm chuyên nghiệp

\begin{lstlisting}
├── 11. configuration
├── 12. documentation
├── 13. test
├── 17. ide
├── 18. package manager
├── 19. build tool
├── 20. make module
└── 21. production (docker)
\end{lstlisting}

\section{Introduction}

Installation (environment, IDE)

Hello world

Courses

Resources


\section{Syntax}

variables and expressions

conditional

loops and Iteration

functions

define, use

parameters

scope of variables

anonymous functions

callbacks

self-invoking functions, inner functions

functions that return functions, functions that redefined themselves

closures

naming convention

comment convention

\section{Cấu trúc dữ liệu}

Number

String

Collection

DateTime

Boolean

Object

\section{Lập trình hướng đối tượng}

Classes & Objects

Inheritance

Encapsulation

Abstraction

Polymorphism

For OOP Example: see Python: OOP

\subsection{Bài tập}

\textbf{Quản lý tài khoản ngân hàng}

\section{Networking}

REST (example with chat app sender, receiver, message)

\subsection{Bài tập}

Guess My Number Game

\section{GUI - Giao diện}

Quản lý hot girl

Quản lý truyện tranh

Create Analog Clock

Chương trình lịch âm dương

Chương trình học từ tiếng Anh

\section{Game}

\begin{itemize}
  \item Create Pong Game
  \item Create flappy bird
  \item Create Bouncing Game
\end{itemize}


\section{Cơ sở dữ liệu}

\subsection{Thử thách}


\section{How to ask a question}

Focus on questions about an actual problem you have faced. Include details about what you have tried and exactly what you are trying to do.

Ask about...

✔ Specific programming problems

✔ Software algorithms

✔ Coding techniques

✔ Software development tools

Not all questions work well in our format. Avoid questions that are primarily opinion-based, or that are likely to generate discussion rather than answers.

Don't ask about...

✖ Questions you haven't tried to find an answer for (show your work!)

✖ Product or service recommendations or comparisons

✖ Requests for lists of things, polls, opinions, discussions, etc.

✖ Anything not directly related to writing computer programs

\section{Các nguyên tắc lập trình}

Generic

KISS (Keep It Simple Stupid)

YAGNI

Do The Simplest Thing That Could Possibly Work

Keep Things DRY

Code For The Maintainer

Avoid Premature Optimization

Inter-Module/Class

Minimise Coupling

Law of Demeter

Composition Over Inheritance

Orthogonality

Module/Class

Maximise Cohesion

Liskov Substitution Principle

Open/Closed Principle

Single Responsibility Principle

Hide Implementation Details

Curly's Law

Software Quality Laws

First Law of Software Quality


\section{Các mô hình lập trình}

Main paradigm approaches 1

1. Imperative


Description:

Computation as statements that directly change a program state (datafields)

Main Characteristics:

Direct assignments, common data structures, global variables

Critics: Edsger W. Dijkstra, Michael A. Jackson

Examples: Assembly, C, C++, Java, PHP, Python

2. Structured

Description:

A style of imperative programming with more logical program structure

Main Characteristics:

Structograms, indentation, either no, or limited use of, goto statements

Examples: C, C++, Java, Python

3. Procedural

Description:

Derived from structured programming, based on the concept of modular programming or the procedure call

Main Characteristics:

Local variables, sequence, selection, iteration, and modularization

Examples: C, C++, Lisp, PHP, Python

4. Functional


Description:

Treats computation as the evaluation of mathematical functions avoiding state and mutable data

Main Characteristics:

Lambda calculus, compositionality, formula, recursion, referential transparency, no side effects

Examples: Clojure, Coffeescript, Elixir, Erlang, F#, Haskell, Lisp, Python, Scala, SequenceL, SML

5. Event-driven including time driven

Description:

Program flow is determined mainly by events, such as mouse clicks or interrupts including timer

Main Characteristics:

Main loop, event handlers, asynchronous processes

Examples: Javascript, ActionScript, Visual Basic

6. Object-oriented

Description:

Treats datafields as objects manipulated through pre-defined methods only

Main Characteristics:

Objects, methods, message passing, information hiding, data abstraction, encapsulation, polymorphism, inheritance, serialization-marshalling

Examples: Common Lisp, C++, C#, Eiffel, Java, PHP, Python, Ruby, Scala

7. Declarative

Description:

Defines computation logic without defining its detailed control flow

Main Characteristics:

4GLs, spreadsheets, report program generators

Examples: SQL, regular expressions, CSS, Prolog

8. Automata-based programming

Description:

Treats programs as a model of a finite state machine or any other formal automata

Main Characteristics:

State enumeration, control variable, state changes, isomorphism, state transition table

Examples: AsmL

\section{Testing}

\includegraphics{programming/introduction/unit_test_tdd}

1. Definition 1 2

Test-driven development (TDD) is a software development process that relies on the repetition of a very short development cycle:

\includegraphics[width=\linewidth]{programming/introduction/tdd.jpg}

Step 1: First the developer writes an (initially failing) automated test case that defines a desired improvement or new function,

Step 2: Then produces the minimum amount of code to pass that test,

Step 3: Finally refactors the new code to acceptable standards.

Kent Beck, who is credited with having developed or 'rediscovered' the technique, stated in 2003 that TDD encourages simple designs and inspires confidence.

2. Principles 2

Kent Beck defines

Never with a single line of code unless you have a failing automated test.
Eliminate duplication
Red: (Automated test fail) Green (Automated test pass because dev code has been written) Refactor (Eliminate duplication, Clean the code)

3. Assertions & Assert Framework

\includegraphics[width=\linewidth]{programming/introduction/tdd_assertion.png}

Assert that the expected results have occurred.
[code lang="java"] @Test public void test() { assertEquals(2, 1 + 1); } [/code]


4. Test Runners 3

\includegraphics[width=\linewidth]{programming/introduction/tdd_test_runner.png}

When testing a large real-world web app there may be tens or hundreds of test cases, and we certainly don't want to run each one manually. In such as scenario we need to use a test runner to find and execute the tests for us, and in this article we'll explore just that.

A test runner provides the a good basis for a real testing framework. A test runner is designed to run tests, tag tests with attributes (annotations), and provide reporting and other features.

In general you break your tests up into 3 standard sections; setUp(), tests, and tearDown(), typical for a test runner setup.

The setUp() and tearDown() methods are run automatically for every test, and contain respectively:

The setup steps you need to take before running the test, such as unlocking the screen and killing open apps.
The cooldown steps you need to run after the test, such as closing the Marionette session.

5. Test Frameworks

Language	Test Frameworks
C++/VisualStudio	C++: Test
Web Service	rest-assured
Web UI	SeleniumHQ

\section{Logging}

Logging is the process of recording application actions and state to a secondary interface.

\includegraphics[width=\linewidth]{programming/introduction/logging}

Logging System

\includegraphics[width=\linewidth]{programming/introduction/logging_system}

Levels

Level	When it’s used
DEBUG	Detailed information, typically of interest only when diagnosing problems.
INFO	Confirmation that things are working as expected.
WARNING	An indication that something unexpected happened, or indicative of some problem in the near future (e.g. ‘disk space low’). The software is still working as expected.

ERROR

Due to a more serious problem, the software has not been able to perform some function.
CRITICAL	A serious error, indicating that the program itself may be unable to continue running.
Best Practices 2 4 5
Logging should always be considered when handling an exception but should never take the place of a real handler.
Keep all logging code in your production code. Have an ability to enable more/less detailed logging in production, preferably per subsystem and without restarting your program.
Make logs easy to parse by grep and by eye. Stick to several common fields at the beginning of each line. Identify time, severity, and subsystem in every line. Clearly formulate the message. Make every log message easy to map to its source code line.
If an error happens, try to collect and log as much information as possible. It may take long but it's OK because normal processing has failed anyway. Not having to wait when the same condition happens in production with a debugger attached is priceless.

\section{Lập trình hàm}

Functional
Without mutable variables, assignment, conditional

Advantages 1
Most functional languages provide a nice, protected environment, somewhat like JavaLanguage. It's good to be able to catch exceptions instead of having CoreDumps in stability-critical applications.
FP encourages safe ways of programming. I've never seen an OffByOne mistake in a functional program, for example... I've seen one. Adding two lengths to get an index but one of them was zero-indexed. Easy to discover though. -- AnonymousDonor
Functional programs tend to be much more terse than their ImperativeLanguage counterparts. Often this leads to enhanced programmer productivity.
FP encourages quick prototyping. As such, I think it is the best software design paradigm for ExtremeProgrammers... but what do I know.
FP is modular in the dimension of functionality, where ObjectOrientedProgramming is modular in the dimension of different components.
Generic routines (also provided by CeePlusPlus) with easy syntax. ParametricPolymorphism
The ability to have your cake and eat it. Imagine you have a complex OO system processing messages - every component might make state changes depending on the message and then forward the message to some objects it has links to. Wouldn't it be just too cool to be able to easily roll back every change if some object deep in the call hierarchy decided the message is flawed? How about having a history of different states?
Many housekeeping tasks made for you: deconstructing data structures (PatternMatching), storing variable bindings (LexicalScope with closures), strong typing (TypeInference), * GarbageCollection, storage allocation, whether to use boxed (pointer-to-value) or unboxed (value directly) representation...
Safe multithreading! Immutable data structures are not subject to data race conditions, and consequently don't have to be protected by locks. If you are always allocating new objects, rather than destructively manipulating existing ones, the locking can be hidden in the allocation and GarbageCollection system.

\section{Lập trình song song}

Paralell/Concurrency Programming
1. Callback Pattern 2
Callback functions are derived from a programming paradigm known as functional programming. At a fundamental level, functional programming specifies the use of functions as arguments. Functional programming was—and still is, though to a much lesser extent today—seen as an esoteric technique of specially trained, master programmers.

Fortunately, the techniques of functional programming have been elucidated so that mere mortals like you and me can understand and use them with ease. One of the chief techniques in functional programming happens to be callback functions. As you will read shortly, implementing callback functions is as easy as passing regular variables as arguments. This technique is so simple that I wonder why it is mostly covered in advanced JavaScript topics.

[code lang="javascript"] function getN(){ return 10; }

var n = getN();

function getAsyncN(callback){ setTimeout(function(){ callback(10); }, 1000); }

function afterGetAsyncN(result){ var n = 10; console.log(n); }

getAsyncN(afterGetAsyncN); [/code]

2. Promise Pattern 1 3
What is a promise?
The core idea behind promises is that a promise represents the result of an asynchronous operation.

A promise is in one of three different states:

pending - The initial state of a promise.
fulfilled - The state of a promise representing a successful operation.
rejected - The state of a promise representing a failed operation.
Once a promise is fulfilled or rejected, it is immutable (i.e. it can never change again).


\begin{lstlisting}[language=Javscript]
function aPromise(message){
  return new Promise(function(fulfill, reject){
    if(message == "success"){
      fulfill("it is a success Promise");
    } if(message == "fail"){
      reject("it is a fail Promise");
    }
  });
}
\end{lstlisting}

Usage:

\begin{lstlisting}[language=Javascript]
aPromise("success").then(function(successMessage){
  console.log(successMessage) }, function(failMessage){
  // it is a success Promise
  console.log(failMessage)
})
\end{lstlisting}

\begin{lstlisting}[language=Javascript]
aPromise("fail").then(function(successMessage){
  console.log(successMessage) }, function(failMessage){
  console.log(failMessage)
}) // it is a fail Promise
\end{lstlisting}

\section{IDE - Môi trường phát triển tích hợp}

An integrated development environment (IDE) is a software application that provides comprehensive facilities to computer programmers for software development. An IDE normally consists of a source code editor, build automation tools and a debugger. Most modern IDEs have intelligent code completion.

1. Navigation

Word Navigation Line Navigation File Navigation

2. Editing

Auto Complete Code Complete Multicursor Template (Snippets)

3. Formatting

Debugging
Custom Rendering for Object
%\chapter{Python}

\section{Giới thiệu}

\begin{item}
  \item `Python` is a widely used general-purpose, high-level programming language. Its design philosophy emphasizes code readability, and its syntax allows programmers to express concepts in fewer lines of code than would be possible in languages such as C++ or Java.
  \item The language provides constructs intended to enable clear programs on both a small and large scale.
\end{item}

Python Tutorial
Python is a general-purpose interpreted, interactive, object-oriented, and high-level programming language. It was created by Guido van Rossum during 1985- 1990. Like Perl, Python source code is also available under the GNU General Public License (GPL). This tutorial gives enough understanding on Python programming language.

Python is Interpreted

Python is processed at runtime by the interpreter. You do not need to compile your program before executing it. This is similar to PERL and PHP.

Python is Interactive

You can actually sit at a Python prompt and interact with the interpreter directly to write your programs.

Python is Object-Oriented

Python supports Object-Oriented style or technique of programming that encapsulates code within objects.

Python is Beginner Friendly

Python is a great language for the beginner-level programmers and supports the development of a wide range of applications from simple text processing to WWW browsers to games.

Audience
This tutorial is designed for software programmers who need to learn Python programming language from scratch.


\textbf{Sách}

\href{https://docs.google.com/document/d/1gQFMXZtynpuTenoOQNGCHttArT4NspTWcyJQr5ps9Mk/edit?usp=sharing}{Tập hợp các sách python}

\textbf{Khoá học}

\href{1frO9QYhgsXbMzcyXoA4czWkxTWF8RBTJVf9uoO1rElU}{Tập hợp các khóa học python}

\textbf{Tham khảo}

\href{http://blog.tryolabs.com/2015/12/15/top-10-python-libraries-of-2015/}{Top 10 Python Libraries Of 2015}

\section{Cài đặt}

Get Started
Welcome! This tutorial details how to get started with Python.

For Windows
Anaconda 4.3.0
Anaconda is BSD licensed which gives you permission to use Anaconda commercially and for redistribution.

1. Download the installer
2. Optional: Verify data integrity with MD5 or SHA-256
3. Double-click the .exe file to install Anaconda and follow the instructions on the screen
Python 3.6 version
64-BIT INSTALLER
Python 2.7 version
64-BIT INSTALLER
Step 2. Discover the Map

https://docs.python.org/2/library/index.html

For CentOS
Developer tools
The Development tools will allow you to build and compile software from source code. Tools for building RPMs are also included, as well as source code management tools like Git, SVN, and CVS.

\begin{lstlisting}[language=bash]
yum groupinstall "Development tools"
yum install zlib-devel
yum install bzip2-devel
yum install openssl-devel
yum install ncurses-devel
yum install sqlite-devel
\end{lstlisting}

Python & Anaconda
Anaconda is BSD licensed which gives you permission to use Anaconda commercially and for redistribution.

\begin{lstlisting}[language=bash]
cd /opt
wget --no-check-certificate https://www.python.org/ftp/python/2.7.6/Python-2.7.6.tar.xz
tar xf Python-2.7.6.tar.xz
cd Python-2.7.6
./configure --prefix=/usr/local
make && make altinstall
## link
ln -s /usr/local/bin/python2.7 /usr/local/bin/python
# final check
which python
python -V
# install Anaconda
cd ~/Downloads
wget https://repo.continuum.io/archive/Anaconda-2.3.0-Linux-x86_64.sh
bash ~/Downloads/Anaconda-2.3.0-Linux-x86_64.sh
\end{lstlisting}

\section{Cơ bản}

\section{Cú pháp cơ bản}

Print, print

\begin{lstlisting}[language=python]
print "Hello World"
\end{lstlisting}


Conditional

\begin{lstlisting}[language=Python]
if you_smart:
    print "learn python"
else:
    print "go away"
\end{lstlisting}

Loop

In general, statements are executed sequentially: The first statement in a function is executed first, followed by the second, and so on. There may be a situation when you need to execute a block of code several number of times.

Programming languages provide various control structures that allow for more complicated execution paths. A loop statement allows us to execute a statement or group of statements multiple times. The following diagram illustrates a loop statement


Python programming language provides following types of loops to handle looping requirements.

while loop	Repeats a statement or group of statements while a given condition is TRUE. It tests the condition before executing the loop body.
for loop	Executes a sequence of statements multiple times and abbreviates the code that manages the loop variable.
nested loops	You can use one or more loop inside any another while, for or do..while loop.
While Loop
A while loop statement in Python programming language repeatedly executes a target statement as long as a given condition is true.

Syntax

The syntax of a while loop in Python programming language is

\begin{lstlisting}[language=Python]
while expression:
   statement(s)
\end{lstlisting}

Example

\begin{lstlisting}[language=Python]
count = 0
while count < 9:
   print 'The count is:', count
   count += 1
print "Good bye!"
\end{lstlisting}


For Loop

It has the ability to iterate over the items of any sequence, such as a list or a string.

Syntax

\begin{lstlisting}[language=Python]
for iterating_var in sequence:
   statements(s)
\end{lstlisting}

If a sequence contains an expression list, it is evaluated first. Then, the first item in the sequence is assigned to the iterating variable iterating_var. Next, the statements block is executed. Each item in the list is assigned to iterating_var, and the statement(s) block is executed until the entire sequence is exhausted.

Example

\begin{lstlisting}[language=Python]
for i in range(10):
    print "hello", i

for letter in 'Python':
   print 'Current letter :', letter

fruits = ['banana', 'apple',  'mango']
for fruit in fruits:
   print 'Current fruit :', fruit

print "Good bye!"
\end{lstlisting}

Yield and Generator

Yield is a keyword that is used like return, except the function will return a generator.

\begin{lstlisting}[language=Python]
def createGenerator():
    yield 1
    yield 2
    yield 3
mygenerator = createGenerator() # create a generator
print(mygenerator) # mygenerator is an object!
# <generator object createGenerator at 0xb7555c34>
for i in mygenerator:
    print(i)
# 1
# 2
# 3
\end{lstlisting}


Visit Yield and Generator explained for more information

Functions

Variable-length arguments

\begin{lstlisting}[language=Python]
def functionname([formal_args,] *var_args_tuple ):
   "function_docstring"
   function_suite
   return [expression]
\end{lstlisting}

Example

\begin{lstlisting}[language=Python]
#!/usr/bin/python

# Function definition is here
def printinfo( arg1, *vartuple ):
   "This prints a variable passed arguments"
   print "Output is: "
   print arg1
   for var in vartuple:
      print var
   return;

# Now you can call printinfo function
printinfo( 10 )
printinfo( 70, 60, 50 )
\end{lstlisting}

Coding Convention
Code layout
Indentation: 4 spaces

Suggest Readings

"Python Functions". www.tutorialspoint.com
"Python Loops". www.tutorialspoint.com
"What does the “yield” keyword do?". stackoverflow.com
"Improve Your Python: 'yield' and Generators Explained". jeffknupp.com

\textbf{Vấn đề với mảng}

\begin{item}
  \item Random Sampling \footnote{tham khảo [pytorch](http://pytorch.org/docs/master/torch.html?highlight=randn#torch.randn), [numpy](https://docs.scipy.org/doc/numpy-1.13.0/reference/routines.random.html))} - sinh ra một mảng ngẫu nhiên trong khoảng (0, 1), mảng ngẫu nhiên số nguyên trong khoảng (x, y), mảng ngẫu nhiên là permutation của số từ 1 đến n
\end{item}

\section{Yield and Generators}

Coroutines and Subroutines
When we call a normal Python function, execution starts at function's first line and continues until a return statement, exception, or the end of the function (which is seen as an implicit return None) is encountered. Once a function returns control to its caller, that's it. Any work done by the function and stored in local variables is lost. A new call to the function creates everything from scratch.

This is all very standard when discussing functions (more generally referred to as subroutines) in computer programming. There are times, though, when it's beneficial to have the ability to create a "function" which, instead of simply returning a single value, is able to yield a series of values. To do so, such a function would need to be able to "save its work," so to speak.

I said, "yield a series of values" because our hypothetical function doesn't "return" in the normal sense. return implies that the function is returning control of execution to the point where the function was called. "Yield," however, implies that the transfer of control is temporary and voluntary, and our function expects to regain it in the future.

In Python, "functions" with these capabilities are called generators, and they're incredibly useful. generators (and the yield statement) were initially introduced to give programmers a more straightforward way to write code responsible for producing a series of values. Previously, creating something like a random number generator required a class or module that both generated values and kept track of state between calls. With the introduction of generators, this became much simpler.

To better understand the problem generators solve, let's take a look at an example. Throughout the example, keep in mind the core problem being solved: generating a series of values.

Note: Outside of Python, all but the simplest generators would be referred to as coroutines. I'll use the latter term later in the post. The important thing to remember is, in Python, everything described here as a coroutine is still a generator. Python formally defines the term generator; coroutine is used in discussion but has no formal definition in the language.

Example: Fun With Prime Numbers
Suppose our boss asks us to write a function that takes a list of ints and returns some Iterable containing the elements which are prime1 numbers.

Remember, an Iterable is just an object capable of returning its members one at a time.

"Simple," we say, and we write the following:

\begin{lstlisting}[language=Python]
def get_primes(input_list):
    result_list = list()
    for element in input_list:
        if is_prime(element):
            result_list.append()

    return result_list
\end{lstlisting}

or better yet...

\begin{lstlisting}[language=Python]
def get_primes(input_list):
    return (element for element in input_list if is_prime(element))

# not germane to the example, but here's a possible implementation of
# is_prime...

def is_prime(number):
    if number > 1:
        if number == 2:
            return True
        if number % 2 == 0:
            return False
        for current in range(3, int(math.sqrt(number) + 1), 2):
            if number % current == 0:
                return False
        return True
    return False
\end{lstlisting}

Either get_primes implementation above fulfills the requirements, so we tell our boss we're done. She reports our function works and is exactly what she wanted.

Dealing With Infinite Sequences
Well, not quite exactly. A few days later, our boss comes back and tells us she's run into a small problem: she wants to use our get_primes function on a very large list of numbers. In fact, the list is so large that merely creating it would consume all of the system's memory. To work around this, she wants to be able to call get_primes with a start value and get all the primes larger than start (perhaps she's solving Project Euler problem 10).

Once we think about this new requirement, it becomes clear that it requires more than a simple change to get_primes. Clearly, we can't return a list of all the prime numbers from start to infinity (operating on infinite sequences, though, has a wide range of useful applications). The chances of solving this problem using a normal function seem bleak.

Before we give up, let's determine the core obstacle preventing us from writing a function that satisfies our boss's new requirements. Thinking about it, we arrive at the following: functions only get one chance to return results, and thus must return all results at once. It seems pointless to make such an obvious statement; "functions just work that way," we think. The real value lies in asking, "but what if they didn't?"

Imagine what we could do if get_primes could simply return the next value instead of all the values at once. It wouldn't need to create a list at all. No list, no memory issues. Since our boss told us she's just iterating over the results, she wouldn't know the difference.

Unfortunately, this doesn't seem possible. Even if we had a magical function that allowed us to iterate from n to infinity, we'd get stuck after returning the first value:

def get_primes(start):
    for element in magical_infinite_range(start):
        if is_prime(element):
            return element
Imagine get_primes is called like so:

def solve_number_10():
    # She *is* working on Project Euler #10, I knew it!
    total = 2
    for next_prime in get_primes(3):
        if next_prime < 2000000:
            total += next_prime
        else:
            print(total)
            return
Clearly, in get_primes, we would immediately hit the case where number = 3 and return at line 4. Instead of return, we need a way to generate a value and, when asked for the next one, pick up where we left off.

Functions, though, can't do this. When they return, they're done for good. Even if we could guarantee a function would be called again, we have no way of saying, "OK, now, instead of starting at the first line like we normally do, start up where we left off at line 4." Functions have a single entry point: the first line.

Enter the Generator
This sort of problem is so common that a new construct was added to Python to solve it: the generator. A generator "generates" values. Creating generators was made as straightforward as possible through the concept of generator functions, introduced simultaneously.

A generator function is defined like a normal function, but whenever it needs to generate a value, it does so with the yield keyword rather than return. If the body of a def contains yield, the function automatically becomes a generator function (even if it also contains a return statement). There's nothing else we need to do to create one.

generator functions create generator iterators. That's the last time you'll see the term generator iterator, though, since they're almost always referred to as "generators". Just remember that a generator is a special type of iterator. To be considered an iterator, generators must define a few methods, one of which is next(). To get the next value from a generator, we use the same built-in function as for iterators: next().

This point bears repeating: to get the next value from a generator, we use the same built-in function as for iterators: next().

(next() takes care of calling the generator's next() method). Since a generator is a type of iterator, it can be used in a for loop.

So whenever next() is called on a generator, the generator is responsible for passing back a value to whomever called next(). It does so by calling yield along with the value to be passed back (e.g. yield 7). The easiest way to remember what yield does is to think of it as return (plus a little magic) for generator functions.**

Again, this bears repeating: yield is just return (plus a little magic) for generator functions.

Here's a simple generator function:

>>> def simple_generator_function():
>>>    yield 1
>>>    yield 2
>>>    yield 3
And here are two simple ways to use it:

>>> for value in simple_generator_function():
>>>     print(value)
1
2
3
>>> our_generator = simple_generator_function()
>>> next(our_generator)
1
>>> next(our_generator)
2
>>> next(our_generator)
3
Magic?
What's the magic part? Glad you asked! When a generator function calls yield, the "state" of the generator function is frozen; the values of all variables are saved and the next line of code to be executed is recorded until next() is called again. Once it is, the generator function simply resumes where it left off. If next() is never called again, the state recorded during the yield call is (eventually) discarded.

Let's rewrite get_primes as a generator function. Notice that we no longer need the magical_infinite_range function. Using a simple while loop, we can create our own infinite sequence:

def get_primes(number):
    while True:
        if is_prime(number):
            yield number
        number += 1
If a generator function calls return or reaches the end its definition, a StopIteration exception is raised. This signals to whoever was calling next() that the generator is exhausted (this is normal iterator behavior). It is also the reason the while True: loop is present in get_primes. If it weren't, the first time next() was called we would check if the number is prime and possibly yield it. If next() were called again, we would uselessly add 1 to number and hit the end of the generator function (causing StopIteration to be raised). Once a generator has been exhausted, calling next() on it will result in an error, so you can only consume all the values of a generator once. The following will not work:

>>> our_generator = simple_generator_function()
>>> for value in our_generator:
>>>     print(value)

>>> # our_generator has been exhausted...
>>> print(next(our_generator))
Traceback (most recent call last):
  File "<ipython-input-13-7e48a609051a>", line 1, in <module>
    next(our_generator)
StopIteration

>>> # however, we can always create a new generator
>>> # by calling the generator function again...

>>> new_generator = simple_generator_function()
>>> print(next(new_generator)) # perfectly valid
1
Thus, the while loop is there to make sure we never reach the end of get_primes. It allows us to generate a value for as long as next() is called on the generator. This is a common idiom when dealing with infinite series (and generators in general).

Visualizing the flow
Let's go back to the code that was calling get_primes: solve_number_10.

def solve_number_10():
    # She *is* working on Project Euler #10, I knew it!
    total = 2
    for next_prime in get_primes(3):
        if next_prime < 2000000:
            total += next_prime
        else:
            print(total)
            return
It's helpful to visualize how the first few elements are created when we call get_primes in solve_number_10's for loop. When the for loop requests the first value from get_primes, we enter get_primes as we would in a normal function.

We enter the while loop on line 3
The if condition holds (3 is prime)
We yield the value 3 and control to solve_number_10.
Then, back in solve_number_10:

The value 3 is passed back to the for loop
The for loop assigns next_prime to this value
next_prime is added to total
The for loop requests the next element from get_primes
This time, though, instead of entering get_primes back at the top, we resume at line 5, where we left off.

def get_primes(number):
    while True:
        if is_prime(number):
            yield number
        number += 1 # <<<<<<<<<<
Most importantly, number still has the same value it did when we called yield (i.e. 3). Remember, yield both passes a value to whoever called next(), and saves the "state" of the generator function. Clearly, then, number is incremented to 4, we hit the top of the while loop, and keep incrementing number until we hit the next prime number (5). Again we yield the value of number to the for loop in solve_number_10. This cycle continues until the for loop stops (at the first prime greater than 2,000,000).

Moar Power
In PEP 342, support was added for passing values into generators. PEP 342 gave generators the power to yield a value (as before), receive a value, or both yield a value and receive a (possibly different) value in a single statement.

To illustrate how values are sent to a generator, let's return to our prime number example. This time, instead of simply printing every prime number greater than number, we'll find the smallest prime number greater than successive powers of a number (i.e. for 10, we want the smallest prime greater than 10, then 100, then 1000, etc.). We start in the same way as get_primes:

def print_successive_primes(iterations, base=10):
    # like normal functions, a generator function
    # can be assigned to a variable

    prime_generator = get_primes(base)
    # missing code...
    for power in range(iterations):
        # missing code...

def get_primes(number):
    while True:
        if is_prime(number):
        # ... what goes here?
The next line of get_primes takes a bit of explanation. While yield number would yield the value of number, a statement of the form other = yield foo means, "yield foo and, when a value is sent to me, set other to that value." You can "send" values to a generator using the generator's send method.

def get_primes(number):
    while True:
        if is_prime(number):
            number = yield number
        number += 1
In this way, we can set number to a different value each time the generator yields. We can now fill in the missing code in print_successive_primes:

def print_successive_primes(iterations, base=10):
    prime_generator = get_primes(base)
    prime_generator.send(None)
    for power in range(iterations):
        print(prime_generator.send(base ** power))
Two things to note here: First, we're printing the result of generator.send, which is possible because send both sends a value to the generator and returns the value yielded by the generator (mirroring how yield works from within the generator function).

Second, notice the prime_generator.send(None) line. When you're using send to "start" a generator (that is, execute the code from the first line of the generator function up to the first yield statement), you must send None. This makes sense, since by definition the generator hasn't gotten to the first yield statement yet, so if we sent a real value there would be nothing to "receive" it. Once the generator is started, we can send values as we do above.

Round-up
In the second half of this series, we'll discuss the various ways in which generators have been enhanced and the power they gained as a result. yield has become one of the most powerful keywords in Python. Now that we've built a solid understanding of how yield works, we have the knowledge necessary to understand some of the more "mind-bending" things that yield can be used for.

Believe it or not, we've barely scratched the surface of the power of yield. For example, while send does work as described above, it's almost never used when generating simple sequences like our example. Below, I've pasted a small demonstration of one common way send is used. I'll not say any more about it as figuring out how and why it works will be a good warm-up for part two.

\begin{lstlisting}[language=Python]
import random

def get_data():
    """Return 3 random integers between 0 and 9"""
    return random.sample(range(10), 3)

def consume():
    """Displays a running average across lists of integers sent to it"""
    running_sum = 0
    data_items_seen = 0

    while True:
        data = yield
        data_items_seen += len(data)
        running_sum += sum(data)
        print('The running average is {}'.format(running_sum / float(data_items_seen)))

def produce(consumer):
    """Produces a set of values and forwards them to the pre-defined consumer
    function"""
    while True:
        data = get_data()
        print('Produced {}'.format(data))
        consumer.send(data)
        yield

if __name__ == '__main__':
    consumer = consume()
    consumer.send(None)
    producer = produce(consumer)

    for _ in range(10):
        print('Producing...')
        next(producer)
\end{lstlisting}

Remember...
There are a few key ideas I hope you take away from this discussion:

generators are used to generate a series of values
yield is like the return of generator functions
The only other thing yield does is save the "state" of a generator function
A generator is just a special type of iterator
Like iterators, we can get the next value from a generator using next()
for gets values by calling next() implicitly

\section{Cấu trúc dữ liệu}

Number
Basic Operation

\begin{lstlisting}[language=Python]
1
1.2
1 + 2
abs(-5)
\end{lstlisting}


\section{Quản lý gói với Anaconda}

\noindent Cài đặt package tại một branch của một project trên github

\begin{lstlisting}[language=Python]
$ pip install git+https://github.com/tangentlabs/django-oscar-paypal.git@issue/34/oscar-0.6#egg=django-oscar-paypal
\end{lstlisting}

\noindent Trích xuất danh sách package

\begin{lstlisting}
$ pip freeze > requirements.txt
\end{lstlisting}

\noindent \textbf{Chạy ipython trong environment anaconda}

\noindent Chạy đống lệnh này

\begin{lstlisting}[language=bash]
  conda install nb_conda
  source activate my_env
  python -m IPython kernelspec install-self --user
  ipython notebook
\end{lstlisting}

\noindent \textbf{Interactive programming với ipython}

\noindent Trích xuất ipython ra slide (không hiểu sao default `--to slides` không work nữa, lại phải thêm tham số `--reveal-prefix` [^1]

\begin{lstlisting}[language=bash]
jupyter nbconvert "file.ipynb"
  --to slides
  --reveal-prefix "https://cdnjs.cloudflare.com/ajax/libs/reveal.js/3.1.0"
\end{lstlisting}

**Tham khảo thêm**

* https://stackoverflow.com/questions/37085665/in-which-conda-environment-is-jupyter-executing
* https://github.com/jupyter/notebook/issues/541#issuecomment-146387578
* https://stackoverflow.com/a/20101940/772391

\noindent \textbf{python 3.4 hay 3.5}

Có lẽ 3.5 là lựa chọn tốt hơn (phải có của tensorflow, pytorch, hỗ trợ mock)

### Quản lý môi trường phát triển với conda

Chạy lệnh `remove` để xóa một môi trường

\begin{lstlisting}[language=bash]
conda remove --name flowers --all
\end{lstlisting}

\section{Test với python}

\textbf{Sử dụng những loại test nào?}

Hiện tại mình đang viết unittest với default class của python là Unittest. Thực ra toàn sử dụng `assertEqual` là chính!

Ngoài ra mình cũng đang sử dụng tox để chạy test trên nhiều phiên bản python (python 2.7, 3.5). Điều hay của tox là mình có thể thiết kế toàn bộ cài đặt project và các dependencies package trong file `tox.ini`

\textbf{Chạy test trên nhiều phiên bản python với tox}

Pycharm hỗ trợ debug tox (quá tuyệt!), chỉ với thao tác đơn giản là nhấn chuột phải vào file tox.ini của project.

\section{Xây dựng docs với readthedocs và sphinx}

\noindent \textbf{20/12/2017}: Tự nhiên hôm nay tất cả các class có khai báo kế thừa ở project languageflow không thể index được. Vãi thật. Làm thằng đệ không biết đâu mà build model.

Thử build lại chục lần, thay đổi file conf.py và package\_reference.rst chán chê không được. Giả thiết đầu tiên là do hai nguyên nhân (1) docstring ghi sai, (2) nội dung trong package\_reference.rst bị sai. Sửa chán chê cũng vẫn thể, thử checkout các commit của git. Không hoạt động!

Mất khoảng vài tiếng mới để ý thằng readthedocs có phần log cho từng build một. Lần mò vào build gần nhất và build (mình nhớ là) thành công cách đây 2 ngày

\noindent Log build gần nhất

\begin{lstlisting}
Running Sphinx v1.6.5
making output directory...
loading translations [en]... done
loading intersphinx inventory from https://docs.python.org/objects.inv...
intersphinx inventory has moved: https://docs.python.org/objects.inv -> https://docs.python.org/2/objects.inv
loading intersphinx inventory from http://docs.scipy.org/doc/numpy/objects.inv...
intersphinx inventory has moved: http://docs.scipy.org/doc/numpy/objects.inv -> https://docs.scipy.org/doc/numpy/objects.inv
building [mo]: targets for 0 po files that are out of date
building [readthedocsdirhtml]: targets for 8 source files that are out of date
updating environment: 8 added, 0 changed, 0 removed
reading sources... [ 12%] authors
reading sources... [ 25%] contributing
reading sources... [ 37%] history
reading sources... [ 50%] index
reading sources... [ 62%] installation
reading sources... [ 75%] package_reference
reading sources... [ 87%] readme
reading sources... [100%] usage

looking for now-outdated files... none found
pickling environment... done
checking consistency... done
preparing documents... done
writing output... [ 12%] authors
writing output... [ 25%] contributing
writing output... [ 37%] history
writing output... [ 50%] index
writing output... [ 62%] installation
writing output... [ 75%] package_reference
writing output... [ 87%] readme
writing output... [100%] usage
\end{lstlisting}

Log build hồi trước

\begin{lstlisting}[language=bash]
Running Sphinx v1.5.6
making output directory...
loading translations [en]... done
loading intersphinx inventory from https://docs.python.org/objects.inv...
intersphinx inventory has moved: https://docs.python.org/objects.inv -> https://docs.python.org/2/objects.inv
loading intersphinx inventory from http://docs.scipy.org/doc/numpy/objects.inv...
intersphinx inventory has moved: http://docs.scipy.org/doc/numpy/objects.inv -> https://docs.scipy.org/doc/numpy/objects.inv
building [mo]: targets for 0 po files that are out of date
building [readthedocs]: targets for 8 source files that are out of date
updating environment: 8 added, 0 changed, 0 removed
reading sources... [ 12%] authors
reading sources... [ 25%] contributing
reading sources... [ 37%] history
reading sources... [ 50%] index
reading sources... [ 62%] installation
reading sources... [ 75%] package_reference
reading sources... [ 87%] readme
reading sources... [100%] usage

/home/docs/checkouts/readthedocs.org/user_builds/languageflow/checkouts/develop/languageflow/transformer/count.py:docstring of languageflow.transformer.count.CountVectorizer:106: WARNING: Definition list ends without a blank line; unexpected unindent.
/home/docs/checkouts/readthedocs.org/user_builds/languageflow/checkouts/develop/languageflow/transformer/tfidf.py:docstring of languageflow.transformer.tfidf.TfidfVectorizer:113: WARNING: Definition list ends without a blank line; unexpected unindent.
../README.rst:7: WARNING: nonlocal image URI found: https://img.shields.io/badge/latest-1.1.6-brightgreen.svg
looking for now-outdated files... none found
pickling environment... done
checking consistency... done
preparing documents... done
writing output... [ 12%] authors
writing output... [ 25%] contributing
writing output... [ 37%] history
writing output... [ 50%] index
writing output... [ 62%] installation
writing output... [ 75%] package_reference
writing output... [ 87%] readme
writing output... [100%] usage
\end{lstlisting}

Đập vào mắt là sự khác biệt giữa documentation type

Lỗi

\begin{lstlisting}[language=bash]
building [readthedocsdirhtml]: targets for 8 source files that are out of date
\end{lstlisting}

Chạy

\begin{lstlisting}[language=bash]
building [readthedocs]: targets for 8 source files that are out of date
\end{lstlisting}

Hí ha hí hửng. Chắc trong cơn bất loạn sửa lại settings đây mà. Sửa lại nó trong phần Settings (Admin &gt; Settings &gt; Documentation type)

![](https://magizbox.files.wordpress.com/2017/10/screenshot-from-2017-12-20-09-54-23.png)

Khi chạy nó đã cho ra log đúng

\begin{lstlisting}[language=bash]
building [readthedocsdirhtml]: targets for 8 source files that are out of date
\end{lstlisting}

Nhưng vẫn lỗi. Vãi!!! Sau khoảng 20 phút tiếp tục bấn loạn, chửi bới readthedocs các kiểu. Thì để ý dòng này

Lỗi

\begin{lstlisting}[language=bash]
Running Sphinx v1.6.5
\end{lstlisting}


Chạy

\begin{lstlisting}[language=bash]
Running Sphinx v1.5.6
\end{lstlisting}

Ngay dòng đầu tiên mà không để ý, ngu thật. Aha, Hóa ra là thằng readthedocs nó tự động update phiên bản sphinx lên 1.6.5. Mình là mình chúa ghét thay đổi phiên bản (code đã mệt rồi, lại còn phải tương thích với nhiều phiên bản nữa thì ăn c** à). Đầu tiên search với Pycharm thấy dòng này trong `conf.py`

\begin{lstlisting}[language=bash]
# If your documentation needs a minimal Sphinx version, state it here.
# needs_sphinx = '1.0'
\end{lstlisting}

Đổi thành

\begin{lstlisting}[language=bash]
# If your documentation needs a minimal Sphinx version, state it here.
needs_sphinx = '1.5.6'
\end{lstlisting}

Vẫn vậy (holy sh*t). Thử sâu một tẹo (thực sự là rất nhiều tẹo). Thấy cái này trong trang Settings

![](https://magizbox.files.wordpress.com/2017/10/screenshot-from-2017-12-20-10-01-39.png)

Ờ há. Thằng đần này cho phép trỏ đường dẫn tới một file trong project để cấu hình dependency. Haha.
Tạo thêm một file `requirements` trong thư mục `docs` với nội dung

\begin{lstlisting}
sphinx==1.5.6
\end{lstlisting}


Sau đó cấu hình nó trên giao diện web của readthedocs

![](https://magizbox.files.wordpress.com/2017/10/screenshot-from-2017-12-20-10-04-49.png)

Build thử. Build thử thôi. Cảm giác đúng lắm rồi đấy. Và... nó chạy. Ahihi

![](https://magizbox.files.wordpress.com/2017/10/screenshot-from-2017-12-20-10-06-32.png)

\textbf{Kinh nghiệm}

* Khi không biết làm gì, hãy làm 3 việc. Đọc LOG. Phân tích LOG. Và cố gắng để LOG thay đổi theo ý mình.

PS: Trong quá trình này, cũng không thèm build thằng PDF với Epub nữa. Tiết kiệm được bao nhiêu thời gian.

\section{Pycharm Pycharm}

01/2018: Pycharm là trình duyệt ưa thích của mình trong suốt 3 năm vừa rồi.

Hôm nay tự nhiên lại gặp lỗi không tự nhận unittest, không resolve được package import bởi relative path. Vụ không tự nhận unittest sửa bằng cách xóa file .idea là xong. Còn vụ không resolve được package import bởi relative path thì vẫn chịu rồi. Nhìn code cứ đỏ lòm khó chịu thật.

\section{Vì sao lại code python?}

\textbf{01/11/2017}
Thích python vì nó quá đơn giản (và quá đẹp).

[^1]: https://github.com/jupyter/nbconvert/issues/91#issuecomment-283736634
%\chapter{C++}


C++ is a general-purpose programming language. It has imperative, object-oriented and generic programming features, while also providing facilities for low-level memory manipulation. It was designed with a bias toward system programming and embedded, resource-constrained and large systems, with performance, efficiency and flexibility of use as its design highlights. C++ has also been found useful in many other contexts, with key strengths being software infrastructure and resource-constrained applications, including desktop applications, servers (e.g. e-commerce, web search or SQL servers), and performance-critical applications (e.g. telephone switches or space probes). C++ is a compiled language, with implementations of it available on many platforms and provided by various organizations, including the Free Software Foundation (FSF's GCC), LLVM, Microsoft, Intel and IBM.

View online \href{http://magizbox.com/training/cpp/site/}{http://magizbox.com/training/cpp/site/}

\section{Get Started}

What do I need to start with CLion?
In general to develop in C/C++ with CLion you need:

CMake, 2.8.11+ (Check JetBrains guide for updates)
GCC/G++/Clang (Linux) or
MinGW 3. or MinGW — w64 3.-4. or Cygwin 1.7.32 (minimum required) up to 2.0. (Windows)
Downloading and Installing CMake
Downloading and installing CMake is pretty simple, just go to the website, download and install by following the recommended guide there or the on Desktop Wizard.

Download and install file cmake-3.9.0-win64-x65.msi
> cmake
Usage

  cmake [options] <path-to-source>
  cmake [options] <path-to-existing-build>

Specify a source directory to (re-)generate a build system for it in the
current working directory.  Specify an existing build directory to
re-generate its build system.

Run 'cmake --help' for more information.
Downloading and Getting Cygwin
Cygwin is a large collection of GNU and Open Source tools which provide functionality similar to a Linux distribution on Windows

Download file setup-x86_64.exe from the website https://cygwin.com/install.html

Install setup-x86_64.exe file



This is the root directory where Cygwin will be located, usually the recommended C:\ works



Choose where to install LOCAL DOWNLOAD PACKAGES: This is not the same as root directory, but rather where packages (ie. extra C libraries and tools) you download using Cygwin will be located



Follow the recommended instructions until you get to packages screen:



Once you get to the packages screen, this is where you customize what libraries or tools you will install. From here on I followed the above guide but here’s the gist:

From this window, choose the Cygwin applications to install. For our purposes, you will select certain GNU C/C++ packages.

Click the + sign next to the Devel category to expand it.

You will see a long list of possible packages that can be downloaded. Scroll the list to see more packages.

Pick each of the following packages by clicking its corresponding “Skip” marker.

gcc-core: C compiler subpackage
gcc-g++: C++ subpackage
libgcc1: C runtime library
gdb: The GNU Debugger
make: The GNU version of the ‘make’ utility
libmpfr4 : A library for multiple-precision floating-point arithmetic with exact rounding
Download and install CLion
Download file CLion-2017.2.exe from website https://www.jetbrains.com/clion/download/#section=windows



Config environment File > Settings... > Build, Execution, Deployment

Choose Cygwin home: C:\cygwin64
Choose CMake executable: Bundled CMake 3.8.2
Run your first C++ program with CLion

\section{Basic Syntax}

C/C++
Hello World
#include <iostream>
using namespace std;

int main() {
    cout << "hello world";
}
Convention
Naming
variable_name_like_this
class_data_memeber_name_like_this_
kConstantNamesLikeThis
ClassNameLikeThis
filenamelikethis_myusefulclass_test.cc
Comment
Class Comment
// Iterates over the contents of a GargantuanTable.
// Example:
//    GargantuanTableIterator* iter = table->NewIterator();
//    for (iter->Seek("foo"); !iter->done(); iter->Next()) {
//      process(iter->key(), iter->value());
//    }
//    delete iter;
class GargantuanTableIterator {
  ...
};
Todo Comment
// TODO(kl@gmail.com): Use a "*" here for concatenation operator.
// TODO(Zeke) change this to use relations.

\section{Cấu trúc dữ liệu}

Data Structure
Number
C++ offer the programmer a rich assortment of built-in as well as user defined data types. Following table lists down seven basic C++ data types:

Boolean - bool
Character - char
Integer - int
Floating point - float
Double floating point - double
Valueless - void
Wide character - wchar_t
Several of the basic types can be modified using one or more of these type modifiers: signed, unsigned, short, long

Following is the example, which will produce correct size of various data types on your computer.

#include <iostream>
using namespace std;

int main() {
   cout << "Size of char : " << sizeof(char) << endl;
   cout << "Size of int : " << sizeof(int) << endl;
   cout << "Size of short int : " << sizeof(short int) << endl;
   cout << "Size of long int : " << sizeof(long int) << endl;
   cout << "Size of float : " << sizeof(float) << endl;
   cout << "Size of double : " << sizeof(double) << endl;
   cout << "Size of wchar_t : " << sizeof(wchar_t) << endl;
   return 0;
}
String
String Basic

#include <iostream>
#include <string>
using namespace std ;

// assign a string
string s1 = "www.java2s.com\n";
cout << s1;

// input a string
string s2;
cin >> s2;

// concatenate two strings
string s_c = s1 + s2;

// compare strings
s1 == s2;
Collection
Pointer
A pointer is a variable whose value is the address of another variable. Like any variable or constant, you must declare a pointer before you can work with it.

The general form of a pointer variable declaration is:

type *variable_name;
// example
int    *ip;    // pointer to an integer
double *dp;    // pointer to a double
float  *fp;    // pointer to a float
char   *ch;    // pointer to character
Pointer Lab



#include <iostream>
using namespace std;

/*
 * Look at these lines
 */
int* a;
a = new int[3];
a[0] = 10;
a[1] = 2;
cout << "Address of pointer a: &a = " << &a << endl;
cout << "Value   of pointer a:  a = " << a << endl << endl;
cout << "Address of a[0]: &a[0] = " << &a[0] << endl;
cout << "Value   of a[0]: a[0]  = " << a[0]  << endl;
cout << "Value   of a[0]: *a    = " << *a    << endl << endl;
cout << "Address of a[1]: &a[1] = " << &a[1] << endl;
cout << "Value   of a[1]: a[1]  = " << a[1]  << endl;
cout << "Value   of a[1]: *(a+1)= " << *(a+1)<< endl << endl;
cout << "Address of a[2]: &a[2] = " << &a[2] << endl;
cout << "Value   of a[2]: a[2]  = " << a[2]  << endl;
cout << "Value   of a[2]: *(a+2)= " << *(a+2)<< endl << endl;
Result:

Address of pointer a: &a = 008FF770
Value   of pointer a:  a = 00C66ED0

Address of a[0]: &a[0] = 00C66ED0
Value   of a[0]: a[0]  = 10
Value   of a[0]: *a    = 10

Address of a[1]: &a[1] = 00C66ED4
Value   of a[1]: a[1]  = 2
Value   of a[1]: *(a+1)= 2

Address of a[2]: &a[2] = 00C66ED8
Value   of a[2]: a[2]  = -842150451
Value   of a[2]: *(a+2)= -842150451
Stack, Queue, Linked List, Array, Deque, List, Map, Set

Datetime
The C++ standard library does not provide a proper date type. C++ inherits the structs and functions for date and time manipulation from C. To access date and time related functions and structures, you would need to include header file in your C++ program.

There are four time-related types: clock_t, time_t, size_t, and tm. The types clock_t, size_t and time_t are capable of representing the system time and date as some sort of integer.

The structure type tm holds the date and time in the form of a C structure having the following elements:

struct tm {
   int tm_sec;   // seconds of minutes from 0 to 61
   int tm_min;   // minutes of hour from 0 to 59
   int tm_hour;  // hours of day from 0 to 24
   int tm_mday;  // day of month from 1 to 31
   int tm_mon;   // month of year from 0 to 11
   int tm_year;  // year since 1900
   int tm_wday;  // days since sunday
   int tm_yday;  // days since January 1st
   int tm_isdst; // hours of daylight savings time
}
Current date and time

Consider you want to retrieve the current system date and time, either as a local time or as a Coordinated Universal Time (UTC). Following is the example to achieve the same:

#include <iostream>
#include <ctime>

using namespace std;

int main( ) {
   // current date/time based on current system
   time_t now = time(0);

   // convert now to string form
   char* dt = ctime(&now);

   cout << "The local date and time is: " << dt << endl;

   // convert now to tm struct for UTC
   tm *gmtm = gmtime(&now);
   dt = asctime(gmtm);
   cout << "The UTC date and time is:"<< dt << endl;
}
When the above code is compiled and executed, it produces the following result:

The local date and time is: Sat Jan  8 20:07:41 2011

The UTC date and time is:Sun Jan  9 03:07:41 2011

\section{Lập trình hướng đối tượng}

Object Oriented Programming
Classes and Objects
#include <iostream>
using namespace std;

class Pacman {

    private:
      int x;
      int y;
    public:
    Pacman(int x, int y);
    void show();
};

Pacman::Pacman(int x, int y){
    this->x = x;
    this->y = y;
}

void Pacman::show(){
    std::cout << "(" << this->x << ", " << this->y << ")";
}

int main() {
    // your code goes here
    Pacman p = Pacman(2, 3);
    p.show();
    return 0;
}
Template
Function Template

#include <iostream>
#include <string>

using namespace std;

template <typename T>

T Max(T a, T b)
{
    return a < b ? b : a;
}

int main()
{

    int i = 39;
    int j = 20;
    cout << Max(i, j) << endl;

    double f1 = 13.5;
    double f2 = 20.7;
    cout << Max(f1, f2) << endl;

    string s1 = "Hello";
    string s2 = "World";
    cout << Max(s1, s2) << endl;

    double n1 = 20.3;
    float n2 = 20.4;
    // it will show an error
    // Error: no instance of function template "Max" matches the argument list
    //        arguments types are: (double, float)
    cout << Max(n1, n2) << endl;
    return 0;
}

\section{Cơ sở dữ liệu}


Database
Sqlite with Visual Studio 2013
Step 1: Create new project 1.1 Create a new C++ Win32 Console application.

Step 2: Download Sqlite DLL

2.1. Download the native SQLite DLL from: http://sqlite.org/sqlite-dll-win32-x86-3070400.zip 2.2. Unzip the DLL and DEF files and place the contents in your project’s source folder (an easy way to find this is to right click on the tab and click the “Open Containing Folder” menu item.

Step 3: Build LIB file

3.1. Open a “Developer Command Prompt” and navigate to your source folder. (If you can't find this tool, follow this post in stackoverflow Where is Developer Command Prompt for VS2013? to create it) 3.2. Create an import library using the following command line: LIB /DEF:sqlite3.def

Step 4: Add Dependencies

4.1. Add the library (i.e. sqlite3.lib) to your Project Properties -> Configuration Properties -> Linker -> Input -> Additional Dependencies. 4.2. Download http://sqlite.org/sqlite-amalgamation-3070400.zip 4.3. Unzip the sqlite3.h header file and place into your source directory. 4.4. Include the the sqlite3.h header file in your source code. 4.5. You will need to include the sqlite3.dll in the same directory as your program (or in a System Folder).

Step 5: Run test code

#include "stdafx.h"
#include <ios>
#include <iostream>
#include "sqlite3.h"

using namespace std;

int _tmain(int argc, _TCHAR* argv[])
{
   int rc;
   char *error;

   // Open Database
   cout << "Opening MyDb.db ..." << endl;
   sqlite3 *db;
   rc = sqlite3_open("MyDb.db", &db);
   if (rc)
   {
      cerr << "Error opening SQLite3 database: " << sqlite3_errmsg(db) << endl << endl;
      sqlite3_close(db);
      return 1;
   }
   else
   {
      cout << "Opened MyDb.db." << endl << endl;
   }

   // Execute SQL
   cout << "Creating MyTable ..." << endl;
   const char *sqlCreateTable = "CREATE TABLE MyTable (id INTEGER PRIMARY KEY, value STRING);";
   rc = sqlite3_exec(db, sqlCreateTable, NULL, NULL, &error);
   if (rc)
   {
      cerr << "Error executing SQLite3 statement: " << sqlite3_errmsg(db) << endl << endl;
      sqlite3_free(error);
   }
   else
   {
      cout << "Created MyTable." << endl << endl;
   }

   // Execute SQL
   cout << "Inserting a value into MyTable ..." << endl;
   const char *sqlInsert = "INSERT INTO MyTable VALUES(NULL, 'A Value');";
   rc = sqlite3_exec(db, sqlInsert, NULL, NULL, &error);
   if (rc)
   {
      cerr << "Error executing SQLite3 statement: " << sqlite3_errmsg(db) << endl << endl;
      sqlite3_free(error);
   }
   else
   {
      cout << "Inserted a value into MyTable." << endl << endl;
   }

   // Display MyTable
   cout << "Retrieving values in MyTable ..." << endl;
   const char *sqlSelect = "SELECT * FROM MyTable;";
   char **results = NULL;
   int rows, columns;
   sqlite3_get_table(db, sqlSelect, &results, &rows, &columns, &error);
   if (rc)
   {
      cerr << "Error executing SQLite3 query: " << sqlite3_errmsg(db) << endl << endl;
      sqlite3_free(error);
   }
   else
   {
      // Display Table
      for (int rowCtr = 0; rowCtr <= rows; ++rowCtr)
      {
         for (int colCtr = 0; colCtr < columns; ++colCtr)
         {
            // Determine Cell Position
            int cellPosition = (rowCtr * columns) + colCtr;

            // Display Cell Value
            cout.width(12);
            cout.setf(ios::left);
            cout << results[cellPosition] << " ";
         }

         // End Line
         cout << endl;

         // Display Separator For Header
         if (0 == rowCtr)
         {
            for (int colCtr = 0; colCtr < columns; ++colCtr)
            {
               cout.width(12);
               cout.setf(ios::left);
               cout << "~~~~~~~~~~~~ ";
            }
            cout << endl;
         }
      }
   }
   sqlite3_free_table(results);

   // Close Database
   cout << "Closing MyDb.db ..." << endl;
   sqlite3_close(db);
   cout << "Closed MyDb.db" << endl << endl;

   // Wait For User To Close Program
   cout << "Please press any key to exit the program ..." << endl;
   cin.get();

   return 0;
}

\section{Testing}

Create Unit Test in Visual Studio 2013
Step 1. Create TDDLab Solution
1.1 Open Visual Studio 2013

1.2 File ->  New Project... ->

Click Visual C++ -> Win32

Choose Win32 Console Application

Fill to Name input text: TDDLab

Click OK -> Next

1.3 In project settings, remove options:

Precompiled Header
Securirty Develoment Lifecyde(SQL) check
1.4 Click Finish

Step 2. Create Counter Class
2.1 Right-click TDDLab -> Add -> Class...

2.2 Choose Visual C++ -> C++ Class -> Add

2.3 Fill in Class name box Counter -> Finish

2.4 In Counter.h file, add this below function

int add(int a, int b);
2.5 In Counter.cpp, add this below function

int Counter::add(int a, int b) {
  return a+b;
}
Your Counter class should look like this



Step 3. Create TDDLabTest Project
3.1 Right-click Solution 'TDDLab' -> Add -> New Project...

3.2 Choose Visual C++ -> Test

3.3 Choose Native Unit Test Project

3.4 Fill to Name input text: TDDLabTest

Step 4. Write unit test
4.1 In unittest1.cpp, add header of Counter class

#include "../TDDLab/Counter.h"
4.2 In TEST_METHOD function

{
  Counter counter;
  Assert::AreEqual(2, counter.add(1, 1));
}
4.3 Click TEST in menu bar -> Run -> `All Test (Ctrl + R, A)

Step 5. Fix error LNK 2019: unresolved external symbol
5.1 Change Configuration Type of TDDLab project

Right click  TDDLab project -> Properties
General -> Configuration Type -> Static library (.lib) -> OK
5.2 Add Reference to TDDLabTest project

Right click TDDLabTest solution -> Properties -> Common Properties -> Add New Reference
Choose TDDLab -> OK -> OK
Step 6. Run Tests
Click TEST in menu bar -> Run -> `All Test (Ctrl + R, A)

Test should be passed.

\section{IDE & Debugging}

Visual Studio 2013
Install Extension

VsVim

googletest guide

Folder Structure with VS 2013

solution
│   README.md
│
|───project1
|   │   file011.txt
|   │   file012.txt
|   │
|───project2
|   │   file011.txt
|   │   file012.txt
|   │
Auto Format

Ctrl + K, Ctrl + D
Git in Visual Studio

https://git-scm.com/book/en/v2/Git-in-Other-Environments-Git-in-Visual-Studio

Online IDE
codechef ide


%\chapter{Javascript}

View online \href{http://magizbox.com/training/java/site/}{http://magizbox.com/training/java/site/}

What is Javascript?
JavaScript is a high-level, dynamic, untyped, and interpreted programming language. It has been standardized in the ECMAScript language specification. Alongside HTML and CSS, it is one of the three core technologies of World Wide Web content production; the majority of websites employ it and it is supported by all modern Web browsers without plug-ins. JavaScript is prototype-based with first-class functions, making it a multi-paradigm language, supporting object-oriented, imperative, and functional programming styles. It has an API for working with text, arrays, dates and regular expressions, but does not include any I/O, such as networking, storage, or graphics facilities, relying for these upon the host environment in which it is embedded.


\section{Installation}

Google Chrome
Pycharm

\section{IDE}

Google Chrome Developer Tools

The Chrome Developer Tools (DevTools for short), are a set of web authoring and debugging tools built into Google Chrome. The DevTools provide web developers deep access into the internals of the browser and their web application. Use the DevTools to efficiently track down layout issues, set JavaScript breakpoints, and get insights for code optimization.

\section{Basic Syntax}

1. Code Formatting
Indent with 2 spaces

// Object initializer.
var inset = {
  top: 10,
  right: 20,
  bottom: 15,
  left: 12
};

// Array initializer.
this.rows_ = [
  '"Slartibartfast" <fjordmaster@magrathea.com>',
  '"Zaphod Beeblebrox" <theprez@universe.gov>',
  '"Ford Prefect" <ford@theguide.com>',
  '"Arthur Dent" <has.no.tea@gmail.com>',
  '"Marvin the Paranoid Android" <marv@googlemail.com>',
  'the.mice@magrathea.com'
];

// Used in a method call.
goog.dom.createDom(goog.dom.TagName.DIV, {
  id: 'foo',
  className: 'some-css-class',
  style: 'display:none'
}, 'Hello, world!');
2. Naming
functionNamesLikeThis
variableNamesLikeThis
ClassNamesLikeThis
EnumNamesLikeThis
methodNamesLikeThis
CONSTANT_VALUES_LIKE_THIS
foo.namespaceNamesLikeThis.bar
filenameslikethis.js.
3. Comment
Use JSDoc

3.1 Class Comment
/**
 * Class making something fun and easy.
 * @param {string} arg1 An argument that makes this more interesting.
 * @param {Array.<number>} arg2 List of numbers to be processed.
 * @constructor
 * @extends {goog.Disposable}
 */
project.MyClass = function(arg1, arg2) {
  // ...
};
goog.inherits(project.MyClass, goog.Disposable);
3.2 Method Comment
/**
 * Operates on an instance of MyClass and returns something.
 * @param {project.MyClass} obj Instance of MyClass which leads to a long
 *     comment that needs to be wrapped to two lines.
 * @return {boolean} Whether something occurred.
 */
function PR_someMethod(obj) {
  // ...
}
4. Expression and Statements
Expression
A fragment of code that produces a value is called an Expression

22
"this is an epression"
(5 > 6) ? false : true
Statements
The Simplest kind of stagement is an expression with a semi colon

!false;
5 + 6;
5. Loop and iteration
while
var number = 0;
while (number <= 12) {
  console.log(number);
  number = number + 2;
}
do..while
do {
  var yourName = prompt("Who are you?");
} while (!yourName);
console.log(yourName);
for
for (var i = 0; i < 10; i++) {
  console.log(i);
}
6. Function
6.1 Defining a Function
var square = function(x) {
  return x * x;
};
square(5);
6.2 Scope
Scope is the area where contains all variable or function are living.
Scope has some rules:
Child Scope can access all variable and function in parent Scope. (E.g: Local Scope can access Global Scope)
function saveName(firstName) {
    var temp = "temp";
    function capitalizeName() {
        temp = temp + " here";
        return firstName.toUpperCase();
    }
    var capitalized = capitalizeName();
    return capitalized;
}
alert(saveName("Robert"));
But parent Scope can access variable and function inside children scope (E.g: Global Scope cannot acces to local Scope)
function talkDirty () {
    var saying = "Oh, you little VB lover, you";
    return alert(saying);
}
alert(saying); //->Error
6.3 Call Stack
The storage where computer stores context is called CALL STACK.

// CALL STACK
function greet(who) {
    console.log("Hello " + who);
    ask("How are you?");
    console.log("I'm fine");
};

function ask(question) {
    console.log("well, " + question);
};

greet("Harry");
console.log("Bye");
Out of Call Stack

function chicken() {
    return egg();
}

function egg() {
    return chicken();
}
console.log(chicken() + " came first");
6.4. Optional Argument
We can pass too many or too few arguments to the function without any SyntaxError.
If we pass too much arguments, the extra ones are ignored
If we pass to few arguments, the missing ones get value undefined
function power(base, exponent) {
    if (exponent == undefined) {
        exponent = 2;
    }
    var result = 1;
    for (var count = 0; count < exponent; count++) {
        result = result * base;
    }
    return result;
}
console.log(power(4));
console.log(power(4,3));
upside: flexible
downside: hard to control the error

6.5 Closure
Look at this example:

function sayHello(name){
    var text = 'Hello' + name;
    var say = function(){
        console.log(text);
    }
    return say;
}
var say2 = sayHello("ahaha");
say2();
if in C program, does say2() work?
The answer is nope! Because in C program, when a function returns, the Stack-flame will be destroyed, and all the local variable such as text will undefinded. So, when say2() is called, there is no text anymore, and the error, will be shown!
But, in JavaScript, This code works!! Because, it provides for us an Object called Closure! Closure is borned when we define a function in another function, it keep all the live local variable. So, when say2() is called, the closure will give all the value of local variable outside it, and text will be identity.!

var globalVariable = 10;
function func(){
    var name = "xxx";
    function getName(){
        return name;
    }
    function speak(){
        var sound = "alo";
        function scream(){
            console.log(globalVariable);
            console.log(name);
            return "aaaaaaaaaa!";
        }
        function talk(){
            var voice = getName() + " speak " + sound;
            console.log(voice);
            return voice;
        }
        scream();
        talk();
    }
    speak();
}
func();
6.6. Recursion
Recursion is function can call itself, as long as it is not overflow

function power(base, exponent){
    if (exponent == 0){
        return 1;
    }
    else{
        return base * power(base, exponent -1);
    }
}
console.log(power(2,3));

function FindSolution(target){
    function Find(start, history){
        if (start == target){
            return history;
        }
        else if (start > target){
            return null;
        }
        else{
            return Find(start + 5, "(" + history + " + 5 ") ||
            Find(start * 3, "(" + history + " * 3)");
        }
    }
    return Find(1, "1");
}
console.log(FindSolution(25));
6.7. Arguments object
The arguments object contains all parameters you pass to a function.

function argumentCounter() {
    console.log("you gave me", arguments.length, "argument.");
}
argumentCounter("Straw man", "Tautology", "Ad hominem");
6.8. Higher-Order Function
###Filter array
var ancestry = JSON.parse(ANCESTRY_FILE);
console.log(ancestry.length);

function filter(array, test) {
    var passed = [];
    for (var i = 0; i < array.length; i++){
        if (test(array[i])){
            passed.push(array[i]);
        }
    }
    return passed;
}
console.log(filter(ancestry, function(person){
    return person.born > 1900 && person.born < 1925;
}));

### TRANSFORMING WITH A MAP
function map(array, transform) {
  var mapped = [];
  for (var i = 0; i < array.length; i++)
    mapped.push(transform(array[i]));
  return mapped;
}

var overNinety = ancestry.filter(function(person) {
  return person.died - person.born > 90;
});
console.log(map(overNinety, function(person) {
  return person.name;
}));


### REDUCE
function reduce(array, combine, start) {
  var current = start;
  for (var i = 0; i < array.length; i++)
    current = combine(current, array[i]);
  return current;
}
console.log(reduce([1, 2, 3, 4], function(a, b) {
  return a + b;
}, 0));
#Problem: using map and reduce, transform [1,2,3,4] to [1,2],[3,4]

var a = [1, 2, 3, 4]
a = _.map(a, function(i){
    if(i % 2 == 0){
        return [[],[i]]
    } else {
        return [[i], []]
    }
});
a = _.reduce(a, function(x, y){
   return [x[0].concat(y[0]), x[1].concat(y[1])]
})


### BINDING FUNCTION
var theSet = ["Carel Haverbeke", "Maria van Brussel",
              "Donald Duck"];
function isInSet(set, person) {
  return set.indexOf(person.name) > -1;
}

console.log(ancestry.filter(function(person) {
  return isInSet(theSet, person);
}));
console.log(ancestry.filter(isInSet.bind(null, theSet)));
What's the cleanest way to write a multiline string in JavaScript? [duplicate] ↩

Google JavaScript Style Guide ↩

\section{Data Structure}

\subsection{Number}

Some example of number: 10, 1.234, 1.99e9, NaN, Infinity, -Infinity

console.log(2.99e9);
console.log(0 /0);
console.log(1 /0);
console.log(-1 /0);
Automatic Conversion

console.log(8 * null); // -> 0
console.log("5" - 1); // -> 4
console.log("5" + 1); //-> 51
console.log(false == 0) //-> true

\subsection{String}

sprintf
In index.html

<script src="cdnjs.cloudflare.com/ajax/libs/sprintf/1.0.3/sprintf.js"/>

In script.js

// arguments
sprintf("%1$s %2$s", "hello", "sprintf")
# hello sprintf

// object
var user = {
    name: "Dolly"
}
sprintf("Hello %(name)s", user)
# Hello Dolly

// array of object
var users = [
    {name: "Dolly"},
    {name: "Molly"}
]
sprintf("Hello %(users[0].name)s and %(users[1].name)s", {users: users})
# Hello Dolly and Molly
Multiline String
str = "\
line 1\
line 2\
line 3";
Regular Expression in JavaScript
This lab is based on Chapter9: EloquentJavaScript

Creating a regular expression
There are 2 ways:

var re1 = new RegExp("abc");
var re2 = /abc/
there are some special characters such as question mark, or plus sign. If you want to use them, you have to use backslash. Like this:

var eighteen = /eighteen\+/;
var question = /question\?/;
Testing for match
Regular Express has a number of method. Simplest is test

console.log(/abc/.test("abcd"));
console.log(/abc/.test("abxde"));
Matching a set of character []: Put a set of characters between 2 square bracket

console.log(/[0123456789]/.test("1245"));
console.log(/[0-9]/.test("1");
console.log(/[0-9]/.test("acd");
console.log(/[0-9]/.test("aaascacas1"));
There are some special character:
\d Any digit character (Like [0-9])

var datetime = /\d\d-\d\d-\d\d\d\d\s\d\d:\d\d/;
console.log(datetime.test("16-06-2016 14:09"));
console.log(dateTime.test("30-jan-2003 15:20"));
\w An alphanumeric character (“word character”)

var word = /\w/;
console.log(word.test("@#@#"));
\s Any whitespace character (space, tab, newline, and similar)

var space = /\d\.\s+abc/;
console.log(space.test("1. abd"));
console.log(space.test("1.     abd"));
console.log(space.test("1.abd"));
\D A character that is not a digit

var notDigit = /\D/;
console.log(notDigit.test("ww"));
console.log(notDigit.test("1a"));
console.log(notDigit.test("1124"));
\W A nonalphanumeric character

var nonAlphanumbericChar = /\W/;
console.log(nonAlphanumbericChar.test("abc12231"));
console.log(nonAlphanumbericChar.test("!@#%{}_"));
\S A nonwhitespace character

var nonWhiteSpace = /\S/;
console.log(nonWhiteSpace.test("abc123"));
console.log(nonWhiteSpace.test("1.  abcd"));
console.log(nonWhiteSpace.test("  "));
"." Any character except for newline

var anyThing = /...\./;
console.log(anyThing.test("abc."));
console.log(anyThing.test("acbacd."));
console.log(anyThing.test("acba"));
"^" Using caret character to match any except the ones

var notBinary = /[^01]/;
console.log(notBinary.test("01101011100"));
console.log(notBinary.test("01021010010"));
Repeating parts of Pattern
The square bracket [] above only match 1 digit. How can regex match more than 1 digit?
"+" Match one or more
"*" Match zero or more

console.log(/\d+/.test(1234));
console.log(/\d+/.test());

console.log(/\d*/.test(1234));
console.log(/\d*/.test())
"?" Question mark test a character exist or not is still oke

var ball = /bal?l/;
console.log(ball.test("ball"));
console.log(ball.test("bal"));
{a,b} the character before exist from a to b times. Check datetime:

var datetime = /\d{1,2}-\d{1,2}-\d{4} \d{1,2}:\d{1,2}/;
console.log(datetime.test("20-12-2015 14:09"));
var checkTimes = /waz{3,5}up/;
console.log(checkTimes.test("wazzzzzup"));
console.log(checkTimes.test("wazzzup"));
console.log(checkTimes.test("wazup"));
Grouping Subexpressions
() using prentheses to make whole group like one character

var cartoonCrying = /boo+(hoo+)+/i; //i to match all Captalize or normal text
console.log(cartoonCrying.test("Boohoooohoohooo"));
console.log(cartoonCrying.test("boohoooohooOOO"));
Matches and group
Test is a simplest method, and it only return true or false.
exec (execute) is anther method in regex. It returns null if no match, and object if match.

var match  = /\d+/.exec("one two 100");
console.log(match);
console.log(match.input);
console.log(match.index);
if in the expression has a group subexpression, then it will return the text contain this subexpress, and the text match this subexpress:

var quotedText = /'([^']*)'/;
console.log(quotedText.exec("she said 'hello'"));
and if the subexpression appears one more times, then the result will be displayed the last match one.

console.log(/bad(ly)?/.exec("bad"));
console.log(/(\d)+/.exec("123"));
The date type
create new Date(). return the current time

var date =  new Date();
console.log(new Date(2009, 11, 9);
console.log(new Date(2009, 11, 9, 23, 59, 61));
<!--TimeStamp-->
console.log(new Date(2009, 11, 9, 23, 59, 61).getTime());
console.log(new Date(1260378001000));
<!--getFullYear, getMonth,...-->
var date = new Date();
console.log(date.getFullYear());
console.log(date.getMonth());
console.log(date.getDate());
console.log(date.getHours());
console.log(date.getMinutes());
console.log(date.getSeconds());
Word and string boundaries
console.log(/cat/.test("concatenate"));
console.log(/cat/.test("con123cat-129e0enate"));
console.log(/\bcat\b/.test("concatenate"));
console.log(/\bcat\b/.test("con123cat-129e0enate"));
Choice patterm
Only one in the list beween the "|" match

var animalCount = /\b\d+ (pig|cow|chicken)s?\b/;
console.log(animalCount.test("15 pigs"));
console.log(animalCount.test("15 pigchickens"));
Replace
Replace will find the first match and replace.if we want to replace all matches, using "g" behind the expresssion

console.log("papa".replace("p", "m"));
console.log("Borobudur".replace(/[ou]/, "a"));
console.log("Borobudur".replace(/[ou]/g, "a"));
Replace can refer back to the matched, and using them

console.log("Le, Khanh\nNguyen, Hung\nDuong, Bach".replace(/([\w]+), ([\w]+)/g, "$1 $2"));
Greed
function stripComments(code) {
  return code.replace(/\/\/.*|\/\*[^]*\*\//g, "");
}
console.log(stripComments("1 + /* 2 */3"));
// → 1 + 3
console.log(stripComments("x = 10;// ten!"));
// → x = 10;
console.log(stripComments("1 /* a */+/* b */ 1"));
// → 1  1
Search method
Search method return the first index if the regular expression match.
And return -1 if not found

console.log("  word".search(/\S/));
// → 2
console.log("    ".search(/\S/));
// → -1
The last index property
In the regular expression has a property is lastIndex. And when this Regex do some method, it will start from the lastIndex. And after doing something, the lastIndex will update to the behind the index of the match exec.

var pattern = /y/g;
pattern.lastIndex = 3; //lastIndex update to 3
var match = pattern.exec("xyzzy"); //lastIndex update to 5
console.log(pattern.lastIndex);

match = pattern.exec("xyzzyxxx"); //Not match any "y" from index 5
console.log(match.index);
console.log(pattern.lastIndex);
Looping Over the Line
Applying the hepoloris of lastIndex, we can using while to do something like this:

var input = "A string with 3 numbers in it... 42 and 88.";
var number = /\b(\d+)\b/g;
var match;
while (match = number.exec(input))
  console.log("Found", match[1], "at", match.index);

\subsection{Collection}

Some useful methods with array
push and pop
var a = [1,2,3,4];
console.log(a.pop(), a);
console.log(a.push(3), a);
shift and unshift
console.log(a.shift(), a);
console.log(a.unshift(1), a);
indexOf and lastIndexOf
var b = [1,2,3,4,2,3,1];
console.log(b.indexOf(1));
console.log(b.lastIndexOf(1));
slice
console.log([0,1,2,3,4].slice(2,4));
console.log([0,1,2,3,4].slice(2));
concat
var a = [1,2,3];
var b = [4,5,6];
a.concat(b);
console.log(a);

\subsection{Datetime}

Current Time
moment().format('MMMM Do YYYY, h:mm:ss a');
Moment.js ↩

\subsection{Boolean}

Boolean has only 2 values: true and false

console.log("Abc" < "Abcd") // -> true
console.log("abc" < "Abcd") // -> false
console.log("123" == "123") // -> true
console.log(NaN == NaN) // -> false
what is the different?

console.log("5" == 5);
console.log("5" === 5);

\subsection{Object}

Object
Define an object
var object = {
  number: 10,
  string: "string",
  array: [1,2,3],
  object: {a: 1, b: 2}
}
Add new property to object
object.newProperty = "value";
object['key'] = 'value';
delete property
delete object.newProperty;
Window object (global object)
The Global scope is stored in an object which called window

function test(){
    var local = 10;
    console.log("local" in window);
    console.log(window.local);
}
test();
var global = 10;
console.log("global" in window);
console.log(window.global);

\section{OOP}


1. Classes and Objects
Constructor
function Ball(position){
    this.position = position;
    this.display = function(){
        console.log(this.position[0], ", ", this.position[1]);
    }
}

ball = new Ball([2, 3]);
ball.display();
2. Inheritance
Person = function (name, birthday, job) {
  this.name = name;
  this.birthday = birthday;
  this.job = job;
};

Person.prototype.display = function () {
  console.log(this.name, "\n");
  console.log(this.birthday, "\n");
  console.log(this.job, "\n");
};

Politician = function (name, birthday) {
  Person.call(this, name, birthday, "Politician");
};
Politician.prototype = Object.create(Person.prototype);
Politician.prototype.constructor = Politician;

var person1 = new Person("Barack Obama", "04/08/1961", "Politician");
var person2 = new Politician("David Cameron", "09/10/1966");
person1.display();
person2.display();

Object-Oriented Programming
var rabbit = {};
rabbit.speak = function(line){
console.log("The rabit says:'" + line + "'");
 };
rabbit.speak("I'm alive");

function speak(line){
   console.log("The "+ this.type + " rabbit says '" + line + "'");
 }

var whiteRabbit = {type: "white", speak: speak};
var fatRabbit = {type: "fat", speak: speak};
whiteRabbit.speak("Oh my ears and whiskers, " + "how late it's getting!");
fatRabbit.speak("I could sure use a carrot right now");

// Prototype
// Prototype is another object that is used as a fallback source of properties
// When object request a property that it does not have, its prototype will be searched for the property
var empty = {};
console.log(empty.toString);
console.log(empty.toString);

// Get prototype of an object 2 ways:
console.log(Object.getPrototypeOf({}) == Object.prototype);
console.log(Object.getPrototypeOf(Object.prototype));

// Using Object.create to create an object with an specific prototype
var protoRabbit = {
  speak: function(line){
    console.log("The " + this.type + " rabbit says '" + line + "'");
  }
};

var killerRabbit = Object.create(protoRabbit);
killerRabbit.type = "Killer";
killerRabbit.speak("Skreeee!");

// Constructor
function Rabbit(type){
   this.type = type;
}
var killerRabbit = new Rabbit("Killer");
var blackRabbit = new Rabbit("black");
console.log(blackRabbit.type);

// using prototype to add a new method
Rabbit.prototype.speak = function(line) {
  console.log("The " + this.type + " rabit says '" + line + "'");
};
blackRabbit.speak("Doom...");


// OVERRIDING DERIVED PROPERTIES
Rabbit.prototype.teeth = "small";
console.log(killerRabbit.teeth);

killerRabbit.teeth = "Long, sharp, and bloody";
console.log(killerRabbit.teeth);
console.log(blackRabbit.teeth);
console.log(Rabbit.prototype.teeth);

// PROTOTYPE INTERFERENCE
// A prototype can be used at any time to add methods, properties
// to all objects based on it
Rabbit.prototype.dance = function (){
  console.log("The " + this.type + " rabbit dances a jig");
};
killerRabbit.dance();
// but there is a problem:
var map = {};
function storePhi(event, phi){
   map[event] = phi;
}

storePhi("pizza", 0.069);
storePhi("touched tree", -0.081);
console.log(map);

Object.prototype.nonsense = "hi";
for (var name in map) {
  console.log(name);
}
console.log("nonsense" in map);
console.log("toString" in map);
delete Object.prototype.nonsense;
//  we can use Object.defineProperty to solve it
Object.defineProperty(Object.prototype, "hiddenNonsense", {
   enumerable: false,
   value: "hi"
});

for (var name in map) {
  console.log(name);
}
console.log(map.hiddenNonsense);
// but there still has a problem
console.log("toString" in map);
console.log(map.hasOwnProperty("toString"));

// PROTOTYPE-LESS OBJECTS
// if we only want to create an fresh object, without prototype then we tranform null to create
var map = Object.create(null);
map["pizza"] = 0.09;
console.log("toString" in map);
console.log("pizza" in map);

// POLYMORPHISM
// laying out a table: example for polymorphism
function rowHeights(rows) {
    return rows.map(function(row){
        return row.reduce(function(max, cell) {
            return Math.max(max, cell.minHeight());
        }, 0);
    });
}

function colWidths(rows) {
    return rows[0].map(function(_, i) {
        return rows.reduce(function(max, row){
            return Math.max(max, row[i].minWidth());
        }, 0);
    });
}

function drawTable(rows) {
    var heights = rowHeights(rows);
    var widths = colWidths(rows);

    function drawLine(blocks, lineNo) {
        return blocks.map(function(block) {
            return block[lineNo];
        }).join(" ");
    }

    function drawRow(row, rowNum){
        var blocks = row.map(function(cell, colNum) {
           return cell.draw(widths[colNum], heights[rowNum]);
        });
        return blocks[0].map(function(_, lineNo) {
            return drawLine(blocks, lineNo);
        }).join("\n");
    }

    return rows.map(drawRow).join("\n");
}

function repeat(string, times){
    var result = "";
    for (var i = 0; i < times; i++){
        result += string;
    }
    return result;
}

function TextCell(text){
    this.text = text.split("\n");
}
TextCell.prototype.minWidth = function(){
    return this.text.reduce(function(width, line){
        return Math.max(width, line.lenght);
    }, 0);
};
TextCell.prototype.minHeight = function(){
    return this.text.length;
}
TextCell.prototype.minHeight = function(){
    return this.text.lenght;
}
TextCell.prototype.minHeight = function(){
    return this.text.length;
}
TextCell.prototype.draw = function(width, height){
    var result = [];
    for (var i = 0; i < height; i++){
        var line = this.text[i] || "";
        result.push(line + repeat(" ", width - line.length));
    }
    return result;
}

var rows = [];
for (var i = 0; i < 5; i++){
    var row = [];
    for (var j = 0; j < 5; j++){
        if ((i + j) % 2 == 0){
            row.push(new TextCell("1234"));
        } else{
            row.push(new TextCell("5"));
        }
    }
    rows.push(row);
}
console.log(drawTable(rows));

// // GETTERS AND SETTERS
// var pile = {
//     elements: ["eggshell", "orange peel", "worm"],
//     get height(){
//         return this.elements.length;
//     },
//     set height(value) {
//         console.log("Ignoring attemp to set high to ", value);
//     }
// };

// console.log(pile.height);
// pile.height = 100;
// console.log(pile.height);

[1]: Introduction to Object-Oriented JavaScript [2]: How to call parent constructor?

\section{Networking}

POST
$.ajax({
  type: "POST",
  url: "http://service.com/items",
  data: JSON.stringify({"name": "new item"}),
  contentType: 'application/json'
}).done(function (data) {
  console.log(data)
}).fail(function () {
});

\section{Logging}


Javascript Logging
Having a fancy JavaScript debugger is great, but sometimes the fastest way to find bugs is just to dump as much information to the console as you can.

console.log
console.assert
console.error

\section{Documentation}

Components
jsdoc (with docdash template)

JSDoc is an API documentation generator for JavaScript, similar to JavaDoc or PHPDoc. You add documentation comments directly to your source code, right along side the code itself. The JSDoc Tool will scan your source code, and generate a complete HTML documentation website for you.

gulp, PyCharm

Usage
Step 1. Install gulp-jsdoc
npm install --save-dev gulp gulp-jsdoc docdash
Step 2. Create documentation task
Create documentation task in gulpfile.js file

var template = {
  "path": "./node_modules/docdash"
};

gulp.task('docs', function(){
  return gulp.src("./src/*.js")
    .pipe(jsdoc('./docs', template));
});
Step 3. Refresh Gulp tasks
In pycharm, click to refresh button in gulp window.

Step 4. Add comment to your code
Add comment to your code, You can see an example: should.js

/**
 * Simple utility function for a bit more easier should assertion
 * extension
 * @param {Function} f So called plugin function. It should accept
 * 2 arguments: `should` function and `Assertion` constructor
 * @memberOf should
 * @returns {Function} Returns `should` function
 * @static
 * @example
 *
 * should.use(function(should, Assertion) {
 *   Assertion.add('asset', function() {
 *      this.params = { operator: 'to be asset' };
 *
 *      this.obj.should.have.property('id').which.is.a.Number();
 *      this.obj.should.have.property('path');
 *  })
 * })
 */
should.use = function(f) {
  f(should, should.Assertion);
  return this;
};

Types: boolean, string, number, Array (see more)

Step 5. Run docs task
In pycharm, click to docs task in gulp window.

\section{Error Handling}

In javascript bugs may be displayed is NaN or underfined and program still run but after that, the wrong value can cause some mistake when we use it So, finding bugs and fix them is the quiet hard work in javascript But we can do, and this job is called debugging

STRICT MODE
This is the way to find errors that javascript ignores. Example is using an undefined variable. if we dont use strick mode, then everything will be ok, but if using, the error will be shown

function SpotProblem(){
//     "use strict";
    for (counter = 0; counter < 10; counter++){
        console.log("Good!");
    }
}
SpotProblem();
console.log(counter);
strick mode can find error when using this in local, but it is still in global. Example: When we forget to declare the key word "new" when create an new Object

"use strict";
function Person(name){
    this.name = name;
}
var john = Person("John");
console.log(name);
And there are another cases, that trick mode is not allowed: Delete an object is not allowed

"use strict";
var x = 3.14;
delete x;

"use strict";
var obj = {v1: 3, v2: 4};
delete pbj;

"use strict";
var func = function(){};
delete func;
Duplicate parameter is not allowed

"use strict";
var func = function(a1, a1){
    console.log(a1);
}
Reserve Word is not allowed to name variable

"use strict";
var arguments = 5;
var eval = 6;
console.log(arguments);
console.log(eval);
TESTING
Testing makes sure that the program working well, and if there are any changes, testing will automatic show us the error, thus, we know where need to fix

function Vector(x, y){
    this.x = x;
    this.y = y;
}
Vector.prototype.plus = function(other){
    return new Vector(this.x + other.x, this.y + other.y);
}

function TestVector(){
    var p1 = new Vector(10, 20);
    var p2 = new Vector(-10, 5);
    var p3 = p1.plus(p2);

    if (p1.x !== 10) return "fail: x property";
    if (p1.y !== 20) return "fail: y property";
    if (p2.x !== -10) return "fail: nagative x property";
    if (p2.y !== 5) return "fail: y property";
    if (p3.x !== 0) return "fail: x property from plus";
    if (p3.y !== 25) return "fail: y property from plus";
    return "Vector is Oke";
}
TestVector();
DEBUGGING
when the testing is fail, we have to debug to find the bugs.
The first we should guess the bug. And then we put break point in the line, we assume it make bug
If that is the exactly bug we want to find, then we fix it, and write more test for this case
In this example code below, the function convert the number in the decima to another. we run and see the result is wrong, so we guess that the error may be caused by the result variable, then we put break point in the line contains result variable.

function ConvertNumber(n, base) {
  var result = "", sign = "";
  if (n < 0) {
    sign = "-";
    n = -n;
  }
  do {
    result = String(n % base) + result;
    n /= base; //-> n = Math.floor(n / base);
  } while (n > 0);
  return sign + result;
}
console.log(ConvertNumber(13, 10));
console.log(ConvertNumber(14, 2));
ERROR PROPAGATION
Sometime our code is working well with normal input. But with special one, they can cause error. So, we have to consider all situation can make Flaws, and handling them.
This example code below has an if..else to handle the wrong input if user types not a number in the prompt input

function promptNumber(question) {
  var result = Number(prompt(question, ""));
  if (isNaN(result)) return null;
  else return result;
}
console.log(promptNumber("How many trees do you see?"));
EXCEPTION
In the Error Propagation, we can control the errors if we know them. But what will happen if we don't know the error? For solving this problem, javascript provides for us an try...catch.. to control error we dont know or not sure

try {
    throw new Error("Invalid defination");
} catch (error){
    console.log(error);
}

function promtDirection(question){
    var result = prompt(question, "");
    if (result.toLowerCase() == "left") return "L";
    if (result.toLowerCase() == "right") return "R";
    throw new Error("Invalid direction: " + result);
}

function look(){
    if (promtDirection("Which way?") == "L") {
        return "a house";
    }
    else{
        return "two angry bears";
    }
}

try {
    console.log("you see", look());
} catch (error) {
    console.log("Something went wrong: " + error);
}
CLEAN UP AFTER EXCEPTIONS
We have a block of code below:

var context = null;
function withContext(newContext, body){
  var oldContext = context;
  context = newContext;
  var result = body();
  context = oldContext;
  return result;
}
withContext("new", function(){
  var a = b/0;
  return a;
});
What would happend with context? It cannot be excute the last line code, because in withContext function, it will throw off the stack by an exception. So javascript provides a try...finally...

var context = null;
function withContext(newContext, body){
  var oldContext = context;
  context = newContext;
  try{
    return body();
  } finally {
    context = oldContext;
  }
}
withContext("new", function(){
  var a = b/0;
  return a;
});
SELECTIVE CATCHING
There are some errors cannot handle by environment. So, if we let the error go through, it can cause broken program.
For examnple, the Error() in environment cannot catch the infinitive loop in the try block, if we dont catch this problem, the programm will crash soon

for (;;) {
  try {
    var dir = promtDirection("Where?");
    console.log("You chose ", dir);
    break;
  } catch (e) {
    console.log("Not a valid direction. Try again.");
  }
}
The loop will break out if the promptDirection() can excute.
But it doesn't. Because it is not defined before, so the environment catch it and go through the catch to show error
The circle again and again will make the program crash.
So we will create a special Exception.

function InputError(message){
  this.message = message;
  this.stack = (new Error()).stack;
}
InputError.prototype = Object.create(Error.prototype);
InputError.prototype.name = "InputError";
Error: has an property is stack. it contains all exception, which environment can catch. Then, we have the promptDirection function to return the result if Enter valid format, or an exception if invalid

function promptDirection(question){
  var result = prompt(question, "");
  if (result.toLowerCase() == "left") return "L";
  if (result.toLowerCase() == "right") return "R";
  throw new InputError("Invalid direction: " + result);
}
Finally, we can catch all exception we want

for (;;){
  try {
    var dir = promptDirection("Where?");
    console.log("You choose ", dir);
    break;
  } catch(e) {
    if (e instanceof InputError){
      console.log("Not a valid direction. Try again. ");
    }
    else {
      throw e;
    }
  }
}
ASSERTIONS
function AssertionFailed(message) {
  this.message = message;
}
AssertionFailed.prototype = Object.create(Error.prototype);

function assert(test, message) {
  if (!test)
    throw new AssertionFailed(message);
}

function lastElement(array) {
  assert(array.length > 0, "empty array in lastElement");
  return array[array.length - 1];
}

\section{Testing}

Mocha
Mocha is a feature-rich JavaScript test framework running on Node.js and the browser, making asynchronous testing simple and fun. Mocha tests run serially, allowing for flexible and accurate reporting, while mapping uncaught exceptions to the correct test cases.

Installation
bower install -D mocha chai
Usage
Step 1. Make index.html

<!DOCTYPE html>
<html>
<head>
  <meta charset="utf-8">
  <title>Tests</title>
  <link rel="stylesheet" media="all" href="mocha.css">
</head>
<body>
  <div id="mocha"></div>
  <script src="mocha.js"></script>
  <script src="chai.js"></script>
  <script src="functions.js"></script>
  <script>mocha.setup('bdd'); chai.should();</script>
  <script src="tests.js"></script>
  <script>mocha.run();</script>
</body>
</html>
Step 2. Edit functions.js

function sum(a, b){
  return a + b;
}

function asynchronusSum(a, b){
  return new Promise(function(fulfill, reject){
    fulfill(a + b);
  });
}
Step 3. Edit tests.js

describe('Calculator', function() {
  this.timeout(5000);
  describe('#sum()', function() {
    it('should return sum of two number', function() {
      sum(2, 3).should.equal(5)
    });
  });

  describe('#asynchronusSum()', function() {
    it('should return sum of two number', function(done) {
      asynchronusSum(2, 3).then(function(output){
          output.should.equal(5);
          done();
      })
    });
  });
});

\section{Package Manager}

Bower
A package manager for the web

Web sites are made of lots of things — frameworks, libraries, assets, utilities, and rainbows. Bower manages all these things for you.

Bower works by fetching and installing packages from all over, taking care of hunting, finding, downloading, and saving the stuff you’re looking for. Bower keeps track of these packages in a manifest file, bower.json. How you use packages is up to you. Bower provides hooks to facilitate using packages in your tools and workflows.

Bower is optimized for the front-end. Bower uses a flat dependency tree, requiring only one version for each package, reducing page load to a minimum.

http://bower.io/

[code] bower install jquery underscore moment sprintf -S [/code]

HTML <bower based>

<script src="./bower_components/jquery/dist/jquery.js"></script>
<script src="./bower_components/moment/moment.js"></script>
<script src="./bower_components/underscore/underscore.js"></script>
<script src="./bower_components/sprintf/src/sprintf.js"></script>
HTML <cdn based>

<script src="//cdnjs.cloudflare.com/ajax/libs/jquery/3.0.0-beta1/jquery.js"></script>
<script src="//cdnjs.cloudflare.com/ajax/libs/underscore.js/1.8.3/underscore.js"></script>
<script src="//cdnjs.cloudflare.com/ajax/libs/sprintf/1.0.3/sprintf.js"></script>

\section{Build Tool}

Gulp

Automate and enhance your workflow

Here's some of the sweet stuff you try out with this repo.

Compile CoffeeScript (with source maps!)
Compile Handlebars Templates
Compile SASS with Compass
LiveReload
require non-CommonJS code, with dependencies
Set up module aliases
Run a static Node server (with logging)
Pop open your app in a Browser
Report Errors through Notification Center
Image processing
Installation
npm install -S gulp gulp-concat
Usage
Watch

var gulp = require('gulp');
var concat = require('gulp-concat');
var uglify = require('gulp-uglify');
var jsdoc = require("gulp-jsdoc");

var third_parties = [
  "bower_components/jquery/dist/jquery.js",
  "bower_components/bootstrap/dist/js/bootstrap.js",
  "bower_components/underscore/underscore.js",
  "bower_components/ring/ring.js",
  "bower_components/moment/moment.js",
  "bower_components/sprintf/src/sprintf.js",
  "bower_components/uri.js/src/URI.js",
  "bower_components/run/run.js"
];

var modules = [
  "modules/your_script.js"
];

gulp.watch(third_parties, ['js_thirdparty']);
gulp.watch(modules, ['js_modules']);

gulp.task('js_thirdparty', function () {
  return gulp
    .src(third_parties)
    .pipe(concat('third_party.uglify.js'))
    .pipe(uglify())
    .pipe(gulp.dest('./scripts'));
});

gulp.task('js_modules', function () {
  return gulp
    .src(modules)
    .pipe(concat('modules.uglify.js'))
    //.pipe(uglify())
    .pipe(gulp.dest('./scripts'));
});

gulp.task('documentation', function () {
  return gulp
    .src("./modules/*/*.js")
    .pipe(jsdoc('./documentation'));
});

gulp.task('default', ['js_thirdparty', 'js_modules']);
http://gulpjs.com/

Deprecated
grunt

\section{Make Module}

Make Module
sample modules: underscore, momentjs

Folder Structure
|- docs
|- test
|- src
|   |-- your_module.js
|- .gitignore
|- bower.json




%\chapter{Java}

01/11/2017: Java đơn giản là gay nhé. Không chơi. Viết java chỉ viết thế này thôi. Không viết hơn. Thề!
\chapter{PHP}

PHP là ngôn ngữ lập trình web dominate tất cả các anh tài khác mà (chắc là) chỉ dịu đi khi mô hình REST xuất hiện. Nhớ lần đầu gặp bạn Laravel mà cảm giác cuộc đời sang trang.

Cuối tuần này lại phải xem làm sao cài được xdebug vào PHPStorm cho thằng em tập tành lập trình. Haizzz

### Tương tác với cơ sở dữ liệu

Liệt kê danh sách các bản ghi trong bảng groups

```
$sql = "SELECT * FROM `groups`";
$groups = mysqli_query($conn, $sql);
```

Xóa một bản ghi trong bảng groups

```
$sql = "DELETE FROM `groups` WHERE id = `5`";
mysqli_query($conn, $sql);
```

### Cài đặt debug trong PHPStorm

https://www.youtube.com/watch?v=mEJ21RB0F14

(1) XAMPP

- Download XAMPP (cho PHP 7.1.x - do XDebug chưa chính thức hỗ trợ 7.2.0)
https://www.apachefriends.org/xampp-files/7.1.12/xampp-win32-7.1.12-0-VC14-installer.exe
- Install XAMPP xampp-win32-7.1.12-0-VC14-installer.exe
- Truy cập vào địa chỉ http://localhost/dashboard/phpinfo.php để kiểm tra cài đặt đã thành công chưa

(2) Tải và cài đặt PHPStorm

- Download PHPStorm https://download-cf.jetbrains.com/webide/PhpStorm-2017.3.2.exe
- Install PHPStorm

(3) Tạo một web project trong PHPStorm
- Chọn interpreter trỏ đến PHP trong xampp

(4) Viết một chương trình add.php

```php
$a = 2;
$b = 3;
$c = $a + $b;

echo $c;
```

Click vào `add.php`, chọn Debug, PHPStorm sẽ báo chưa cài XDebug

(5) Cài đặt XDebug theo hướng dẫn tại https://gist.github.com/odan/1abe76d373a9cbb15bed

Click vào add.php, chọn Debug

(6) Cài đặt XDebug với PHPStorm Marklets
Vào trang https://www.jetbrains.com/phpstorm/marklets/
Trong phần Zend Debugger
- chọn cổng 9000
- IP: 127.0.0.1
Nhấn nút Generate

Bookmark các link &quot;Start debugger&quot;, &quot;Stop debugger&quot; lên trình duyệt

(7) Debug PHP từ trình duyệt

* Vào trang http://localhost/untitled/add.php
* Click vào bookmark Start debugger
* Trong PHPStorm, nhấn vào biểu tượng &quot;Start Listening for PHP Debug Connections&quot;
* Đặt breakpoint tại dòng thứ 5
* Refresh lại trang http://localhost/untitled/add.php, lúc này, breakpoint sẽ dừng ở dòng 5

%\chapter{R}

View online \href{http://magizbox.com/training/r/site/}{http://magizbox.com/training/r/site/}

R
R is a programming language and software environment for statistical computing and graphics supported by the R Foundation for Statistical Computing. The R language is widely used among statisticians and data miners for developing statistical software and data analysis. Polls, surveys of data miners, and studies of scholarly literature databases show that R's popularity has increased substantially in recent years.

R is a GNU package. The source code for the R software environment is written primarily in C, Fortran, and R. R is freely available under the GNU General Public License, and pre-compiled binary versions are provided for various operating systems. While R has a command line interface, there are several graphical front-ends available.[

R was created by Ross Ihaka and Robert Gentleman at the University of Auckland, New Zealand, and is currently developed by the R Development Core Team, of which Chambers is a member. R is named partly after the first names of the first two R authors and partly as a play on the name of S. The project was conceived in 1992, with an initial version released in 1995 and a stable beta version in 2000.

\section{R Courses}

I'm going to give a course about R, but it's take a lot of time to finish. I will give at least one lesson a week. You can track it here

(next) Data visualization with R
Everything you need to know about R
Read and Write Data
Importing data from JSON into R
Manipulate Data
Manipulate String and Datetime
Actually, beside my works, there are a lot of excellent and free courses in the internet for you

Beginner

tryr from codeschool

tryr is a course for beginners created by codeschool. This course contains R Syntax, Vectors, Matrices, Summary Statistics, Factors, Data Frames and Working With Real-World Data sections.

Introduction to R from datacamp

This course created by datacamp - a "online learning platform that focuses on building the best learning experience for Data Science in specific". Here is the introduction about this course quoted from authors "In this introduction to R, you will master the basics of this beautiful open source language such as factors, lists and data frames. With the knowledge gained in this course, you will be ready to undertake your first very own data analysis." It contains 6 chapters: Intro to basics, Vectors, Matrices, Factors, Data frames and Lists.

Intermediate and Advanced

R Programming of Johns Hopkins University in coursera Learn how to program in R and how to use R for effective data analysis. This is the second course in the Johns Hopkins Data Science Specialization. It's a 4-weeks course, contains: Overview of R, R data types and objects, reading and writing data (week 1),  Control structures, functions, scoping rules, dates and times (week 2), Loop functions, debugging tools (week 3) and Simulation, code profiling (week 4)

An Introduction to Statistical Learning with Applications in R of two experts Trevor Hastie and Rob Tibshirani from Standfor Unitiversity

This course was introduced by Kevin Markham in r-blogger in september 2014. "I found it to be an excellent course in statistical learning (also known as “machine learning”), largely due to the high quality of both the textbook and the video lectures. And as an R user, it was extremely helpful that they included R code to demonstrate most of the techniques described in the book." In this course you will learn about Statistical Learning, Linear Regression, Classification, Resampling Methods, Linear Model Selection and Regularization, Moving Beyond Linearity, Tree-Based Methods, Support Vector Machines and Unsupervised Learning

Cheatsheet – Python & R codes for common Machine Learning Algorithms

\section{Everything you need to know about R}

In this post I maintain all useful references for someone want to write nice R code.

Google’s R Style Guide at google
R is a high-level programming language used primarily for statistical computing and graphics. The goal of the R Programming Style Guide is to make our R code easier to read, share, and verify. The rules below were designed in collaboration with the entire R user community at Google.

Installing R packages at r-bloggers
https://www.r-bloggers.com/installing-r-packages/

This is a short post giving steps on how to actually install R packages.

Managing your projects in a reproducible fashion at nicercode
https://nicercode.github.io/blog/2013-04-05-projects/

Managing your projects in a reproducible fashion doesn’t just make your science reproducible, it makes your life easier.

Creating R Packages
http://cran.r-project.org/doc/contrib/Leisch-CreatingPackages.pdf

This tutorial gives a practical introduction to creating R packages. We discuss how object oriented programming and S formulas can be used to give R code the usual look and feel, how to start a package from a collection of R functions, and how to test the code once the package has been created. As running example we use functions for standard linear regression analysis which are developed from scratch

How to write trycatch in R
http://stackoverflow.com/questions/12193779/how-to-write-trycatch-in-r

Welcome to the R world 😉

Debugging with RStudio
https://support.rstudio.com/hc/en-us/articles/200713843-Debugging-with-RStudio

RStudio includes a visual debugger that can help you understand code and find bugs.

Optimising code
http://adv-r.had.co.nz/Profiling.html#performance-profiling

Optimising code to make it run faster is an iterative process:

Find the biggest bottleneck (the slowest part of your code). Try to eliminate it (you may not succeed but that’s ok). Repeat until your code is “fast enough.” This sounds easy, but it’s not.
%\chapter{Scala}

View online \href{http://magizbox.com/training/scala/site/}{http://magizbox.com/training/scala/site/}

Scala is a programming language for general software applications. Scala has full support for functional programming and a very strong static type system. This allows programs written in Scala to be very concise and thus smaller in size than other general-purpose programming languages. Many of Scala's design decisions were inspired by criticism of the shortcomings of Java.

\section{Installation}

Windows
Step 1. Download scala from http://www.scala-lang.org/downloads

Step 2. Run installer

Step 3. Verify

Open terminal and check which version of scala

$ scala -version

Scala code runner version 2.11.5 -- Copyright 2002-2013, LAMP/EPFL

\section{IDE}

I use IntelliJ IDEA 2016.2 as scala IDE

IntelliJ IDEA Installation Guide
Online IDE
You can use tryscala as an online IDE

http://www.tryscala.com/

\section{Basic Syntax}

Print print
> println("Hello, Scala!");

Hello, Scala!
Conditional
if Statement

if statement consists of a Boolean expression followed by one or more statements.

var x = 10;
if( x < 20 ){
 println("This is if statement");
}
if-else Statement

var x = 30
if( x < 20 ){
  println("This is if statement");
} else {
  println("This is else statement");
}
if-else if-else Statement

 var x = 30;
if( x == 10 ){
 println("Value of X is 10");
} else if( x == 20 ){
 println("Value of X is 20");
} else if( x == 30 ){
 println("Value of X is 30");
} else{
 println("This is else statement");
}
Coding Convention 1
Keep It Simple
Don't pack two much in one expression
/*
 * It's amazing what you can get done in a single statement
 * But that does not mean you have to do it.
 */
jp.getRawClasspath.filter(
  _.getEntryKind == IClasspathEntry.CPE_SOURCE).
  iterator.flatMap(entry =>
    flatten(ResourcesPlugin.getWorkspace.
      getRoot.findMember(entry.getPath)))
Refactor
There's a lot of value in meaningfull names.
Easy to add them using inline vals and defs
val sources = jp.getRawClasspath.filter(
  _.getEntryKind == IClasspathEntry.CPE_SOURCE)
def workspaceRoot =
  ResourcesPlugin.getWorkspace.getRoot
def filesOfEntry(entry: Set[File]) =
  flatten(worspaceRoot.findMember(entry.getPath)
sources.iterator flatMap filesOfEntry
Prefer Functional
By default

use vals, not vars
use recursions or combinators, not loops
use immutable collections
concentrate on transformations, not CRUD
When to deviate from the default - sometimes, mutable gives better performance. - sometimes (but not that often!) it adds convenience

But don't diablolize local state
Why does mutable state lead to complexity?

It interacts with different program parts in ways that are hard to track.

=> Local state is less harmful than global state.

"Var" Shortcuts
var interfaces = parseClassHeader()...
if (isAnnotation) interfaces += ClassFileAnnotation
Refactor

val parsedIfaces = parseClassHeader()
val interfaces =
  if (isAnnotation) parsedIfaces + ClassFileAnnotation
  else parsedIfaces
Martin Odersky - Scala with Style ↩
%\chapter{NodeJS}

View online \href{http://magizbox.com/training/nodejs/site/}{http://magizbox.com/training/nodejs/site/}

Node.js is an open-source, cross-platform JavaScript runtime environment for developing a diverse variety of tools and applications. Although Node.js is not a JavaScript framework, many of its basic modules are written in JavaScript, and developers can write new modules in JavaScript. The runtime environment interprets JavaScript using Google's V8 JavaScript engine. Node.js has an event-driven architecture capable of asynchronous I/O. These design choices aim to optimize throughput and scalability in Web applications with many input/output operations, as well as for real-time Web applications (e.g., real-time communication programs and browser games). Node.js was originally written in 2009 by Ryan Dahl. The initial release supported only Linux. Its development and maintenance was led by Dahl and later sponsored by Joyent.

\section{Get Started}

Installation
Windows
In this section I will show you how to Install Node.js® and NPM on Windows

Prerequisites
Node isn’t a program that you simply launch like Word or Photoshop: you won’t find it pinned to the taskbar or in your list of Apps. To use Node you must type command-line instructions, so you need to be comfortable with (or at least know how to start) a command-line tool like the Windows Command Prompt, PowerShell, Cygwin, or the Git shell (which is installed along with Github for Windows).

Installation Overview
Installing Node and NPM is pretty straightforward using the installer package available from the Node.js® web site.

Installation Steps
1. Download the Windows installer from the Nodes.js® web site.

2. Run the installer (the .msi file you downloaded in the previous step.)

3. Follow the prompts in the installer (Accept the license agreement, click the NEXT button a bunch of times and accept the default installation settings).



4. Restart your computer. You won’t be able to run Node.js® until you restart your computer.

Ubuntu
In this section I will show you how to Install Node.js® and NPM on Ubuntu

# update os
sudo apt-get update
# install node with apt-get
sudo apt-get install nodejs
# install npm with apt-get
sudo apt-get install npm
Test
Make sure you have Node and NPM installed by running simple commands to see what version of each is installed and to run a simple test program:

> node -v
v6.9.5

> npm -v
3.10.10
Suggested Readings
How To Install Node.js on an Ubuntu 14.04 server
How to Install Node.js® and NPM on Windows

\section{Basic Syntax}

Print
console.log("Hello World");
Conditional
if(you_smart){
    console.log("learn nodejs");
} else {
    console.log("go away");
}
Loop
for(var count = 0; count < 10; count++){
    console.log(count);
}
Function
function print_info(arg1, arg2){
    console.log(arg1);
    console.log(arg2);
}

\section{File System & IO}

File System & IO
Node implements File I/O using simple wrappers around standard POSIX functions. The Node File System (fs) module can be imported using the following syntax −

var fs = require("fs")
Synchronous vs Asynchronous
Every method in the fs module has synchronous as well as asynchronous forms. Asynchronous methods take the last parameter as the completion function callback and the first parameter of the callback function as error. It is better to use an asynchronous method instead of a synchronous method, as the former never blocks a program during its execution, whereas the second one does.

Example

Create a text file named input.txt with the following content −

Tutorials Point is giving self learning content
to teach the world in simple and easy way!!!!!
Let us create a js file named main.js with the following code −

var fs = require("fs");

// Asynchronous read
fs.readFile('input.txt', function (err, data) {
   if (err) {
      return console.error(err);
   }
   console.log("Asynchronous read: " + data.toString());
});

// Synchronous read
var data = fs.readFileSync('input.txt');
console.log("Synchronous read: " + data.toString());

console.log("Program Ended");
Now run the main.js to see the result −

$ node main.js
Verify the Output.

Synchronous read: Tutorials Point is giving self learning content
to teach the world in simple and easy way!!!!!

Program Ended
Asynchronous read: Tutorials Point is giving self learning content
to teach the world in simple and easy way!!!!!
The following sections in this chapter provide a set of good examples on major File I/O methods.
Open a File
Syntax

Following is the syntax of the method to open a file in asynchronous mode −

fs.open(path, flags[, mode], callback)
Parameters

Here is the description of the parameters used −

path − This is the string having file name including path.
flags − Flags indicate the behavior of the file to be opened. All possible values have been mentioned below.
mode − It sets the file mode (permission and sticky bits), but only if the file was created. It defaults to 0666, readable and writeable.
callback − This is the callback function which gets two arguments (err, fd).
Flags

Flags for read/write operations are −

r - Open file for reading. An exception occurs if the file does not exist.
r+ - Open file for reading and writing. An exception occurs if the file does not exist.
rs - Open file for reading in synchronous mode.
rs+ - Open file for reading and writing, asking the OS to open it synchronously. See notes for 'rs' about using this with caution.
w - Open file for writing. The file is created (if it does not exist) or truncated (if it exists).
wx - Like 'w' but fails if the path exists.
w+ - Open file for reading and writing. The file is created (if it does not exist) or truncated (if it exists).
wx+ - Like 'w+' but fails if path exists.
a - Open file for appending. The file is created if it does not exist.
ax - Like 'a' but fails if the path exists.
a+ - Open file for reading and appending. The file is created if it does not exist.
ax+ - Like 'a+' but fails if the the path exists.
Example

Let us create a js file named main.js having the following code to open a file input.txt for reading and writing.

var fs = require("fs");

// Asynchronous - Opening File
console.log("Going to open file!");
fs.open('input.txt', 'r+', function(err, fd) {
   if (err) {
      return console.error(err);
   }
  console.log("File opened successfully!");
});
Now run the main.js to see the result −

$ node main.js
Verify the Output.

Going to open file!
File opened successfully!
Get File Information
Syntax

Following is the syntax of the method to get the information about a file −

fs.stat(path, callback)
Parameters

Here is the description of the parameters used −

path − This is the string having file name including path.
callback − This is the callback function which gets two arguments (err, stats) where stats is an object of fs.Stats type which is printed below in the example.
Apart from the important attributes which are printed below in the example, there are several useful methods available in fs.Stats class which can be used to check file type. These methods are given in the following table.

Method Description

stats.isFile() - Returns true if file type of a simple file.
stats.isDirectory() - Returns true if file type of a directory.
stats.isBlockDevice() - Returns true if file type of a block device.
stats.isCharacterDevice() - Returns true if file type of a character device.
stats.isSymbolicLink() - Returns true if file type of a symbolic link.
stats.isFIFO() - Returns true if file type of a FIFO.
stats.isSocket() - Returns true if file type of asocket.
Example

Let us create a js file named main.js with the following code −

var fs = require("fs");

console.log("Going to get file info!");
fs.stat('input.txt', function (err, stats) {
   if (err) {
       return console.error(err);
   }
   console.log(stats);
   console.log("Got file info successfully!");

   // Check file type
   console.log("isFile ? " + stats.isFile());
   console.log("isDirectory ? " + stats.isDirectory());
});
Now run the main.js to see the result −

$ node main.js
Verify the Output.

Going to get file info!
{
   dev: 1792,
   mode: 33188,
   nlink: 1,
   uid: 48,
   gid: 48,
   rdev: 0,
   blksize: 4096,
   ino: 4318127,
   size: 97,
   blocks: 8,
   atime: Sun Mar 22 2015 13:40:00 GMT-0500 (CDT),
   mtime: Sun Mar 22 2015 13:40:57 GMT-0500 (CDT),
   ctime: Sun Mar 22 2015 13:40:57 GMT-0500 (CDT)
}
Got file info successfully!
isFile ? true
isDirectory ? false
Writing a File
Syntax

Following is the syntax of one of the methods to write into a file −

fs.writeFile(filename, data[, options], callback)
This method will over-write the file if the file already exists. If you want to write into an existing file then you should use another method available.

Parameters

Here is the description of the parameters used −

path − This is the string having the file name including path.
data − This is the String or Buffer to be written into the file.
options − The third parameter is an object which will hold {encoding, mode, flag}. By default. encoding is utf8, mode is octal value 0666. and flag is 'w'
callback − This is the callback function which gets a single parameter err that returns an error in case of any writing error.
Example

Let us create a js file named main.js having the following code −

var fs = require("fs");

console.log("Going to write into existing file");
fs.writeFile('input.txt', 'Simply Easy Learning!',  function(err) {
   if (err) {
      return console.error(err);
   }

   console.log("Data written successfully!");
   console.log("Let's read newly written data");
   fs.readFile('input.txt', function (err, data) {
      if (err) {
         return console.error(err);
      }
      console.log("Asynchronous read: " + data.toString());
   });
});
Now run the main.js to see the result −

$ node main.js
Verify the Output.

Going to write into existing file
Data written successfully!
Let's read newly written data
Asynchronous read: Simply Easy Learning!
Reading a File
Syntax

Following is the syntax of one of the methods to read from a file −

fs.read(fd, buffer, offset, length, position, callback)
This method will use file descriptor to read the file. If you want to read the file directly using the file name, then you should use another method available.

Parameters

Here is the description of the parameters used −

fd − This is the file descriptor returned by fs.open().
buffer − This is the buffer that the data will be written to.
offset − This is the offset in the buffer to start writing at.
length − This is an integer specifying the number of bytes to read.
position − This is an integer specifying where to begin reading from in the file. * If position is null, data will be read from the current file position. callback − This is the callback function which gets the three arguments, (err, bytesRead, buffer).
Example

Let us create a js file named main.js with the following code −

var fs = require("fs");
var buf = new Buffer(1024);

console.log("Going to open an existing file");
fs.open('input.txt', 'r+', function(err, fd) {
   if (err) {
      return console.error(err);
   }
   console.log("File opened successfully!");
   console.log("Going to read the file");
   fs.read(fd, buf, 0, buf.length, 0, function(err, bytes){
      if (err){
         console.log(err);
      }
      console.log(bytes + " bytes read");

      // Print only read bytes to avoid junk.
      if(bytes > 0){
         console.log(buf.slice(0, bytes).toString());
      }
   });
});
Now run the main.js to see the result −

$ node main.js
Verify the Output.

Going to open an existing file
File opened successfully!
Going to read the file
97 bytes read
Tutorials Point is giving self learning content
to teach the world in simple and easy way!!!!!
Closing a File
Syntax

Following is the syntax to close an opened file −

fs.close(fd, callback)
Parameters

Here is the description of the parameters used −

fd − This is the file descriptor returned by file fs.open() method.
callback − This is the callback function No arguments other than a possible exception are given to the completion callback.
Example Let us create a js file named main.js having the following code −

var fs = require("fs");
var buf = new Buffer(1024);

console.log("Going to open an existing file");
fs.open('input.txt', 'r+', function(err, fd) {
   if (err) {
      return console.error(err);
   }
   console.log("File opened successfully!");
   console.log("Going to read the file");

   fs.read(fd, buf, 0, buf.length, 0, function(err, bytes){
      if (err){
         console.log(err);
      }

      // Print only read bytes to avoid junk.
      if(bytes > 0){
         console.log(buf.slice(0, bytes).toString());
      }

      // Close the opened file.
      fs.close(fd, function(err){
         if (err){
            console.log(err);
         }
         console.log("File closed successfully.");
      });
   });
});
Now run the main.js to see the result −

$ node main.js
Verify the Output.

Going to open an existing file
File opened successfully!
Going to read the file
Tutorials Point is giving self learning content
to teach the world in simple and easy way!!!!!

File closed successfully.
Truncate a File
Syntax

Following is the syntax of the method to truncate an opened file −

fs.ftruncate(fd, len, callback)
Parameters

Here is the description of the parameters used −

fd − This is the file descriptor returned by fs.open().
len − This is the length of the file after which the file will be truncated.
callback − This is the callback function No arguments other than a possible ekxception are given to the completion callback.
Example

Let us create a js file named main.js having the following code −

var fs = require("fs");
var buf = new Buffer(1024);

console.log("Going to open an existing file");
fs.open('input.txt', 'r+', function(err, fd) {
   if (err) {
      return console.error(err);
   }
   console.log("File opened successfully!");
   console.log("Going to truncate the file after 10 bytes");

   // Truncate the opened file.
   fs.ftruncate(fd, 10, function(err){
      if (err){
         console.log(err);
      }
      console.log("File truncated successfully.");
      console.log("Going to read the same file");

      fs.read(fd, buf, 0, buf.length, 0, function(err, bytes){
         if (err){
            console.log(err);
         }

         // Print only read bytes to avoid junk.
         if(bytes > 0){
            console.log(buf.slice(0, bytes).toString());
         }

         // Close the opened file.
         fs.close(fd, function(err){
            if (err){
               console.log(err);
            }
            console.log("File closed successfully.");
         });
      });
   });
});
Now run the main.js to see the result −

$ node main.js
Verify the Output.

Going to open an existing file
File opened successfully!
Going to truncate the file after 10 bytes
File truncated successfully.
Going to read the same file
Tutorials
File closed successfully.
Delete a File
Syntax Following is the syntax of the method to delete a file −

fs.unlink(path, callback)
Parameters

Here is the description of the parameters used −

path − This is the file name including path.
callback − This is the callback function No arguments other than a possible exception are given to the completion callback.
Example

Let us create a js file named main.js having the following code −

var fs = require("fs");

console.log("Going to delete an existing file");
fs.unlink('input.txt', function(err) {
   if (err) {
      return console.error(err);
   }
   console.log("File deleted successfully!");
});
Now run the main.js to see the result −

$ node main.js
Verify the Output.

Going to delete an existing file
File deleted successfully!
Create a Directory
Syntax

Following is the syntax of the method to create a directory −

fs.mkdir(path[, mode], callback)
Parameters

Here is the description of the parameters used −

path − This is the directory name including path.
mode − This is the directory permission to be set. Defaults to 0777.
callback − This is the callback function No arguments other than a possible exception are given to the completion callback.
Example

Let us create a js file named main.js having the following code −

var fs = require("fs");

console.log("Going to create directory /tmp/test");
fs.mkdir('/tmp/test',function(err){
   if (err) {
      return console.error(err);
   }
   console.log("Directory created successfully!");
});
Now run the main.js to see the result −

$ node main.js
Verify the Output.

Going to create directory /tmp/test
Directory created successfully!
Read a Directory
Syntax

Following is the syntax of the method to read a directory −

fs.readdir(path, callback)
Parameters

Here is the description of the parameters used −

path − This is the directory name including path.
callback − This is the callback function which gets two arguments (err, files) where files is an array of the names of the files in the directory excluding '.' and '..'.
Example

Let us create a js file named main.js having the following code −

var fs = require("fs");

console.log("Going to read directory /tmp");
fs.readdir("/tmp/",function(err, files){
   if (err) {
      return console.error(err);
   }
   files.forEach( function (file){
      console.log( file );
   });
});
Now run the main.js to see the result −

$ node main.js
Verify the Output.

Going to read directory /tmp
ccmzx99o.out
ccyCSbkF.out
employee.ser
hsperfdata_apache
test
test.txt
Remove a Directory
Syntax

Following is the syntax of the method to remove a directory −

fs.rmdir(path, callback)
Parameters

Here is the description of the parameters used −

path − This is the directory name including path.
callback − This is the callback function No argume nts other than a possible exception are given to the completion callback.
Example

Let us create a js file named main.js having the following code −

var fs = require("fs");

console.log("Going to delete directory /tmp/test");
fs.rmdir("/tmp/test",function(err){
   if (err) {
      return console.error(err);
   }
   console.log("Going to read directory /tmp");

   fs.readdir("/tmp/",function(err, files){
      if (err) {
         return console.error(err);
      }
      files.forEach( function (file){
         console.log( file );
      });
   });
});
Now run the main.js to see the result −

$ node main.js
Verify the Output.

Going to read directory /tmp
ccmzx99o.out
ccyCSbkF.out
employee.ser
hsperfdata_apache
test.txt

\section{Package Manager}

Package Manager: NPM
Node Package Manager (NPM) provides two main functionalities −

Online repositories for node.js packages/modules which are searchable on search.nodejs.org
Command line utility to install Node.js packages, do version management and dependency management of Node.js packages.
NPM comes bundled with Node.js installables after v0.6.3 version. To verify the same, open console and type the following command and see the result −

$ npm --version
2.7.1
If you are running an old version of NPM then it is quite easy to update it to the latest version. Just use the following command from root −

$ sudo npm install npm -g
/usr/bin/npm -> /usr/lib/node_modules/npm/bin/npm-cli.js
npm@2.7.1 /usr/lib/node_modules/npm
Installing Modules
There is a simple syntax to install any Node.js module −

$ npm install <Module Name>
For example, following is the command to install a famous Node.js web framework module called express −

$ npm install express
Now you can use this module in your js file as following −

var express = require('express');
Global vs Local Installation
By default, NPM installs any dependency in the local mode. Here local mode refers to the package installation in node_modules directory lying in the folder where Node application is present. Locally deployed packages are accessible via require() method. For example, when we installed express module, it created node_modules directory in the current directory where it installed the express module.

$ ls -l
total 0
drwxr-xr-x 3 root root 20 Mar 17 02:23 node_modules
Alternatively, you can use npm ls command to list down all the locally installed modules.

Globally installed packages/dependencies are stored in system directory. Such dependencies can be used in CLI (Command Line Interface) function of any node.js but cannot be imported using require() in Node application directly. Now let's try installing the express module using global installation.

$ npm install express -g
This will produce a similar result but the module will be installed globally. Here, the first line shows the module version and the location where it is getting installed.

express@4.12.2 /usr/lib/node_modules/express
├── merge-descriptors@1.0.0
├── utils-merge@1.0.0
├── cookie-signature@1.0.6
├── methods@1.1.1
├── fresh@0.2.4
├── cookie@0.1.2
├── escape-html@1.0.1
├── range-parser@1.0.2
├── content-type@1.0.1
├── finalhandler@0.3.3
├── vary@1.0.0
├── parseurl@1.3.0
├── content-disposition@0.5.0
├── path-to-regexp@0.1.3
├── depd@1.0.0
├── qs@2.3.3
├── on-finished@2.2.0 (ee-first@1.1.0)
├── etag@1.5.1 (crc@3.2.1)
├── debug@2.1.3 (ms@0.7.0)
├── proxy-addr@1.0.7 (forwarded@0.1.0, ipaddr.js@0.1.9)
├── send@0.12.1 (destroy@1.0.3, ms@0.7.0, mime@1.3.4)
├── serve-static@1.9.2 (send@0.12.2)
├── accepts@1.2.5 (negotiator@0.5.1, mime-types@2.0.10)
└── type-is@1.6.1 (media-typer@0.3.0, mime-types@2.0.10)
You can use the following command to check all the modules installed globally −

$ npm ls -g
Using package.json
package.json is present in the root directory of any Node application/module and is used to define the properties of a package. Let's open package.json of express package present in node_modules/express/

{
   "name": "express",
      "description": "Fast, unopinionated, minimalist web framework",
      "version": "4.11.2",
      "author": {

         "name": "TJ Holowaychuk",
         "email": "tj@vision-media.ca"
      },

   "contributors": [{
      "name": "Aaron Heckmann",
      "email": "aaron.heckmann+github@gmail.com"
   },

   {
      "name": "Ciaran Jessup",
      "email": "ciaranj@gmail.com"
   },

   {
      "name": "Douglas Christopher Wilson",
      "email": "doug@somethingdoug.com"
   },

   {
      "name": "Guillermo Rauch",
      "email": "rauchg@gmail.com"
   },

   {
      "name": "Jonathan Ong",
      "email": "me@jongleberry.com"
   },

   {
      "name": "Roman Shtylman",
      "email": "shtylman+expressjs@gmail.com"
   },

   {
      "name": "Young Jae Sim",
      "email": "hanul@hanul.me"
   } ],
   "license": "MIT", "repository": {
      "type": "git",
      "url": "https://github.com/strongloop/express"
   },
   "homepage": "https://expressjs.com/", "keywords": [
      "express",
      "framework",
      "sinatra",
      "web",
      "rest",
      "restful",
      "router",
      "app",
      "api"
   ],
   "dependencies": {
      "accepts": "~1.2.3",
      "content-disposition": "0.5.0",
      "cookie-signature": "1.0.5",
      "debug": "~2.1.1",
      "depd": "~1.0.0",
      "escape-html": "1.0.1",
      "etag": "~1.5.1",
      "finalhandler": "0.3.3",
      "fresh": "0.2.4",
      "media-typer": "0.3.0",
      "methods": "~1.1.1",
      "on-finished": "~2.2.0",
      "parseurl": "~1.3.0",
      "path-to-regexp": "0.1.3",
      "proxy-addr": "~1.0.6",
      "qs": "2.3.3",
      "range-parser": "~1.0.2",
      "send": "0.11.1",
      "serve-static": "~1.8.1",
      "type-is": "~1.5.6",
      "vary": "~1.0.0",
      "cookie": "0.1.2",
      "merge-descriptors": "0.0.2",
      "utils-merge": "1.0.0"
   },
   "devDependencies": {
      "after": "0.8.1",
      "ejs": "2.1.4",
      "istanbul": "0.3.5",
      "marked": "0.3.3",
      "mocha": "~2.1.0",
      "should": "~4.6.2",
      "supertest": "~0.15.0",
      "hjs": "~0.0.6",
      "body-parser": "~1.11.0",
      "connect-redis": "~2.2.0",
      "cookie-parser": "~1.3.3",
      "express-session": "~1.10.2",
      "jade": "~1.9.1",
      "method-override": "~2.3.1",
      "morgan": "~1.5.1",
      "multiparty": "~4.1.1",
      "vhost": "~3.0.0"
   },
   "engines": {
      "node": ">= 0.10.0"
   },
   "files": [
      "LICENSE",
      "History.md",
      "Readme.md",
      "index.js",
      "lib/"
   ],
   "scripts": {
      "test": "mocha --require test/support/env
         --reporter spec --bail --check-leaks test/ test/acceptance/",
      "test-cov": "istanbul cover node_modules/mocha/bin/_mocha
         -- --require test/support/env --reporter dot --check-leaks test/ test/acceptance/",
      "test-tap": "mocha --require test/support/env
         --reporter tap --check-leaks test/ test/acceptance/",
      "test-travis": "istanbul cover node_modules/mocha/bin/_mocha
         --report lcovonly -- --require test/support/env
         --reporter spec --check-leaks test/ test/acceptance/"
   },
   "gitHead": "63ab25579bda70b4927a179b580a9c580b6c7ada",
   "bugs": {
      "url": "https://github.com/strongloop/express/issues"
   },
   "_id": "express@4.11.2",
   "_shasum": "8df3d5a9ac848585f00a0777601823faecd3b148",
   "_from": "express@*",
   "_npmVersion": "1.4.28",
   "_npmUser": {
      "name": "dougwilson",
      "email": "doug@somethingdoug.com"
   },
   "maintainers": [
      {
         "name": "tjholowaychuk",
         "email": "tj@vision-media.ca"
      },
      {
         "name": "jongleberry",
         "email": "jonathanrichardong@gmail.com"
      },
      {
         "name": "shtylman",
         "email": "shtylman@gmail.com"
      },
      {
         "name": "dougwilson",
         "email": "doug@somethingdoug.com"
      },
      {
         "name": "aredridel",
         "email": "aredridel@nbtsc.org"
      },
      {
         "name": "strongloop",
         "email": "callback@strongloop.com"
      },
      {
         "name": "rfeng",
         "email": "enjoyjava@gmail.com"
      }
   ],
   "dist": {
      "shasum": "8df3d5a9ac848585f00a0777601823faecd3b148",
      "tarball": "https://registry.npmjs.org/express/-/express-4.11.2.tgz"
   },
   "directories": {},
      "_resolved": "https://registry.npmjs.org/express/-/express-4.11.2.tgz",
      "readme": "ERROR: No README data found!"
}
Attributes of Package.json
name − name of the package
version − version of the package
description − description of the package
homepage − homepage of the package
author − author of the package
contributors − name of the contributors to the package
dependencies − list of dependencies. NPM automatically installs all the dependencies mentioned here in the node_module folder of the package. repository − repository type and URL of the package
main − entry point of the package
keywords − keywords
Uninstalling a Module
Use the following command to uninstall a Node.js module.

$ npm uninstall express
Once NPM uninstalls the package, you can verify it by looking at the content of /node_modules/ directory or type the following command −

$ npm ls
Updating a Module
Update package.json and change the version of the dependency to be updated and run the following command.

$ npm update express
Search a Module
Search a package name using NPM.

$ npm search express
Create a Module
Creating a module requires package.json to be generated. Let's generate package.json using NPM, which will generate the basic skeleton of the package.json.

$ npm init

This utility will walk you through creating a package.json file.
It only covers the most common items, and tries to guess sane defaults.

See 'npm help json' for definitive documentation on these fields
and exactly what they do.

Use 'npm install <pkg> --save' afterwards to install a package and
save it as a dependency in the package.json file.

Press ^C at any time to quit.
name: (webmaster)
You will need to provide all the required information about your module. You can take help from the above-mentioned package.json file to understand the meanings of various information demanded. Once package.json is generated, use the following command to register yourself with NPM repository site using a valid email address.

$ npm adduser
Username: mcmohd
Password:
Email: (this IS public) mcmohd@gmail.com
It is time now to publish your module −

$ npm publish
If everything is fine with your module, then it will be published in the repository and will be accessible to install using NPM like any other Node.js module.

\section{Command Line}

Pass command line arguments
The arguments are stored in process.argv

Here are the node docs on handling command line args:

 process.argv is an array containing the command line arguments. The first element will be 'node', the second element will be the name of the JavaScript file. The next elements will be any additional command line arguments.

// print process.argv
process.argv.forEach(function (val, index, array) {
  console.log(index + ': ' + val);
});
This will generate:

$ node process-2.js one two=three four
0: node
1: /Users/mjr/work/node/process-2.js
2: one
3: two=three
4: four





%\input{programming/octave.tex}
%\chapter{Toolbox}

View online \href{http://magizbox.com/training/toolbox/site/}{http://magizbox.com/training/toolbox/site/}

Toolbox by MG
The Toolbox contains all the little tools you never know where to find.

Text Editor
Vim : Vim is a clone of Bill Joy's vi text editor program for Unix. It was written by Bram Moolenaar based on source for a port of the Stevie editor to the Amiga and first released publicly in 1991. Vim is designed for use both from a command-line interface and as a standalone application in a graphical user interface. Vim is free and open source software and is released under a license that includes some charityware clauses, encouraging users who enjoy the software to consider donating to children in Uganda. The license is compatible with the GNU General Public License. Although it was originally released for the Amiga, Vim has since been developed to be cross-platform, supporting many other platforms. In 2006, it was voted the most popular editor amongst Linux Journal readers; in 2015 the Stack Overflow developer survey found it to be the third most popular text editor; and in 2016 the Stack Overflow developer survey found it to be the fourth most popular development environment.

Virtual Machine
VirtualBox : Oracle VM VirtualBox (formerly Sun VirtualBox, Sun xVM VirtualBox and Innotek VirtualBox) is a free and open-source hypervisor for x86 computers currently being developed by Oracle Corporation. Developed initially by Innotek GmbH, it was acquired by Sun Microsystems in 2008 which was in turn acquired by Oracle in 2010. VirtualBox may be installed on a number of host operating systems, including: Linux, macOS, Windows, Solaris, and OpenSolaris. There are also ports to FreeBSD and Genode. It supports the creation and management of guest virtual machines running versions and derivations of Windows, Linux, BSD, OS/2, Solaris, Haiku, OSx86 and others, and limited virtualization of macOS guests on Apple hardware. For some guest operating systems, a "Guest Additions" package of device drivers and system applications is available which typically improves performance, especially of graphics.

VMWare : VMware, Inc. is a subsidiary of Dell Technologies that provides cloud computing and platform virtualization software and services. It was the first commercially successful company to virtualize the x86 architecture. VMware's desktop software runs on Microsoft Windows, Linux, and macOS, while its enterprise software hypervisor for servers, VMware ESXi, is a bare-metal hypervisor that runs directly on server hardware without requiring an additional underlying operating system.

\section{Vim}

Vim
Running Vim for the First Time
To start Vim, enter this command:

gvim file.txt
In UNIX you can type this at any command prompt. If you are running Microsoft Windows, open an MS-DOS prompt window and enter the command. In either case, Vim starts editing a file called file.txt. Because this is a new file, you get a blank window. This is what your screen will look like:

+---------------------------------------+
|#                  |
|~                  |
|~                  |
|~                  |
|~                  |
|"file.txt" [New file]          |
+---------------------------------------+
    ('#" is the cursor position.)
The tilde (~) lines indicate lines not in the file. In other words, when Vim runs out of file to display, it displays tilde lines. At the bottom of the screen, a message line indicates the file is named file.txt and shows that you are creating a new file. The message information is temporary and other information overwrites it.

THE VIM COMMAND

The gvim command causes the editor to create a new window for editing. If you use this command:

vim file.txt
the editing occurs inside your command window. In other words, if you are running inside an xterm, the editor uses your xterm window. If you are using an MS-DOS command prompt window under Microsoft Windows, the editing occurs inside this window. The text in the window will look the same for both versions, but with gvim you have extra features, like a menu bar. More about that later.

Inserting text
The Vim editor is a modal editor. That means that the editor behaves differently, depending on which mode you are in. The two basic modes are called Normal mode and Insert mode. In Normal mode the characters you type are commands. In Insert mode the characters are inserted as text. Since you have just started Vim it will be in Normal mode. To start Insert mode you type the "i" command (i for Insert). Then you can enter the text. It will be inserted into the file. Do not worry if you make mistakes; you can correct them later. To enter the following programmer's limerick, this is what you type:

iA very intelligent turtle
Found programming UNIX a hurdle
After typing "turtle" you press the key to start a new line. Finally you press the key to stop Insert mode and go back to Normal mode. You now have two lines of text in your Vim window:

+---------------------------------------+
|A very intelligent turtle      |
|Found programming UNIX a hurdle    |
|~                  |
|~                  |
|                   |
+---------------------------------------+
WHAT IS THE MODE?

To be able to see what mode you are in, type this command:

:set showmode
You will notice that when typing the colon Vim moves the cursor to the last line of the window. That's where you type colon commands (commands that start with a colon). Finish this command by pressing the <Enter> key (all commands that start with a colon are finished this way). Now, if you type the "i" command Vim will display --INSERT-- at the bottom of the window. This indicates you are in Insert mode.

+---------------------------------------+
|A very intelligent turtle      |
|Found programming UNIX a hurdle    |
|~                  |
|~                  |
|-- INSERT --               |
+---------------------------------------+
If you press <Esc> to go back to Normal mode the last line will be made blank.

GETTING OUT OF TROUBLE

One of the problems for Vim novices is mode confusion, which is caused by forgetting which mode you are in or by accidentally typing a command that switches modes. To get back to Normal mode, no matter what mode you are in, press the key. Sometimes you have to press it twice. If Vim beeps back at you, you already are in Normal mode.

==============================================================================

Moving around
After you return to Normal mode, you can move around by using these keys:

h   left                        *hjkl*
j   down
k   up
l   right
At first, it may appear that these commands were chosen at random. After all, who ever heard of using l for right? But actually, there is a very good reason for these choices: Moving the cursor is the most common thing you do in an editor, and these keys are on the home row of your right hand. In other words, these commands are placed where you can type them the fastest (especially when you type with ten fingers).

Note:
You can also move the cursor by using the arrow keys.  If you do,
however, you greatly slow down your editing because to press the arrow
keys, you must move your hand from the text keys to the arrow keys.
Considering that you might be doing it hundreds of times an hour, this
can take a significant amount of time.
   Also, there are keyboards which do not have arrow keys, or which
locate them in unusual places; therefore, knowing the use of the hjkl
keys helps in those situations.
One way to remember these commands is that h is on the left, l is on the right and j points down. In a picture:

           k
       h     l
         j
The best way to learn these commands is by using them. Use the "i" command to insert some more lines of text. Then use the hjkl keys to move around and insert a word somewhere. Don't forget to press to go back to Normal mode. The |vimtutor| is also a nice way to learn by doing.

For Japanese users, Hiroshi Iwatani suggested using this:

        Komsomolsk
            ^
            |
   Huan Ho  <--- --->  Los Angeles
(Yellow river)      |
            v
          Java (the island, not the programming language)
==============================================================================

Deleting characters
To delete a character, move the cursor over it and type "x". (This is a throwback to the old days of the typewriter, when you deleted things by typing xxxx over them.) Move the cursor to the beginning of the first line, for example, and type xxxxxxx (seven x's) to delete "A very ". The result should look like this:

+---------------------------------------+
|intelligent turtle         |
|Found programming UNIX a hurdle    |
|~                  |
|~                  |
|                   |
+---------------------------------------+
Now you can insert new text, for example by typing:

iA young <Esc>
This begins an insert (the i), inserts the words "A young", and then exits insert mode (the final ). The result:

+---------------------------------------+
|A young intelligent turtle     |
|Found programming UNIX a hurdle    |
|~                  |
|~                  |
|                   |
+---------------------------------------+
DELETING A LINE

To delete a whole line use the "dd" command. The following line will then move up to fill the gap:

+---------------------------------------+
|Found programming UNIX a hurdle    |
|~                  |
|~                  |
|~                  |
|                   |
+---------------------------------------+
DELETING A LINE BREAK

In Vim you can join two lines together, which means that the line break between them is deleted. The "J" command does this. Take these two lines:

A young intelligent
turtle
Move the cursor to the first line and press "J":

A young intelligent turtle
==============================================================================

Undo and Redo
Suppose you delete too much. Well, you can type it in again, but an easier way exists. The "u" command undoes the last edit. Take a look at this in action: After using "dd" to delete the first line, "u" brings it back. Another one: Move the cursor to the A in the first line:

A young intelligent turtle
Now type xxxxxxx to delete "A young". The result is as follows:

 intelligent turtle
Type "u" to undo the last delete. That delete removed the g, so the undo restores the character.

g intelligent turtle
The next u command restores the next-to-last character deleted:

ng intelligent turtle
The next u command gives you the u, and so on:

ung intelligent turtle
oung intelligent turtle
young intelligent turtle
 young intelligent turtle
A young intelligent turtle

Note:
If you type "u" twice, and the result is that you get the same text
back, you have Vim configured to work Vi compatible.  Look here to fix
this: |not-compatible|.
   This text assumes you work "The Vim Way".  You might prefer to use
the good old Vi way, but you will have to watch out for small
differences in the text then.
REDO

If you undo too many times, you can press CTRL-R (redo) to reverse the preceding command. In other words, it undoes the undo. To see this in action, press CTRL-R twice. The character A and the space after it disappear:

young intelligent turtle
There's a special version of the undo command, the "U" (undo line) command. The undo line command undoes all the changes made on the last line that was edited. Typing this command twice cancels the preceding "U".

A very intelligent turtle
  xxxx              Delete very

A intelligent turtle
          xxxxxx        Delete turtle

A intelligent
                Restore line with "U"
A very intelligent turtle
                Undo "U" with "u"
A intelligent
The "U" command is a change by itself, which the "u" command undoes and CTRL-R redoes. This might be a bit confusing. Don't worry, with "u" and CTRL-R you can go to any of the situations you had. More about that in section |32.2|.

Reference: http://vimdoc.sourceforge.net/htmldoc/usr_02.html

\section{Virtual Box}

Virtual Box
Export and Import VirtualBox VM images?
Export
Open VirtualBox and enter into the File option to choice Export Appliance...



You will then get an assistance window to help you generating the image.

Select the VM to export
Enter the output file path and name


You can choice a format, which I always leave the default OVF 1.

Finally you can write metadata like Version and Description
Now you have an OVA file that you can carry to whatever machine to use it.

Import
Open VirtualBox and enter into the File option to choice Import

You will then get an assistance window to help you loading the image.

Enter the path to the file that you have previously exported


Then you can modify the settings of the VM like RAM size, CPU, etc.


My recommendation on this is to enable the Reinitialize the MAC address of all the network cards option

Press Import and done!
Now you have cloned the VM from the host machine into another one

Reference: https://askubuntu.com/questions/588426/how-to-export-and-import-virtualbox-vm-images

Install Guest Additions
Guest Additions installs on the guest system and includes device drivers and system applications that optimize performance of the machine. Launch the guest OS in VirtualBox and click on Devices and Install Guest Additions.



The AutoPlay window opens on the guest OS and click on the Run VBox Windows Additions executable.



Click yes when the UAC screen comes up.



Now simply follow through the installation wizard.



During the installation wizard you can choose the Direct3D acceleration if you would like it. Remember this is going to take up more of your Host OS’s resources and is still experimental possibly making the guest unstable.



When the installation starts you will need to allow the Sun display adapters to be installed.



After everything has completed a reboot is required.

\section{VMWare}

VMWare
VMware Workstation is a program that allows you to run a virtual computer within your physical computer. The virtual computer runs as if it was its own machine. A virtual machine is great for trying out new operating systems such as Linux, visiting websites you don't trust, creating a computing environment specifically for children, testing the effects of computer viruses, and much more. You can even print and plug in USB drives. Read this guide to get the most out of VMware Workstation.

Installing VMware Workstation


1. Make sure your computer meets the system requirements. Because you will be running an operating system from within your own operating system, VMware Workstation has fairly high system requirements. If you don’t meet these, you may not be able to run VMware effectively. You must have a 64-bit processor. VMware supports Windows and Linux operating systems. You must have enough memory to run your operating system, the virtual operating system, and any programs inside that operating system. 1 GB is the minimum, but 3 or more is recommended. You must have a 16-bit or 32-bit display adapter. 3D effects will most likely not work well inside the virtual operating system, so gaming is not always efficient. You need at least 1.5 GB of free space to install VMware Workstation, along with at least 1 GB per operating system that you install.



2. Download the VMware software. You can download the VMware installer from the Download Center on the VMware website. Select the newest version and click the link for the installer. You will need to login with your VMware username. You will be asked to read and review the license agreement before you can download the file. You can only have one version of VMware Workstation installed at a time.



3. Install VMware Workstation. Once you have downloaded the file, right-click on the file and select “Run as administrator”. You will be asked to review the license again. Most users can use the Typical installation option. At the end of the installation, you will be prompted for your license key. Once the installation is finished, restart the computer. Part

Installing an Operating System


1. Open VMware. Installing a virtual operating system is much like installing it on a regular PC. You will need to have the installation disc or ISO image as well as any necessary licenses for the operating system that you want to install.

You can install most distributions of Linux as well as any version of Windows.


2. Click File. Select New Virtual Machine and then choose Typical. VMware will prompt you for the installation media. If it recognizes the operating system, it will enable Easy Installation:

Physical disc – Insert the installation disc for the operating system you want to install and then select the drive in VMware.
ISO image – Browse to the location of the ISO file on your computer.
Install operating system later. This will create a blank virtual disk. You will need to manually install the operating system later.


3. Enter in the details for the operating system. For Windows and other licensed operating systems, you will need to enter your product key. You will also need to enter your preferred username and a password if you want one. * If you are not using Easy Install, you will need to browse the list for the operating system you are installing.



4. Name your virtual machine. The name will help you identify it on your physical computer. It will also help distinguish between multiple virtual computers running different operating systems.



5. Set the disk size. You can allocate any amount of free space on your computer to the virtual machine to act as the installed operating system’s hard drive. Make sure to set enough to install any programs that you want to run in the virtual machine.



6. Customize your virtual machine’s virtual hardware. You can set the virtual machine to emulate specific hardware by clicking the “Customize Hardware” button. This can be useful if you are trying to run an older program that only supports certain hardware. Setting this is optional.



7. Set the virtual machine to start. Check the box labeled “Power on this virtual machine after creation” if you want the virtual machine to start up as soon as you finish making it. If you don’t check this box, you can select your virtual machine from the list in VMware and click the Power On button.



8. Wait for your installation to complete. Once you’ve powered on the virtual machine for the first time, the operating system will begin to install automatically. If you provided all of the correct information during the setup of the virtual machine, then you should not have to do anything. If you didn’t enter your product key or create a username during the virtual machine setup, you will most likely be prompted during the installation of the operating system.



9. Check that VMware Tools is installed. Once the operating system is installed, the program VMware Tools should be automatically installed. Check that it appears on the desktop or in the program files for the newly installed operating system.

VMware tools are configuration options for your virtual machine, and keeps your virtual machine up to date with any software changes.

Navigating VMware


1. Start a virtual machine. To start a virtual machine, click the VM menu and select the virtual machine that you want to turn on. You can choose to start the virtual machine normally, or boot directly to the virtual BIOS.



2. Stop a virtual machine. To stop a virtual machine, select it and then click the VM menu. Select the Power option.

Power Off – The virtual machine turns off as if the power was cut out.
Shut Down Guest – This sends a shutdown signal to the virtual machine which causes the virtual machine to shut down as if you had selected the shutdown option.
You can also turn off the virtual machine by using the shutdown option in the virtual operating system.


3. Move files between the virtual machine and your physical computer. Moving files between your computer and the virtual machine is as simple as dragging and dropping. Files can be moved in both directions between the computer and the virtual machine, and can also be dragged from one virtual machine to another.

When you drag and drop, the original will stay in the original location and a copy will be created in the new location.
You can also move files by copying and pasting.
Virtual machines can connect to shared folders as well.


4. Add a printer to your virtual machine. You can add any printer to your virtual machine without having to install any extra drivers, as long as it is already installed on your host computer.

Select the virtual machine that you want to add the printer to.
Click the VM menu and select Settings.
Click the Hardware tab, and then click Add. This will start the Add Hardware wizard.
Select Printer and then click Finish. Your virtual printer will be enabled the next time you turn the virtual machine on.


5. Connect a USB drive to the virtual machine. Virtual machines can interact with a USB drive the same way that your normal operating system does. The USB drive cannot be accessed on both the host computer and the virtual machine at the same time.

If the virtual machine is the active window, the USB drive will be automatically connected to the virtual machine when it is plugged in.
If the virtual machine is not the active window or is not running, select the virtual machine and click the VM menu. Select Removable Devices and then click Connect. The USB drive will automatically connect to your virtual machine.


6. Take a snapshot of a virtual machine. A snapshot is a saved state and will allow you to load the virtual machine to that precise moment as many times as you need.

Select your virtual machine, click the VM menu, hover over Snapshot and select Take Snapshot.
Give your Snapshot a name. You can also give it a description, though this is optional.
Click OK to save the Snapshot.
Load a saved Snapshot by clicking the VM menu and then selecting Snapshot. Choose the Snapshot you wish to load from the list and click Go To.


7. Become familiar with keyboard shortcuts. A combination of the "Ctrl" and other keys are used to navigate virtual machines. For example, "Ctrl," "Alt" and "Enter" puts the current virtual machine in full screen mode or moves through multiple machines. "Ctrl," "Alt" and "Tab" will move between virtual machines when the mouse is being used by 1 machine.



  \bibliography{mybib}{}
  \bibliographystyle{ksfh_nat}
  \addcontentsline{toc}{part}{Tài liệu}

  \printindex
  \addcontentsline{toc}{part}{Chỉ mục}

  \listoftodos[Ghi chú]
  \addcontentsline{toc}{part}{Ghi chú}

\end{document}